% !Mode:: "TeX:UTF-8"
\documentclass{article}
\input{../../en_preamble.tex}
\input{../../xecjk_preamble.tex}
\usepackage{algorithm}
\usepackage{algorithmic}
\begin{document}
\title{液滴模型建立}
\author{王鹏祥}
\date{2019.7.14}
\maketitle

		\section{preface}
		这星期因为老师回来了,所以把之前对变分问题的建立给整理了一下。
		\section{水滴模型}
		\begin{equation*}
		F=\gamma_{LV} A_{LV}+(\gamma_{SL}-\gamma_{SV})A_{SL}
		\end{equation*}
		其中$\gamma_{LV}, \gamma_{SL},\gamma_{SV}$分别代表的是液—气,固—液,固—气表面之间的张力。 $A_{LV},A_{SL}$表示的是液—气,固—液之间的接触面积。F为最小表面自由能能量泛函。
        
        原式可变为
        \begin{equation*}
        F=\gamma_{LV} A_{LV}+\cos \theta A_{SL}\gamma_{LV}
        \end{equation*}
        $\theta$为接触角,对于给定的材料系统而言,是一个固定值
\begin{figure}[H]
	\centering
	\includegraphics[height=6.0cm,width=9.5cm]{figures/2-1.png}
	\caption{半圆内表示液体,阴影表示固体,半圆外表示气体}
\end{figure}

\section{二维变分问题建立}
如图1建立坐标系,液滴与y轴交点为(0,h),与x轴交点为(-a,0),(a,0),曲线方程为y=f(x)。

那么$A_{LV}$为一条弧线,其数值为$\int^{a}_{-a} \sqrt{1+y'}dx$

$A_{SL}$为一条直线,长度为2a

液滴的面积为固定值$\int_{-a}^{a}ydx$

原方程变为
\begin{equation*}
\varPi=\gamma_{LV} \int^{a}_{-a} \sqrt{1+y'}dx - 2a\cos \theta \gamma_{LV}
\end{equation*}

约束条件为
\begin{equation*}
\left\{
\begin{aligned}
y(-a)    & =  0 \\
y(a)     & =  0 \\
\oint ds & =  \int_{-a}^{a}ydx = V
\end{aligned}
\right.
\end{equation*}

其变分问题可以描述为,找出一个y,使得在给定面积条件下,使得液滴的自由能达到最小
\begin{equation}
\varPi=\gamma_{LV} \int^{a}_{-a} \sqrt{1+y'(t)}x'(t)dt - 2a\cos \theta \gamma_{LV}+\lambda(\int_{-a}^{a}y(t)x'(t)dt-V)
\end{equation}

\section{三维变分问题建立}
$A_{LV}$为一个近似于半球的的表面积,可看成曲线f(x)绕y轴转一圈所形成的的图形,其数值为$\int^{h}_{0} 2\pi x \sqrt{1+y'}dy$

$A_{SL}$为半径为a的圆的面积$\pi a^{2}$

液滴的体积为固定值$\int_{0}^{h}\pi x^{2}dx$

原方程变为
\begin{equation*}
\varPi=\gamma_{LV} \int^{h}_{0} 2\pi x \sqrt{1+y'}dy - \pi a^{2}\gamma_{LV}\cos \theta 
\end{equation*}
约束条件为
\begin{equation*}
\left\{
\begin{aligned}
y(-a)    & =  0 \\
y(a)     & =  0 \\
y(0)     & =  h \\
\oint ds & =  \int_{0}^{h}\pi x^{2}dx = V
\end{aligned}
\right.
\end{equation*}
其变分问题可以描述为,找出一个y,使得在给定体积条件下,使得液滴的自由能达到最小
\begin{equation}
\varPi=\gamma_{LV} \int^{h}_{0} 2\pi x \sqrt{1+y'}dy - \pi a^{2}\gamma_{LV}\cos \theta +\lambda(\int_{0}^{h}\pi x^{2}dx-V)
\end{equation}

\subsection{对二维问题的变分求解}
通过变分公式
\begin{equation*}
\begin{aligned} \delta \Pi &=\prod(y+\delta y)-\prod(y) \\ &=\int_{\alpha}^{\beta}\left\{F\left(x, y+\delta y, y^{\prime}+\delta y^{\prime}\right)-F\left(x, y, y^{\prime}\right)\right\} \mathrm{d} x \end{aligned}
\end{equation*}
将(1)式进行变分
\begin{align*}
\delta \Pi=&\gamma_{LV}(\int_{0}^{\alpha}\sqrt{1+\left(y^{\prime}(t)+\delta y^{\prime}(t)\right)^{2}} x'(t)\mathrm{d} t-\int_{0}^{\alpha}\sqrt{1+y^{\prime 2}(t)} x'(t)\mathrm{d} t) - 2a\cos \theta \gamma_{LV}\\&+\delta\lambda(\int_{-a}^{a}y(t)x'(t)dt-V)
 +\pi\lambda(\int_{-a}^{a}\delta y(t)x'(t)dt)
\end{align*}
\section{体会}\setlength{\parindent}{2em}
我也试着对(1)(2)式子进行泛函的变分,仔细看了那篇论文,但感觉论文里有好多东西是错的,有一些自己也没搞懂,依葫芦画瓢,对二维问题进行了变分,但是感觉不对,自己也计算往下计算了一下,但结果很明显的不对,估计是我自己的哪里推错了(可能一开始就错了)。那本书也继续往下看了一些,知道了很多算子,感觉微分几何中算子是真的多,符号真的多,又是方块又是三角的,很多依旧看不明白,只能理解个大概吧。

\end{document} 