% !Mode:: "TeX:UTF-8"
\documentclass{article}
\input{../../en_preamble.tex}
\input{../../xecjk_preamble.tex}
\usepackage{algorithm}
\usepackage{algorithmic}
\begin{document}
\title{湿润现象}
\author{王鹏祥}
\date{2019.7.2}
\maketitle

		\section{研究模型}\setlength{\parindent}{2em}
		液滴滴在一个表面上,达到平衡状态时能量的变化
		\begin{align}
		F=\gamma_{LV}A_{LV}+(\gamma_{SL}-\gamma{SV})A_{SL}\\
		\end{align}
		其中$\gamma_{LV}\,\gamma_{SL}\,\gamma{SV}$分别表示固-气,固-液和液-气表面张力\\
		A距离,面积\\
		F极小泛函能量
		
	   对于液滴问题上式改写为
	   \begin{align}
	   F/\gamma_{LV}=A_{LV}-\cos\theta A_{SL}
	   \end{align}
	   且带有约束条件
	   $\oint dv=v$(体积不变)
		\section{一些定义和概念}
		\subsection{湿润}
		1.固体(或液体)表面上气体被液体(或另一种互不相容液体)取代的现象.\\ 
		  液体与固体表面接触的能力.
		  
		2.一定温度和压力下,湿润的程度可用湿润的过程的吉布斯函数(吉布斯自由能)来衡量。吉布斯函数越低越多,越易湿润。
		
		3.吉布斯自由能:系统减少的内能中可以转化为对外做功的部分\\
		\begin{align}
		G = U - TS + pV = H - TS
		\end{align}
		其中U是系统的内能,T是温度(绝对温度,K),S是熵,p是压强,V是体积,H是焓。
		吉布斯自由能的微分形式是:
		\begin{align}
     	dG = - SdT + Vdp + \mu dN
		\end{align}
		$\mu$为化学势
		
		4.化学势:化学势是物理内容丰富的热力学强度量,化学势是表征系统与媒质,或系统相与相之间,或系统组元之间粒子转移的趋势。粒子总是从高化学势向低化学势区域、相或组元转移,直到两者相等才相互处于化学平衡。
		\subsection{一些性质}
		1.同向异性:如果各个方向的测量结果是相同的,说明其物理性质与取向无关,就称为各向同性。如果物理性质和取向密切相关,不同取向的测量结果迥异,就称为各向异性。
		
		2.刚性:指两个物体相碰撞不会发生变形
		
		\subsection{接触角}
		1.当系统达到平衡时,在气液固三相交界点,气-液与气-固界面张立之间的夹角.\\
		 
\begin{figure}[H]
\centering
\includegraphics[height=6.0cm,width=9.5cm]{figures/1-1.png}
\caption{}
\end{figure}

       2.yong方程(湿润方程)\\
       
\begin{figure}[H]
	\centering
	\includegraphics[height=6.0cm,width=9.5cm]{figures/1-2.png}
	\label{fig:2}
\end{figure}
        
        3.接触角滞后:实固体表面在一定程度上或者粗糙不平或者化学组成不均一,这就使得实际物体表面上的接触角并非如Young 方程所预示的取值唯一。而是在相对稳定的两个角度之间变化,这种现象被称为接触角滞后现象。
\section{一些问题和体会}\setlength{\parindent}{2em}
首先最大的问题是不知道最小表面自由能量泛函是如何建立的,可能是由于还没有看泛函和变分原理导引导致的。关于GitHub还有很多需要学习的,比如一些Markdowm语言,还有起一些先关操作。

有几个我在学习中发现的不错的网址

1.关于湿润模型的ppt,http://www.doc88.com/p-4127669482377.html

2.关于GitHub的https://segmentfault.com/blog/stormzhang。

3.关于Markdown语言 https://www.jianshu.com/p/q81RER。

4.两份关于湿润的文献  湿润现象的附着力与内聚力 庞礼军\\

\end{document} 