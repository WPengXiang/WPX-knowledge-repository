\documentclass{article}

%%%%%%%%------------------------------------------------------------------------
%%%% 日常所用宏包

%% 控制页边距
% 如果是beamer文档类, 则不用geometry
\makeatletter
\@ifclassloaded{beamer}{}{\usepackage[top=2.5cm, bottom=2.5cm, left=2.5cm, right=2.5cm]{geometry}}
\makeatother

\makeatletter
\@ifclassloaded{beamer}{
\makeatletter
\def\th@mystyle{%
    \normalfont % body font
    \setbeamercolor{block title example}{bg=orange,fg=white}
    \setbeamercolor{block body example}{bg=orange!20,fg=black}
    \def\inserttheoremblockenv{exampleblock}
  }
\makeatother
\theoremstyle{mystyle}
\newtheorem*{remark}{Remark}

\newcommand{\propnumber}{} % initialize
\newtheorem*{prop}{Proposition \propnumber}
\newenvironment{propc}[1]
  {\renewcommand{\propnumber}{#1}%
   \begin{shaded}\begin{prop}}
  {\end{prop}\end{shaded}}

\makeatletter
\newenvironment<>{proofs}[1][\proofname]{%
    \par
    \def\insertproofname{#1\@addpunct{.}}%
    \usebeamertemplate{proof begin}#2}
  {\usebeamertemplate{proof end}}
\makeatother

}{
}
\makeatother
\usepackage{amsthm}

%\DeclareMathOperator{\sech}{sech}
%\DeclareMathOperator{\csch}{csch}
%\DeclareMathOperator{\arcsec}{arcsec}
%\DeclareMathOperator{\arccot}{arccot}
%\DeclareMathOperator{\arccsc}{arccsc}
%\DeclareMathOperator{\arccosh}{arccosh}
%\DeclareMathOperator{\arcsinh}{arcsinh}
%\DeclareMathOperator{\arctanh}{arctanh}
%\DeclareMathOperator{\arcsech}{arcsech}
%\DeclareMathOperator{\arccsch}{arccsch}
%\DeclareMathOperator{\arccoth}{arccoth}
%% 控制项目列表
\usepackage{enumerate}

%% Todo list
\usepackage{enumitem}
\newlist{todolist}{itemize}{2}
\setlist[todolist]{label=$\square$}
\usepackage{pifont}
\newcommand{\cmark}{\ding{51}}%
\newcommand{\xmark}{\ding{55}}%
\newcommand{\done}{\rlap{$\square$}{\raisebox{2pt}{\large\hspace{1pt}\cmark}}%
\hspace{-2.5pt}}
\newcommand{\wontfix}{\rlap{$\square$}{\large\hspace{1pt}\xmark}}

\usepackage[utf8]{inputenc}
\usepackage[english]{babel}

\usepackage{framed}

%% 多栏显示
\usepackage{multicol}

%% 算法环境
\usepackage{algorithm}
\usepackage{algorithmic}
\usepackage{float}

%% 网址引用
\usepackage{url}

%% 控制矩阵行距
\renewcommand\arraystretch{1.4}

%% 粗体
\usepackage{lmodern}
\usepackage{bm}


%% hyperref宏包,生成可定位点击的超链接,并且会生成pdf书签
\makeatletter
\@ifclassloaded{beamer}{
\usepackage{hyperref}
\usepackage{ragged2e} % 对齐
}{
\usepackage[%
    pdfstartview=FitH,%
    CJKbookmarks=true,%
    bookmarks=true,%
    bookmarksnumbered=true,%
    bookmarksopen=true,%
    colorlinks=true,%
    citecolor=blue,%
    linkcolor=blue,%
    anchorcolor=green,%
    urlcolor=blue%
]{hyperref}
}
\makeatother



\makeatletter % 如果是 beamer 不需要下面两个包
\@ifclassloaded{beamer}{
\mode<presentation>
{
}
}{
%% 控制标题
\usepackage{titlesec}
%% 控制目录
\usepackage{titletoc}
}
\makeatother

%% 控制表格样式
\usepackage{booktabs}

%% 控制字体大小
\usepackage{type1cm}

%% 首行缩进,用\noindent取消某段缩进
\usepackage{indentfirst}

%% 支持彩色文本、底色、文本框等
\usepackage{color,xcolor}

%% AMS LaTeX宏包: http://zzg34b.w3.c361.com/package/maths.htm#amssymb
\usepackage{amsmath,amssymb}
%% 多个图形并排
\usepackage{subfig}
%%%% 基本插图方法
%% 图形宏包
\usepackage{graphicx}


%%%% 基本插图方法结束

%%%% pgf/tikz绘图宏包设置
\usepackage{pgf,tikz}
\usetikzlibrary{shapes,automata,snakes,backgrounds,arrows}
\usetikzlibrary{mindmap}
%% 可以直接在latex文档中使用graphviz/dot语言,
%% 也可以用dot2tex工具将dot文件转换成tex文件再include进来
%% \usepackage[shell,pgf,outputdir={docgraphs/}]{dot2texi}
%%%% pgf/tikz设置结束


\makeatletter % 如果是 beamer 不需要下面两个包
\@ifclassloaded{beamer}{

}{
%%%% fancyhdr设置页眉页脚
%% 页眉页脚宏包
\usepackage{fancyhdr}
%% 页眉页脚风格
\pagestyle{plain}
}

%% 有时会出现\headheight too small的warning
\setlength{\headheight}{15pt}

%% 清空当前页眉页脚的默认设置
%\fancyhf{}
%%%% fancyhdr设置结束


%% 设置listings宏包的一些全局样式
%% 参考http://hi.baidu.com/shawpinlee/blog/item/9ec431cbae28e41cbe09e6e4.html
\usepackage{listings}
\lstloadlanguages{[LaTeX]TeX}

\usepackage{fancyvrb}

\newenvironment{latexample}[1][language={[LaTeX]TeX}]
{\lstset{breaklines=true,
    prebreak = \raisebox{0ex}[0ex][0ex]{\ensuremath{\hookleftarrow}},
    frame=single,
    language={[LaTeX]TeX},
    showstringspaces=false,              %% 设定是否显示代码之间的空格符号
    numbers=left,                        %% 在左边显示行号
    numberstyle=\tiny,                   %% 设定行号字体的大小
    basicstyle=\scriptsize,                    %% 设定字体大小\tiny, \small, \Large等等
    keywordstyle=\color{blue!70}, commentstyle=\color{red!50!green!50!blue!50},
                                         %% 关键字高亮
    frame=shadowbox,                     %% 给代码加框
    rulesepcolor=\color{red!20!green!20!blue!20},
    escapechar=`,                        %% 中文逃逸字符,用于中英混排
    xleftmargin=2em,xrightmargin=2em, aboveskip=1em,
    %breaklines,                          %% 这条命令可以让LaTeX自动将长的代码行换行排版
    extendedchars=false                  %% 这一条命令可以解决代码跨页时,章节标题,页眉等汉字不显示的问题
    basicstyle=\footnotesize\ttfamily, #1}
  \VerbatimEnvironment\begin{VerbatimOut}{latexample.verb.out}}
  {\end{VerbatimOut}\noindent
  \begin{minipage}{1.05\linewidth}
    \lstinputlisting[]{latexample.verb.out}%
  \end{minipage}\qquad
  \begin{minipage}{1\linewidth}
    \input{latexample.verb.out}
  \end{minipage}\\
}

\usepackage{minted}
\renewcommand{\listingscaption}{Python code} \newminted{python}{
    escapeinside=||,
    mathescape=true,
    numbersep=5pt,
    linenos=true,
    autogobble,
    framesep=3mm}
%%%% listings宏包设置结束


%%%% 附录设置
\makeatletter % 对 beamer 要重新设置
\@ifclassloaded{beamer}{

}{
\usepackage[title,titletoc,header]{appendix}
}
\makeatother
%%%% 附录设置结束





%% 设定行距
\linespread{1}

%% 颜色
\newcommand{\red}{\color{red} }
\newcommand{\blue}{\color{blue} }
\newcommand{\brown}{\color{brown} }
\newcommand{\green}{\color{green} }

\newcommand{\bred}{\bf\color{red} }
\newcommand{\bblue}{\bf\color{blue} }
\newcommand{\bbrown}{\bf\color{brown} }
\newcommand{\bgreen}{\bf\color{green} }
%% 1. 小写的英文或希腊字母表示 标量或标量函数
%% 2. 大写的英文或希腊字母表示 集合或空间
%% 3. 粗体的小写字母代表向量或向量形式的常量和函数
%% 4. 粗体的大写字母代表矩阵或张量形式的常量和函数
%% 5. 空心大写字母代表特殊的空间 \mbR 实数 \mbC 复数 \mbP 多项式
%% 6. 花体的大写字母代表算子

%% 粗体的小写字母代表向量或向量函数
\newcommand{\bfa}{{\boldsymbol a}}
\newcommand{\bfb}{{\boldsymbol b}}
\newcommand{\bfc}{{\boldsymbol c}}
\newcommand{\bfd}{{\boldsymbol d}}
\newcommand{\bfe}{{\boldsymbol e}}
\newcommand{\bff}{{\boldsymbol f}}
\newcommand{\bfg}{{\boldsymbol g}}
\newcommand{\bfh}{{\boldsymbol h}}
\newcommand{\bfi}{{\boldsymbol i}}
\newcommand{\bfj}{{\boldsymbol j}}
\newcommand{\bfk}{{\boldsymbol k}}
\newcommand{\bfl}{{\boldsymbol l}}
\newcommand{\bfm}{{\boldsymbol m}}
\newcommand{\bfn}{{\boldsymbol n}}
\newcommand{\bfo}{{\boldsymbol o}}
\newcommand{\bfp}{{\boldsymbol p}}
\newcommand{\bfq}{{\boldsymbol q}}
\newcommand{\bfr}{{\boldsymbol r}}
\newcommand{\bfs}{{\boldsymbol s}}
\newcommand{\bft}{{\boldsymbol t}}
\newcommand{\bfu}{{\boldsymbol u}}
\newcommand{\bfv}{{\boldsymbol v}}
\newcommand{\bfw}{{\boldsymbol w}}
\newcommand{\bfx}{{\boldsymbol x}}
\newcommand{\bfy}{{\boldsymbol y}}
\newcommand{\bfz}{{\boldsymbol z}}

%  算子
\newcommand{\mca}{{\mathcal a}}
\newcommand{\mcb}{{\mathcal b}}
\newcommand{\mcc}{{\mathcal c}}
\newcommand{\mcd}{{\mathcal d}}
\newcommand{\mce}{{\mathcal e}}
\newcommand{\mcf}{{\mathcal f}}
\newcommand{\mcg}{{\mathcal g}}
\newcommand{\mch}{{\mathcal h}}
\newcommand{\mci}{{\mathcal i}}
\newcommand{\mcj}{{\mathcal j}}
\newcommand{\mck}{{\mathcal k}}
\newcommand{\mcl}{{\mathcal l}}
\newcommand{\mcm}{{\mathcal m}}
\newcommand{\mcn}{{\mathcal n}}
\newcommand{\mco}{{\mathcal o}}
\newcommand{\mcp}{{\mathcal p}}
\newcommand{\mcq}{{\mathcal q}}
\newcommand{\mcr}{{\mathcal r}}
\newcommand{\mcs}{{\mathcal s}}
\newcommand{\mct}{{\mathcal t}}
\newcommand{\mcu}{{\mathcal u}}
\newcommand{\mcv}{{\mathcal v}}
\newcommand{\mcw}{{\mathcal w}}
\newcommand{\mcx}{{\mathcal x}}
\newcommand{\mcy}{{\mathcal y}}
\newcommand{\mcz}{{\mathcal z}}

% \rmd
\newcommand{\mra}{{\mathrm a}}
\newcommand{\mrb}{{\mathrm b}}
\newcommand{\mrc}{{\mathrm c}}
\newcommand{\mrd}{{\mathrm d}}
\newcommand{\mre}{{\mathrm e}}
\newcommand{\mrf}{{\mathrm f}}
\newcommand{\mrg}{{\mathrm g}}
\newcommand{\mrh}{{\mathrm h}}
\newcommand{\mri}{{\mathrm i}}
\newcommand{\mrj}{{\mathrm j}}
\newcommand{\mrk}{{\mathrm k}}
\newcommand{\mrl}{{\mathrm l}}
\newcommand{\mrm}{{\mathrm m}}
\newcommand{\mrn}{{\mathrm n}}
\newcommand{\mro}{{\mathrm o}}
\newcommand{\mrp}{{\mathrm p}}
\newcommand{\mrq}{{\mathrm q}}
\newcommand{\mrr}{{\mathrm r}}
\newcommand{\mrs}{{\mathrm s}}
\newcommand{\mrt}{{\mathrm t}}
\newcommand{\mru}{{\mathrm u}}
\newcommand{\mrv}{{\mathrm v}}
\newcommand{\mrw}{{\mathrm w}}
\newcommand{\mrx}{{\mathrm x}}
\newcommand{\mry}{{\mathrm y}}
\newcommand{\mrz}{{\mathrm z}}

%% 粗体的大写字母一般表示矩阵和张量
\newcommand{\bfA}{{\boldsymbol A}}
\newcommand{\bfB}{{\boldsymbol B}}
\newcommand{\bfC}{{\boldsymbol C}}
\newcommand{\bfD}{{\boldsymbol D}}
\newcommand{\bfE}{{\boldsymbol E}}
\newcommand{\bfF}{{\boldsymbol F}}
\newcommand{\bfG}{{\boldsymbol G}}
\newcommand{\bfH}{{\boldsymbol H}}
\newcommand{\bfI}{{\boldsymbol I}}
\newcommand{\bfJ}{{\boldsymbol J}}
\newcommand{\bfK}{{\boldsymbol K}}
\newcommand{\bfL}{{\boldsymbol L}}
\newcommand{\bfM}{{\boldsymbol M}}
\newcommand{\bfN}{{\boldsymbol N}}
\newcommand{\bfO}{{\boldsymbol O}}
\newcommand{\bfP}{{\boldsymbol P}}
\newcommand{\bfQ}{{\boldsymbol Q}}
\newcommand{\bfR}{{\boldsymbol R}}
\newcommand{\bfS}{{\boldsymbol S}}
\newcommand{\bfT}{{\boldsymbol T}}
\newcommand{\bfU}{{\boldsymbol U}}
\newcommand{\bfV}{{\boldsymbol V}}
\newcommand{\bfW}{{\boldsymbol W}}
\newcommand{\bfX}{{\boldsymbol X}}
\newcommand{\bfY}{{\boldsymbol Y}}
\newcommand{\bfZ}{{\boldsymbol Z}}

%% 花体大写字母
\newcommand{\mcA}{{\mathcal A}}
\newcommand{\mcB}{{\mathcal B}}
\newcommand{\mcC}{{\mathcal C}}
\newcommand{\mcD}{{\mathcal D}}
\newcommand{\mcE}{{\mathcal E}}
\newcommand{\mcF}{{\mathcal F}}
\newcommand{\mcG}{{\mathcal G}}
\newcommand{\mcH}{{\mathcal H}}
\newcommand{\mcI}{{\mathcal I}}
\newcommand{\mcJ}{{\mathcal J}}
\newcommand{\mcK}{{\mathcal K}}
\newcommand{\mcL}{{\mathcal L}}
\newcommand{\mcM}{{\mathcal M}}
\newcommand{\mcN}{{\mathcal N}}
\newcommand{\mcO}{{\mathcal O}}
\newcommand{\mcP}{{\mathcal P}}
\newcommand{\mcQ}{{\mathcal Q}}
\newcommand{\mcR}{{\mathcal R}}
\newcommand{\mcS}{{\mathcal S}}
\newcommand{\mcT}{{\mathcal T}}
\newcommand{\mcU}{{\mathcal U}}
\newcommand{\mcV}{{\mathcal V}}
\newcommand{\mcW}{{\mathcal W}}
\newcommand{\mcX}{{\mathcal X}}
\newcommand{\mcY}{{\mathcal Y}}
\newcommand{\mcZ}{{\mathcal Z}}

%% 空心大写字母
\newcommand{\mbA}{{\mathbb A}}
\newcommand{\mbB}{{\mathbb B}}
\newcommand{\mbC}{{\mathbb C}}
\newcommand{\mbD}{{\mathbb D}}
\newcommand{\mbE}{{\mathbb E}}
\newcommand{\mbF}{{\mathbb F}}
\newcommand{\mbG}{{\mathbb G}}
\newcommand{\mbH}{{\mathbb H}}
\newcommand{\mbI}{{\mathbb I}}
\newcommand{\mbJ}{{\mathbb J}}
\newcommand{\mbK}{{\mathbb K}}
\newcommand{\mbL}{{\mathbb L}}
\newcommand{\mbM}{{\mathbb M}}
\newcommand{\mbN}{{\mathbb N}}
\newcommand{\mbO}{{\mathbb O}}
\newcommand{\mbP}{{\mathbb P}}
\newcommand{\mbQ}{{\mathbb Q}}
\newcommand{\mbR}{{\mathbb R}}
\newcommand{\mbS}{{\mathbb S}}
\newcommand{\mbT}{{\mathbb T}}
\newcommand{\mbU}{{\mathbb U}}
\newcommand{\mbV}{{\mathbb V}}
\newcommand{\mbW}{{\mathbb W}}
\newcommand{\mbX}{{\mathbb X}}
\newcommand{\mbY}{{\mathbb Y}}
\newcommand{\mbZ}{{\mathbb Z}}

\newcommand{\mrA}{{\mathrm A}}
\newcommand{\mrB}{{\mathrm B}}
\newcommand{\mrC}{{\mathrm C}}
\newcommand{\mrD}{{\mathrm D}}
\newcommand{\mrE}{{\mathrm E}}
\newcommand{\mrF}{{\mathrm F}}
\newcommand{\mrG}{{\mathrm G}}
\newcommand{\mrH}{{\mathrm H}}
\newcommand{\mrI}{{\mathrm I}}
\newcommand{\mrJ}{{\mathrm J}}
\newcommand{\mrK}{{\mathrm K}}
\newcommand{\mrL}{{\mathrm L}}
\newcommand{\mrM}{{\mathrm M}}
\newcommand{\mrN}{{\mathrm N}}
\newcommand{\mrO}{{\mathrm O}}
\newcommand{\mrP}{{\mathrm P}}
\newcommand{\mrQ}{{\mathrm Q}}
\newcommand{\mrR}{{\mathrm R}}
\newcommand{\mrS}{{\mathrm S}}
\newcommand{\mrT}{{\mathrm T}}
\newcommand{\mrU}{{\mathrm U}}
\newcommand{\mrV}{{\mathrm V}}
\newcommand{\mrW}{{\mathrm W}}
\newcommand{\mrX}{{\mathrm X}}
\newcommand{\mrY}{{\mathrm Y}}
\newcommand{\mrZ}{{\mathrm Z}}


% 粗体的 Greek 字母
\newcommand{\balpha}{{\bm \alpha}}
\newcommand{\bbeta}{{\bm \beta}}
\newcommand{\bgamma}{{\bm \gamma}}
\newcommand{\bdelta}{{\bm \delta}}
\newcommand{\bepsilon}{{\bm \epsilon}}
\newcommand{\bvarepsilon}{{\bm \varepsilon}}
\newcommand{\bzeta}{{\bm \zeta}}
\newcommand{\bfeta}{{\bm \eta}}
\newcommand{\btheta}{{\bm \theta}}
\newcommand{\biota}{{\bm \iota}}
\newcommand{\bkappa}{{\bm \kappa}}
\newcommand{\blambda}{{\bm \lambda}}
\newcommand{\bmu}{{\bm \mu}}
\newcommand{\bnu}{{\bm \nu}}
\newcommand{\bxi}{{\bm \xi}}
\newcommand{\bomicron}{{\bm \omicron}}
\newcommand{\bpi}{{\bm \pi}}
\newcommand{\brho}{{\bm \rho}}
\newcommand{\bsigma}{{\bm \sigma}}
\newcommand{\btau}{{\bm \tau}}
\newcommand{\bupsilon}{{\bm \upsilon}}
\newcommand{\bphi}{{\bm \phi}}
\newcommand{\bvarphi}{{\bm \varphi}}
\newcommand{\bchi}{{\bm \chi}}
\newcommand{\bpsi}{{\bm \psi}}

\newcommand{\bAlpha}{{\bm \Alpha}}
\newcommand{\bBeta}{{\bm \Beta}}
\newcommand{\bGamma}{{\bm \Gamma}}
\newcommand{\bDelta}{{\bm \Delta}}
\newcommand{\bEpsilon}{{\bm \Psilon}}
\newcommand{\bVarepsilon}{{\bm \Varepsilon}}
\newcommand{\bZeta}{{\bm \Zeta}}
\newcommand{\bEta}{{\bm \Eta}}
\newcommand{\bTheta}{{\bm \Theta}}
\newcommand{\bIota}{{\bm \Iota}}
\newcommand{\bKappa}{{\bm \Kappa}}
\newcommand{\bLambda}{{\bm \Lambda}}
\newcommand{\bMu}{{\bm \Mu}}
\newcommand{\bNu}{{\bm \Nu}}
\newcommand{\bXi}{{\bm \Xi}}
\newcommand{\bOmicron}{{\bm \Omicron}}
\newcommand{\bPi}{{\bm \Pi}}
\newcommand{\bRho}{{\bm \Rho}}
\newcommand{\bSigma}{{\bm \Sigma}}
\newcommand{\bTau}{{\bm \Tau}}
\newcommand{\bUpsilon}{{\bm \Upsilon}}
\newcommand{\bPhi}{{\bm \Phi}}
\newcommand{\bChi}{{\bm \Chi}}
\newcommand{\bPsi}{{\bm \Psi}}

% \int_\Omega \bfx^2 \rmd \bfx
\newcommand{\rmd}{\,\mathrm d}
\newcommand{\bfzero}{\mathbf 0}

%% 算子
\newcommand{\ospan}{\operatorname{span}}
\newcommand{\odiv}{\operatorname{div}}
\newcommand{\otr}{\operatorname{tr}}
\newcommand{\ograd}{\operatorname{grad}}
\newcommand{\orot}{\operatorname{rot}}
\newcommand{\ocurl}{\operatorname{curl}}
\newcommand{\odist}{\operatorname{dist}}
\newcommand{\osign}{\operatorname{sign}}
\newcommand{\odiag}{\operatorname{diag}}
\newcommand{\oran}{\operatorname{Ran}} % 像空间
\newcommand{\oker}{\operatorname{Ker}} % 核空间
\newcommand{\ore}{\operatorname{Re}} % 实部
\newcommand{\oim}{\operatorname{Im}} % 虚部
\newcommand{\orank}{\operatorname{rank}}
\newcommand{\ovec}{\operatorname{vec}}
\newcommand{\odet}{\operatorname{det}}
\newcommand{\odim}{\operatorname{dim}}
\newcommand{\osym}{\operatorname{sym}}

\newcommand{\obcurl}{\operatorname{\bf curl}}
%%%% 个性设置结束
%%%%%%%%------------------------------------------------------------------------


%%%%%%%%------------------------------------------------------------------------
%%%% bibtex设置

%% 设定参考文献显示风格
% 下面是几种常见的样式
% * plain: 按字母的顺序排列,比较次序为作者、年度和标题
% * unsrt: 样式同plain,只是按照引用的先后排序
% * alpha: 用作者名首字母+年份后两位作标号,以字母顺序排序
% * abbrv: 类似plain,将月份全拼改为缩写,更显紧凑
% * apalike: 美国心理学学会期刊样式, 引用样式 [Tailper and Zang, 2006]

%\makeatletter
%\@ifclassloaded{beamer}{
%\bibliographystyle{apalike}
%}{
%\bibliographystyle{abbrv}
%}
%\makeatother


%%%% bibtex设置结束
%%%%%%%%------------------------------------------------------------------------

%%%%%%%%------------------------------------------------------------------------
%%%% xeCJK相关宏包

\usepackage{xltxtra, fontenc, xunicode}
\usepackage[slantfont, boldfont]{xeCJK}

\setlength{\parindent}{1.5em}%中文缩进两个汉字位

%% 针对中文进行断行
\XeTeXlinebreaklocale "zh"

%% 给予TeX断行一定自由度
\XeTeXlinebreakskip = 0pt plus 1pt minus 0.1pt

%%%% xeCJK设置结束
%%%%%%%%------------------------------------------------------------------------

%%%%%%%%------------------------------------------------------------------------
%%%% xeCJK字体设置

%% 设置中文标点样式,支持quanjiao、banjiao、kaiming等多种方式
\punctstyle{kaiming}

%% 设置缺省中文字体
\setCJKmainfont[BoldFont={Adobe Heiti Std}, ItalicFont={Adobe Kaiti Std}]{Adobe Song Std}
%\setCJKmainfont{Adobe Kaiti Std}
%% 设置中文无衬线字体
%\setCJKsansfont[BoldFont={Adobe Heiti Std}]{Adobe Kaiti Std}
%% 设置等宽字体
%\setCJKmonofont{Adobe Heiti Std}

%% 英文衬线字体
\setmainfont{DejaVu Serif}
%% 英文等宽字体
\setmonofont{DejaVu Sans Mono}
%% 英文无衬线字体
\setsansfont{DejaVu Sans}

%% 定义新字体
\setCJKfamilyfont{song}{Adobe Song Std}
\setCJKfamilyfont{kai}{Adobe Kaiti Std}
\setCJKfamilyfont{hei}{Adobe Heiti Std}
\setCJKfamilyfont{fangsong}{Adobe Fangsong Std}
\setCJKfamilyfont{lisu}{LiSu}
\setCJKfamilyfont{youyuan}{YouYuan}

%% 自定义宋体
\newcommand{\song}{\CJKfamily{song}}
%% 自定义楷体
\newcommand{\kai}{\CJKfamily{kai}}
%% 自定义黑体
\newcommand{\hei}{\CJKfamily{hei}}
%% 自定义仿宋体
\newcommand{\fangsong}{\CJKfamily{fangsong}}
%% 自定义隶书
\newcommand{\lisu}{\CJKfamily{lisu}}
%% 自定义幼圆
\newcommand{\youyuan}{\CJKfamily{youyuan}}

%%%% xeCJK字体设置结束
%%%%%%%%------------------------------------------------------------------------

%%%%%%%%------------------------------------------------------------------------
%%%% 一些关于中文文档的重定义
\newcommand{\chntoday}{\number\year\,年\,\number\month\,月\,\number\day\,日}
%% 数学公式定理的重定义

%% 中文破折号,据说来自清华模板
\newcommand{\pozhehao}{\kern0.3ex\rule[0.8ex]{2em}{0.1ex}\kern0.3ex}

\makeatletter %
\@ifclassloaded{beamer}{

}{
\newtheorem{example}{例}
\newtheorem{theorem}{定理}[section]
\newtheorem{definition}{定义}
\newtheorem{axiom}{公理}
\newtheorem{property}{性质}
\newtheorem{proposition}{命题}
\newtheorem{lemma}{引理}
\newtheorem{corollary}{推论}
\newtheorem{remark}{注解}
\newtheorem{condition}{条件}
\newtheorem{conclusion}{结论}
\newtheorem{assumption}{假设}
}
\makeatother

\makeatletter %
\@ifclassloaded{beamer}{

}{
%% 章节等名称重定义
\renewcommand{\contentsname}{目录}
\renewcommand{\indexname}{索引}
\renewcommand{\listfigurename}{插图目录}
\renewcommand{\listtablename}{表格目录}
\renewcommand{\appendixname}{附录}
\renewcommand{\appendixpagename}{附录}
\renewcommand{\appendixtocname}{附录}
\@ifclassloaded{book}{

}{
\renewcommand{\abstractname}{摘要}
}
}
\makeatother

\renewcommand{\figurename}{图}
\renewcommand{\tablename}{表}

\makeatletter
\@ifclassloaded{book}{
\renewcommand{\bibname}{参考文献}
}{
\renewcommand{\refname}{参考文献}
}
\makeatother

\floatname{algorithm}{算法}
\renewcommand{\algorithmicrequire}{\textbf{输入:}}
\renewcommand{\algorithmicensure}{\textbf{输出:}}

\renewcommand{\today}{\number\year 年 \number\month 月 \number\day 日}

%%%% 中文重定义结束
%%%%%%%%------------------------------------------------------------------------

\begin{document}
\title{二维和三维空间的缩放单项式空间}
\author{魏华祎}
\date{\chntoday}
\maketitle

\section{多边形上的缩放单项式空间}

\subsection{缩放单项式基函数及其计算性质}

\begin{table}[H]
\begin{tabular}{rccccccccccccc}
    $k=0$:&    &    &    &    &    &    & 0: $1$\\\noalign{\smallskip\smallskip}
    $k=1$:&    &    &    &    &    &  1:$\bar x$ &    &  2:$\bar y$\\\noalign{\smallskip\smallskip}
    $k=2$:&    &    &    &    &  3:${\bar x}^2$ &    &  4:$\bar x\bar y$ &    & 5:${\bar y}^2$\\\noalign{\smallskip\smallskip}
    $k=3$:&    &    &    &  6:${\bar x}^3$ &    &  7:${\bar x}^2\bar y$ &    & 8:$\bar x{\bar y}^2$ &    &  9:${\bar y}^3$\\\noalign{\smallskip\smallskip}
    $k=4$:&    &    &  10:${\bar x}^4$ &    &  11:${\bar x}^3\bar y$ &    & 12:${\bar x}^2{\bar y}^2$ &    &  13:${\bar x}{\bar y}^3$ &    & 14:${\bar y}^4$ \\\noalign{\smallskip\smallskip}
    $k=5$:&    &  15:${\bar x}^5$ &    &  16:${\bar x}^4\bar y$ &    & 17:${\bar x}^3{\bar y}^2$ &    & 18:${\bar x}^2{\bar y}^3$ &    & 19:${\bar x}{\bar y}^4$ &    &  20:${\bar y}^5$
    \\\noalign{\smallskip\smallskip}
\end{tabular}
    \caption{缩放单项式函数及其编号规则, 其中 $\bar x = \frac{x - x_K}{h_K}$,
    $\bar y = \frac{ y - y_K}{h_K}$.}\label{tb:scalep}
\end{table}


记 $K\subset\mbR^2$ 为一多边形单元,  $N_K$ 为其顶点的个数, $|K|$ 为其面积, $h_K
= \sqrt{|K|}$ 为其尺寸, $\bfx_K =(x_K, y_K)$ 为其重心. 记 $\balpha = (\alpha_0,
\alpha_1)$ 为任一二重非负整数指标, 则可定义 $K$ 上的{\bf 缩放单项式} 如下:
\begin{equation}
    m_\balpha = 
    \frac{(x - x_K)^{\alpha_0}(y - y_K)^{\alpha_1}}{h_K^{|\balpha|}},\quad
    \forall (x, y) = \bfx \in K
    \label{eq:sp}
\end{equation}
其中 $|\balpha| = \alpha_0 + \alpha_1$, 它是该缩放单项式对应的次数. 给定正整数
$k$, 把所有 $|\balpha|\leq k$ 对应的缩放单项式按表 \ref{tb:scalep} 中的方式排序, 
可得到一个长度为 $n_k = \frac{(k+1)(k+2)}{2}$ 向量函数
\begin{equation*}
    \bfm_k = [m_0, m_1, m_2, \cdots, m_{n_k-1}]
\end{equation*}
它们组成了标量空间 $\mbP_k(K)$ 的一组基. 同时记 $\mcI^k = \{0, 1, \cdots,
n_k-1\}$ 为向量函数 $\bfm_k$ 的下标集合.


 为了更方便编 写程序, 下面详细讨论 $\bfm_k$ 在求导运算下的计算规律. 仔细观察表
\ref{tb:scalep},  可得一阶导数的计算公式如下:
\begin{align*}
    \frac{\partial\bfm_k}{\partial x}[\mcI_{x, 0}^k] =&  0 \\
    \frac{\partial\bfm_k}{\partial x}[\mcI_{x, 1}^k] =& \frac{1}{h_K}\bfm_{k-1}\bfC_{k-1, x}  \\
    \frac{\partial\bfm_k}{\partial y}[\mcI_{y, 0}^k] =&  0 \\
    \frac{\partial\bfm_k}{\partial y}[\mcI_{y, 1}^k] =& \frac{1}{h_K}\bfm_{k-1}\bfC_{k-1, y}  \\
\end{align*}
其中
\begin{align*}
    \mcI_{x, 0}^k = &\{0, 2, 5, 9, 14, \cdots, n_k-1\} \subset \mcI^k\\
    \mcI_{x, 1}^k = &\{
        \overbrace{1}^1, 
        \overbrace{3, 4}^2, 
        \overbrace{6, 7, 8}^3, 
        \cdots, 
        \overbrace{n_k - k-1, \cdots, n_k - 2}^{k}\}\subset \mcI^k\\
    \mcI_{y, 0}^k = &\{0, 1, 3, 6, 10, \cdots, n_k-k-1\}\subset \mcI^k\\
    \mcI_{y, 1}^k = &\{
        \overbrace{2}^1, 
        \overbrace{4, 5}^2, 
        \overbrace{7, 8, 9}^3, 
        \cdots, 
        \overbrace{n_k-k, \cdots, n_k - 1}^{k}\} \subset \mcI^k\\
    \bfC_{k-1, x} = &\odiag[
        \overbrace{1}^1, 
        \overbrace{2, 1}^2, 
        \overbrace{3, 2, 1}^3, 
        \cdots, 
        \overbrace{k, \cdots, 1}^{k}] \\
    \bfC_{k-1, y} = &\odiag[
        \overbrace{1}^1, 
        \overbrace{1, 2}^2, 
        \overbrace{1, 2, 3}^3, 
        \cdots, 
    \overbrace{1, \cdots, k}^{k}] \\
\end{align*}

二阶导数的计算公式如下:
\begin{align*}
    \frac{\partial^2\bfm_k}{\partial x^2}[\mcI_{x^2, 0}^k] =&  0 \\
    \frac{\partial^2\bfm_k}{\partial x^2}[\mcI_{x^2, 1}^k] =& 
    \frac{1}{h_K^2}\bfm_{k-2}\bfC_{k-2, x^2}  \\
    \frac{\partial^2\bfm_k}{\partial y^2}[\mcI_{y^2, 0}^k] =&  0 \\
    \frac{\partial^2\bfm_k}{\partial y^2}[\mcI_{y^2, 1}^k] =& 
    \frac{1}{h_K^2}\bfm_{k-2}\bfC_{k-2, y^2}  \\
    \frac{\partial^2\bfm_k}{\partial x\partial y}[\mcI_{xy, 0}^k] =&  0 \\
    \frac{\partial^2\bfm_k}{\partial x\partial y}[\mcI_{xy, 1}^k] =&
    \frac{1}{h_K^2}\bfm_{k-2}\bfC_{k-2, xy}
\end{align*}
其中
\begin{align*}
    \mcI_{x^2, 0}^k = &\{0, 1, 2, 4, 5, 8, 9, \cdots, n_k-2, n_k-1\}
    \subset \mcI^k\\
    \mcI_{x^2, 1}^k = &\{
        \overbrace{3}^1, 
        \overbrace{6, 7}^2, 
        \overbrace{10, 11, 12}^3, 
        \cdots, 
        \overbrace{n_k-k-1, \cdots, n_k - 3}^{k-1}\}\subset\mcI^k\\
    \mcI_{y^2, 0}^k = &\{0, 1, 2, 3, 4, 6, 7, \cdots, n_k-k-1, n_k-k\}
    \subset\mcI^k\\
    \mcI_{y^2, 1}^k = &\{
        \overbrace{5}^1, 
        \overbrace{8, 9}^2, 
        \overbrace{12, 13, 14}^3, 
        \cdots, 
        \overbrace{n_k-k+1, \cdots, n_k - 1}^{k-1}\}
        \subset\mcI^k\\
    \mcI_{xy, 0}^k = &\{0, 1, 2, 3, 5, 6, 9, \cdots, n_k-k-1, n_k-1\}
    \subset\mcI^k\\
    \mcI_{xy, 1}^k = &\{
        \overbrace{2}^1, 
        \overbrace{4, 5}^2, 
        \overbrace{7, 8, 9}^3, 
        \cdots, 
        \overbrace{n_k-k, \cdots, n_k - 1}^{k-1}\}
        \subset\mcI^k \\
    \bfC_{k-2, x^2} = &\odiag[
        \overbrace{2}^1, 
        \overbrace{6, 2}^2, 
        \overbrace{12, 6, 2}^3, 
        \overbrace{20, 12, 6, 2}^4, 
        \cdots, 
        \overbrace{k\cdot(k-1), \cdots, 2\cdot1}^{k-1}] \\
    \bfC_{k-2, y^2} = &\odiag[
        \overbrace{2}^1, 
        \overbrace{2, 6}^2, 
        \overbrace{2, 6, 12}^3, 
        \overbrace{2, 6, 12, 20}^4, 
        \cdots, 
    \overbrace{2\cdot 1, \cdots, k\cdot(k-1)}^{k-1}] \\
    \bfC_{k-2, xy} = &\odiag[
        \overbrace{1}^1, 
        \overbrace{2, 2}^2, 
        \overbrace{3, 4, 3}^3, 
        \overbrace{4, 6, 6, 4}^4, 
        \cdots, 
    \overbrace{(k-1)\cdot 1, \cdots, 1\cdot (k-1)}^{k-1}] \\
\end{align*}

向量空间 $\mbP_k(K;\mbR^2)$ 的基函数
$$
\bfM_k = \begin{bmatrix}
    \bfm_k & 0 \\ 0 & \bfm_k
\end{bmatrix}
$$
其旋度的计算公式为
$$
\ocurl(\bfM_k) = \orot(\bfM_k) = 
\left[
    -\frac{\partial \bfm_k}{\partial y}, \frac{\partial \bfm_k}{\partial x}
\right]
$$
其对称梯度的计算公式为
$$
\bvarepsilon(\bfM_k) = 
\left[
    \begin{bmatrix}
        {\partial m_0 \over \partial x} & 
        \frac{1}{2}{\partial m_0 \over \partial y} \\
        \frac{1}{2}{\partial m_0 \over \partial y} & 0\\
    \end{bmatrix}
    , \cdots, 
    \begin{bmatrix}
        {\partial m_{n_k-1} \over \partial x} & 
        \frac{1}{2}{\partial m_{n_k-1} \over \partial y} \\
        \frac{1}{2}{\partial m_{n_k-1} \over \partial y} & 0\\
    \end{bmatrix},
    \begin{bmatrix}
        0 & \frac{1}{2}{\partial m_0 \over \partial x} \\
        \frac{1}{2}{\partial m_0 \over \partial x} & 
        {\partial m_0 \over \partial y}\\ 
    \end{bmatrix}
    , \cdots, 
    \begin{bmatrix}
        0 & \frac{1}{2}{\partial m_{n_k-1} \over \partial x} \\
        \frac{1}{2}{\partial m_{n_k-1} \over \partial x} & 
        {\partial m_{n_k-1} \over \partial y} \\
    \end{bmatrix}
\right]
$$
对称梯度散度的计算公式为
\begin{align*}
    \odiv\bvarepsilon(\bfM_k) =
    \begin{bmatrix}
        \frac{\partial^2\bfm_k}{\partial x^2}+
        \frac{1}{2}\frac{\partial^2\bfm_k}{\partial y^2} &
        \frac{1}{2}\frac{\partial^2\bfm_k}{\partial x \partial y} \\
        \frac{1}{2}\frac{\partial^2\bfm_k}{\partial x \partial y} &
        \frac{1}{2}\frac{\partial^2\bfm_k}{\partial x^2} + 
        \frac{\partial^2\bfm_k}{\partial y^2}
    \end{bmatrix}
\end{align*}
散度的计算公式为
$$
\odiv(\bfM_k) = \left[
    {\partial \bfm_k \over \partial x},
    {\partial \bfm_k \over \partial y} 
\right]
$$

$\mbP_{k-2}(K;\mbR^2)$ 空间可以正交分解为 $\mbG^\perp_{k-2}(K) \oplus
\mbG_{k-2}(K)$, 其中 $\mbG^\perp_{k-2}(K)$ 基函数可选为如下形式
$$
\bfM_{k-2}^\perp = 
\begin{bmatrix}
    m_2 \\ -m_1
\end{bmatrix} \bfm_{k-3} = 
\begin{bmatrix}
    m_2 \\ -m_1
\end{bmatrix} 
[m_0, m_1, \cdots, m_{n_{k-3}-1}]
= 
\begin{bmatrix}
    \bfm_{k-2}[\mcI_{y, 1}^{k-2}] \\ -\bfm_{k-2}[\mcI_{x, 1}^{k-2}]
\end{bmatrix}
$$
$\mbG_{k-2}(K)$ 基函数可选为如下形式
$$
\bfM_{k-2}^\nabla = [h_K\nabla m_1, h_K\nabla m_2, \cdots, h_K\nabla m_{n_{k-1}-1}]
$$

记对称 $2\times 2$ 张量 $k-1$ 次多项式空间为 $\mbN_{k-1}(K;\mbS)$, 它的的基函数为
$$
\bfN_{k-1} = 
\left[
\begin{bmatrix}
    1 & 0 \\ 0 & 0 
\end{bmatrix} 
\bfm_{k-1}, 
\begin{bmatrix}
    0 & 1 \\ 1 & 0 
\end{bmatrix} 
\bfm_{k-1},
\begin{bmatrix}
    0 & 0 \\ 0 & 1 
\end{bmatrix} 
\bfm_{k-1}\right] 
$$
其散度为
$$
\odiv\left(\bfN_{k-1}\right) = 
\begin{bmatrix}
 \frac{\partial\bfm_{k-1}}{\partial x} & \frac{\partial\bfm_{k-1}}{\partial y} &
 \bfzero \\
 \bfzero & \frac{\partial\bfm_{k-1}}{\partial x} & 
 \frac{\partial\bfm_{k-1}}{\partial y} 
\end{bmatrix}, 
$$

$\mbP_{k-2}(K;\mbR^2)$ 空间可以正交分解为 $\mbG^\perp_{k-2}(K) \oplus
\mbG_{k-2}(K)$, 基函数为
$$
\bfM_{k-2} = 
\begin{bmatrix}
    \bfm_{k-2} & \bfzero\\
    \bfzero & \bfm_{k-2}
\end{bmatrix}
$$
可以做如下分解
\begin{equation} \label{eq:split}
\bfM_{k-2} = h_K\left[ 
\nabla \bfm_{k-1}[\mcI^{k-1}_{x, 1}]+
\begin{bmatrix}
   m_2 \\ -m_1 
\end{bmatrix}
\frac{\partial \bfm_{k-2}}{\partial y}
,
\nabla \bfm_{k-1}[\mcI^{k-1}_{y, 1}] - 
\begin{bmatrix}
   m_2 \\ -m_1 
\end{bmatrix}
\frac{\partial \bfm_{k-2}}{\partial x}
\right]\bfC_{k-2}
\end{equation}
其中
$$
\bfC_{k-2} = 
\odiag\left[\overbrace{\frac{1}{1}}^1, \overbrace{\frac{1}{2}, \frac{1}{2}}^2,
\cdots, \overbrace{\frac{1}{k-1}, \cdots, \frac{1}{k-1}}^{k-1}\right] \\
$$

\subsection{程序实现}
\cite{fealpy}
\section{多面体单元上的缩放单项式空间}

\section{缩放单项式基函数及其计算性质}
记 $K\subset\mbR^3$ 为一多面体单元,  $N_K$ 为其顶点的个数, $|K|$ 为其体积, $h_K
= |K|^{1/3}$ 为其尺寸, $\bfx_K =(x_K, y_K, z_K)$ 为其重心. 记 $\balpha = (\alpha_0,
\alpha_1, \alpha_2)$ 为任一三重非负整数指标, 则可定义 $K$ 上的{\bf 缩放单项式} 如下:
\begin{equation}
    m_\balpha = 
    \frac{(x - x_K)^{\alpha_0}(y - y_K)^{\alpha_1}(z - z_K)^{\alpha_2}}{h_K^{|\balpha|}},\quad
    \forall (x, y) = \bfx \in K
    \label{eq:sp}
\end{equation}
其中 $|\balpha| = \alpha_0 + \alpha_1 + \alpha_2$, 它是该缩放单项式对应的次数.  
给定正整数 $k$, 把所有 $|\balpha|\leq k$ 对应的缩放单项式自然的方式排序, 
可得到一个长度为 $n_k = \frac{(k+1)(k+2)(k+3)}{6}$ 向量函数
\begin{equation*}
    \bfm_k = [m_0, m_1, m_2, \cdots, m_{n_k-1}]
\end{equation*}
它们组成了标量空间 $\mbP_k(K)$ 的一组基. 


同时记 $\mcI^k = \{0, 1, \cdots,
n_k-1\}$ 为向量函数 $\bfm_k$ 的下标集合.

\bibliographystyle{abbrv}
\bibliography{scaledspace}
\end{document}
