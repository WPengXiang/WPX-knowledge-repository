\documentclass{article}
\input{../en_preamble.tex}
\input{../xecjk_preamble.tex}
\begin{document}
\title{二维和三维空间的缩放单项式空间}
\author{魏华祎}
\date{\chntoday}
\maketitle

\section{多边形上的缩放单项式空间}

\subsection{缩放单项式基函数及其计算性质}

\begin{table}[H]
\begin{tabular}{rccccccccccccc}
    $k=0$:&    &    &    &    &    &    & 0: $1$\\\noalign{\smallskip\smallskip}
    $k=1$:&    &    &    &    &    &  1:$\bar x$ &    &  2:$\bar y$\\\noalign{\smallskip\smallskip}
    $k=2$:&    &    &    &    &  3:${\bar x}^2$ &    &  4:$\bar x\bar y$ &    & 5:${\bar y}^2$\\\noalign{\smallskip\smallskip}
    $k=3$:&    &    &    &  6:${\bar x}^3$ &    &  7:${\bar x}^2\bar y$ &    & 8:$\bar x{\bar y}^2$ &    &  9:${\bar y}^3$\\\noalign{\smallskip\smallskip}
    $k=4$:&    &    &  10:${\bar x}^4$ &    &  11:${\bar x}^3\bar y$ &    & 12:${\bar x}^2{\bar y}^2$ &    &  13:${\bar x}{\bar y}^3$ &    & 14:${\bar y}^4$ \\\noalign{\smallskip\smallskip}
    $k=5$:&    &  15:${\bar x}^5$ &    &  16:${\bar x}^4\bar y$ &    & 17:${\bar x}^3{\bar y}^2$ &    & 18:${\bar x}^2{\bar y}^3$ &    & 19:${\bar x}{\bar y}^4$ &    &  20:${\bar y}^5$
    \\\noalign{\smallskip\smallskip}
\end{tabular}
    \caption{缩放单项式函数及其编号规则, 其中 $\bar x = \frac{x - x_K}{h_K}$,
    $\bar y = \frac{ y - y_K}{h_K}$.}\label{tb:scalep}
\end{table}


记 $K\subset\mbR^2$ 为一多边形单元,  $N_K$ 为其顶点的个数, $|K|$ 为其面积, $h_K
= \sqrt{|K|}$ 为其尺寸, $\bfx_K =(x_K, y_K)$ 为其重心. 记 $\balpha = (\alpha_0,
\alpha_1)$ 为任一二重非负整数指标, 则可定义 $K$ 上的{\bf 缩放单项式} 如下:
\begin{equation}
    m_\balpha = 
    \frac{(x - x_K)^{\alpha_0}(y - y_K)^{\alpha_1}}{h_K^{|\balpha|}},\quad
    \forall (x, y) = \bfx \in K
    \label{eq:sp}
\end{equation}
其中 $|\balpha| = \alpha_0 + \alpha_1$, 它是该缩放单项式对应的次数. 给定正整数
$k$, 把所有 $|\balpha|\leq k$ 对应的缩放单项式按表 \ref{tb:scalep} 中的方式排序, 
可得到一个长度为 $n_k = \frac{(k+1)(k+2)}{2}$ 向量函数
\begin{equation*}
    \bfm_k = [m_0, m_1, m_2, \cdots, m_{n_k-1}]
\end{equation*}
它们组成了标量空间 $\mbP_k(K)$ 的一组基. 同时记 $\mcI^k = \{0, 1, \cdots,
n_k-1\}$ 为向量函数 $\bfm_k$ 的下标集合.


 为了更方便编 写程序, 下面详细讨论 $\bfm_k$ 在求导运算下的计算规律. 仔细观察表
\ref{tb:scalep},  可得一阶导数的计算公式如下:
\begin{align*}
    \frac{\partial\bfm_k}{\partial x}[\mcI_{x, 0}^k] =&  0 \\
    \frac{\partial\bfm_k}{\partial x}[\mcI_{x, 1}^k] =& \frac{1}{h_K}\bfm_{k-1}\bfC_{k-1, x}  \\
    \frac{\partial\bfm_k}{\partial y}[\mcI_{y, 0}^k] =&  0 \\
    \frac{\partial\bfm_k}{\partial y}[\mcI_{y, 1}^k] =& \frac{1}{h_K}\bfm_{k-1}\bfC_{k-1, y}  \\
\end{align*}
其中
\begin{align*}
    \mcI_{x, 0}^k = &\{0, 2, 5, 9, 14, \cdots, n_k-1\} \subset \mcI^k\\
    \mcI_{x, 1}^k = &\{
        \overbrace{1}^1, 
        \overbrace{3, 4}^2, 
        \overbrace{6, 7, 8}^3, 
        \cdots, 
        \overbrace{n_k - k-1, \cdots, n_k - 2}^{k}\}\subset \mcI^k\\
    \mcI_{y, 0}^k = &\{0, 1, 3, 6, 10, \cdots, n_k-k-1\}\subset \mcI^k\\
    \mcI_{y, 1}^k = &\{
        \overbrace{2}^1, 
        \overbrace{4, 5}^2, 
        \overbrace{7, 8, 9}^3, 
        \cdots, 
        \overbrace{n_k-k, \cdots, n_k - 1}^{k}\} \subset \mcI^k\\
    \bfC_{k-1, x} = &\odiag[
        \overbrace{1}^1, 
        \overbrace{2, 1}^2, 
        \overbrace{3, 2, 1}^3, 
        \cdots, 
        \overbrace{k, \cdots, 1}^{k}] \\
    \bfC_{k-1, y} = &\odiag[
        \overbrace{1}^1, 
        \overbrace{1, 2}^2, 
        \overbrace{1, 2, 3}^3, 
        \cdots, 
    \overbrace{1, \cdots, k}^{k}] \\
\end{align*}

二阶导数的计算公式如下:
\begin{align*}
    \frac{\partial^2\bfm_k}{\partial x^2}[\mcI_{x^2, 0}^k] =&  0 \\
    \frac{\partial^2\bfm_k}{\partial x^2}[\mcI_{x^2, 1}^k] =& 
    \frac{1}{h_K^2}\bfm_{k-2}\bfC_{k-2, x^2}  \\
    \frac{\partial^2\bfm_k}{\partial y^2}[\mcI_{y^2, 0}^k] =&  0 \\
    \frac{\partial^2\bfm_k}{\partial y^2}[\mcI_{y^2, 1}^k] =& 
    \frac{1}{h_K^2}\bfm_{k-2}\bfC_{k-2, y^2}  \\
    \frac{\partial^2\bfm_k}{\partial x\partial y}[\mcI_{xy, 0}^k] =&  0 \\
    \frac{\partial^2\bfm_k}{\partial x\partial y}[\mcI_{xy, 1}^k] =&
    \frac{1}{h_K^2}\bfm_{k-2}\bfC_{k-2, xy}
\end{align*}
其中
\begin{align*}
    \mcI_{x^2, 0}^k = &\{0, 1, 2, 4, 5, 8, 9, \cdots, n_k-2, n_k-1\}
    \subset \mcI^k\\
    \mcI_{x^2, 1}^k = &\{
        \overbrace{3}^1, 
        \overbrace{6, 7}^2, 
        \overbrace{10, 11, 12}^3, 
        \cdots, 
        \overbrace{n_k-k-1, \cdots, n_k - 3}^{k-1}\}\subset\mcI^k\\
    \mcI_{y^2, 0}^k = &\{0, 1, 2, 3, 4, 6, 7, \cdots, n_k-k-1, n_k-k\}
    \subset\mcI^k\\
    \mcI_{y^2, 1}^k = &\{
        \overbrace{5}^1, 
        \overbrace{8, 9}^2, 
        \overbrace{12, 13, 14}^3, 
        \cdots, 
        \overbrace{n_k-k+1, \cdots, n_k - 1}^{k-1}\}
        \subset\mcI^k\\
    \mcI_{xy, 0}^k = &\{0, 1, 2, 3, 5, 6, 9, \cdots, n_k-k-1, n_k-1\}
    \subset\mcI^k\\
    \mcI_{xy, 1}^k = &\{
        \overbrace{2}^1, 
        \overbrace{4, 5}^2, 
        \overbrace{7, 8, 9}^3, 
        \cdots, 
        \overbrace{n_k-k, \cdots, n_k - 1}^{k-1}\}
        \subset\mcI^k \\
    \bfC_{k-2, x^2} = &\odiag[
        \overbrace{2}^1, 
        \overbrace{6, 2}^2, 
        \overbrace{12, 6, 2}^3, 
        \overbrace{20, 12, 6, 2}^4, 
        \cdots, 
        \overbrace{k\cdot(k-1), \cdots, 2\cdot1}^{k-1}] \\
    \bfC_{k-2, y^2} = &\odiag[
        \overbrace{2}^1, 
        \overbrace{2, 6}^2, 
        \overbrace{2, 6, 12}^3, 
        \overbrace{2, 6, 12, 20}^4, 
        \cdots, 
    \overbrace{2\cdot 1, \cdots, k\cdot(k-1)}^{k-1}] \\
    \bfC_{k-2, xy} = &\odiag[
        \overbrace{1}^1, 
        \overbrace{2, 2}^2, 
        \overbrace{3, 4, 3}^3, 
        \overbrace{4, 6, 6, 4}^4, 
        \cdots, 
    \overbrace{(k-1)\cdot 1, \cdots, 1\cdot (k-1)}^{k-1}] \\
\end{align*}

向量空间 $\mbP_k(K;\mbR^2)$ 的基函数
$$
\bfM_k = \begin{bmatrix}
    \bfm_k & 0 \\ 0 & \bfm_k
\end{bmatrix}
$$
其旋度的计算公式为
$$
\ocurl(\bfM_k) = \orot(\bfM_k) = 
\left[
    -\frac{\partial \bfm_k}{\partial y}, \frac{\partial \bfm_k}{\partial x}
\right]
$$
其对称梯度的计算公式为
$$
\bvarepsilon(\bfM_k) = 
\left[
    \begin{bmatrix}
        {\partial m_0 \over \partial x} & 
        \frac{1}{2}{\partial m_0 \over \partial y} \\
        \frac{1}{2}{\partial m_0 \over \partial y} & 0\\
    \end{bmatrix}
    , \cdots, 
    \begin{bmatrix}
        {\partial m_{n_k-1} \over \partial x} & 
        \frac{1}{2}{\partial m_{n_k-1} \over \partial y} \\
        \frac{1}{2}{\partial m_{n_k-1} \over \partial y} & 0\\
    \end{bmatrix},
    \begin{bmatrix}
        0 & \frac{1}{2}{\partial m_0 \over \partial x} \\
        \frac{1}{2}{\partial m_0 \over \partial x} & 
        {\partial m_0 \over \partial y}\\ 
    \end{bmatrix}
    , \cdots, 
    \begin{bmatrix}
        0 & \frac{1}{2}{\partial m_{n_k-1} \over \partial x} \\
        \frac{1}{2}{\partial m_{n_k-1} \over \partial x} & 
        {\partial m_{n_k-1} \over \partial y} \\
    \end{bmatrix}
\right]
$$
对称梯度散度的计算公式为
\begin{align*}
    \odiv\bvarepsilon(\bfM_k) =
    \begin{bmatrix}
        \frac{\partial^2\bfm_k}{\partial x^2}+
        \frac{1}{2}\frac{\partial^2\bfm_k}{\partial y^2} &
        \frac{1}{2}\frac{\partial^2\bfm_k}{\partial x \partial y} \\
        \frac{1}{2}\frac{\partial^2\bfm_k}{\partial x \partial y} &
        \frac{1}{2}\frac{\partial^2\bfm_k}{\partial x^2} + 
        \frac{\partial^2\bfm_k}{\partial y^2}
    \end{bmatrix}
\end{align*}
散度的计算公式为
$$
\odiv(\bfM_k) = \left[
    {\partial \bfm_k \over \partial x},
    {\partial \bfm_k \over \partial y} 
\right]
$$

$\mbP_{k-2}(K;\mbR^2)$ 空间可以正交分解为 $\mbG^\perp_{k-2}(K) \oplus
\mbG_{k-2}(K)$, 其中 $\mbG^\perp_{k-2}(K)$ 基函数可选为如下形式
$$
\bfM_{k-2}^\perp = 
\begin{bmatrix}
    m_2 \\ -m_1
\end{bmatrix} \bfm_{k-3} = 
\begin{bmatrix}
    m_2 \\ -m_1
\end{bmatrix} 
[m_0, m_1, \cdots, m_{n_{k-3}-1}]
= 
\begin{bmatrix}
    \bfm_{k-2}[\mcI_{y, 1}^{k-2}] \\ -\bfm_{k-2}[\mcI_{x, 1}^{k-2}]
\end{bmatrix}
$$
$\mbG_{k-2}(K)$ 基函数可选为如下形式
$$
\bfM_{k-2}^\nabla = [h_K\nabla m_1, h_K\nabla m_2, \cdots, h_K\nabla m_{n_{k-1}-1}]
$$

记对称 $2\times 2$ 张量 $k-1$ 次多项式空间为 $\mbN_{k-1}(K;\mbS)$, 它的的基函数为
$$
\bfN_{k-1} = 
\left[
\begin{bmatrix}
    1 & 0 \\ 0 & 0 
\end{bmatrix} 
\bfm_{k-1}, 
\begin{bmatrix}
    0 & 1 \\ 1 & 0 
\end{bmatrix} 
\bfm_{k-1},
\begin{bmatrix}
    0 & 0 \\ 0 & 1 
\end{bmatrix} 
\bfm_{k-1}\right] 
$$
其散度为
$$
\odiv\left(\bfN_{k-1}\right) = 
\begin{bmatrix}
 \frac{\partial\bfm_{k-1}}{\partial x} & \frac{\partial\bfm_{k-1}}{\partial y} &
 \bfzero \\
 \bfzero & \frac{\partial\bfm_{k-1}}{\partial x} & 
 \frac{\partial\bfm_{k-1}}{\partial y} 
\end{bmatrix}, 
$$

$\mbP_{k-2}(K;\mbR^2)$ 空间可以正交分解为 $\mbG^\perp_{k-2}(K) \oplus
\mbG_{k-2}(K)$, 基函数为
$$
\bfM_{k-2} = 
\begin{bmatrix}
    \bfm_{k-2} & \bfzero\\
    \bfzero & \bfm_{k-2}
\end{bmatrix}
$$
可以做如下分解
\begin{equation} \label{eq:split}
\bfM_{k-2} = h_K\left[ 
\nabla \bfm_{k-1}[\mcI^{k-1}_{x, 1}]+
\begin{bmatrix}
   m_2 \\ -m_1 
\end{bmatrix}
\frac{\partial \bfm_{k-2}}{\partial y}
,
\nabla \bfm_{k-1}[\mcI^{k-1}_{y, 1}] - 
\begin{bmatrix}
   m_2 \\ -m_1 
\end{bmatrix}
\frac{\partial \bfm_{k-2}}{\partial x}
\right]\bfC_{k-2}
\end{equation}
其中
$$
\bfC_{k-2} = 
\odiag\left[\overbrace{\frac{1}{1}}^1, \overbrace{\frac{1}{2}, \frac{1}{2}}^2,
\cdots, \overbrace{\frac{1}{k-1}, \cdots, \frac{1}{k-1}}^{k-1}\right] \\
$$

\subsection{程序实现}
\cite{fealpy}
\section{多面体单元上的缩放单项式空间}

\section{缩放单项式基函数及其计算性质}
记 $K\subset\mbR^3$ 为一多面体单元,  $N_K$ 为其顶点的个数, $|K|$ 为其体积, $h_K
= |K|^{1/3}$ 为其尺寸, $\bfx_K =(x_K, y_K, z_K)$ 为其重心. 记 $\balpha = (\alpha_0,
\alpha_1, \alpha_2)$ 为任一三重非负整数指标, 则可定义 $K$ 上的{\bf 缩放单项式} 如下:
\begin{equation}
    m_\balpha = 
    \frac{(x - x_K)^{\alpha_0}(y - y_K)^{\alpha_1}(z - z_K)^{\alpha_2}}{h_K^{|\balpha|}},\quad
    \forall (x, y) = \bfx \in K
    \label{eq:sp}
\end{equation}
其中 $|\balpha| = \alpha_0 + \alpha_1 + \alpha_2$, 它是该缩放单项式对应的次数.  
给定正整数 $k$, 把所有 $|\balpha|\leq k$ 对应的缩放单项式自然的方式排序, 
可得到一个长度为 $n_k = \frac{(k+1)(k+2)(k+3)}{6}$ 向量函数
\begin{equation*}
    \bfm_k = [m_0, m_1, m_2, \cdots, m_{n_k-1}]
\end{equation*}
它们组成了标量空间 $\mbP_k(K)$ 的一组基. 


同时记 $\mcI^k = \{0, 1, \cdots,
n_k-1\}$ 为向量函数 $\bfm_k$ 的下标集合.

\bibliographystyle{abbrv}
\bibliography{scaledspace}
\end{document}
