% !Mode:: "TeX:UTF-8"
\documentclass{article}

%%%%%%%%------------------------------------------------------------------------
%%%% 日常所用宏包

%% 控制页边距
% 如果是beamer文档类, 则不用geometry
\makeatletter
\@ifclassloaded{beamer}{}{\usepackage[top=2.5cm, bottom=2.5cm, left=2.5cm, right=2.5cm]{geometry}}
\makeatother

\makeatletter
\@ifclassloaded{beamer}{
\makeatletter
\def\th@mystyle{%
    \normalfont % body font
    \setbeamercolor{block title example}{bg=orange,fg=white}
    \setbeamercolor{block body example}{bg=orange!20,fg=black}
    \def\inserttheoremblockenv{exampleblock}
  }
\makeatother
\theoremstyle{mystyle}
\newtheorem*{remark}{Remark}

\newcommand{\propnumber}{} % initialize
\newtheorem*{prop}{Proposition \propnumber}
\newenvironment{propc}[1]
  {\renewcommand{\propnumber}{#1}%
   \begin{shaded}\begin{prop}}
  {\end{prop}\end{shaded}}

\makeatletter
\newenvironment<>{proofs}[1][\proofname]{%
    \par
    \def\insertproofname{#1\@addpunct{.}}%
    \usebeamertemplate{proof begin}#2}
  {\usebeamertemplate{proof end}}
\makeatother

}{
}
\makeatother
\usepackage{amsthm}

%\DeclareMathOperator{\sech}{sech}
%\DeclareMathOperator{\csch}{csch}
%\DeclareMathOperator{\arcsec}{arcsec}
%\DeclareMathOperator{\arccot}{arccot}
%\DeclareMathOperator{\arccsc}{arccsc}
%\DeclareMathOperator{\arccosh}{arccosh}
%\DeclareMathOperator{\arcsinh}{arcsinh}
%\DeclareMathOperator{\arctanh}{arctanh}
%\DeclareMathOperator{\arcsech}{arcsech}
%\DeclareMathOperator{\arccsch}{arccsch}
%\DeclareMathOperator{\arccoth}{arccoth}
%% 控制项目列表
\usepackage{enumerate}

%% Todo list
\usepackage{enumitem}
\newlist{todolist}{itemize}{2}
\setlist[todolist]{label=$\square$}
\usepackage{pifont}
\newcommand{\cmark}{\ding{51}}%
\newcommand{\xmark}{\ding{55}}%
\newcommand{\done}{\rlap{$\square$}{\raisebox{2pt}{\large\hspace{1pt}\cmark}}%
\hspace{-2.5pt}}
\newcommand{\wontfix}{\rlap{$\square$}{\large\hspace{1pt}\xmark}}

\usepackage[utf8]{inputenc}
\usepackage[english]{babel}

\usepackage{framed}

%% 多栏显示
\usepackage{multicol}

%% 算法环境
\usepackage{algorithm}
\usepackage{algorithmic}
\usepackage{float}

%% 网址引用
\usepackage{url}

%% 控制矩阵行距
\renewcommand\arraystretch{1.4}

%% 粗体
\usepackage{lmodern}
\usepackage{bm}


%% hyperref宏包,生成可定位点击的超链接,并且会生成pdf书签
\makeatletter
\@ifclassloaded{beamer}{
\usepackage{hyperref}
\usepackage{ragged2e} % 对齐
}{
\usepackage[%
    pdfstartview=FitH,%
    CJKbookmarks=true,%
    bookmarks=true,%
    bookmarksnumbered=true,%
    bookmarksopen=true,%
    colorlinks=true,%
    citecolor=blue,%
    linkcolor=blue,%
    anchorcolor=green,%
    urlcolor=blue%
]{hyperref}
}
\makeatother



\makeatletter % 如果是 beamer 不需要下面两个包
\@ifclassloaded{beamer}{
\mode<presentation>
{
}
}{
%% 控制标题
\usepackage{titlesec}
%% 控制目录
\usepackage{titletoc}
}
\makeatother

%% 控制表格样式
\usepackage{booktabs}

%% 控制字体大小
\usepackage{type1cm}

%% 首行缩进,用\noindent取消某段缩进
\usepackage{indentfirst}

%% 支持彩色文本、底色、文本框等
\usepackage{color,xcolor}

%% AMS LaTeX宏包: http://zzg34b.w3.c361.com/package/maths.htm#amssymb
\usepackage{amsmath,amssymb}
%% 多个图形并排
\usepackage{subfig}
%%%% 基本插图方法
%% 图形宏包
\usepackage{graphicx}


%%%% 基本插图方法结束

%%%% pgf/tikz绘图宏包设置
\usepackage{pgf,tikz}
\usetikzlibrary{shapes,automata,snakes,backgrounds,arrows}
\usetikzlibrary{mindmap}
%% 可以直接在latex文档中使用graphviz/dot语言,
%% 也可以用dot2tex工具将dot文件转换成tex文件再include进来
%% \usepackage[shell,pgf,outputdir={docgraphs/}]{dot2texi}
%%%% pgf/tikz设置结束


\makeatletter % 如果是 beamer 不需要下面两个包
\@ifclassloaded{beamer}{

}{
%%%% fancyhdr设置页眉页脚
%% 页眉页脚宏包
\usepackage{fancyhdr}
%% 页眉页脚风格
\pagestyle{plain}
}

%% 有时会出现\headheight too small的warning
\setlength{\headheight}{15pt}

%% 清空当前页眉页脚的默认设置
%\fancyhf{}
%%%% fancyhdr设置结束


%% 设置listings宏包的一些全局样式
%% 参考http://hi.baidu.com/shawpinlee/blog/item/9ec431cbae28e41cbe09e6e4.html
\usepackage{listings}
\lstloadlanguages{[LaTeX]TeX}

\usepackage{fancyvrb}

\newenvironment{latexample}[1][language={[LaTeX]TeX}]
{\lstset{breaklines=true,
    prebreak = \raisebox{0ex}[0ex][0ex]{\ensuremath{\hookleftarrow}},
    frame=single,
    language={[LaTeX]TeX},
    showstringspaces=false,              %% 设定是否显示代码之间的空格符号
    numbers=left,                        %% 在左边显示行号
    numberstyle=\tiny,                   %% 设定行号字体的大小
    basicstyle=\scriptsize,                    %% 设定字体大小\tiny, \small, \Large等等
    keywordstyle=\color{blue!70}, commentstyle=\color{red!50!green!50!blue!50},
                                         %% 关键字高亮
    frame=shadowbox,                     %% 给代码加框
    rulesepcolor=\color{red!20!green!20!blue!20},
    escapechar=`,                        %% 中文逃逸字符,用于中英混排
    xleftmargin=2em,xrightmargin=2em, aboveskip=1em,
    %breaklines,                          %% 这条命令可以让LaTeX自动将长的代码行换行排版
    extendedchars=false                  %% 这一条命令可以解决代码跨页时,章节标题,页眉等汉字不显示的问题
    basicstyle=\footnotesize\ttfamily, #1}
  \VerbatimEnvironment\begin{VerbatimOut}{latexample.verb.out}}
  {\end{VerbatimOut}\noindent
  \begin{minipage}{1.05\linewidth}
    \lstinputlisting[]{latexample.verb.out}%
  \end{minipage}\qquad
  \begin{minipage}{1\linewidth}
    \input{latexample.verb.out}
  \end{minipage}\\
}

\usepackage{minted}
\renewcommand{\listingscaption}{Python code} \newminted{python}{
    escapeinside=||,
    mathescape=true,
    numbersep=5pt,
    linenos=true,
    autogobble,
    framesep=3mm}
%%%% listings宏包设置结束


%%%% 附录设置
\makeatletter % 对 beamer 要重新设置
\@ifclassloaded{beamer}{

}{
\usepackage[title,titletoc,header]{appendix}
}
\makeatother
%%%% 附录设置结束





%% 设定行距
\linespread{1}

%% 颜色
\newcommand{\red}{\color{red} }
\newcommand{\blue}{\color{blue} }
\newcommand{\brown}{\color{brown} }
\newcommand{\green}{\color{green} }

\newcommand{\bred}{\bf\color{red} }
\newcommand{\bblue}{\bf\color{blue} }
\newcommand{\bbrown}{\bf\color{brown} }
\newcommand{\bgreen}{\bf\color{green} }
%% 1. 小写的英文或希腊字母表示 标量或标量函数
%% 2. 大写的英文或希腊字母表示 集合或空间
%% 3. 粗体的小写字母代表向量或向量形式的常量和函数
%% 4. 粗体的大写字母代表矩阵或张量形式的常量和函数
%% 5. 空心大写字母代表特殊的空间 \mbR 实数 \mbC 复数 \mbP 多项式
%% 6. 花体的大写字母代表算子

%% 粗体的小写字母代表向量或向量函数
\newcommand{\bfa}{{\boldsymbol a}}
\newcommand{\bfb}{{\boldsymbol b}}
\newcommand{\bfc}{{\boldsymbol c}}
\newcommand{\bfd}{{\boldsymbol d}}
\newcommand{\bfe}{{\boldsymbol e}}
\newcommand{\bff}{{\boldsymbol f}}
\newcommand{\bfg}{{\boldsymbol g}}
\newcommand{\bfh}{{\boldsymbol h}}
\newcommand{\bfi}{{\boldsymbol i}}
\newcommand{\bfj}{{\boldsymbol j}}
\newcommand{\bfk}{{\boldsymbol k}}
\newcommand{\bfl}{{\boldsymbol l}}
\newcommand{\bfm}{{\boldsymbol m}}
\newcommand{\bfn}{{\boldsymbol n}}
\newcommand{\bfo}{{\boldsymbol o}}
\newcommand{\bfp}{{\boldsymbol p}}
\newcommand{\bfq}{{\boldsymbol q}}
\newcommand{\bfr}{{\boldsymbol r}}
\newcommand{\bfs}{{\boldsymbol s}}
\newcommand{\bft}{{\boldsymbol t}}
\newcommand{\bfu}{{\boldsymbol u}}
\newcommand{\bfv}{{\boldsymbol v}}
\newcommand{\bfw}{{\boldsymbol w}}
\newcommand{\bfx}{{\boldsymbol x}}
\newcommand{\bfy}{{\boldsymbol y}}
\newcommand{\bfz}{{\boldsymbol z}}

%  算子
\newcommand{\mca}{{\mathcal a}}
\newcommand{\mcb}{{\mathcal b}}
\newcommand{\mcc}{{\mathcal c}}
\newcommand{\mcd}{{\mathcal d}}
\newcommand{\mce}{{\mathcal e}}
\newcommand{\mcf}{{\mathcal f}}
\newcommand{\mcg}{{\mathcal g}}
\newcommand{\mch}{{\mathcal h}}
\newcommand{\mci}{{\mathcal i}}
\newcommand{\mcj}{{\mathcal j}}
\newcommand{\mck}{{\mathcal k}}
\newcommand{\mcl}{{\mathcal l}}
\newcommand{\mcm}{{\mathcal m}}
\newcommand{\mcn}{{\mathcal n}}
\newcommand{\mco}{{\mathcal o}}
\newcommand{\mcp}{{\mathcal p}}
\newcommand{\mcq}{{\mathcal q}}
\newcommand{\mcr}{{\mathcal r}}
\newcommand{\mcs}{{\mathcal s}}
\newcommand{\mct}{{\mathcal t}}
\newcommand{\mcu}{{\mathcal u}}
\newcommand{\mcv}{{\mathcal v}}
\newcommand{\mcw}{{\mathcal w}}
\newcommand{\mcx}{{\mathcal x}}
\newcommand{\mcy}{{\mathcal y}}
\newcommand{\mcz}{{\mathcal z}}

% \rmd
\newcommand{\mra}{{\mathrm a}}
\newcommand{\mrb}{{\mathrm b}}
\newcommand{\mrc}{{\mathrm c}}
\newcommand{\mrd}{{\mathrm d}}
\newcommand{\mre}{{\mathrm e}}
\newcommand{\mrf}{{\mathrm f}}
\newcommand{\mrg}{{\mathrm g}}
\newcommand{\mrh}{{\mathrm h}}
\newcommand{\mri}{{\mathrm i}}
\newcommand{\mrj}{{\mathrm j}}
\newcommand{\mrk}{{\mathrm k}}
\newcommand{\mrl}{{\mathrm l}}
\newcommand{\mrm}{{\mathrm m}}
\newcommand{\mrn}{{\mathrm n}}
\newcommand{\mro}{{\mathrm o}}
\newcommand{\mrp}{{\mathrm p}}
\newcommand{\mrq}{{\mathrm q}}
\newcommand{\mrr}{{\mathrm r}}
\newcommand{\mrs}{{\mathrm s}}
\newcommand{\mrt}{{\mathrm t}}
\newcommand{\mru}{{\mathrm u}}
\newcommand{\mrv}{{\mathrm v}}
\newcommand{\mrw}{{\mathrm w}}
\newcommand{\mrx}{{\mathrm x}}
\newcommand{\mry}{{\mathrm y}}
\newcommand{\mrz}{{\mathrm z}}

%% 粗体的大写字母一般表示矩阵和张量
\newcommand{\bfA}{{\boldsymbol A}}
\newcommand{\bfB}{{\boldsymbol B}}
\newcommand{\bfC}{{\boldsymbol C}}
\newcommand{\bfD}{{\boldsymbol D}}
\newcommand{\bfE}{{\boldsymbol E}}
\newcommand{\bfF}{{\boldsymbol F}}
\newcommand{\bfG}{{\boldsymbol G}}
\newcommand{\bfH}{{\boldsymbol H}}
\newcommand{\bfI}{{\boldsymbol I}}
\newcommand{\bfJ}{{\boldsymbol J}}
\newcommand{\bfK}{{\boldsymbol K}}
\newcommand{\bfL}{{\boldsymbol L}}
\newcommand{\bfM}{{\boldsymbol M}}
\newcommand{\bfN}{{\boldsymbol N}}
\newcommand{\bfO}{{\boldsymbol O}}
\newcommand{\bfP}{{\boldsymbol P}}
\newcommand{\bfQ}{{\boldsymbol Q}}
\newcommand{\bfR}{{\boldsymbol R}}
\newcommand{\bfS}{{\boldsymbol S}}
\newcommand{\bfT}{{\boldsymbol T}}
\newcommand{\bfU}{{\boldsymbol U}}
\newcommand{\bfV}{{\boldsymbol V}}
\newcommand{\bfW}{{\boldsymbol W}}
\newcommand{\bfX}{{\boldsymbol X}}
\newcommand{\bfY}{{\boldsymbol Y}}
\newcommand{\bfZ}{{\boldsymbol Z}}

%% 花体大写字母
\newcommand{\mcA}{{\mathcal A}}
\newcommand{\mcB}{{\mathcal B}}
\newcommand{\mcC}{{\mathcal C}}
\newcommand{\mcD}{{\mathcal D}}
\newcommand{\mcE}{{\mathcal E}}
\newcommand{\mcF}{{\mathcal F}}
\newcommand{\mcG}{{\mathcal G}}
\newcommand{\mcH}{{\mathcal H}}
\newcommand{\mcI}{{\mathcal I}}
\newcommand{\mcJ}{{\mathcal J}}
\newcommand{\mcK}{{\mathcal K}}
\newcommand{\mcL}{{\mathcal L}}
\newcommand{\mcM}{{\mathcal M}}
\newcommand{\mcN}{{\mathcal N}}
\newcommand{\mcO}{{\mathcal O}}
\newcommand{\mcP}{{\mathcal P}}
\newcommand{\mcQ}{{\mathcal Q}}
\newcommand{\mcR}{{\mathcal R}}
\newcommand{\mcS}{{\mathcal S}}
\newcommand{\mcT}{{\mathcal T}}
\newcommand{\mcU}{{\mathcal U}}
\newcommand{\mcV}{{\mathcal V}}
\newcommand{\mcW}{{\mathcal W}}
\newcommand{\mcX}{{\mathcal X}}
\newcommand{\mcY}{{\mathcal Y}}
\newcommand{\mcZ}{{\mathcal Z}}

%% 空心大写字母
\newcommand{\mbA}{{\mathbb A}}
\newcommand{\mbB}{{\mathbb B}}
\newcommand{\mbC}{{\mathbb C}}
\newcommand{\mbD}{{\mathbb D}}
\newcommand{\mbE}{{\mathbb E}}
\newcommand{\mbF}{{\mathbb F}}
\newcommand{\mbG}{{\mathbb G}}
\newcommand{\mbH}{{\mathbb H}}
\newcommand{\mbI}{{\mathbb I}}
\newcommand{\mbJ}{{\mathbb J}}
\newcommand{\mbK}{{\mathbb K}}
\newcommand{\mbL}{{\mathbb L}}
\newcommand{\mbM}{{\mathbb M}}
\newcommand{\mbN}{{\mathbb N}}
\newcommand{\mbO}{{\mathbb O}}
\newcommand{\mbP}{{\mathbb P}}
\newcommand{\mbQ}{{\mathbb Q}}
\newcommand{\mbR}{{\mathbb R}}
\newcommand{\mbS}{{\mathbb S}}
\newcommand{\mbT}{{\mathbb T}}
\newcommand{\mbU}{{\mathbb U}}
\newcommand{\mbV}{{\mathbb V}}
\newcommand{\mbW}{{\mathbb W}}
\newcommand{\mbX}{{\mathbb X}}
\newcommand{\mbY}{{\mathbb Y}}
\newcommand{\mbZ}{{\mathbb Z}}

\newcommand{\mrA}{{\mathrm A}}
\newcommand{\mrB}{{\mathrm B}}
\newcommand{\mrC}{{\mathrm C}}
\newcommand{\mrD}{{\mathrm D}}
\newcommand{\mrE}{{\mathrm E}}
\newcommand{\mrF}{{\mathrm F}}
\newcommand{\mrG}{{\mathrm G}}
\newcommand{\mrH}{{\mathrm H}}
\newcommand{\mrI}{{\mathrm I}}
\newcommand{\mrJ}{{\mathrm J}}
\newcommand{\mrK}{{\mathrm K}}
\newcommand{\mrL}{{\mathrm L}}
\newcommand{\mrM}{{\mathrm M}}
\newcommand{\mrN}{{\mathrm N}}
\newcommand{\mrO}{{\mathrm O}}
\newcommand{\mrP}{{\mathrm P}}
\newcommand{\mrQ}{{\mathrm Q}}
\newcommand{\mrR}{{\mathrm R}}
\newcommand{\mrS}{{\mathrm S}}
\newcommand{\mrT}{{\mathrm T}}
\newcommand{\mrU}{{\mathrm U}}
\newcommand{\mrV}{{\mathrm V}}
\newcommand{\mrW}{{\mathrm W}}
\newcommand{\mrX}{{\mathrm X}}
\newcommand{\mrY}{{\mathrm Y}}
\newcommand{\mrZ}{{\mathrm Z}}


% 粗体的 Greek 字母
\newcommand{\balpha}{{\bm \alpha}}
\newcommand{\bbeta}{{\bm \beta}}
\newcommand{\bgamma}{{\bm \gamma}}
\newcommand{\bdelta}{{\bm \delta}}
\newcommand{\bepsilon}{{\bm \epsilon}}
\newcommand{\bvarepsilon}{{\bm \varepsilon}}
\newcommand{\bzeta}{{\bm \zeta}}
\newcommand{\bfeta}{{\bm \eta}}
\newcommand{\btheta}{{\bm \theta}}
\newcommand{\biota}{{\bm \iota}}
\newcommand{\bkappa}{{\bm \kappa}}
\newcommand{\blambda}{{\bm \lambda}}
\newcommand{\bmu}{{\bm \mu}}
\newcommand{\bnu}{{\bm \nu}}
\newcommand{\bxi}{{\bm \xi}}
\newcommand{\bomicron}{{\bm \omicron}}
\newcommand{\bpi}{{\bm \pi}}
\newcommand{\brho}{{\bm \rho}}
\newcommand{\bsigma}{{\bm \sigma}}
\newcommand{\btau}{{\bm \tau}}
\newcommand{\bupsilon}{{\bm \upsilon}}
\newcommand{\bphi}{{\bm \phi}}
\newcommand{\bvarphi}{{\bm \varphi}}
\newcommand{\bchi}{{\bm \chi}}
\newcommand{\bpsi}{{\bm \psi}}

\newcommand{\bAlpha}{{\bm \Alpha}}
\newcommand{\bBeta}{{\bm \Beta}}
\newcommand{\bGamma}{{\bm \Gamma}}
\newcommand{\bDelta}{{\bm \Delta}}
\newcommand{\bEpsilon}{{\bm \Psilon}}
\newcommand{\bVarepsilon}{{\bm \Varepsilon}}
\newcommand{\bZeta}{{\bm \Zeta}}
\newcommand{\bEta}{{\bm \Eta}}
\newcommand{\bTheta}{{\bm \Theta}}
\newcommand{\bIota}{{\bm \Iota}}
\newcommand{\bKappa}{{\bm \Kappa}}
\newcommand{\bLambda}{{\bm \Lambda}}
\newcommand{\bMu}{{\bm \Mu}}
\newcommand{\bNu}{{\bm \Nu}}
\newcommand{\bXi}{{\bm \Xi}}
\newcommand{\bOmicron}{{\bm \Omicron}}
\newcommand{\bPi}{{\bm \Pi}}
\newcommand{\bRho}{{\bm \Rho}}
\newcommand{\bSigma}{{\bm \Sigma}}
\newcommand{\bTau}{{\bm \Tau}}
\newcommand{\bUpsilon}{{\bm \Upsilon}}
\newcommand{\bPhi}{{\bm \Phi}}
\newcommand{\bChi}{{\bm \Chi}}
\newcommand{\bPsi}{{\bm \Psi}}

% \int_\Omega \bfx^2 \rmd \bfx
\newcommand{\rmd}{\,\mathrm d}
\newcommand{\bfzero}{\mathbf 0}

%% 算子
\newcommand{\ospan}{\operatorname{span}}
\newcommand{\odiv}{\operatorname{div}}
\newcommand{\otr}{\operatorname{tr}}
\newcommand{\ograd}{\operatorname{grad}}
\newcommand{\orot}{\operatorname{rot}}
\newcommand{\ocurl}{\operatorname{curl}}
\newcommand{\odist}{\operatorname{dist}}
\newcommand{\osign}{\operatorname{sign}}
\newcommand{\odiag}{\operatorname{diag}}
\newcommand{\oran}{\operatorname{Ran}} % 像空间
\newcommand{\oker}{\operatorname{Ker}} % 核空间
\newcommand{\ore}{\operatorname{Re}} % 实部
\newcommand{\oim}{\operatorname{Im}} % 虚部
\newcommand{\orank}{\operatorname{rank}}
\newcommand{\ovec}{\operatorname{vec}}
\newcommand{\odet}{\operatorname{det}}
\newcommand{\odim}{\operatorname{dim}}
\newcommand{\osym}{\operatorname{sym}}

\newcommand{\obcurl}{\operatorname{\bf curl}}
%%%% 个性设置结束
%%%%%%%%------------------------------------------------------------------------


%%%%%%%%------------------------------------------------------------------------
%%%% bibtex设置

%% 设定参考文献显示风格
% 下面是几种常见的样式
% * plain: 按字母的顺序排列,比较次序为作者、年度和标题
% * unsrt: 样式同plain,只是按照引用的先后排序
% * alpha: 用作者名首字母+年份后两位作标号,以字母顺序排序
% * abbrv: 类似plain,将月份全拼改为缩写,更显紧凑
% * apalike: 美国心理学学会期刊样式, 引用样式 [Tailper and Zang, 2006]

%\makeatletter
%\@ifclassloaded{beamer}{
%\bibliographystyle{apalike}
%}{
%\bibliographystyle{abbrv}
%}
%\makeatother


%%%% bibtex设置结束
%%%%%%%%------------------------------------------------------------------------

%%%%%%%%------------------------------------------------------------------------
%%%% xeCJK相关宏包

\usepackage{xltxtra, fontenc, xunicode}
\usepackage[slantfont, boldfont]{xeCJK}

\setlength{\parindent}{1.5em}%中文缩进两个汉字位

%% 针对中文进行断行
\XeTeXlinebreaklocale "zh"

%% 给予TeX断行一定自由度
\XeTeXlinebreakskip = 0pt plus 1pt minus 0.1pt

%%%% xeCJK设置结束
%%%%%%%%------------------------------------------------------------------------

%%%%%%%%------------------------------------------------------------------------
%%%% xeCJK字体设置

%% 设置中文标点样式,支持quanjiao、banjiao、kaiming等多种方式
\punctstyle{kaiming}

%% 设置缺省中文字体
\setCJKmainfont[BoldFont={Adobe Heiti Std}, ItalicFont={Adobe Kaiti Std}]{Adobe Song Std}
%\setCJKmainfont{Adobe Kaiti Std}
%% 设置中文无衬线字体
%\setCJKsansfont[BoldFont={Adobe Heiti Std}]{Adobe Kaiti Std}
%% 设置等宽字体
%\setCJKmonofont{Adobe Heiti Std}

%% 英文衬线字体
\setmainfont{DejaVu Serif}
%% 英文等宽字体
\setmonofont{DejaVu Sans Mono}
%% 英文无衬线字体
\setsansfont{DejaVu Sans}

%% 定义新字体
\setCJKfamilyfont{song}{Adobe Song Std}
\setCJKfamilyfont{kai}{Adobe Kaiti Std}
\setCJKfamilyfont{hei}{Adobe Heiti Std}
\setCJKfamilyfont{fangsong}{Adobe Fangsong Std}
\setCJKfamilyfont{lisu}{LiSu}
\setCJKfamilyfont{youyuan}{YouYuan}

%% 自定义宋体
\newcommand{\song}{\CJKfamily{song}}
%% 自定义楷体
\newcommand{\kai}{\CJKfamily{kai}}
%% 自定义黑体
\newcommand{\hei}{\CJKfamily{hei}}
%% 自定义仿宋体
\newcommand{\fangsong}{\CJKfamily{fangsong}}
%% 自定义隶书
\newcommand{\lisu}{\CJKfamily{lisu}}
%% 自定义幼圆
\newcommand{\youyuan}{\CJKfamily{youyuan}}

%%%% xeCJK字体设置结束
%%%%%%%%------------------------------------------------------------------------

%%%%%%%%------------------------------------------------------------------------
%%%% 一些关于中文文档的重定义
\newcommand{\chntoday}{\number\year\,年\,\number\month\,月\,\number\day\,日}
%% 数学公式定理的重定义

%% 中文破折号,据说来自清华模板
\newcommand{\pozhehao}{\kern0.3ex\rule[0.8ex]{2em}{0.1ex}\kern0.3ex}

\makeatletter %
\@ifclassloaded{beamer}{

}{
\newtheorem{example}{例}
\newtheorem{theorem}{定理}[section]
\newtheorem{definition}{定义}
\newtheorem{axiom}{公理}
\newtheorem{property}{性质}
\newtheorem{proposition}{命题}
\newtheorem{lemma}{引理}
\newtheorem{corollary}{推论}
\newtheorem{remark}{注解}
\newtheorem{condition}{条件}
\newtheorem{conclusion}{结论}
\newtheorem{assumption}{假设}
}
\makeatother

\makeatletter %
\@ifclassloaded{beamer}{

}{
%% 章节等名称重定义
\renewcommand{\contentsname}{目录}
\renewcommand{\indexname}{索引}
\renewcommand{\listfigurename}{插图目录}
\renewcommand{\listtablename}{表格目录}
\renewcommand{\appendixname}{附录}
\renewcommand{\appendixpagename}{附录}
\renewcommand{\appendixtocname}{附录}
\@ifclassloaded{book}{

}{
\renewcommand{\abstractname}{摘要}
}
}
\makeatother

\renewcommand{\figurename}{图}
\renewcommand{\tablename}{表}

\makeatletter
\@ifclassloaded{book}{
\renewcommand{\bibname}{参考文献}
}{
\renewcommand{\refname}{参考文献}
}
\makeatother

\floatname{algorithm}{算法}
\renewcommand{\algorithmicrequire}{\textbf{输入:}}
\renewcommand{\algorithmicensure}{\textbf{输出:}}

\renewcommand{\today}{\number\year 年 \number\month 月 \number\day 日}

%%%% 中文重定义结束
%%%%%%%%------------------------------------------------------------------------

\begin{document}
\title{单纯形网格上的拉格朗日有限元空间}
\author{魏华祎}
\date{\chntoday}
\maketitle
\section{简介}
在数学分析中, 函数是核心的研究对象, 用函数可以表示很多我们关心的量. 但实际应用
当中涉及到的大部分函数, 是不能用初等函数或者的它们组合表达出来. 

在有界区域 $\Omega \in \mbR^d$ 上, 定义一个函数 $u(\bfx)$

\section{重心坐标函数与单纯形网格}

下面介绍重心坐标的相关知识.

设 $\{\bfx_i = (x_{i,0}, x_{i, 1}, \cdots, x_{i, d-1})^T\}_{i=0}^{d}$ 是
$\mathbb R^d$ 空间中的 $d+1$ 个点,如果它们不在一个超平面上, 即 $d$ 维向量集合
$\{\bfx_0\bfx_i\}_{i=1}^d$ 是线性独立的,也等价于下面的矩阵非奇异
\begin{equation}
    \bfA =\begin{pmatrix}
        x_{0, 0} & x_{1, 0} & \cdots & x_{d, 0} \\
        x_{0, 1} & x_{1, 1} & \cdots & x_{d, 1} \\
        \vdots   & \vdots   & \ddots & \vdots \\
        x_{0, d-1} & x_{1, d-1} & \cdots & x_{d, d-1}\\
        1 & 1 & \cdots & 1
    \end{pmatrix}
\end{equation}

 
给定任意点 $\bfx=(x_0, x_1, \cdots, x_{d-1})^T\in \mathbb R^d$, 可以得到一组实数
值 $\blambda := (\lambda_0(\bfx), \lambda_1(\bfx), \cdots, \lambda_d(\bfx))^T$, 满足如下的方程
\begin{equation}
    \bfA \blambda=
    \begin{pmatrix}
        \bfx \\ 1
    \end{pmatrix}
    \label{eq:lbc}
\end{equation}
即
\begin{equation}
    \bfx = \sum\limits_{i=0}^{d}\lambda_i(\bfx) \bfx_i,
    \text{ with} \sum\limits_{i=0}^{d}\lambda_i(\bfx) = 1.
\end{equation}

点集 $\{\bfx_i\}_{i=0}^d$ 的凸壳
\begin{equation}
    \tau = \{\bfx = \sum_{i=0}^{d}\lambda_i\bfx_i | 0\leq \lambda_i \leq
    1, \sum_{i=0}^d\lambda_i = 1\}
\end{equation}
就称为由点集 $\{\bfx_i\}_{i=0}^d$ 生成的几何 $d$-单纯形。 例如, 区间是 1-单纯形,三
角形是一个 2-单纯形, 四面体是一个 3-单纯形。

而 $\lambda_0(\bfx)$, $\lambda_1(\bfx)$, $\cdots$, $\lambda_{d}(\bfx)$ 就称为
$\bfx$
关于点集 $\{\bfx_i\}_{i=0}^d$ 重心坐标。  易知 $\lambda_0(\bfx)$,
$\lambda_1(\bfx)$, $\cdots$, $\lambda_{d}(\bfx)$ 是关于 $\bfx$ 的线性函数并且,
\begin{equation}
    \lambda_i(\bfx_j) = 
    \begin{cases}
        1, & i = j\\
        0, & i\not= j
    \end{cases}, 
    i, j = 0, \cdots, d
\end{equation}

\subsection{Silvester 插值多项式}

Silvester 插值多项式\cite{sheng2008} 是构造高次基函数的基础,它的定义如下
$$
R_i(k,\lambda)=
\begin{cases}
\frac{1}{i!}\prod_{l=0}^{i-1} (k\lambda-l),~& 1\leq i\leq k\\
1,& i=0\\
\end{cases}
$$
其中 $\lambda \in [0,1]$,$k$ 表示区间 $[0,1]$ 的等分数。易知 

\begin{itemize}
    \item $R_i(k, \lambda)$ 是关于 $\lambda$ 的 $i$ 次多项式。
    \item 当 $\lambda= \frac{l}{k}, l=0, 1, \cdots, i-1$ 时, $R_i(k, \lambda) =0$。
    \item 当 $\lambda=\frac{i}{k}$ 时 $R_i(k, \lambda) = 1$。
\end{itemize}

\section{单纯形上的拉格朗日有限元基函数构造}

在这一节,我们讨论几何$d$-单纯形上的任意$k$次拉格朗日形函数的构造。

\subsection{任意维任意次的公式}

记 $\bfm$ 为一个 $d+1$ 维的多重指标向量 $(m_0, m_1, \cdots, m_d)$,满足  
\begin{equation*}
    m_i \geq 0, i=0, 1, \cdots, d, \text{ and } \sum_{i=0}^d m_i=k.
\end{equation*}
$\bfm$ 所有可能取值的个数为: 
\begin{equation*}
    n_k := \begin{pmatrix}
        d \\ k+d 
    \end{pmatrix}
\end{equation*}

记 $\alpha$ 为多重指标 $\bfm$ 从 0 到 $n_k-1$ 的一维自然编号, 下面表格
\ref{tb:num} 展示了具体编号规则:
\begin{table}[H]
    \centering
    \begin{tabular}{| l | c | c | c | c | c|}
    \hline
    $\alpha$ & \multicolumn{5}{c|}{$\bfm_\alpha$} \\\hline
    0 & k   & 0 & 0 & $\cdots$ & 0 \\\hline
    1 & k-1 & 1 & 0 & $\cdots$ & 0 \\\hline
    2 & k-1 & 0 & 1 & $\cdots$ & 0 \\\hline
    $\vdots$ & $\vdots$ & $\vdots$ & $\vdots$ & $\ddots$ & $\vdots$ \\\hline
    d & k-1 & 0 & 0 & $\cdots$ & 1 \\\hline
    d+1 & k-2 & 2 & 0 & $\cdots$ & 0 \\\hline
    d+2 & k-2 & 1 & 1 & $\cdots$ & 0 \\\hline
    $\vdots$ & $\vdots$ & $\vdots$ & $\vdots$ & $\ddots$ & $\vdots$ \\\hline
    2d-1 & k-2 & 1 & 0 & $\cdots$ & 1 \\\hline
    2d & k-2 & 0 & 2 & $\cdots$ & 0 \\\hline
    $\vdots$ & $\vdots$ & $\vdots$ & $\vdots$ & $\ddots$ & $\vdots$ \\\hline
    $n_p$ & 0 & 0 & 0 & $\cdots$ & k \\
    \hline
    \end{tabular}
    \caption{多重指标 $\bfm_\alpha$ 自然编号规则。}\label{tb:num}
\end{table}

给定第 $\alpha$ 个多重指标 $\bfm_\alpha$, 可以用如下方式构造一个 $d$-单纯形
$\tau$ 上的 $k$ 次多项式函数

\begin{equation}
    \phi_{\alpha}(\bfx) = \frac{1}{\bfm_\alpha!}\prod_{i=0}^{d}\prod_{j =
    0}^{m_i - 1} (k\lambda_i(\bfx) - j) = \prod_{i=0}^dR_{m_i}(k, \lambda_i(\bfx))
    \label{eq:phi0}
\end{equation}
其中
\begin{align*}
    \bfm_\alpha! = m_0!m_1!\cdots m_d! 
\end{align*}
对于每一个多重指标 $\bfm_\alpha$, 都可以找到一个在 $d$-单纯形 $\tau$ 中的点
$\bfx_\alpha$ 满足:
\begin{equation*}
    \bfx_\alpha = \sum_{i=0}^d \frac{m_i}{k} \bfx_i. 
\end{equation*}
其中 $m_i$ 是 $\bfm_\alpha$ 的第 $i$ 个分量。容易验证 $\bfx_\alpha$ 即是
$\phi_{\alpha}$ 对应的插值点,满足 
\begin{equation}
    \phi_\alpha(\bfx_\beta) = 
    \begin{cases}
        1, & \alpha = \beta\\
        0, & \alpha \ne \beta
    \end{cases}
    \label{eq:phi1}
\end{equation}
其中 $\alpha, \beta = 0, 1, \cdots, n_k-1$。 

给定一个 $\mathbb R^d$ 空间中的三角形网格 $\mcT$, 基于 \eqref{eq:phi0},就可以构
造任意次的分片连续拉格朗日有限元空间。

\begin{figure}[ht]
    \centering
    \includegraphics[scale=0.4]{./figures/int.pdf}
    \includegraphics[scale=0.4]{./figures/intdof4.pdf}
    \caption{区间单元上的 4 次有限元自由度插值节点分布与编号。}
    \label{fig:int4}
\end{figure}

\begin{figure}[ht]
    \centering
    \includegraphics[scale=0.4]{./figures/tri.pdf}
    \includegraphics[scale=0.4]{./figures/tridof4.pdf}
    \caption{三角形单元上的 4 次有限元自由度插值节点分布与编号。}
    \label{fig:tri4}
\end{figure}

\begin{figure}[ht]
    \centering
    \includegraphics[scale=0.4]{./figures/tet.pdf}
    \includegraphics[scale=0.4]{./figures/tetdof4.pdf}
    \caption{四面体单元上的 4 次有限元自由度插值节点分布与编号。}
    \label{fig:tet4}
\end{figure}
\newpage 

\subsection{$\phi_\alpha$ 和 $\nabla \phi_\alpha$ 面向数组的计算}

这一节,我们讨论如何用面向数组的方式来计算 $\phi_\alpha$ 和 $\nabla \phi_\alpha$
在任意一个重心坐标点 $(\lambda_0, \lambda_1, \cdots, \lambda_d)$ 处的值

首先构造一个向量
\begin{equation*}
    \bfP = (\frac{1}{0!}, \frac{1}{1!}, \frac{1}{2!}, \cdots, \frac{1}{k!}),
\end{equation*}
和一个矩阵
\begin{equation}\label{eq:A}
\bfA :=                                                                            
\begin{pmatrix}  
1  &  1  & \cdot & 1 \\
k\lambda_0 & k\lambda_1 & \cdots & k\lambda_d\\                                             
k\lambda_0 - 1 & k\lambda_1 - 1 & \cdots & k\lambda_d - 1\\   
\vdots & \vdots & \ddots & \vdots \\                                                     
k\lambda_0 - (k - 1) & k\lambda_1 - (k - 1) & \vdots & k\lambda_d - (k - 1)
\end{pmatrix},
\end{equation}
然后矩阵 $\bfA$ 每一列做累乘运算,并左乘一个以 $\bfP$ 为对角线的对角矩阵,可以得
到:

\begin{equation}\label{eq:B}
\bfB = \mathrm{diag}(\bfP)
\begin{pmatrix}
1 & 1 & \cdots & 1 \\
k\lambda_0 & k\lambda_1 & \cdots & k\lambda_d\\
\prod_{j=0}^{1}(k\lambda_0 - j) & \prod_{j=0}^{1}(k\lambda_1 - j)
& \cdots & \prod_{j=0}^{1}(k\lambda_d - j) \\
\vdots & \vdots & \ddots & \vdots \\
\prod_{j=0}^{k-1}(k\lambda_0 - j) & \prod_{j=0}^{k-1}(k\lambda_1 - j) & \cdots & \prod_{j=0}^{k-1}(k\lambda_d - j) 
\end{pmatrix}
\end{equation}
注意到矩阵 $\bfB$ 含有公式 \eqref{eq:phi0} 中的所有可能的基础模块,且有 
\begin{equation*}
    \phi_\alpha = \prod_{i=0}^d\bfB_{m_i, i}
\end{equation*}
其中 $m_i$ 是 $\bfm_\alpha$ 的第 $i$ 分量。

下面讨论 $\nabla \phi_\alpha$ 的面向数组计算问题。利用函数乘积的求导法则,可得   
\begin{equation*}
    \nabla \prod_{j = 0}^{m_i - 1} (k\lambda_i - j)
    = k\sum_{j=0}^{m_i - 1}\prod_{0\le l \le m_i-1, l\not= j}(k\lambda_i -
    l)\nabla \lambda_i.
\end{equation*}
$$
\bfD^i = 
\begin{pmatrix}
k & k\lambda_i & \cdots & k\lambda_i \\
k\lambda_i - 1 & k & \cdots & k\lambda_i - 1 \\
\vdots & \vdots & \ddots & \vdots \\
k\lambda_i - (k-1) & k\lambda_i - (k-1) & \cdots & k 
\end{pmatrix}
, \quad 0 \le i \le d, 
$$
则可得到矩阵 $\bfD$,其元素为 
$$
\bfD_{i,j} = \sum_{m=0}^j\prod_{k=0}^j D^i_{k, m},\quad 0 \le i \le d,
, 0 \le j \le k-1.
$$
最后,可以用如下的方式来计算 $\bfB$ 的梯度:
\begin{equation*}
\begin{aligned}
\nabla \bfB = & \mathrm{diag}(\bfP)
\begin{pmatrix}
0 & 0 & \cdots & 0 \\
\bfD_{0,0} \nabla \lambda_0 & 
\bfD_{1,0} \nabla \lambda_1 & \cdots& 
\bfD_{d,0} \nabla \lambda_d \\
\vdots & \vdots & \ddots & \vdots \\
\bfD_{0, k-1} \nabla \lambda_0 &
\bfD_{1, k-1} \nabla \lambda_1 & \cdots &
\bfD_{d, k-1} \nabla \lambda_d 
\end{pmatrix}\\
= & \mathrm{diag}(\bfP)
\begin{pmatrix}
\mathbf 0\\
\bfD
\end{pmatrix}
\begin{pmatrix}
\nabla \lambda_0 &  &  & \\
 & \nabla \lambda_1 & & \\
 & & \ddots & \\
 & & & \nabla \lambda_d
\end{pmatrix}\\
= & \bfF 
\begin{pmatrix}
\nabla \lambda_0 &  &  & \\
 & \nabla \lambda_1 & & \\
 & & \ddots & \\
 & & & \nabla \lambda_d
\end{pmatrix},
\end{aligned}
\end{equation*}
其中
\begin{equation}\label{eq:F}
    \bfF = \mathrm{diag}(\bfP)
\begin{pmatrix} 
    \mathbf 0\\ \bfD
\end{pmatrix}.
\end{equation}
注意,上面公式中的 $\nabla \lambda_i$ 要看成一个整体。

\subsection{区间 $ [x_0, x_1] $ 单元上的基函数构造}

给定区间单元上的一个重心坐标 $(\lambda_0, \lambda_1)$, 存在
$ x \in [x_0, x_1]$, 使得:

\[
\lambda_0 := \frac{x_1 - x}{x_1 - x_0}, \quad
\lambda_1 := \frac{x  - x_0}{x_1 - x_0}
\]

显然 

\[
\lambda_0 + \lambda_1 = 1
\]

重心坐标关于$ x $ 的导数为:

\[
\frac{\mathrm d \lambda_0}{\mathrm d x} = -\frac{1}{x_1 - x_0},\quad
\frac{\mathrm d \lambda_1}{\mathrm d x} = \frac{1}{x_1 - x_0}
\]

区间 $[x_0, x_1]$ 上的个 $ k\geq 1 $ 次基函数共有 

\[
n_{dof} = k+1,
\]

其计算公式如下:

\[
\phi_{m,n} = \frac{p^p}{m!n!}\prod_{l_0 = 0}^{m - 1}
(\lambda_0 - \frac{l_0}{p}) \prod_{l_1 = 0}^{n-1}(\lambda_1 -
\frac{l_1}{p}).
\]

其中 $ m\geq 0$, $ n\geq 0 $, 且 $ m+n=p $, 这里规定:

\[
 \prod_{l_i=0}^{-1}(\lambda_i - \frac{l_i}{p}) := 1,\quad i=0, 1
\]

$ p $ 次基函数的面向数组的计算
构造向量: 

\[
P = ( \frac{1}{0!},  \frac{1}{1!}, \frac{1}{2!}, \cdots, \frac{1}{k!})
\]

构造矩阵: 

\[
A :=
\begin{pmatrix}
1  &  1  \\
\lambda_0 & \lambda_1\\
\lambda_0 - \frac{1}{k} & \lambda_1 - \frac{1}{k}\\
\vdots & \vdots \\
\lambda_0 - \frac{k - 1}{k} & \lambda_1 - \frac{k - 1}{k}
\end{pmatrix}
\]

对 $ A $ 的每一列做累乘运算, 并左乘由 \(P\) 形成的对角矩阵, 得矩阵:

\[
B = \mathrm{diag}(P)
\begin{pmatrix}
1 & 1\\
\lambda_0 & \lambda_1\\
\prod_{l=0}^{1}(\lambda_0 - \frac{l}{p}) & \prod_{l=0}^{1}(\lambda_1 - \frac{l}{p})\\
\vdots & \vdots \\
\prod_{l=0}^{p-1}(\lambda_0 - \frac{l}{p}) & \prod_{l=0}^{p-1}(\lambda_1 - \frac{l}{p})
\end{pmatrix}
\]

易知, 只需从 \(B\) 的每一列中各选择一项相乘(要求二项次数之和为 \(p\)),
再乘以 \(p^p\) 即可得到相应的基函数, 其中取法共有

\[
n_{dof} = {p+1}
\]

构造指标矩阵:

\[
I = \begin{pmatrix}
p  & 0 \\ p-1 & 1 \\ \vdots & \vdots \\ 0 & p
\end{pmatrix}
\]

则 \(p+1\) 个 \(p\) 次基函数可写成如下形式

\[
\phi_i = p^pB_{I_{i,0}, 0}B_{I_{i, 1},1}, \quad i = 0, 1, \cdots, n_{dof}
\]

\cite{wei_fealpy}
\bibliographystyle{abbrv}
\bibliography{ref}
\end{document}
