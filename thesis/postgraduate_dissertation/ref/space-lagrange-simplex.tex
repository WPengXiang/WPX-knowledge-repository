% !Mode:: "TeX:UTF-8"
\documentclass{article}
\input{../en_preamble.tex}
\input{../xecjk_preamble.tex}
\begin{document}
\title{单纯形网格上的拉格朗日有限元空间}
\author{魏华祎}
\date{\chntoday}
\maketitle
\section{简介}
在数学分析中, 函数是核心的研究对象, 用函数可以表示很多我们关心的量. 但实际应用
当中涉及到的大部分函数, 是不能用初等函数或者的它们组合表达出来. 

在有界区域 $\Omega \in \mbR^d$ 上, 定义一个函数 $u(\bfx)$

\section{重心坐标函数与单纯形网格}

下面介绍重心坐标的相关知识.

设 $\{\bfx_i = (x_{i,0}, x_{i, 1}, \cdots, x_{i, d-1})^T\}_{i=0}^{d}$ 是
$\mathbb R^d$ 空间中的 $d+1$ 个点,如果它们不在一个超平面上, 即 $d$ 维向量集合
$\{\bfx_0\bfx_i\}_{i=1}^d$ 是线性独立的,也等价于下面的矩阵非奇异
\begin{equation}
    \bfA =\begin{pmatrix}
        x_{0, 0} & x_{1, 0} & \cdots & x_{d, 0} \\
        x_{0, 1} & x_{1, 1} & \cdots & x_{d, 1} \\
        \vdots   & \vdots   & \ddots & \vdots \\
        x_{0, d-1} & x_{1, d-1} & \cdots & x_{d, d-1}\\
        1 & 1 & \cdots & 1
    \end{pmatrix}
\end{equation}

 
给定任意点 $\bfx=(x_0, x_1, \cdots, x_{d-1})^T\in \mathbb R^d$, 可以得到一组实数
值 $\blambda := (\lambda_0(\bfx), \lambda_1(\bfx), \cdots, \lambda_d(\bfx))^T$, 满足如下的方程
\begin{equation}
    \bfA \blambda=
    \begin{pmatrix}
        \bfx \\ 1
    \end{pmatrix}
    \label{eq:lbc}
\end{equation}
即
\begin{equation}
    \bfx = \sum\limits_{i=0}^{d}\lambda_i(\bfx) \bfx_i,
    \text{ with} \sum\limits_{i=0}^{d}\lambda_i(\bfx) = 1.
\end{equation}

点集 $\{\bfx_i\}_{i=0}^d$ 的凸壳
\begin{equation}
    \tau = \{\bfx = \sum_{i=0}^{d}\lambda_i\bfx_i | 0\leq \lambda_i \leq
    1, \sum_{i=0}^d\lambda_i = 1\}
\end{equation}
就称为由点集 $\{\bfx_i\}_{i=0}^d$ 生成的几何 $d$-单纯形。 例如, 区间是 1-单纯形,三
角形是一个 2-单纯形, 四面体是一个 3-单纯形。

而 $\lambda_0(\bfx)$, $\lambda_1(\bfx)$, $\cdots$, $\lambda_{d}(\bfx)$ 就称为
$\bfx$
关于点集 $\{\bfx_i\}_{i=0}^d$ 重心坐标。  易知 $\lambda_0(\bfx)$,
$\lambda_1(\bfx)$, $\cdots$, $\lambda_{d}(\bfx)$ 是关于 $\bfx$ 的线性函数并且,
\begin{equation}
    \lambda_i(\bfx_j) = 
    \begin{cases}
        1, & i = j\\
        0, & i\not= j
    \end{cases}, 
    i, j = 0, \cdots, d
\end{equation}

\subsection{Silvester 插值多项式}

Silvester 插值多项式\cite{sheng2008} 是构造高次基函数的基础,它的定义如下
$$
R_i(k,\lambda)=
\begin{cases}
\frac{1}{i!}\prod_{l=0}^{i-1} (k\lambda-l),~& 1\leq i\leq k\\
1,& i=0\\
\end{cases}
$$
其中 $\lambda \in [0,1]$,$k$ 表示区间 $[0,1]$ 的等分数。易知 

\begin{itemize}
    \item $R_i(k, \lambda)$ 是关于 $\lambda$ 的 $i$ 次多项式。
    \item 当 $\lambda= \frac{l}{k}, l=0, 1, \cdots, i-1$ 时, $R_i(k, \lambda) =0$。
    \item 当 $\lambda=\frac{i}{k}$ 时 $R_i(k, \lambda) = 1$。
\end{itemize}

\section{单纯形上的拉格朗日有限元基函数构造}

在这一节,我们讨论几何$d$-单纯形上的任意$k$次拉格朗日形函数的构造。

\subsection{任意维任意次的公式}

记 $\bfm$ 为一个 $d+1$ 维的多重指标向量 $(m_0, m_1, \cdots, m_d)$,满足  
\begin{equation*}
    m_i \geq 0, i=0, 1, \cdots, d, \text{ and } \sum_{i=0}^d m_i=k.
\end{equation*}
$\bfm$ 所有可能取值的个数为: 
\begin{equation*}
    n_k := \begin{pmatrix}
        d \\ k+d 
    \end{pmatrix}
\end{equation*}

记 $\alpha$ 为多重指标 $\bfm$ 从 0 到 $n_k-1$ 的一维自然编号, 下面表格
\ref{tb:num} 展示了具体编号规则:
\begin{table}[H]
    \centering
    \begin{tabular}{| l | c | c | c | c | c|}
    \hline
    $\alpha$ & \multicolumn{5}{c|}{$\bfm_\alpha$} \\\hline
    0 & k   & 0 & 0 & $\cdots$ & 0 \\\hline
    1 & k-1 & 1 & 0 & $\cdots$ & 0 \\\hline
    2 & k-1 & 0 & 1 & $\cdots$ & 0 \\\hline
    $\vdots$ & $\vdots$ & $\vdots$ & $\vdots$ & $\ddots$ & $\vdots$ \\\hline
    d & k-1 & 0 & 0 & $\cdots$ & 1 \\\hline
    d+1 & k-2 & 2 & 0 & $\cdots$ & 0 \\\hline
    d+2 & k-2 & 1 & 1 & $\cdots$ & 0 \\\hline
    $\vdots$ & $\vdots$ & $\vdots$ & $\vdots$ & $\ddots$ & $\vdots$ \\\hline
    2d-1 & k-2 & 1 & 0 & $\cdots$ & 1 \\\hline
    2d & k-2 & 0 & 2 & $\cdots$ & 0 \\\hline
    $\vdots$ & $\vdots$ & $\vdots$ & $\vdots$ & $\ddots$ & $\vdots$ \\\hline
    $n_p$ & 0 & 0 & 0 & $\cdots$ & k \\
    \hline
    \end{tabular}
    \caption{多重指标 $\bfm_\alpha$ 自然编号规则。}\label{tb:num}
\end{table}

给定第 $\alpha$ 个多重指标 $\bfm_\alpha$, 可以用如下方式构造一个 $d$-单纯形
$\tau$ 上的 $k$ 次多项式函数

\begin{equation}
    \phi_{\alpha}(\bfx) = \frac{1}{\bfm_\alpha!}\prod_{i=0}^{d}\prod_{j =
    0}^{m_i - 1} (k\lambda_i(\bfx) - j) = \prod_{i=0}^dR_{m_i}(k, \lambda_i(\bfx))
    \label{eq:phi0}
\end{equation}
其中
\begin{align*}
    \bfm_\alpha! = m_0!m_1!\cdots m_d! 
\end{align*}
对于每一个多重指标 $\bfm_\alpha$, 都可以找到一个在 $d$-单纯形 $\tau$ 中的点
$\bfx_\alpha$ 满足:
\begin{equation*}
    \bfx_\alpha = \sum_{i=0}^d \frac{m_i}{k} \bfx_i. 
\end{equation*}
其中 $m_i$ 是 $\bfm_\alpha$ 的第 $i$ 个分量。容易验证 $\bfx_\alpha$ 即是
$\phi_{\alpha}$ 对应的插值点,满足 
\begin{equation}
    \phi_\alpha(\bfx_\beta) = 
    \begin{cases}
        1, & \alpha = \beta\\
        0, & \alpha \ne \beta
    \end{cases}
    \label{eq:phi1}
\end{equation}
其中 $\alpha, \beta = 0, 1, \cdots, n_k-1$。 

给定一个 $\mathbb R^d$ 空间中的三角形网格 $\mcT$, 基于 \eqref{eq:phi0},就可以构
造任意次的分片连续拉格朗日有限元空间。

\begin{figure}[ht]
    \centering
    \includegraphics[scale=0.4]{./figures/int.pdf}
    \includegraphics[scale=0.4]{./figures/intdof4.pdf}
    \caption{区间单元上的 4 次有限元自由度插值节点分布与编号。}
    \label{fig:int4}
\end{figure}

\begin{figure}[ht]
    \centering
    \includegraphics[scale=0.4]{./figures/tri.pdf}
    \includegraphics[scale=0.4]{./figures/tridof4.pdf}
    \caption{三角形单元上的 4 次有限元自由度插值节点分布与编号。}
    \label{fig:tri4}
\end{figure}

\begin{figure}[ht]
    \centering
    \includegraphics[scale=0.4]{./figures/tet.pdf}
    \includegraphics[scale=0.4]{./figures/tetdof4.pdf}
    \caption{四面体单元上的 4 次有限元自由度插值节点分布与编号。}
    \label{fig:tet4}
\end{figure}
\newpage 

\subsection{$\phi_\alpha$ 和 $\nabla \phi_\alpha$ 面向数组的计算}

这一节,我们讨论如何用面向数组的方式来计算 $\phi_\alpha$ 和 $\nabla \phi_\alpha$
在任意一个重心坐标点 $(\lambda_0, \lambda_1, \cdots, \lambda_d)$ 处的值

首先构造一个向量
\begin{equation*}
    \bfP = (\frac{1}{0!}, \frac{1}{1!}, \frac{1}{2!}, \cdots, \frac{1}{k!}),
\end{equation*}
和一个矩阵
\begin{equation}\label{eq:A}
\bfA :=                                                                            
\begin{pmatrix}  
1  &  1  & \cdot & 1 \\
k\lambda_0 & k\lambda_1 & \cdots & k\lambda_d\\                                             
k\lambda_0 - 1 & k\lambda_1 - 1 & \cdots & k\lambda_d - 1\\   
\vdots & \vdots & \ddots & \vdots \\                                                     
k\lambda_0 - (k - 1) & k\lambda_1 - (k - 1) & \vdots & k\lambda_d - (k - 1)
\end{pmatrix},
\end{equation}
然后矩阵 $\bfA$ 每一列做累乘运算,并左乘一个以 $\bfP$ 为对角线的对角矩阵,可以得
到:

\begin{equation}\label{eq:B}
\bfB = \mathrm{diag}(\bfP)
\begin{pmatrix}
1 & 1 & \cdots & 1 \\
k\lambda_0 & k\lambda_1 & \cdots & k\lambda_d\\
\prod_{j=0}^{1}(k\lambda_0 - j) & \prod_{j=0}^{1}(k\lambda_1 - j)
& \cdots & \prod_{j=0}^{1}(k\lambda_d - j) \\
\vdots & \vdots & \ddots & \vdots \\
\prod_{j=0}^{k-1}(k\lambda_0 - j) & \prod_{j=0}^{k-1}(k\lambda_1 - j) & \cdots & \prod_{j=0}^{k-1}(k\lambda_d - j) 
\end{pmatrix}
\end{equation}
注意到矩阵 $\bfB$ 含有公式 \eqref{eq:phi0} 中的所有可能的基础模块,且有 
\begin{equation*}
    \phi_\alpha = \prod_{i=0}^d\bfB_{m_i, i}
\end{equation*}
其中 $m_i$ 是 $\bfm_\alpha$ 的第 $i$ 分量。

下面讨论 $\nabla \phi_\alpha$ 的面向数组计算问题。利用函数乘积的求导法则,可得   
\begin{equation*}
    \nabla \prod_{j = 0}^{m_i - 1} (k\lambda_i - j)
    = k\sum_{j=0}^{m_i - 1}\prod_{0\le l \le m_i-1, l\not= j}(k\lambda_i -
    l)\nabla \lambda_i.
\end{equation*}
$$
\bfD^i = 
\begin{pmatrix}
k & k\lambda_i & \cdots & k\lambda_i \\
k\lambda_i - 1 & k & \cdots & k\lambda_i - 1 \\
\vdots & \vdots & \ddots & \vdots \\
k\lambda_i - (k-1) & k\lambda_i - (k-1) & \cdots & k 
\end{pmatrix}
, \quad 0 \le i \le d, 
$$
则可得到矩阵 $\bfD$,其元素为 
$$
\bfD_{i,j} = \sum_{m=0}^j\prod_{k=0}^j D^i_{k, m},\quad 0 \le i \le d,
, 0 \le j \le k-1.
$$
最后,可以用如下的方式来计算 $\bfB$ 的梯度:
\begin{equation*}
\begin{aligned}
\nabla \bfB = & \mathrm{diag}(\bfP)
\begin{pmatrix}
0 & 0 & \cdots & 0 \\
\bfD_{0,0} \nabla \lambda_0 & 
\bfD_{1,0} \nabla \lambda_1 & \cdots& 
\bfD_{d,0} \nabla \lambda_d \\
\vdots & \vdots & \ddots & \vdots \\
\bfD_{0, k-1} \nabla \lambda_0 &
\bfD_{1, k-1} \nabla \lambda_1 & \cdots &
\bfD_{d, k-1} \nabla \lambda_d 
\end{pmatrix}\\
= & \mathrm{diag}(\bfP)
\begin{pmatrix}
\mathbf 0\\
\bfD
\end{pmatrix}
\begin{pmatrix}
\nabla \lambda_0 &  &  & \\
 & \nabla \lambda_1 & & \\
 & & \ddots & \\
 & & & \nabla \lambda_d
\end{pmatrix}\\
= & \bfF 
\begin{pmatrix}
\nabla \lambda_0 &  &  & \\
 & \nabla \lambda_1 & & \\
 & & \ddots & \\
 & & & \nabla \lambda_d
\end{pmatrix},
\end{aligned}
\end{equation*}
其中
\begin{equation}\label{eq:F}
    \bfF = \mathrm{diag}(\bfP)
\begin{pmatrix} 
    \mathbf 0\\ \bfD
\end{pmatrix}.
\end{equation}
注意,上面公式中的 $\nabla \lambda_i$ 要看成一个整体。

\subsection{区间 $ [x_0, x_1] $ 单元上的基函数构造}

给定区间单元上的一个重心坐标 $(\lambda_0, \lambda_1)$, 存在
$ x \in [x_0, x_1]$, 使得:

\[
\lambda_0 := \frac{x_1 - x}{x_1 - x_0}, \quad
\lambda_1 := \frac{x  - x_0}{x_1 - x_0}
\]

显然 

\[
\lambda_0 + \lambda_1 = 1
\]

重心坐标关于$ x $ 的导数为:

\[
\frac{\mathrm d \lambda_0}{\mathrm d x} = -\frac{1}{x_1 - x_0},\quad
\frac{\mathrm d \lambda_1}{\mathrm d x} = \frac{1}{x_1 - x_0}
\]

区间 $[x_0, x_1]$ 上的个 $ k\geq 1 $ 次基函数共有 

\[
n_{dof} = k+1,
\]

其计算公式如下:

\[
\phi_{m,n} = \frac{p^p}{m!n!}\prod_{l_0 = 0}^{m - 1}
(\lambda_0 - \frac{l_0}{p}) \prod_{l_1 = 0}^{n-1}(\lambda_1 -
\frac{l_1}{p}).
\]

其中 $ m\geq 0$, $ n\geq 0 $, 且 $ m+n=p $, 这里规定:

\[
 \prod_{l_i=0}^{-1}(\lambda_i - \frac{l_i}{p}) := 1,\quad i=0, 1
\]

$ p $ 次基函数的面向数组的计算
构造向量: 

\[
P = ( \frac{1}{0!},  \frac{1}{1!}, \frac{1}{2!}, \cdots, \frac{1}{k!})
\]

构造矩阵: 

\[
A :=
\begin{pmatrix}
1  &  1  \\
\lambda_0 & \lambda_1\\
\lambda_0 - \frac{1}{k} & \lambda_1 - \frac{1}{k}\\
\vdots & \vdots \\
\lambda_0 - \frac{k - 1}{k} & \lambda_1 - \frac{k - 1}{k}
\end{pmatrix}
\]

对 $ A $ 的每一列做累乘运算, 并左乘由 \(P\) 形成的对角矩阵, 得矩阵:

\[
B = \mathrm{diag}(P)
\begin{pmatrix}
1 & 1\\
\lambda_0 & \lambda_1\\
\prod_{l=0}^{1}(\lambda_0 - \frac{l}{p}) & \prod_{l=0}^{1}(\lambda_1 - \frac{l}{p})\\
\vdots & \vdots \\
\prod_{l=0}^{p-1}(\lambda_0 - \frac{l}{p}) & \prod_{l=0}^{p-1}(\lambda_1 - \frac{l}{p})
\end{pmatrix}
\]

易知, 只需从 \(B\) 的每一列中各选择一项相乘(要求二项次数之和为 \(p\)),
再乘以 \(p^p\) 即可得到相应的基函数, 其中取法共有

\[
n_{dof} = {p+1}
\]

构造指标矩阵:

\[
I = \begin{pmatrix}
p  & 0 \\ p-1 & 1 \\ \vdots & \vdots \\ 0 & p
\end{pmatrix}
\]

则 \(p+1\) 个 \(p\) 次基函数可写成如下形式

\[
\phi_i = p^pB_{I_{i,0}, 0}B_{I_{i, 1},1}, \quad i = 0, 1, \cdots, n_{dof}
\]

\cite{wei_fealpy}
\bibliographystyle{abbrv}
\bibliography{ref}
\end{document}
