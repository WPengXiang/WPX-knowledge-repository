% !Mode:: "TeX:UTF-8"
\documentclass{article}

%%%%%%%%------------------------------------------------------------------------
%%%% 日常所用宏包

%% 控制页边距
% 如果是beamer文档类, 则不用geometry
\makeatletter
\@ifclassloaded{beamer}{}{\usepackage[top=2.5cm, bottom=2.5cm, left=2.5cm, right=2.5cm]{geometry}}
\makeatother

\makeatletter
\@ifclassloaded{beamer}{
\makeatletter
\def\th@mystyle{%
    \normalfont % body font
    \setbeamercolor{block title example}{bg=orange,fg=white}
    \setbeamercolor{block body example}{bg=orange!20,fg=black}
    \def\inserttheoremblockenv{exampleblock}
  }
\makeatother
\theoremstyle{mystyle}
\newtheorem*{remark}{Remark}

\newcommand{\propnumber}{} % initialize
\newtheorem*{prop}{Proposition \propnumber}
\newenvironment{propc}[1]
  {\renewcommand{\propnumber}{#1}%
   \begin{shaded}\begin{prop}}
  {\end{prop}\end{shaded}}

\makeatletter
\newenvironment<>{proofs}[1][\proofname]{%
    \par
    \def\insertproofname{#1\@addpunct{.}}%
    \usebeamertemplate{proof begin}#2}
  {\usebeamertemplate{proof end}}
\makeatother

}{
}
\makeatother
\usepackage{amsthm}

%\DeclareMathOperator{\sech}{sech}
%\DeclareMathOperator{\csch}{csch}
%\DeclareMathOperator{\arcsec}{arcsec}
%\DeclareMathOperator{\arccot}{arccot}
%\DeclareMathOperator{\arccsc}{arccsc}
%\DeclareMathOperator{\arccosh}{arccosh}
%\DeclareMathOperator{\arcsinh}{arcsinh}
%\DeclareMathOperator{\arctanh}{arctanh}
%\DeclareMathOperator{\arcsech}{arcsech}
%\DeclareMathOperator{\arccsch}{arccsch}
%\DeclareMathOperator{\arccoth}{arccoth}
%% 控制项目列表
\usepackage{enumerate}

%% Todo list
\usepackage{enumitem}
\newlist{todolist}{itemize}{2}
\setlist[todolist]{label=$\square$}
\usepackage{pifont}
\newcommand{\cmark}{\ding{51}}%
\newcommand{\xmark}{\ding{55}}%
\newcommand{\done}{\rlap{$\square$}{\raisebox{2pt}{\large\hspace{1pt}\cmark}}%
\hspace{-2.5pt}}
\newcommand{\wontfix}{\rlap{$\square$}{\large\hspace{1pt}\xmark}}

\usepackage[utf8]{inputenc}
\usepackage[english]{babel}

\usepackage{framed}

%% 多栏显示
\usepackage{multicol}

%% 算法环境
\usepackage{algorithm}
\usepackage{algorithmic}
\usepackage{float}

%% 网址引用
\usepackage{url}

%% 控制矩阵行距
\renewcommand\arraystretch{1.4}

%% 粗体
\usepackage{lmodern}
\usepackage{bm}


%% hyperref宏包,生成可定位点击的超链接,并且会生成pdf书签
\makeatletter
\@ifclassloaded{beamer}{
\usepackage{hyperref}
\usepackage{ragged2e} % 对齐
}{
\usepackage[%
    pdfstartview=FitH,%
    CJKbookmarks=true,%
    bookmarks=true,%
    bookmarksnumbered=true,%
    bookmarksopen=true,%
    colorlinks=true,%
    citecolor=blue,%
    linkcolor=blue,%
    anchorcolor=green,%
    urlcolor=blue%
]{hyperref}
}
\makeatother



\makeatletter % 如果是 beamer 不需要下面两个包
\@ifclassloaded{beamer}{
\mode<presentation>
{
}
}{
%% 控制标题
\usepackage{titlesec}
%% 控制目录
\usepackage{titletoc}
}
\makeatother

%% 控制表格样式
\usepackage{booktabs}

%% 控制字体大小
\usepackage{type1cm}

%% 首行缩进,用\noindent取消某段缩进
\usepackage{indentfirst}

%% 支持彩色文本、底色、文本框等
\usepackage{color,xcolor}

%% AMS LaTeX宏包: http://zzg34b.w3.c361.com/package/maths.htm#amssymb
\usepackage{amsmath,amssymb}
%% 多个图形并排
\usepackage{subfig}
%%%% 基本插图方法
%% 图形宏包
\usepackage{graphicx}


%%%% 基本插图方法结束

%%%% pgf/tikz绘图宏包设置
\usepackage{pgf,tikz}
\usetikzlibrary{shapes,automata,snakes,backgrounds,arrows}
\usetikzlibrary{mindmap}
%% 可以直接在latex文档中使用graphviz/dot语言,
%% 也可以用dot2tex工具将dot文件转换成tex文件再include进来
%% \usepackage[shell,pgf,outputdir={docgraphs/}]{dot2texi}
%%%% pgf/tikz设置结束


\makeatletter % 如果是 beamer 不需要下面两个包
\@ifclassloaded{beamer}{

}{
%%%% fancyhdr设置页眉页脚
%% 页眉页脚宏包
\usepackage{fancyhdr}
%% 页眉页脚风格
\pagestyle{plain}
}

%% 有时会出现\headheight too small的warning
\setlength{\headheight}{15pt}

%% 清空当前页眉页脚的默认设置
%\fancyhf{}
%%%% fancyhdr设置结束


%% 设置listings宏包的一些全局样式
%% 参考http://hi.baidu.com/shawpinlee/blog/item/9ec431cbae28e41cbe09e6e4.html
\usepackage{listings}
\lstloadlanguages{[LaTeX]TeX}

\usepackage{fancyvrb}

\newenvironment{latexample}[1][language={[LaTeX]TeX}]
{\lstset{breaklines=true,
    prebreak = \raisebox{0ex}[0ex][0ex]{\ensuremath{\hookleftarrow}},
    frame=single,
    language={[LaTeX]TeX},
    showstringspaces=false,              %% 设定是否显示代码之间的空格符号
    numbers=left,                        %% 在左边显示行号
    numberstyle=\tiny,                   %% 设定行号字体的大小
    basicstyle=\scriptsize,                    %% 设定字体大小\tiny, \small, \Large等等
    keywordstyle=\color{blue!70}, commentstyle=\color{red!50!green!50!blue!50},
                                         %% 关键字高亮
    frame=shadowbox,                     %% 给代码加框
    rulesepcolor=\color{red!20!green!20!blue!20},
    escapechar=`,                        %% 中文逃逸字符,用于中英混排
    xleftmargin=2em,xrightmargin=2em, aboveskip=1em,
    %breaklines,                          %% 这条命令可以让LaTeX自动将长的代码行换行排版
    extendedchars=false                  %% 这一条命令可以解决代码跨页时,章节标题,页眉等汉字不显示的问题
    basicstyle=\footnotesize\ttfamily, #1}
  \VerbatimEnvironment\begin{VerbatimOut}{latexample.verb.out}}
  {\end{VerbatimOut}\noindent
  \begin{minipage}{1.05\linewidth}
    \lstinputlisting[]{latexample.verb.out}%
  \end{minipage}\qquad
  \begin{minipage}{1\linewidth}
    \input{latexample.verb.out}
  \end{minipage}\\
}

\usepackage{minted}
\renewcommand{\listingscaption}{Python code} \newminted{python}{
    escapeinside=||,
    mathescape=true,
    numbersep=5pt,
    linenos=true,
    autogobble,
    framesep=3mm}
%%%% listings宏包设置结束


%%%% 附录设置
\makeatletter % 对 beamer 要重新设置
\@ifclassloaded{beamer}{

}{
\usepackage[title,titletoc,header]{appendix}
}
\makeatother
%%%% 附录设置结束





%% 设定行距
\linespread{1}

%% 颜色
\newcommand{\red}{\color{red} }
\newcommand{\blue}{\color{blue} }
\newcommand{\brown}{\color{brown} }
\newcommand{\green}{\color{green} }

\newcommand{\bred}{\bf\color{red} }
\newcommand{\bblue}{\bf\color{blue} }
\newcommand{\bbrown}{\bf\color{brown} }
\newcommand{\bgreen}{\bf\color{green} }
%% 1. 小写的英文或希腊字母表示 标量或标量函数
%% 2. 大写的英文或希腊字母表示 集合或空间
%% 3. 粗体的小写字母代表向量或向量形式的常量和函数
%% 4. 粗体的大写字母代表矩阵或张量形式的常量和函数
%% 5. 空心大写字母代表特殊的空间 \mbR 实数 \mbC 复数 \mbP 多项式
%% 6. 花体的大写字母代表算子

%% 粗体的小写字母代表向量或向量函数
\newcommand{\bfa}{{\boldsymbol a}}
\newcommand{\bfb}{{\boldsymbol b}}
\newcommand{\bfc}{{\boldsymbol c}}
\newcommand{\bfd}{{\boldsymbol d}}
\newcommand{\bfe}{{\boldsymbol e}}
\newcommand{\bff}{{\boldsymbol f}}
\newcommand{\bfg}{{\boldsymbol g}}
\newcommand{\bfh}{{\boldsymbol h}}
\newcommand{\bfi}{{\boldsymbol i}}
\newcommand{\bfj}{{\boldsymbol j}}
\newcommand{\bfk}{{\boldsymbol k}}
\newcommand{\bfl}{{\boldsymbol l}}
\newcommand{\bfm}{{\boldsymbol m}}
\newcommand{\bfn}{{\boldsymbol n}}
\newcommand{\bfo}{{\boldsymbol o}}
\newcommand{\bfp}{{\boldsymbol p}}
\newcommand{\bfq}{{\boldsymbol q}}
\newcommand{\bfr}{{\boldsymbol r}}
\newcommand{\bfs}{{\boldsymbol s}}
\newcommand{\bft}{{\boldsymbol t}}
\newcommand{\bfu}{{\boldsymbol u}}
\newcommand{\bfv}{{\boldsymbol v}}
\newcommand{\bfw}{{\boldsymbol w}}
\newcommand{\bfx}{{\boldsymbol x}}
\newcommand{\bfy}{{\boldsymbol y}}
\newcommand{\bfz}{{\boldsymbol z}}

%  算子
\newcommand{\mca}{{\mathcal a}}
\newcommand{\mcb}{{\mathcal b}}
\newcommand{\mcc}{{\mathcal c}}
\newcommand{\mcd}{{\mathcal d}}
\newcommand{\mce}{{\mathcal e}}
\newcommand{\mcf}{{\mathcal f}}
\newcommand{\mcg}{{\mathcal g}}
\newcommand{\mch}{{\mathcal h}}
\newcommand{\mci}{{\mathcal i}}
\newcommand{\mcj}{{\mathcal j}}
\newcommand{\mck}{{\mathcal k}}
\newcommand{\mcl}{{\mathcal l}}
\newcommand{\mcm}{{\mathcal m}}
\newcommand{\mcn}{{\mathcal n}}
\newcommand{\mco}{{\mathcal o}}
\newcommand{\mcp}{{\mathcal p}}
\newcommand{\mcq}{{\mathcal q}}
\newcommand{\mcr}{{\mathcal r}}
\newcommand{\mcs}{{\mathcal s}}
\newcommand{\mct}{{\mathcal t}}
\newcommand{\mcu}{{\mathcal u}}
\newcommand{\mcv}{{\mathcal v}}
\newcommand{\mcw}{{\mathcal w}}
\newcommand{\mcx}{{\mathcal x}}
\newcommand{\mcy}{{\mathcal y}}
\newcommand{\mcz}{{\mathcal z}}

% \rmd
\newcommand{\mra}{{\mathrm a}}
\newcommand{\mrb}{{\mathrm b}}
\newcommand{\mrc}{{\mathrm c}}
\newcommand{\mrd}{{\mathrm d}}
\newcommand{\mre}{{\mathrm e}}
\newcommand{\mrf}{{\mathrm f}}
\newcommand{\mrg}{{\mathrm g}}
\newcommand{\mrh}{{\mathrm h}}
\newcommand{\mri}{{\mathrm i}}
\newcommand{\mrj}{{\mathrm j}}
\newcommand{\mrk}{{\mathrm k}}
\newcommand{\mrl}{{\mathrm l}}
\newcommand{\mrm}{{\mathrm m}}
\newcommand{\mrn}{{\mathrm n}}
\newcommand{\mro}{{\mathrm o}}
\newcommand{\mrp}{{\mathrm p}}
\newcommand{\mrq}{{\mathrm q}}
\newcommand{\mrr}{{\mathrm r}}
\newcommand{\mrs}{{\mathrm s}}
\newcommand{\mrt}{{\mathrm t}}
\newcommand{\mru}{{\mathrm u}}
\newcommand{\mrv}{{\mathrm v}}
\newcommand{\mrw}{{\mathrm w}}
\newcommand{\mrx}{{\mathrm x}}
\newcommand{\mry}{{\mathrm y}}
\newcommand{\mrz}{{\mathrm z}}

%% 粗体的大写字母一般表示矩阵和张量
\newcommand{\bfA}{{\boldsymbol A}}
\newcommand{\bfB}{{\boldsymbol B}}
\newcommand{\bfC}{{\boldsymbol C}}
\newcommand{\bfD}{{\boldsymbol D}}
\newcommand{\bfE}{{\boldsymbol E}}
\newcommand{\bfF}{{\boldsymbol F}}
\newcommand{\bfG}{{\boldsymbol G}}
\newcommand{\bfH}{{\boldsymbol H}}
\newcommand{\bfI}{{\boldsymbol I}}
\newcommand{\bfJ}{{\boldsymbol J}}
\newcommand{\bfK}{{\boldsymbol K}}
\newcommand{\bfL}{{\boldsymbol L}}
\newcommand{\bfM}{{\boldsymbol M}}
\newcommand{\bfN}{{\boldsymbol N}}
\newcommand{\bfO}{{\boldsymbol O}}
\newcommand{\bfP}{{\boldsymbol P}}
\newcommand{\bfQ}{{\boldsymbol Q}}
\newcommand{\bfR}{{\boldsymbol R}}
\newcommand{\bfS}{{\boldsymbol S}}
\newcommand{\bfT}{{\boldsymbol T}}
\newcommand{\bfU}{{\boldsymbol U}}
\newcommand{\bfV}{{\boldsymbol V}}
\newcommand{\bfW}{{\boldsymbol W}}
\newcommand{\bfX}{{\boldsymbol X}}
\newcommand{\bfY}{{\boldsymbol Y}}
\newcommand{\bfZ}{{\boldsymbol Z}}

%% 花体大写字母
\newcommand{\mcA}{{\mathcal A}}
\newcommand{\mcB}{{\mathcal B}}
\newcommand{\mcC}{{\mathcal C}}
\newcommand{\mcD}{{\mathcal D}}
\newcommand{\mcE}{{\mathcal E}}
\newcommand{\mcF}{{\mathcal F}}
\newcommand{\mcG}{{\mathcal G}}
\newcommand{\mcH}{{\mathcal H}}
\newcommand{\mcI}{{\mathcal I}}
\newcommand{\mcJ}{{\mathcal J}}
\newcommand{\mcK}{{\mathcal K}}
\newcommand{\mcL}{{\mathcal L}}
\newcommand{\mcM}{{\mathcal M}}
\newcommand{\mcN}{{\mathcal N}}
\newcommand{\mcO}{{\mathcal O}}
\newcommand{\mcP}{{\mathcal P}}
\newcommand{\mcQ}{{\mathcal Q}}
\newcommand{\mcR}{{\mathcal R}}
\newcommand{\mcS}{{\mathcal S}}
\newcommand{\mcT}{{\mathcal T}}
\newcommand{\mcU}{{\mathcal U}}
\newcommand{\mcV}{{\mathcal V}}
\newcommand{\mcW}{{\mathcal W}}
\newcommand{\mcX}{{\mathcal X}}
\newcommand{\mcY}{{\mathcal Y}}
\newcommand{\mcZ}{{\mathcal Z}}

%% 空心大写字母
\newcommand{\mbA}{{\mathbb A}}
\newcommand{\mbB}{{\mathbb B}}
\newcommand{\mbC}{{\mathbb C}}
\newcommand{\mbD}{{\mathbb D}}
\newcommand{\mbE}{{\mathbb E}}
\newcommand{\mbF}{{\mathbb F}}
\newcommand{\mbG}{{\mathbb G}}
\newcommand{\mbH}{{\mathbb H}}
\newcommand{\mbI}{{\mathbb I}}
\newcommand{\mbJ}{{\mathbb J}}
\newcommand{\mbK}{{\mathbb K}}
\newcommand{\mbL}{{\mathbb L}}
\newcommand{\mbM}{{\mathbb M}}
\newcommand{\mbN}{{\mathbb N}}
\newcommand{\mbO}{{\mathbb O}}
\newcommand{\mbP}{{\mathbb P}}
\newcommand{\mbQ}{{\mathbb Q}}
\newcommand{\mbR}{{\mathbb R}}
\newcommand{\mbS}{{\mathbb S}}
\newcommand{\mbT}{{\mathbb T}}
\newcommand{\mbU}{{\mathbb U}}
\newcommand{\mbV}{{\mathbb V}}
\newcommand{\mbW}{{\mathbb W}}
\newcommand{\mbX}{{\mathbb X}}
\newcommand{\mbY}{{\mathbb Y}}
\newcommand{\mbZ}{{\mathbb Z}}

\newcommand{\mrA}{{\mathrm A}}
\newcommand{\mrB}{{\mathrm B}}
\newcommand{\mrC}{{\mathrm C}}
\newcommand{\mrD}{{\mathrm D}}
\newcommand{\mrE}{{\mathrm E}}
\newcommand{\mrF}{{\mathrm F}}
\newcommand{\mrG}{{\mathrm G}}
\newcommand{\mrH}{{\mathrm H}}
\newcommand{\mrI}{{\mathrm I}}
\newcommand{\mrJ}{{\mathrm J}}
\newcommand{\mrK}{{\mathrm K}}
\newcommand{\mrL}{{\mathrm L}}
\newcommand{\mrM}{{\mathrm M}}
\newcommand{\mrN}{{\mathrm N}}
\newcommand{\mrO}{{\mathrm O}}
\newcommand{\mrP}{{\mathrm P}}
\newcommand{\mrQ}{{\mathrm Q}}
\newcommand{\mrR}{{\mathrm R}}
\newcommand{\mrS}{{\mathrm S}}
\newcommand{\mrT}{{\mathrm T}}
\newcommand{\mrU}{{\mathrm U}}
\newcommand{\mrV}{{\mathrm V}}
\newcommand{\mrW}{{\mathrm W}}
\newcommand{\mrX}{{\mathrm X}}
\newcommand{\mrY}{{\mathrm Y}}
\newcommand{\mrZ}{{\mathrm Z}}


% 粗体的 Greek 字母
\newcommand{\balpha}{{\bm \alpha}}
\newcommand{\bbeta}{{\bm \beta}}
\newcommand{\bgamma}{{\bm \gamma}}
\newcommand{\bdelta}{{\bm \delta}}
\newcommand{\bepsilon}{{\bm \epsilon}}
\newcommand{\bvarepsilon}{{\bm \varepsilon}}
\newcommand{\bzeta}{{\bm \zeta}}
\newcommand{\bfeta}{{\bm \eta}}
\newcommand{\btheta}{{\bm \theta}}
\newcommand{\biota}{{\bm \iota}}
\newcommand{\bkappa}{{\bm \kappa}}
\newcommand{\blambda}{{\bm \lambda}}
\newcommand{\bmu}{{\bm \mu}}
\newcommand{\bnu}{{\bm \nu}}
\newcommand{\bxi}{{\bm \xi}}
\newcommand{\bomicron}{{\bm \omicron}}
\newcommand{\bpi}{{\bm \pi}}
\newcommand{\brho}{{\bm \rho}}
\newcommand{\bsigma}{{\bm \sigma}}
\newcommand{\btau}{{\bm \tau}}
\newcommand{\bupsilon}{{\bm \upsilon}}
\newcommand{\bphi}{{\bm \phi}}
\newcommand{\bvarphi}{{\bm \varphi}}
\newcommand{\bchi}{{\bm \chi}}
\newcommand{\bpsi}{{\bm \psi}}

\newcommand{\bAlpha}{{\bm \Alpha}}
\newcommand{\bBeta}{{\bm \Beta}}
\newcommand{\bGamma}{{\bm \Gamma}}
\newcommand{\bDelta}{{\bm \Delta}}
\newcommand{\bEpsilon}{{\bm \Psilon}}
\newcommand{\bVarepsilon}{{\bm \Varepsilon}}
\newcommand{\bZeta}{{\bm \Zeta}}
\newcommand{\bEta}{{\bm \Eta}}
\newcommand{\bTheta}{{\bm \Theta}}
\newcommand{\bIota}{{\bm \Iota}}
\newcommand{\bKappa}{{\bm \Kappa}}
\newcommand{\bLambda}{{\bm \Lambda}}
\newcommand{\bMu}{{\bm \Mu}}
\newcommand{\bNu}{{\bm \Nu}}
\newcommand{\bXi}{{\bm \Xi}}
\newcommand{\bOmicron}{{\bm \Omicron}}
\newcommand{\bPi}{{\bm \Pi}}
\newcommand{\bRho}{{\bm \Rho}}
\newcommand{\bSigma}{{\bm \Sigma}}
\newcommand{\bTau}{{\bm \Tau}}
\newcommand{\bUpsilon}{{\bm \Upsilon}}
\newcommand{\bPhi}{{\bm \Phi}}
\newcommand{\bChi}{{\bm \Chi}}
\newcommand{\bPsi}{{\bm \Psi}}

% \int_\Omega \bfx^2 \rmd \bfx
\newcommand{\rmd}{\,\mathrm d}
\newcommand{\bfzero}{\mathbf 0}

%% 算子
\newcommand{\ospan}{\operatorname{span}}
\newcommand{\odiv}{\operatorname{div}}
\newcommand{\otr}{\operatorname{tr}}
\newcommand{\ograd}{\operatorname{grad}}
\newcommand{\orot}{\operatorname{rot}}
\newcommand{\ocurl}{\operatorname{curl}}
\newcommand{\odist}{\operatorname{dist}}
\newcommand{\osign}{\operatorname{sign}}
\newcommand{\odiag}{\operatorname{diag}}
\newcommand{\oran}{\operatorname{Ran}} % 像空间
\newcommand{\oker}{\operatorname{Ker}} % 核空间
\newcommand{\ore}{\operatorname{Re}} % 实部
\newcommand{\oim}{\operatorname{Im}} % 虚部
\newcommand{\orank}{\operatorname{rank}}
\newcommand{\ovec}{\operatorname{vec}}
\newcommand{\odet}{\operatorname{det}}
\newcommand{\odim}{\operatorname{dim}}
\newcommand{\osym}{\operatorname{sym}}

\newcommand{\obcurl}{\operatorname{\bf curl}}
%%%% 个性设置结束
%%%%%%%%------------------------------------------------------------------------


%%%%%%%%------------------------------------------------------------------------
%%%% bibtex设置

%% 设定参考文献显示风格
% 下面是几种常见的样式
% * plain: 按字母的顺序排列,比较次序为作者、年度和标题
% * unsrt: 样式同plain,只是按照引用的先后排序
% * alpha: 用作者名首字母+年份后两位作标号,以字母顺序排序
% * abbrv: 类似plain,将月份全拼改为缩写,更显紧凑
% * apalike: 美国心理学学会期刊样式, 引用样式 [Tailper and Zang, 2006]

%\makeatletter
%\@ifclassloaded{beamer}{
%\bibliographystyle{apalike}
%}{
%\bibliographystyle{abbrv}
%}
%\makeatother


%%%% bibtex设置结束
%%%%%%%%------------------------------------------------------------------------

%%%%%%%%------------------------------------------------------------------------
%%%% xeCJK相关宏包

\usepackage{xltxtra, fontenc, xunicode}
\usepackage[slantfont, boldfont]{xeCJK}

\setlength{\parindent}{1.5em}%中文缩进两个汉字位

%% 针对中文进行断行
\XeTeXlinebreaklocale "zh"

%% 给予TeX断行一定自由度
\XeTeXlinebreakskip = 0pt plus 1pt minus 0.1pt

%%%% xeCJK设置结束
%%%%%%%%------------------------------------------------------------------------

%%%%%%%%------------------------------------------------------------------------
%%%% xeCJK字体设置

%% 设置中文标点样式,支持quanjiao、banjiao、kaiming等多种方式
\punctstyle{kaiming}

%% 设置缺省中文字体
\setCJKmainfont[BoldFont={Adobe Heiti Std}, ItalicFont={Adobe Kaiti Std}]{Adobe Song Std}
%\setCJKmainfont{Adobe Kaiti Std}
%% 设置中文无衬线字体
%\setCJKsansfont[BoldFont={Adobe Heiti Std}]{Adobe Kaiti Std}
%% 设置等宽字体
%\setCJKmonofont{Adobe Heiti Std}

%% 英文衬线字体
\setmainfont{DejaVu Serif}
%% 英文等宽字体
\setmonofont{DejaVu Sans Mono}
%% 英文无衬线字体
\setsansfont{DejaVu Sans}

%% 定义新字体
\setCJKfamilyfont{song}{Adobe Song Std}
\setCJKfamilyfont{kai}{Adobe Kaiti Std}
\setCJKfamilyfont{hei}{Adobe Heiti Std}
\setCJKfamilyfont{fangsong}{Adobe Fangsong Std}
\setCJKfamilyfont{lisu}{LiSu}
\setCJKfamilyfont{youyuan}{YouYuan}

%% 自定义宋体
\newcommand{\song}{\CJKfamily{song}}
%% 自定义楷体
\newcommand{\kai}{\CJKfamily{kai}}
%% 自定义黑体
\newcommand{\hei}{\CJKfamily{hei}}
%% 自定义仿宋体
\newcommand{\fangsong}{\CJKfamily{fangsong}}
%% 自定义隶书
\newcommand{\lisu}{\CJKfamily{lisu}}
%% 自定义幼圆
\newcommand{\youyuan}{\CJKfamily{youyuan}}

%%%% xeCJK字体设置结束
%%%%%%%%------------------------------------------------------------------------

%%%%%%%%------------------------------------------------------------------------
%%%% 一些关于中文文档的重定义
\newcommand{\chntoday}{\number\year\,年\,\number\month\,月\,\number\day\,日}
%% 数学公式定理的重定义

%% 中文破折号,据说来自清华模板
\newcommand{\pozhehao}{\kern0.3ex\rule[0.8ex]{2em}{0.1ex}\kern0.3ex}

\makeatletter %
\@ifclassloaded{beamer}{

}{
\newtheorem{example}{例}
\newtheorem{theorem}{定理}[section]
\newtheorem{definition}{定义}
\newtheorem{axiom}{公理}
\newtheorem{property}{性质}
\newtheorem{proposition}{命题}
\newtheorem{lemma}{引理}
\newtheorem{corollary}{推论}
\newtheorem{remark}{注解}
\newtheorem{condition}{条件}
\newtheorem{conclusion}{结论}
\newtheorem{assumption}{假设}
}
\makeatother

\makeatletter %
\@ifclassloaded{beamer}{

}{
%% 章节等名称重定义
\renewcommand{\contentsname}{目录}
\renewcommand{\indexname}{索引}
\renewcommand{\listfigurename}{插图目录}
\renewcommand{\listtablename}{表格目录}
\renewcommand{\appendixname}{附录}
\renewcommand{\appendixpagename}{附录}
\renewcommand{\appendixtocname}{附录}
\@ifclassloaded{book}{

}{
\renewcommand{\abstractname}{摘要}
}
}
\makeatother

\renewcommand{\figurename}{图}
\renewcommand{\tablename}{表}

\makeatletter
\@ifclassloaded{book}{
\renewcommand{\bibname}{参考文献}
}{
\renewcommand{\refname}{参考文献}
}
\makeatother

\floatname{algorithm}{算法}
\renewcommand{\algorithmicrequire}{\textbf{输入:}}
\renewcommand{\algorithmicensure}{\textbf{输出:}}

\renewcommand{\today}{\number\year 年 \number\month 月 \number\day 日}

%%%% 中文重定义结束
%%%%%%%%------------------------------------------------------------------------

\begin{document}
\title{FEALPy 中的高阶网格}
\author{魏华祎}
\date{\chntoday}
\maketitle

\section{数学基础}
设$\bfx$是曲面$\tau$上的一点,且曲面$\tau$有参数表示$\bfx(\xi,\eta) 
= [x(\xi,\eta),y(\xi,\eta),z(\xi,\eta)]$,\\
$\bfn$为曲面$\tau$的
单位法向量
$$
\bfn = \frac{\bfx_{\xi} \times \bfx_{\eta}}{\| \bfx_{\xi} \times \bfx_{\eta} \|}
$$
\subsection{第一基本形式}
\begin{definition}{第一基本形式}
	曲面$\tau$的第一基本形式为二次微分式
	$$
	\uppercase\expandafter{\romannumeral1} = 
	<d\bfx,d\bfx> = g_{00}d\xi d\xi + 2g_{01}d\xi d\eta + g_{11}d\eta d\eta 
	$$
	其中$d\bfx = x_{\xi}d\xi+x_{\eta}d\eta$,
	$$
	g_{00} = <x_{\xi},x_{\xi}>,
	g_{01} = <x_{\xi},x_{\eta}>,
	g_{11} = <x_{\eta},x_{\eta}>,
	$$
	称为第一基本形式的系数。
\end{definition}

\begin{property}
曲面的第一基本形式是不依赖于曲面的具体参数化的。
\end{property}

\begin{property}
曲面的第一基本形式在$\mcR^3$的合同变化下不变。
\end{property}

\subsection{第二基本形式}
\begin{definition}{第二基本形式}
曲面的第二基本式为二次微分式为
$$
\uppercase\expandafter{\romannumeral2} = 
-<d\bfx,d\bfn> = b_{00}d\xi d\xi + 2b_{01}d\xi d\eta + b_{11}d\eta d\eta 
$$
其中$d\bfn = \bfn_{\xi} d\xi + \bfn_{\eta}d\eta$
\begin{align*}
b_{00} = <\bfx_{\xi \xi},\bfn> &= -<\bfx_{\xi},\bfn_{\xi}> \\
b_{01} = <\bfx_{\xi \eta},\bfn> &= -<\bfx_{\xi},\bfn_{\eta}> = 
- <\bfx_{\eta},\bfn_{\xi}> \\
b_{11} = <\bfx_{\eta \eta},\bfn> &= -<\bfx_{\eta},\bfn_{\eta}> 
\end{align*}
称为第二基本形式的系数
\end{definition}

\begin{property}
设$\bfx = \bfx(\xi,\eta)$和$\bfx = \bfx(\overline{\xi},
\overline{\eta})$ 是曲面$\tau$的两个不同的参数表示,当变换
$(\xi,\eta) \rightarrow (\overline{\xi},
\overline{\eta})$的雅克比矩阵的行列式大于零,第二基本形式不遍;
当变换
$(\xi,\eta) \rightarrow (\overline{\xi},
\overline{\eta})$的雅克比矩阵的行列式小于零,第二基本形式改变符号.
\end{property}

\begin{property}
设$\tau$是曲面$R^3$的一张曲面,$\bfx(\xi,\eta)$是它的参数表述;$\mcT$是$R^3$
的一个合同变换,则曲面$\widetilde{\tau}:\widetilde{\xi,\eta} = 
\mcT \circ \bfx(\xi,\eta)$的第二基本形式
$\widetilde{\uppercase\expandafter{\romannumeral2}}$与曲面$\tau$的
第二基本形式有如下关系:当$\mcT$为刚体运动(合同变换分解的正交变换的矩阵的行列式为1)时$\widetilde{\uppercase\expandafter{\romannumeral2}}(\xi,\eta) = 
\uppercase\expandafter{\romannumeral2}(\xi,\eta)$ ,当$\mcT$为反向刚
体运动(合同变换分解的正交变换的矩阵的行列式为-1)时$\widetilde{\uppercase\expandafter{\romannumeral2}}(\xi,\eta) = 
-\uppercase\expandafter{\romannumeral2}(\xi,\eta)$ .
\end{property}

\subsection{形函数}
\begin{definition}{形函数}
形函数定义于单元内部的、坐标的连续函数,它满足下列条件:
\begin{itemize}
\item 在节点i处,$\lambda_i =1$;在其他节点处,$\lambda_i = 0$;
\item 能保证用它定义的未知量(u、v或x、y)在相邻单元之间的连续性;
\item 应包含任意线性项,使用它定义的单元唯一可满足常应变条件;
\item 应满足下列等式:$\sum \lambda_i(\bfx_i)= 1$。
\end{itemize}
\end{definition}

\section{区间}

给定 $p$ 次拉格朗日曲线 $l = [\bfx_0, \bfx_1, \cdots, \bfx_p]$, 共有 $p+1$ 个插值节点
. 参考单元为 $[0, 1]$, 坐标变量记为 $\xi$. 曲线 $l$ 相对于参考单元的重心坐标函数
是
\begin{align*}
    \lambda_0 = 1-\xi, \lambda_1 = \xi
\end{align*}
由此可构造 $p$ 次的拉格朗日形函数 
\begin{align*}
    \bphi_p = [\phi_0, \phi_1, \cdots, \phi_p]
\end{align*}

对于任意的 $\bfx \in l$, 存在 $\xi \in [0, 1]$
\begin{align*}
    \bfx(\xi) = \bfx_0 \phi_0(\xi) + \cdots + \bfx_p\phi_p(\xi)\\
    \bfx_\xi = \bfx_0 (\phi_0)_\xi + \cdots + \bfx_p(\phi_p)_\xi
\end{align*}

在曲线 $l$ 上可定义 $q$ 次的拉格朗日多项式空间, 其基函数记为
\begin{align*}
    \bvarphi_q = [\varphi_0, \varphi_1, \cdots, \varphi_q]
\end{align*}
对应的插值点记为
\begin{align*}
    [\bfy_0, \bfy_1, \cdots, \bfy_q]
\end{align*}
其中 $\bfy_0 = \bfx_0$, $\bfy_q = \bfx_p$.

对于任意的 $\bfx \in l$, 一一对应一个 $\xi \in [0, 1]$
\begin{align*}
    [\varphi_0(\bfx), \varphi_1(\bfx), \cdots, \varphi_q(\bfx)] = 
    [\varphi_0(\xi), \varphi_1(\xi), \cdots, \varphi_q(\xi)] 
\end{align*}

下面考虑 $\varphi_i$ 关于 $\bfx \in l$ 的导数

\begin{align*}
    \nabla_\bfx \varphi_i = \bfx_\xi \bfG^{-1} (\varphi_i)_\xi
\end{align*}
其中,由于$\bfx_{\xi} \cdot \xi_{\bfx} = 1$,所以
\begin{align*}
	x_{\xi}\bfG^{-1} = \frac{1}{\xi_\bfx} \cdot <\xi_\bfx, \xi_\bfx> = \xi_\bfx
\end{align*}

其中 $\bfG = <\bfx_\xi, \bfx_\xi>$.





\section{三角形网格}
给定一个 $p$ 次的拉格朗日三角形 $\tau$, 它是由 $ldof = (p+1)(p+2)/2$ 个节点
$\{\bfx_i\}_{i=0}^{ldof-1}$ 组成。

\begin{figure}[H]
\includegraphics[scale=0.7]{figures/ltri.png}
\end{figure}


设 $\{\phi_i\}_{i=0}^{ldof-1}$ 为对应的拉格朗日形函数,满足 
$$
\phi_i(\bfx_j) = \delta_{ij}
$$
给一个参考元(等腰直角三角形)
\begin{align*}
    \bfu = 
    \begin{bmatrix}
        \xi \\ \eta
    \end{bmatrix}
\end{align*}
上的重心坐标为

\begin{align*}
    \lambda_0 &= 1 - \xi - \eta\\
    \lambda_1 &= \xi\\
    \lambda_2 &= \eta\\
\end{align*}

\begin{align*}
    \nabla_\bfu \lambda_0 &= \begin{bmatrix}
        -1 \\ -1
    \end{bmatrix}\\
    \nabla_\bfu \lambda_1 &= \begin{bmatrix}
        1 \\ 0 
    \end{bmatrix}\\
    \nabla_\bfu \lambda_2 &= \begin{bmatrix}
        0 \\ 1
    \end{bmatrix}
\end{align*}

曲面 $\tau_h$ 上任意一点可表示为
\begin{align*}
    \bfx = \bfx_0 \phi_0 + \bfx_1\phi_1 + \cdots + \bfx_{ldof-1}\phi_{ldof-1}
\end{align*}

\begin{align*}
    \nabla_\bfu \bfx = \bfx_0 \nabla_\bfu^T \phi_0 + \bfx_1\nabla_\bfu^T \phi_1
    + \cdots + \bfx_{ldof-1}\nabla_\bfu^T\phi_{ldof-1}
\end{align*}

所以只需要求每个拉格朗日形函数 $\phi_i$ 关于 $\bfu$ 的导数。

\begin{align*}
    \nabla_\bfu \phi_i = 
    \frac{\partial \phi_i}{\partial \lambda_0}\nabla_\bfu \lambda_0 + 
    \frac{\partial \phi_i}{\partial \lambda_1}\nabla_\bfu \lambda_1 + 
    \frac{\partial \phi_i}{\partial \lambda_2}\nabla_\bfu \lambda_2 
\end{align*}



下面讨论球面上的 $\tau$ 上任意一点可表示为
\begin{align*}
    \overline{\bfx} = \frac{\bfx}{\|\bfx\|}
\end{align*}

\begin{align*}
    \nabla_\bfu \overline{\bfx} = \frac{\nabla_\bfu\bfx}{\|\bfx\|} +
    \bfx\nabla_\bfu^T \|\bfx\|^{-1}
\end{align*}

\begin{align*}
    \nabla_\bfu^T \|\bfx\|^{-1} = -\frac{\bfx^T[\bfx_\xi, \bfx_\eta]}{\|\bfx\|^3} 
\end{align*}

\begin{align*}
    \nabla_\bfu \overline{\bfx} = \frac{\nabla_\bfu\bfx}{\|\bfx\|} - 
    \frac{[\bfx_\xi, \bfx_\eta]}{\|\bfx\|^2} = \frac{\|\bfx\| -
    1}{\|\bfx\|^2}\nabla_\bfu\bfx
\end{align*}

点 $\bfx=(x,y,z)$,则 $\|\bfx\|^{-1}=(x^2+y^2+z^2)^{-\frac{1}{2}}$.

\begin{align*}
    \frac{\partial \|\bfx\|^{-1}}{\partial \xi} &=
    -\frac{1}{2}(x^2+y^2+z^2)^{-\frac{3}{2}}\left[2x x_\xi + 2y y_\xi +2z
    z_\xi\right]\\
                             &=  -\frac{1}{2}\|\bfx\|^{-3} 2\bfx \cdot \bfx_\xi\\
                             &=  -\|\bfx\|^{-3} \bfx\cdot \bfx_\xi
\end{align*}

\begin{align*}
    \frac{\partial \|\bfx\|^{-1}}{\partial \eta} &=
    -\frac{1}{2}(x^2+y^2+z^2)^{-\frac{3}{2}}\left[2x x_\eta + 2y y_\eta +2z
    z_\eta\right]\\
                             &=  -\frac{1}{2}\|\bfx\|^{-3} 2\bfx\cdot\bfx_\eta\\
                             &=  -\|\bfx\|^{-3} \bfx\cdot \bfx_\eta
\end{align*}

\begin{align*}
    \nabla_\bfu \|\bfx\|^{-1} =    
    \begin{bmatrix}
     \nabla_\xi \|\bfx\|^{-1} \\ \nabla_\eta \|\bfx\|^{-1}
    \end{bmatrix}
                             =  -\|\bfx\|^{-3} \bfx 
    \begin{bmatrix}
      \bfx_\xi \\ \bfx_\eta
    \end{bmatrix}
    =-\frac{\bfx \nabla_\bfu \bfx}{\|\bfx\|^3}
\end{align*}

$\tau$ 上的第一基本形式为:
\begin{align*}
    \uppercase\expandafter{\romannumeral1} =  = 
    \begin{bmatrix}
        d\xi , d\eta
    \end{bmatrix}
     \begin{bmatrix}
        g_{00} & g_{01} \\
        g_{10} & g_{11} 
    \end{bmatrix}   
    \begin{bmatrix}
        d\xi \\ d\eta
    \end{bmatrix}
\end{align*}

即

$$
\uppercase\expandafter{\romannumeral1} 
= \langle d\bfx, d\bfx \rangle = g_{00} d\xi d\xi+  2g_{01} d\xi d\eta  + g_{11} d\eta d\eta 
$$

其中

\begin{align*}
	d\bfx = & \bfx_u du +\bfx_v dv,\\
    \bfG = & 
    \begin{bmatrix}
        g_{00} & g_{01} \\
        g_{10} & g_{11} 
    \end{bmatrix}\\
    g_{00} = & <\bfx_\xi, \bfx_\xi> \\
    g_{01} = & <\bfx_\xi, \bfx_\eta> = g_{10}\\
    g_{11} = & <\bfx_\eta, \bfx_\eta> 
\end{align*}

$\tau$ 上的第二基本形式为:
\begin{align*}
    \uppercase\expandafter{\romannumeral2} =
    \begin{bmatrix}
        d\xi , d\eta
    \end{bmatrix}
     \begin{bmatrix}
        b_{00} & b_{01} \\
        b_{10} & b_{11} 
    \end{bmatrix}   
    \begin{bmatrix}
        d\xi \\ d\eta
    \end{bmatrix}
\end{align*}

即

$$
\uppercase\expandafter{\romannumeral1} = -\langle d\bfx, d\bfn \rangle = b_{00} d\xi d\xi+  2b_{01} d\xi d\eta  +
b_{11} d\eta d\eta 
$$

其中

\begin{align*}
	d\bfx = & \bfx_\xi d\xi +\bfx_\eta d\eta, \\
	d\bfn = & \bfn_\xi d\xi + \bfn_\eta d\eta \\
    \bfB = & 
    \begin{bmatrix}
        b_{00} & b_{01} \\
        b_{10} & b_{11} 
    \end{bmatrix}\\
	b_{00} = & <\bfx_{\xi \xi},\bfn> = -<\bfx_\xi,\bfn_\xi> \\
	b_{01} = & <\bfx_{\xi \eta},\bfn> = -<\bfx_\xi,\bfn_\eta> = - <\bfx_\eta,\bfn_\xi> =b_{10}\\
	b_{11} = & <\bfx_{\eta \eta},\bfn> = -<\bfx_\eta,\bfn_\eta> 
\end{align*}

$\tau$ 上的切梯度算子:

\begin{align*}
    \nabla_\tau \phi_i = 
    \nabla_\bfu \bfx \bfG^{-1} \nabla_\bfu \phi_i =
    [\bfx_\xi, \bfx_\eta] \bfG^{-1} \nabla_\bfu \phi_i
\end{align*}

$\tau$ 上的单位法向量为:

$$
\bfn = \frac{\bfx_{\xi} \times \bfx_{\eta}}{\|\bfx_{\xi} \times \bfx_{\eta}\|}
$$

$\tau$ 上的面积为:
\begin{align*}
    |\tau| = \int_\tau 1 \rmd\bfx = \int_{\bar\tau} |\bfx_\xi \times \bfx_\eta| \rmd\bfu
\end{align*}

其中
\begin{align*}
|\bfx_\xi \times \bfx_\eta|^2 = |\bfx_\eta|^2 |\bfx_\xi|^2 \cos^2 \theta  = |\bfx_\eta|^2 |\bfx_\xi|^2 -|\bfx_\xi|^2 |\bfx_{\eta}|^2 
\sin^2 \theta = | \bfG |
\end{align*}

在三角形 $\tau$ 上可定义 $q$ 次的拉格朗日多项式空间, 其基函数记为
\begin{align*}
    \bvarphi_q = [\varphi_0, \varphi_1, \cdots, \varphi_q]
\end{align*}
对应的插值点记为
\begin{align*}
    [\bfy_0, \bfy_1, \cdots, \bfy_q]
\end{align*}

对于任意的 $\bfx \in \tau$, 一一对应一対参考单元的 $(\xi , \eta) $
\begin{align*}
    [\varphi_0(\bfx), \varphi_1(\bfx), \cdots, \varphi_q(\bfx)] = 
    [\varphi_0(\xi,\eta), \varphi_1(\xi,\eta), \cdots, \varphi_q(\xi,\eta)] 
\end{align*}

 $\varphi_i$ 关于 $\bfx \in \tau$ 的导数

\begin{align*}
    \nabla_\bfx \varphi_i = [\bfx_\xi, \bfx_\eta] \bfG^{-1} \nabla_\bfu \varphi_i
\end{align*}

\begin{align*}
    \nabla^2_\bfx \varphi_i = \nabla^2_u \bfx (\bfG^{-1})^2 \nabla_u \varphi_i + \nabla_u \bfx (\bfG^{-1})^2 \nabla_{uu}\varphi_i \nabla_u \bfx
\end{align*}

\section{四边形}

给定一个 $p$ 次的拉格朗日四边形 $\tau$, 它由 $ldof = (p+1)^2 $ 个节点$\{\bfx_i\}_{i=0}^{ldof-1}$ 组成,它是由两个一维单纯形张成的。

设第一个曲线 $l_1$ 的坐标变量为 $ \xi $,其相对于参考单元的重心坐标函数是
$$
\lambda_0 = 1- \xi , \lambda_1 = \xi
$$

由此构造其p次lagrange形函数$\{\phi_i\}_{i=0}^{p}$,其上的插值点为$\{y_i\}_{i=0}^{p}$
\begin{align*}
\phi_i(\bfy_j) &= \delta_{ij} \\
\Phi &= \{ \phi_0 , \phi_1 , \cdots ,\phi_p \}
\end{align*}

设另一个曲线 $l_2$ 的坐标变量为 $ \eta $,其相对于参考单元的重心坐标函数是
$$
\lambda_2 = 1- \eta , \lambda_3 = \eta
$$

由此构造其p次lagrange形函数$\{\psi_i\}_{i=0}^{p}$,其上的插值点为$\{z_i\}_{i=0}^{p}$
\begin{align*}
\psi_i(\bfz_j) &= \delta_{ij} \\
\Psi &= \{ \psi_0 , \psi_1 , \cdots ,\psi_p \}
\end{align*}



那么四边形的参考单元为
$$
 \bfu = \begin{bmatrix}
        	\xi \\ \eta
 		\end{bmatrix}
$$

重心坐标函数是

\begin{align*}
	\begin{bmatrix}
		\Lambda_0 & \Lambda_1 \\
		\Lambda_2 & \Lambda_3
	\end{bmatrix}
	=
     \begin{bmatrix}
       \lambda_0  \\ \lambda_1
    \end{bmatrix}
  	\cdot
    \begin{bmatrix}
        \lambda_2 & \lambda_3
    \end{bmatrix}
    =
    \begin{bmatrix}
        \lambda_0 \lambda_2 & \lambda_0 \lambda_3 \\
        \lambda_1 \lambda_2 & \lambda_1 \lambda_3
    \end{bmatrix}
    =
	\begin{bmatrix}
    	(1 - \xi )(1 - \eta ) & (1 - \xi ) \eta \\
    	\xi (1 - \eta)       & \xi \eta	
    \end{bmatrix}   
\end{align*}

\begin{align*}
    \nabla_{\bfu} \Lambda_0 &= \begin{bmatrix}
    								\eta-1 \\ \xi-1
								\end{bmatrix} \\
    \nabla_{\bfu} \Lambda_1 &= \begin{bmatrix}
    								-\eta \\ 1-\xi
								\end{bmatrix} \\
    \nabla_{\bfu} \Lambda_2 &= \begin{bmatrix}
    								1-\eta \\ -\xi
								\end{bmatrix} \\
    \nabla_{\bfu} \Lambda_2 &= \begin{bmatrix}
    								\eta \\  \xi
								\end{bmatrix}
\end{align*}

形函数是
\begin{align*}
 \alpha = \Phi \otimes \Psi = 
  \begin{bmatrix}
   \phi_0 \psi_0 & \phi_0 \psi_1  & \cdots & \phi_0 \psi_p \\ 
   \vdots        &   \ddots       &        &  \vdots       \\
   \phi_p \psi_0 &  \phi_p \psi_1 &  \cdots&  \phi_p\psi_p  
  \end{bmatrix}
  :=
  \begin{bmatrix}
   \alpha_{00}   & \alpha_{01}    & \cdots & \alpha_{0p} \\ 
   \vdots        &   \ddots       &        &  \vdots       \\
   \alpha_{p0}   &  \alpha_{p1}   &  \cdots&  \alpha_{pp}  
  \end{bmatrix}
\end{align*}

那么四边形上任意一点可以表示为

$$
x = \sum \alpha_{ij}x_{ij} 
$$

其导数为
\begin{align*}
   \alpha_\xi = \Phi_\xi \otimes \Psi = 
   \begin{bmatrix}
   (\phi_0)_\xi \psi_0 & (\phi_0)_\xi \psi_1  & \cdots & (\phi_0)_\xi \psi_p \\ 
   \vdots        &   \ddots       &        &  \vdots       \\
   (\phi_p)_\xi \psi_0 &  (\phi_p)_\xi \psi_1 &  \cdots&  (\phi_p)_\xi\psi_p  
   \end{bmatrix} \\
   \alpha_\eta = \Phi_ \otimes \Psi_\eta = 
   \begin{bmatrix}
   \phi_0 (\psi_0)_\eta & \phi_0 (\psi_1)_\eta  & \cdots & \phi_0 (\psi_p)_\eta \\ 
   \vdots        &   \ddots       &        &  \vdots       \\
   \phi_p (\psi_0)_\eta &  \phi_p (\psi_1)_\eta &  \cdots&  \phi_p (\psi_p)_\eta
   \end{bmatrix} 
\end{align*}


\section{四面体网格}
给定一个 $p$ 次的拉格郎日四面体 $\tau$,它由 $ldof = (p+1)(p+2)(p+3)/6$ 个节点组
成。设 $\{\phi\}_{i=0}^{ldof-1}$ 为对应的拉格郎日形函数,满足
$$
\phi_i(\bfx_j) = \delta_{ij}
$$

\begin{align*}
    \bfu = 
    \begin{bmatrix}
        \xi \\ \eta \\ \zeta
    \end{bmatrix}
\end{align*}

\begin{align*}
    \lambda_0 &= 1-\xi-\eta-\zeta \\
    \lambda_1 &= \xi \\
    \lambda_2 &= \eta \\
    \lambda_3 &= \zeta \\
\end{align*}

\begin{align*}
    \nabla_\bfu\lambda_0  & = 
    \begin{bmatrix}
        -1 \\ -1 \\ -1
    \end{bmatrix}\\
    \nabla_\bfu\lambda_1 & = 
    \begin{bmatrix}
        1 \\ 0 \\0
    \end{bmatrix}\\
    \nabla_\bfu\lambda_2 & = 
    \begin{bmatrix}
        0 \\ 1 \\ 0
    \end{bmatrix}\\
    \nabla_\bfu\lambda_3 & = 
    \begin{bmatrix}
        0 \\ 0 \\ 1
    \end{bmatrix}
\end{align*}

\begin{align*}
    \bfx = \bfx_0\phi_0 +\bfx_1\phi_1+\cdots+\bfx_{ldof-1}\phi_{ldof-1}
\end{align*}

\begin{align*}
    \nabla_{\bfu}\bfx = \bfx_0\nabla_{\bfu}^T\phi_0+\bfx_1\nabla_{\bfu}^T\phi_1
     + \cdots+\bfx_{ldof-1}\nabla_{\bfu}^T\phi_{ldof-1}
\end{align*}

所以只需要求每个拉格朗日形函数 $\phi_i$ 关于 $\bfu$ 的导数。
\begin{align*}
    \nabla_{\bfu}\phi_i =
    \frac{\partial\phi_i}{\partial\lambda_0}\nabla_{\bfu}\lambda_0 + 
    \frac{\partial\phi_i}{\partial\lambda_1}\nabla_{\bfu}\lambda_1 +
    \frac{\partial\phi_i}{\partial\lambda_2}\nabla_{\bfu}\lambda_2 +
    \frac{\partial\phi_i}{\partial\lambda_3}\nabla_{\bfu}\lambda_3 
\end{align*}

$\tau$ 上的第一基本形式为:


\begin{align*}
	I = & \bfv\bfG\bfv^T \\
	\bfv = & 
	\begin{bmatrix}
	d\xi & d\eta & d\zeta
	\end{bmatrix}	\\
    \bfG = & 
    \begin{bmatrix}
        g_{00} & g_{01} & g_{02} \\
        g_{10} & g_{11} & g_{12} \\
        g_{20} & g_{21} & g_{22}
    \end{bmatrix}\\
    g_{00} = & <\bfx_\xi, \bfx_\xi> \\
    g_{01} = & <\bfx_\xi, \bfx_\eta> \\
    g_{02} = & <\bfx_\xi, \bfx_\zeta> \\
    g_{10} = & <\bfx_\eta,\bfx_\xi> \\
    g_{11} = & <\bfx_\eta,\bfx_\eta> \\
    g_{12} = & <\bfx_\eta,\bfx_\zeta> \\
    g_{20} = & <\bfx_\zeta,\bfx_\xi> \\
    g_{21} = & <\bfx_\zeta,\bfx_\eta> \\
    g_{22} = & <\bfx_\zeta,\bfx_\zeta>
\end{align*}

$\tau$ 上的切梯度算子:

\begin{align*}
    \nabla_\tau \phi_i = 
    \nabla_\bfu \bfx \bfG^{-1} \nabla_\bfu \phi_i =
    [\bfx_\xi, \bfx_\eta] \bfG^{-1} \nabla_\bfu \phi_i
\end{align*}

$\tau$ 上的体积为:
\begin{align*}
    |\tau| = \int_{\tau}1d\bfx =
    \int_{\overline{\tau}}|\bfx_{\xi}\times\bfx_{\eta}\cdot\bfx_{\zeta}|d\bfu
\end{align*}
\section{六面体}
给定一个 $p$ 次的拉格朗日六面体 $\tau$, 它由 $ldof = (p+1)^3 $ 个节点
$\{\bfx_i\}_{i=0}^{ldof-1}$ 组成。

设 $\{\phi_i\}_{i=0}^{ldof-1}$ 为对应的拉格朗日形函数,满足 
$$
\phi_i(\bfx_j) = \delta_{ij}
$$

\begin{align*}
    \bfu_0 = 
    \begin{bmatrix}
        \xi \\ 0 \\0
    \end{bmatrix} ,
    \bfu_1 = 
    \begin{bmatrix}
        0 \\ \eta \\ 0
    \end{bmatrix} ,
    \bfu_2 = 
    \begin{bmatrix}
        0 \\ 0 \\ \zeta
    \end{bmatrix}
\end{align*}


\begin{align*}
    \lambda_0 &=  \xi \\
    \lambda_1 &= 1 - \xi\\
    \lambda_2 &= \eta\\
    \lambda_3 &= 1 - \eta\\   
    \lambda_4 &= \zeta\\
    \lambda_5 &= 1 - \zeta\\  
\end{align*}

\begin{align*}
     \begin{bmatrix}
       \lambda_0 , \lambda_1 
    \end{bmatrix}
  	\otimes
    \begin{bmatrix}
        \lambda_2 , \lambda_3
    \end{bmatrix}
    \otimes
    \begin{bmatrix}
        \lambda_4 , \lambda_5
    \end{bmatrix}
    =
    \begin{bmatrix}
        \qquad \qquad \qquad \qquad  \lambda_5\lambda_0\lambda_2 , \lambda_5\lambda_0\lambda_3 \\ 
        \qquad \qquad \qquad \qquad  \lambda_5\lambda_1\lambda_2 ,  \lambda_5\lambda_1\lambda_3 \\   
        \lambda_4\lambda_0\lambda2 , \lambda_4\lambda_0\lambda_3 \qquad \qquad \qquad \qquad \\
         \lambda_4\lambda_1\lambda2 , \lambda_4\lambda_1\lambda_3 \qquad \qquad \qquad \qquad \\
    \end{bmatrix}
\end{align*}

\begin{align*}
    \nabla_{\bfu_0} \lambda_0 &= \begin{bmatrix}
        1 \\ 0 \\ 0
    \end{bmatrix}\\
    \nabla_{\bfu_0} \lambda_1 &= \begin{bmatrix}
        -1 \\ 0 \\ 0 
    \end{bmatrix}\\
    \nabla_{\bfu_1} \lambda_2 &= \begin{bmatrix}
        0 \\ 1 \\0 
    \end{bmatrix} \\
     \nabla_{\bfu_1} \lambda_3 &= \begin{bmatrix}
        0 \\ -1 \\ 0
    \end{bmatrix}\\
     \nabla_{\bfu_2} \lambda_4 &= \begin{bmatrix}
        0 \\ 0 \\ 1 
    \end{bmatrix} \\
     \nabla_{\bfu_1} \lambda_5 &= \begin{bmatrix}
        0 \\ 0 \\ -1
    \end{bmatrix}
\end{align*}

\begin{align*}
    \bfx = \bfx_0 \phi_0 + \bfx_1\phi_1 + \cdots + \bfx_{ldof-1}\phi_{ldof-1}
\end{align*}



所以只需要求每个拉格朗日形函数 $\phi_i$ 关于 $\bfu$ 的导数。

\begin{align*}
    \nabla_\bfu \phi_i = 
    \frac{\partial \phi_i}{\partial \lambda_0}\nabla_\bfu \lambda_0 + 
    \frac{\partial \phi_i}{\partial \lambda_1}\nabla_\bfu \lambda_1 + 
    \frac{\partial \phi_i}{\partial \lambda_2}\nabla_\bfu \lambda_2 +
    \frac{\partial \phi_i}{\partial \lambda_3}\nabla_\bfu \lambda_3 +
    \frac{\partial \phi_i}{\partial \lambda_4}\nabla_\bfu \lambda_4 +
    \frac{\partial \phi_i}{\partial \lambda_5}\nabla_\bfu \lambda_5 
\end{align*}


\begin{align*}
    |\tau| = \int_{\tau}1d\bfx =
    \int_{\overline{\tau}}|\bfx_{\xi}\times\bfx_{\eta}\cdot\bfx_{\zeta}|d\bfu
\end{align*}


$\tau$ 上的第一基本形式为:

\begin{align*}
	I = & \bfv\bfG\bfv^T \\
	\bfv = & 
	\begin{bmatrix}
	d\xi & d\eta & d\zeta
	\end{bmatrix}	\\
    \bfG = & 
    \begin{bmatrix}
        g_{00} & g_{01} & g_{02} \\
        g_{10} & g_{11} & g_{12} \\
        g_{20} & g_{21} & g_{22}
    \end{bmatrix}\\
    g_{00} = & <\bfx_\xi, \bfx_\xi> \\
    g_{01} = & <\bfx_\xi, \bfx_\eta> \\
    g_{02} = & <\bfx_\xi, \bfx_\zeta> \\
    g_{10} = & <\bfx_\eta,\bfx_\xi> \\
    g_{11} = & <\bfx_\eta,\bfx_\eta> \\
    g_{12} = & <\bfx_\eta,\bfx_\zeta> \\
    g_{20} = & <\bfx_\zeta,\bfx_\xi> \\
    g_{21} = & <\bfx_\zeta,\bfx_\eta> \\
    g_{22} = & <\bfx_\zeta,\bfx_\zeta>
\end{align*}

$\tau$ 上的第二基本形式为:

\begin{align*}
    \nabla_\tau \phi_i = 
    \nabla_\bfu \bfx \bfG^{-1} \nabla_\bfu \phi_i =
    [\bfx_\xi, \bfx_\eta, \bfx_\zeta] \bfG^{-1} \nabla_\bfu \phi_i
\end{align*}
\cite{fealpy}

\newpage

\section{高次的多边形网格}


在坐标变量为 $\bfu = (\xi, \eta)$ 的平面坐标系中, 引入一个中心在坐标原
点正 $N$ 边形做为参考单元 $\bar K$, 它共有 $N$ 条曲边, 第 $i$ 条边 $\bar e_i = (\bfu_i^0,
\bfu_i^1, \cdots, \bfu_i^p)$, $0\leq i < N$. 组成 $\bar e_i$ 的点 $\bfu_i^j$ 是
等距分布的.

在 $\bar K$ 上定义 $p$ 次的单项式空间的基函数 
$$
\bar\bfm_p =
\begin{bmatrix} 
    \bar m_0 & \bar m_1 & \cdots & \bar m_{n_p-1}
\end{bmatrix} 
= 
\begin{bmatrix} 
    1 & \xi & \eta & \cdots & \xi\eta^{p-1} & \eta^p
\end{bmatrix}
$$
其中 $n_p = (p+1)(p+2)/2$. 

接着可以在 $\bar K$ 定义标准协调的虚单元空间, 其基函数记为
$$
\bphi_p(\xi, \eta) = 
\begin{bmatrix} 
    \bphi_p^{\bar V} & \bphi_p^{\bar E} & \bphi_p^{\bar K}
\end{bmatrix} 
= 
\begin{bmatrix} 
    \phi_{p, 0} & \phi_{p, 1} & \cdots & \phi_{p, N_K-1}
\end{bmatrix}
$$
其中 $\bphi_p^{\bar V}$ 是定义在 $\bar K$ 的顶点处的基函数, $\bphi_p^{\bar E}$ 是定义在
$\bphi_p^{\bar K}$ 是定义在 $\bar K$ 内部的基函数. $\bar K$ 顶点和边内部点处的基
函数是插值型的基函数, 在相应插值点处的取值为 1, 其它插值点为 0. $\bar K$ 内部的基函
数是积分型的基函数, 共有 $n_{p-2} = (p-1)p/2$ 个, 在 $\bar K$ 边界上的插值点取值
为 0.

可以在给定曲面 $S$ 上, 构造的一个局部逼近曲面 $S$ 的曲多边形单元 $K_p$, 设其有 $N$ 条曲边, 每条边 $e_i$ 取 $p+1$ 点
$\{\bfx_i^0, \bfx_i^1, \cdots, \bfx_i^p\}\subset S$,其中 $\bfx_i^0$ 和 $\bfx_i^p$
是 $K_p$ 逆时针相邻的两个顶点(角点), $0\leq i < N$.对于任意的 $\bfx_p \in K_p$,  可以找到一个 $\bfu \in \bar K$, 使得:
$$
\bfx_p = \bfx_0\phi_{p, 0} + \cdots + \bfx_{N_K-1}\phi_{p, N_K-1} \in K_p,
$$
满足如下性质 
$$
\min \|\bfx_p - a(\bfx)\|_{\bar K}
$$
其中 $a(\bfx)$ 是曲面 $S$ 离 $\bfx_p$ 最近的点. 注意到这个极小化问题中, 只有单元内
部基函数的系数需要确定. 下面的问题是如何有效计算这个极小化问题?


当然, 如果限定为三角形单元, 完全可可以借助拉格朗日形函数来获得一个逼近曲面的曲面
三角 形单元, 然后在这个曲面三角形上构造 VEM 的空间, 对应的多项式一种可能是定义在
参考三角形上, 也即关于参考单元的 $(\xi, \eta)$ 是多项式, 关于实际空间的 $\bfx$
坐标不是多项式.






\bibliographystyle{abbrv}
\bibliography{lagrangemesh}
\end{document}
