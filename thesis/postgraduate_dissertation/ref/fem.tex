% !Mode:: "TeX:UTF-8"
\documentclass{article}
\input{../en_preamble.tex}
\input{../xecjk_preamble.tex}
\begin{document}
\title{椭圆问题的有限元方法}
\author{魏华祎}
\date{\chntoday}
\maketitle
\section{简介}

有限元方法是一种基于 PDE (partial differential equations) 的变分形式
(variational formulation) 求解 PDE 近似解的方法。 本章主要但要拉格朗日有限元方法
的算法与程序设计细节。

\section{Poisson 模型}

考虑如下D氏边界条件的Poisson方程:

\begin{eqnarray}
    -\Delta u(\mathbf x) &=& f(\mathbf x),\text{ on } \Omega.\label{eq:P}\\
    u|_{\partial\Omega} &=& 0.
\end{eqnarray}

\section{Galerkin 方法}

引入函数空间 $H^1(\Omega)$, 对于任意 $v \in H^1(\Omega)$, $v$和它的一阶导数都
在 $\Omega$ 上 $L^2$ 可积。 这里的 $H^1(\Omega)$ 是一个无限维的空间。 另外, 引
入空间 $H^1_0(\Omega) := \{v\in H^1(\Omega), v|_{\partial\Omega} = 0\}$.

对于任意的 $v\in H^1_0(\Omega)$, 同乘以方程 \eqref{eq:P} 的两端, 然后做分部积
分可得:
\begin{equation}\label{eq:wg}
    \int_{\Omega}\nabla u\cdot\nabla v\mathrm{d}x = \int_{\Omega}fv\mathrm{d}x,
    \quad\forall v \in H^1_0(\Omega).
\end{equation}
原问题就转化为: 求解 $u\in H_0^1(\Omega)$, 满足
\begin{equation}\label{eq:W}
    a(u, v) = (f,v) \text{ for all }v\in H_0^1(\Omega).
\end{equation}
其中
$$
a(u, v) = \int_{\Omega}\nabla u\cdot\nabla v\mathrm{d}x,
\quad (f,v) =  \int_{\Omega}fv\mathrm{d}x.
$$

下面我们考虑所谓 Galerkin 方法来求 \eqref{eq:W} 的逼近解。 上面 $H_0^1(\Omega)$
是一个无限维的空间, 为了把无限维的问题转化为有限维的问题, 引入 $H_0^1(\Omega)$
的一个有限维的子空间 $V$, 比如 $V=\mathrm{span}\{\phi_0,\phi_1,\ldots,
\phi_{N-1}\}$。 对任何 $v \in V$, 它都有唯一的表示

$$
v = \sum\limits_{i=0}^{N-1} v_i\phi_i.
$$
可以看出空间 $V$ 和 $N$ 维向量空间 $\mathbb{R}^N$ 是同构的, 即
$$
v = \sum\limits_{i=0}^{N-1} v_i\phi_i\leftrightarrow\mathbf{v} =
\begin{pmatrix}
    v_0 \\ v_1 \\ \vdots \\ v_{N-1}
\end{pmatrix}
$$
其中列向量 $\mathbf{v}$ 是 $v$ 在基 $\{\phi_i\}_{i=1}^N$ 的坐标。 

下面可以构造一个离散的问题: 求 $ \tilde u = \sum_{i=1}^{N}u_i \phi_i \in V$, 其
对应的向量为 $\mathbf u$, 满足
\begin{equation}
    \label{eq:d}
    a(\tilde u, v) = (f, v),
    \quad\forall~v\in V.
\end{equation}
方程 \eqref{eq:d} 中仍然包含有无穷多个方程。 但 $V$ 是一个有限维空间, 本质上
$\tilde u= \sum_{i=0}^{N-1}u_i \phi_i$ 只需要满足下面 $N$ 方程即可
$$
\begin{cases}
    a(\tilde u, \phi_0) = (f, \phi_0) \\
    a(\tilde u, \phi_1) = (f, \phi_1) \\
    \vdots \\
    a(\tilde u, \phi_{N-1}) = (f, \phi_{N-1})
\end{cases}
$$
即
$$
\begin{cases}
    a(\sum_{i=0}^{N-1}u_i \phi_i, \phi_0) = (f, \phi_0) \\
    a(\sum_{i=0}^{N-1}u_i \phi_i, \phi_1) = (f, \phi_1) \\
    \vdots \\
    a(\sum_{i=0}^{N-1}u_i \phi_i, \phi_{N-1}) = (f, \phi_{N-1})
\end{cases}
$$
上面的方程可以改写为下面的形式:
$$
\begin{pmatrix}
    a(\phi_0, \phi_0) & a(\phi_1, \phi_0) & \cdots & a(\phi_{N-1}, \phi_0) \\
    a(\phi_0, \phi_1) & a(\phi_1, \phi_1) & \cdots & a(\phi_{N-1}, \phi_1) \\
    \vdots & \vdots & \ddots & \vdots \\
    a(\phi_0, \phi_{N-1}) & a(\phi_1, \phi_{N-1}) & \cdots & a(\phi_{N-1},
    \phi_{N-1}) \\
\end{pmatrix}
\begin{pmatrix}
    u_0 \\ u_1 \\ \vdots \\ u_{N-1}
\end{pmatrix}
=
\begin{pmatrix}
    (f, \phi_0) \\ (f, \phi_1) \\ \vdots \\ (f, \phi_{N-1})
\end{pmatrix}
$$
上面的左端矩阵称为{\bf 刚度矩阵}(stiff matrix),简记为 $\mathbf A$;右端向量
称为 {\bf 载荷向量} (load vector), 简记为 $\mathbf f$。 上面整个线性代数系统
简记为:
$$
\mathbf A\mathbf u = \mathbf f.
$$
求解可得原问题的逼近解:
$$
\tilde u = \sum\limits_{i=0}^{N-1} u_i\phi_i.
$$

\cite{wei_fealpy}
\bibliographystyle{abbrv}
\bibliography{ref}
\end{document}
