% !Mode:: "TeX:UTF-8"
\documentclass{article}
\input{../en_preamble.tex}
\input{../xecjk_preamble.tex}
\begin{document}
\title{偏微分方程数解中的数学基础}
\author{魏华祎}
\date{\chntoday}
\maketitle
\section{简介}

\section{数学分析}

\begin{equation*}
    \begin{cases}
        -\frac{\rmd^2 u(x)}{\rmd x^2} = f, \quad 0 < x < 1 \\
        u(0) = \alpha, u(1) = \beta
    \end{cases},
\end{equation*}

\begin{equation*}
    \begin{cases}
        \frac{\partial u(x, t)}{\partial t} = 
        \frac{\partial^2 u(x, t)}{\partial x^2} + f(x), & 0< x < L, \leq t \leq T \\
        u(x, 0) = \phi(x), & 0 < x < L,\\
        u(0, t) = u(L, t) = 0, & 0 \leq t \leq T,
    \end{cases},
\end{equation*}

\begin{equation*}
    \begin{cases}
        \frac{\partial u(x, t)}{\partial t} + 
        \frac{\partial u(x, t)}{\partial x} = 0, & 0 < x < 2, t > 0\\
        u(x, 0) = |x - 1|, u(0, t) = 1
    \end{cases}
\end{equation*}

\begin{equation*}
    \begin{cases}
        \frac{\rmd}{\rmd x}\left(x\frac{\rmd u(x)}{\rmd x}\right) = x, \quad 0 < x < 1 \\
        u(0) = 0, u(1) = 0 
    \end{cases},
\end{equation*}



\subsection{散度定理}
给定向量函数 $\bfF(x)$, 其定义域为 $\Omega\in\mathbb R^n$, $\bfn$ 是 $\Omega$ 边界 $\partial \Omega$ 上的单位外法线向量.
$$
\int_{\Omega} \nabla\cdot\bfF~ \rmd \bfx = \int_{\partial \Omega}\bfF\cdot\bfn ~\rmd s
$$
这就是{\bf 散度定理}。 下面给出一些简单的应用:
$$
\begin{aligned}
\int_{\Omega} \nabla\cdot(v\nabla u)~\rmd \bfx &= \int_{\partial\Omega} v\nabla
u\cdot\bfn~\rmd s\\
\int_{\Omega} v\Delta u~\rmd \bfx + \int_{\Omega}\nabla u\cdot\nabla v~\rmd \bfx
&= \int_{\partial\Omega} v\nabla u\cdot\bfn~\rmd s
\end{aligned}
$$

$$
\int_\Omega v_x \rmd\bfx = \int_\Omega \nabla\cdot \begin{pmatrix}
v\\0
\end{pmatrix} \rmd \bfx =
\int_{\partial \Omega} vn_x \mathrm d s
$$

\section{高等代数}

\section{解析几何}

\cite{wei_fealpy}
\bibliographystyle{abbrv}
\bibliography{ref}
\end{document}
