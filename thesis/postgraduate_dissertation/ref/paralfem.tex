% !Mode:: "TeX:UTF-8"
\documentclass{beamer}

\usetheme{Darmstadt}
\useinnertheme{rounded}

\usecolortheme{beaver}
%\usecolortheme{albatross}
%\usecolortheme{beetle}
%\usecolortheme{crane}
%\usecolortheme{dolphin}
%\usecolortheme{dove}
%\usecolortheme{fly}
%\usecolortheme{lily}
%\usecolortheme{orchid}
%\usecolortheme{rose}
%\usecolortheme{seagull}
%\usecolortheme{seahorse}
%\usecolortheme{whale}
%\usecolortheme{wolverine}
%\usecolortheme{default}


\setbeamerfont*{frametitle}{size=\normalsize,series=\bfseries}
\setbeamertemplate{navigation symbols}{}


%%%%%%%%------------------------------------------------------------------------
%%%% 日常所用宏包

%% 控制页边距
% 如果是beamer文档类, 则不用geometry
\makeatletter
\@ifclassloaded{beamer}{}{\usepackage[top=2.5cm, bottom=2.5cm, left=2.5cm, right=2.5cm]{geometry}}
\makeatother

\makeatletter
\@ifclassloaded{beamer}{
\makeatletter
\def\th@mystyle{%
    \normalfont % body font
    \setbeamercolor{block title example}{bg=orange,fg=white}
    \setbeamercolor{block body example}{bg=orange!20,fg=black}
    \def\inserttheoremblockenv{exampleblock}
  }
\makeatother
\theoremstyle{mystyle}
\newtheorem*{remark}{Remark}

\newcommand{\propnumber}{} % initialize
\newtheorem*{prop}{Proposition \propnumber}
\newenvironment{propc}[1]
  {\renewcommand{\propnumber}{#1}%
   \begin{shaded}\begin{prop}}
  {\end{prop}\end{shaded}}

\makeatletter
\newenvironment<>{proofs}[1][\proofname]{%
    \par
    \def\insertproofname{#1\@addpunct{.}}%
    \usebeamertemplate{proof begin}#2}
  {\usebeamertemplate{proof end}}
\makeatother

}{
}
\makeatother
\usepackage{amsthm}

%\DeclareMathOperator{\sech}{sech}
%\DeclareMathOperator{\csch}{csch}
%\DeclareMathOperator{\arcsec}{arcsec}
%\DeclareMathOperator{\arccot}{arccot}
%\DeclareMathOperator{\arccsc}{arccsc}
%\DeclareMathOperator{\arccosh}{arccosh}
%\DeclareMathOperator{\arcsinh}{arcsinh}
%\DeclareMathOperator{\arctanh}{arctanh}
%\DeclareMathOperator{\arcsech}{arcsech}
%\DeclareMathOperator{\arccsch}{arccsch}
%\DeclareMathOperator{\arccoth}{arccoth}
%% 控制项目列表
\usepackage{enumerate}

%% Todo list
\usepackage{enumitem}
\newlist{todolist}{itemize}{2}
\setlist[todolist]{label=$\square$}
\usepackage{pifont}
\newcommand{\cmark}{\ding{51}}%
\newcommand{\xmark}{\ding{55}}%
\newcommand{\done}{\rlap{$\square$}{\raisebox{2pt}{\large\hspace{1pt}\cmark}}%
\hspace{-2.5pt}}
\newcommand{\wontfix}{\rlap{$\square$}{\large\hspace{1pt}\xmark}}

\usepackage[utf8]{inputenc}
\usepackage[english]{babel}

\usepackage{framed}

%% 多栏显示
\usepackage{multicol}

%% 算法环境
\usepackage{algorithm}
\usepackage{algorithmic}
\usepackage{float}

%% 网址引用
\usepackage{url}

%% 控制矩阵行距
\renewcommand\arraystretch{1.4}

%% 粗体
\usepackage{lmodern}
\usepackage{bm}


%% hyperref宏包,生成可定位点击的超链接,并且会生成pdf书签
\makeatletter
\@ifclassloaded{beamer}{
\usepackage{hyperref}
\usepackage{ragged2e} % 对齐
}{
\usepackage[%
    pdfstartview=FitH,%
    CJKbookmarks=true,%
    bookmarks=true,%
    bookmarksnumbered=true,%
    bookmarksopen=true,%
    colorlinks=true,%
    citecolor=blue,%
    linkcolor=blue,%
    anchorcolor=green,%
    urlcolor=blue%
]{hyperref}
}
\makeatother



\makeatletter % 如果是 beamer 不需要下面两个包
\@ifclassloaded{beamer}{
\mode<presentation>
{
}
}{
%% 控制标题
\usepackage{titlesec}
%% 控制目录
\usepackage{titletoc}
}
\makeatother

%% 控制表格样式
\usepackage{booktabs}

%% 控制字体大小
\usepackage{type1cm}

%% 首行缩进,用\noindent取消某段缩进
\usepackage{indentfirst}

%% 支持彩色文本、底色、文本框等
\usepackage{color,xcolor}

%% AMS LaTeX宏包: http://zzg34b.w3.c361.com/package/maths.htm#amssymb
\usepackage{amsmath,amssymb}
%% 多个图形并排
\usepackage{subfig}
%%%% 基本插图方法
%% 图形宏包
\usepackage{graphicx}


%%%% 基本插图方法结束

%%%% pgf/tikz绘图宏包设置
\usepackage{pgf,tikz}
\usetikzlibrary{shapes,automata,snakes,backgrounds,arrows}
\usetikzlibrary{mindmap}
%% 可以直接在latex文档中使用graphviz/dot语言,
%% 也可以用dot2tex工具将dot文件转换成tex文件再include进来
%% \usepackage[shell,pgf,outputdir={docgraphs/}]{dot2texi}
%%%% pgf/tikz设置结束


\makeatletter % 如果是 beamer 不需要下面两个包
\@ifclassloaded{beamer}{

}{
%%%% fancyhdr设置页眉页脚
%% 页眉页脚宏包
\usepackage{fancyhdr}
%% 页眉页脚风格
\pagestyle{plain}
}

%% 有时会出现\headheight too small的warning
\setlength{\headheight}{15pt}

%% 清空当前页眉页脚的默认设置
%\fancyhf{}
%%%% fancyhdr设置结束


%% 设置listings宏包的一些全局样式
%% 参考http://hi.baidu.com/shawpinlee/blog/item/9ec431cbae28e41cbe09e6e4.html
\usepackage{listings}
\lstloadlanguages{[LaTeX]TeX}

\usepackage{fancyvrb}

\newenvironment{latexample}[1][language={[LaTeX]TeX}]
{\lstset{breaklines=true,
    prebreak = \raisebox{0ex}[0ex][0ex]{\ensuremath{\hookleftarrow}},
    frame=single,
    language={[LaTeX]TeX},
    showstringspaces=false,              %% 设定是否显示代码之间的空格符号
    numbers=left,                        %% 在左边显示行号
    numberstyle=\tiny,                   %% 设定行号字体的大小
    basicstyle=\scriptsize,                    %% 设定字体大小\tiny, \small, \Large等等
    keywordstyle=\color{blue!70}, commentstyle=\color{red!50!green!50!blue!50},
                                         %% 关键字高亮
    frame=shadowbox,                     %% 给代码加框
    rulesepcolor=\color{red!20!green!20!blue!20},
    escapechar=`,                        %% 中文逃逸字符,用于中英混排
    xleftmargin=2em,xrightmargin=2em, aboveskip=1em,
    %breaklines,                          %% 这条命令可以让LaTeX自动将长的代码行换行排版
    extendedchars=false                  %% 这一条命令可以解决代码跨页时,章节标题,页眉等汉字不显示的问题
    basicstyle=\footnotesize\ttfamily, #1}
  \VerbatimEnvironment\begin{VerbatimOut}{latexample.verb.out}}
  {\end{VerbatimOut}\noindent
  \begin{minipage}{1.05\linewidth}
    \lstinputlisting[]{latexample.verb.out}%
  \end{minipage}\qquad
  \begin{minipage}{1\linewidth}
    \input{latexample.verb.out}
  \end{minipage}\\
}

\usepackage{minted}
\renewcommand{\listingscaption}{Python code} \newminted{python}{
    escapeinside=||,
    mathescape=true,
    numbersep=5pt,
    linenos=true,
    autogobble,
    framesep=3mm}
%%%% listings宏包设置结束


%%%% 附录设置
\makeatletter % 对 beamer 要重新设置
\@ifclassloaded{beamer}{

}{
\usepackage[title,titletoc,header]{appendix}
}
\makeatother
%%%% 附录设置结束





%% 设定行距
\linespread{1}

%% 颜色
\newcommand{\red}{\color{red} }
\newcommand{\blue}{\color{blue} }
\newcommand{\brown}{\color{brown} }
\newcommand{\green}{\color{green} }

\newcommand{\bred}{\bf\color{red} }
\newcommand{\bblue}{\bf\color{blue} }
\newcommand{\bbrown}{\bf\color{brown} }
\newcommand{\bgreen}{\bf\color{green} }
%% 1. 小写的英文或希腊字母表示 标量或标量函数
%% 2. 大写的英文或希腊字母表示 集合或空间
%% 3. 粗体的小写字母代表向量或向量形式的常量和函数
%% 4. 粗体的大写字母代表矩阵或张量形式的常量和函数
%% 5. 空心大写字母代表特殊的空间 \mbR 实数 \mbC 复数 \mbP 多项式
%% 6. 花体的大写字母代表算子

%% 粗体的小写字母代表向量或向量函数
\newcommand{\bfa}{{\boldsymbol a}}
\newcommand{\bfb}{{\boldsymbol b}}
\newcommand{\bfc}{{\boldsymbol c}}
\newcommand{\bfd}{{\boldsymbol d}}
\newcommand{\bfe}{{\boldsymbol e}}
\newcommand{\bff}{{\boldsymbol f}}
\newcommand{\bfg}{{\boldsymbol g}}
\newcommand{\bfh}{{\boldsymbol h}}
\newcommand{\bfi}{{\boldsymbol i}}
\newcommand{\bfj}{{\boldsymbol j}}
\newcommand{\bfk}{{\boldsymbol k}}
\newcommand{\bfl}{{\boldsymbol l}}
\newcommand{\bfm}{{\boldsymbol m}}
\newcommand{\bfn}{{\boldsymbol n}}
\newcommand{\bfo}{{\boldsymbol o}}
\newcommand{\bfp}{{\boldsymbol p}}
\newcommand{\bfq}{{\boldsymbol q}}
\newcommand{\bfr}{{\boldsymbol r}}
\newcommand{\bfs}{{\boldsymbol s}}
\newcommand{\bft}{{\boldsymbol t}}
\newcommand{\bfu}{{\boldsymbol u}}
\newcommand{\bfv}{{\boldsymbol v}}
\newcommand{\bfw}{{\boldsymbol w}}
\newcommand{\bfx}{{\boldsymbol x}}
\newcommand{\bfy}{{\boldsymbol y}}
\newcommand{\bfz}{{\boldsymbol z}}

%  算子
\newcommand{\mca}{{\mathcal a}}
\newcommand{\mcb}{{\mathcal b}}
\newcommand{\mcc}{{\mathcal c}}
\newcommand{\mcd}{{\mathcal d}}
\newcommand{\mce}{{\mathcal e}}
\newcommand{\mcf}{{\mathcal f}}
\newcommand{\mcg}{{\mathcal g}}
\newcommand{\mch}{{\mathcal h}}
\newcommand{\mci}{{\mathcal i}}
\newcommand{\mcj}{{\mathcal j}}
\newcommand{\mck}{{\mathcal k}}
\newcommand{\mcl}{{\mathcal l}}
\newcommand{\mcm}{{\mathcal m}}
\newcommand{\mcn}{{\mathcal n}}
\newcommand{\mco}{{\mathcal o}}
\newcommand{\mcp}{{\mathcal p}}
\newcommand{\mcq}{{\mathcal q}}
\newcommand{\mcr}{{\mathcal r}}
\newcommand{\mcs}{{\mathcal s}}
\newcommand{\mct}{{\mathcal t}}
\newcommand{\mcu}{{\mathcal u}}
\newcommand{\mcv}{{\mathcal v}}
\newcommand{\mcw}{{\mathcal w}}
\newcommand{\mcx}{{\mathcal x}}
\newcommand{\mcy}{{\mathcal y}}
\newcommand{\mcz}{{\mathcal z}}

% \rmd
\newcommand{\mra}{{\mathrm a}}
\newcommand{\mrb}{{\mathrm b}}
\newcommand{\mrc}{{\mathrm c}}
\newcommand{\mrd}{{\mathrm d}}
\newcommand{\mre}{{\mathrm e}}
\newcommand{\mrf}{{\mathrm f}}
\newcommand{\mrg}{{\mathrm g}}
\newcommand{\mrh}{{\mathrm h}}
\newcommand{\mri}{{\mathrm i}}
\newcommand{\mrj}{{\mathrm j}}
\newcommand{\mrk}{{\mathrm k}}
\newcommand{\mrl}{{\mathrm l}}
\newcommand{\mrm}{{\mathrm m}}
\newcommand{\mrn}{{\mathrm n}}
\newcommand{\mro}{{\mathrm o}}
\newcommand{\mrp}{{\mathrm p}}
\newcommand{\mrq}{{\mathrm q}}
\newcommand{\mrr}{{\mathrm r}}
\newcommand{\mrs}{{\mathrm s}}
\newcommand{\mrt}{{\mathrm t}}
\newcommand{\mru}{{\mathrm u}}
\newcommand{\mrv}{{\mathrm v}}
\newcommand{\mrw}{{\mathrm w}}
\newcommand{\mrx}{{\mathrm x}}
\newcommand{\mry}{{\mathrm y}}
\newcommand{\mrz}{{\mathrm z}}

%% 粗体的大写字母一般表示矩阵和张量
\newcommand{\bfA}{{\boldsymbol A}}
\newcommand{\bfB}{{\boldsymbol B}}
\newcommand{\bfC}{{\boldsymbol C}}
\newcommand{\bfD}{{\boldsymbol D}}
\newcommand{\bfE}{{\boldsymbol E}}
\newcommand{\bfF}{{\boldsymbol F}}
\newcommand{\bfG}{{\boldsymbol G}}
\newcommand{\bfH}{{\boldsymbol H}}
\newcommand{\bfI}{{\boldsymbol I}}
\newcommand{\bfJ}{{\boldsymbol J}}
\newcommand{\bfK}{{\boldsymbol K}}
\newcommand{\bfL}{{\boldsymbol L}}
\newcommand{\bfM}{{\boldsymbol M}}
\newcommand{\bfN}{{\boldsymbol N}}
\newcommand{\bfO}{{\boldsymbol O}}
\newcommand{\bfP}{{\boldsymbol P}}
\newcommand{\bfQ}{{\boldsymbol Q}}
\newcommand{\bfR}{{\boldsymbol R}}
\newcommand{\bfS}{{\boldsymbol S}}
\newcommand{\bfT}{{\boldsymbol T}}
\newcommand{\bfU}{{\boldsymbol U}}
\newcommand{\bfV}{{\boldsymbol V}}
\newcommand{\bfW}{{\boldsymbol W}}
\newcommand{\bfX}{{\boldsymbol X}}
\newcommand{\bfY}{{\boldsymbol Y}}
\newcommand{\bfZ}{{\boldsymbol Z}}

%% 花体大写字母
\newcommand{\mcA}{{\mathcal A}}
\newcommand{\mcB}{{\mathcal B}}
\newcommand{\mcC}{{\mathcal C}}
\newcommand{\mcD}{{\mathcal D}}
\newcommand{\mcE}{{\mathcal E}}
\newcommand{\mcF}{{\mathcal F}}
\newcommand{\mcG}{{\mathcal G}}
\newcommand{\mcH}{{\mathcal H}}
\newcommand{\mcI}{{\mathcal I}}
\newcommand{\mcJ}{{\mathcal J}}
\newcommand{\mcK}{{\mathcal K}}
\newcommand{\mcL}{{\mathcal L}}
\newcommand{\mcM}{{\mathcal M}}
\newcommand{\mcN}{{\mathcal N}}
\newcommand{\mcO}{{\mathcal O}}
\newcommand{\mcP}{{\mathcal P}}
\newcommand{\mcQ}{{\mathcal Q}}
\newcommand{\mcR}{{\mathcal R}}
\newcommand{\mcS}{{\mathcal S}}
\newcommand{\mcT}{{\mathcal T}}
\newcommand{\mcU}{{\mathcal U}}
\newcommand{\mcV}{{\mathcal V}}
\newcommand{\mcW}{{\mathcal W}}
\newcommand{\mcX}{{\mathcal X}}
\newcommand{\mcY}{{\mathcal Y}}
\newcommand{\mcZ}{{\mathcal Z}}

%% 空心大写字母
\newcommand{\mbA}{{\mathbb A}}
\newcommand{\mbB}{{\mathbb B}}
\newcommand{\mbC}{{\mathbb C}}
\newcommand{\mbD}{{\mathbb D}}
\newcommand{\mbE}{{\mathbb E}}
\newcommand{\mbF}{{\mathbb F}}
\newcommand{\mbG}{{\mathbb G}}
\newcommand{\mbH}{{\mathbb H}}
\newcommand{\mbI}{{\mathbb I}}
\newcommand{\mbJ}{{\mathbb J}}
\newcommand{\mbK}{{\mathbb K}}
\newcommand{\mbL}{{\mathbb L}}
\newcommand{\mbM}{{\mathbb M}}
\newcommand{\mbN}{{\mathbb N}}
\newcommand{\mbO}{{\mathbb O}}
\newcommand{\mbP}{{\mathbb P}}
\newcommand{\mbQ}{{\mathbb Q}}
\newcommand{\mbR}{{\mathbb R}}
\newcommand{\mbS}{{\mathbb S}}
\newcommand{\mbT}{{\mathbb T}}
\newcommand{\mbU}{{\mathbb U}}
\newcommand{\mbV}{{\mathbb V}}
\newcommand{\mbW}{{\mathbb W}}
\newcommand{\mbX}{{\mathbb X}}
\newcommand{\mbY}{{\mathbb Y}}
\newcommand{\mbZ}{{\mathbb Z}}

\newcommand{\mrA}{{\mathrm A}}
\newcommand{\mrB}{{\mathrm B}}
\newcommand{\mrC}{{\mathrm C}}
\newcommand{\mrD}{{\mathrm D}}
\newcommand{\mrE}{{\mathrm E}}
\newcommand{\mrF}{{\mathrm F}}
\newcommand{\mrG}{{\mathrm G}}
\newcommand{\mrH}{{\mathrm H}}
\newcommand{\mrI}{{\mathrm I}}
\newcommand{\mrJ}{{\mathrm J}}
\newcommand{\mrK}{{\mathrm K}}
\newcommand{\mrL}{{\mathrm L}}
\newcommand{\mrM}{{\mathrm M}}
\newcommand{\mrN}{{\mathrm N}}
\newcommand{\mrO}{{\mathrm O}}
\newcommand{\mrP}{{\mathrm P}}
\newcommand{\mrQ}{{\mathrm Q}}
\newcommand{\mrR}{{\mathrm R}}
\newcommand{\mrS}{{\mathrm S}}
\newcommand{\mrT}{{\mathrm T}}
\newcommand{\mrU}{{\mathrm U}}
\newcommand{\mrV}{{\mathrm V}}
\newcommand{\mrW}{{\mathrm W}}
\newcommand{\mrX}{{\mathrm X}}
\newcommand{\mrY}{{\mathrm Y}}
\newcommand{\mrZ}{{\mathrm Z}}


% 粗体的 Greek 字母
\newcommand{\balpha}{{\bm \alpha}}
\newcommand{\bbeta}{{\bm \beta}}
\newcommand{\bgamma}{{\bm \gamma}}
\newcommand{\bdelta}{{\bm \delta}}
\newcommand{\bepsilon}{{\bm \epsilon}}
\newcommand{\bvarepsilon}{{\bm \varepsilon}}
\newcommand{\bzeta}{{\bm \zeta}}
\newcommand{\bfeta}{{\bm \eta}}
\newcommand{\btheta}{{\bm \theta}}
\newcommand{\biota}{{\bm \iota}}
\newcommand{\bkappa}{{\bm \kappa}}
\newcommand{\blambda}{{\bm \lambda}}
\newcommand{\bmu}{{\bm \mu}}
\newcommand{\bnu}{{\bm \nu}}
\newcommand{\bxi}{{\bm \xi}}
\newcommand{\bomicron}{{\bm \omicron}}
\newcommand{\bpi}{{\bm \pi}}
\newcommand{\brho}{{\bm \rho}}
\newcommand{\bsigma}{{\bm \sigma}}
\newcommand{\btau}{{\bm \tau}}
\newcommand{\bupsilon}{{\bm \upsilon}}
\newcommand{\bphi}{{\bm \phi}}
\newcommand{\bvarphi}{{\bm \varphi}}
\newcommand{\bchi}{{\bm \chi}}
\newcommand{\bpsi}{{\bm \psi}}

\newcommand{\bAlpha}{{\bm \Alpha}}
\newcommand{\bBeta}{{\bm \Beta}}
\newcommand{\bGamma}{{\bm \Gamma}}
\newcommand{\bDelta}{{\bm \Delta}}
\newcommand{\bEpsilon}{{\bm \Psilon}}
\newcommand{\bVarepsilon}{{\bm \Varepsilon}}
\newcommand{\bZeta}{{\bm \Zeta}}
\newcommand{\bEta}{{\bm \Eta}}
\newcommand{\bTheta}{{\bm \Theta}}
\newcommand{\bIota}{{\bm \Iota}}
\newcommand{\bKappa}{{\bm \Kappa}}
\newcommand{\bLambda}{{\bm \Lambda}}
\newcommand{\bMu}{{\bm \Mu}}
\newcommand{\bNu}{{\bm \Nu}}
\newcommand{\bXi}{{\bm \Xi}}
\newcommand{\bOmicron}{{\bm \Omicron}}
\newcommand{\bPi}{{\bm \Pi}}
\newcommand{\bRho}{{\bm \Rho}}
\newcommand{\bSigma}{{\bm \Sigma}}
\newcommand{\bTau}{{\bm \Tau}}
\newcommand{\bUpsilon}{{\bm \Upsilon}}
\newcommand{\bPhi}{{\bm \Phi}}
\newcommand{\bChi}{{\bm \Chi}}
\newcommand{\bPsi}{{\bm \Psi}}

% \int_\Omega \bfx^2 \rmd \bfx
\newcommand{\rmd}{\,\mathrm d}
\newcommand{\bfzero}{\mathbf 0}

%% 算子
\newcommand{\ospan}{\operatorname{span}}
\newcommand{\odiv}{\operatorname{div}}
\newcommand{\otr}{\operatorname{tr}}
\newcommand{\ograd}{\operatorname{grad}}
\newcommand{\orot}{\operatorname{rot}}
\newcommand{\ocurl}{\operatorname{curl}}
\newcommand{\odist}{\operatorname{dist}}
\newcommand{\osign}{\operatorname{sign}}
\newcommand{\odiag}{\operatorname{diag}}
\newcommand{\oran}{\operatorname{Ran}} % 像空间
\newcommand{\oker}{\operatorname{Ker}} % 核空间
\newcommand{\ore}{\operatorname{Re}} % 实部
\newcommand{\oim}{\operatorname{Im}} % 虚部
\newcommand{\orank}{\operatorname{rank}}
\newcommand{\ovec}{\operatorname{vec}}
\newcommand{\odet}{\operatorname{det}}
\newcommand{\odim}{\operatorname{dim}}
\newcommand{\osym}{\operatorname{sym}}

\newcommand{\obcurl}{\operatorname{\bf curl}}
%%%% 个性设置结束
%%%%%%%%------------------------------------------------------------------------


%%%%%%%%------------------------------------------------------------------------
%%%% bibtex设置

%% 设定参考文献显示风格
% 下面是几种常见的样式
% * plain: 按字母的顺序排列,比较次序为作者、年度和标题
% * unsrt: 样式同plain,只是按照引用的先后排序
% * alpha: 用作者名首字母+年份后两位作标号,以字母顺序排序
% * abbrv: 类似plain,将月份全拼改为缩写,更显紧凑
% * apalike: 美国心理学学会期刊样式, 引用样式 [Tailper and Zang, 2006]

%\makeatletter
%\@ifclassloaded{beamer}{
%\bibliographystyle{apalike}
%}{
%\bibliographystyle{abbrv}
%}
%\makeatother


%%%% bibtex设置结束
%%%%%%%%------------------------------------------------------------------------

%%%%%%%%------------------------------------------------------------------------
%%%% xeCJK相关宏包

\usepackage{xltxtra, fontenc, xunicode}
\usepackage[slantfont, boldfont]{xeCJK}

\setlength{\parindent}{1.5em}%中文缩进两个汉字位

%% 针对中文进行断行
\XeTeXlinebreaklocale "zh"

%% 给予TeX断行一定自由度
\XeTeXlinebreakskip = 0pt plus 1pt minus 0.1pt

%%%% xeCJK设置结束
%%%%%%%%------------------------------------------------------------------------

%%%%%%%%------------------------------------------------------------------------
%%%% xeCJK字体设置

%% 设置中文标点样式,支持quanjiao、banjiao、kaiming等多种方式
\punctstyle{kaiming}

%% 设置缺省中文字体
\setCJKmainfont[BoldFont={Adobe Heiti Std}, ItalicFont={Adobe Kaiti Std}]{Adobe Song Std}
%\setCJKmainfont{Adobe Kaiti Std}
%% 设置中文无衬线字体
%\setCJKsansfont[BoldFont={Adobe Heiti Std}]{Adobe Kaiti Std}
%% 设置等宽字体
%\setCJKmonofont{Adobe Heiti Std}

%% 英文衬线字体
\setmainfont{DejaVu Serif}
%% 英文等宽字体
\setmonofont{DejaVu Sans Mono}
%% 英文无衬线字体
\setsansfont{DejaVu Sans}

%% 定义新字体
\setCJKfamilyfont{song}{Adobe Song Std}
\setCJKfamilyfont{kai}{Adobe Kaiti Std}
\setCJKfamilyfont{hei}{Adobe Heiti Std}
\setCJKfamilyfont{fangsong}{Adobe Fangsong Std}
\setCJKfamilyfont{lisu}{LiSu}
\setCJKfamilyfont{youyuan}{YouYuan}

%% 自定义宋体
\newcommand{\song}{\CJKfamily{song}}
%% 自定义楷体
\newcommand{\kai}{\CJKfamily{kai}}
%% 自定义黑体
\newcommand{\hei}{\CJKfamily{hei}}
%% 自定义仿宋体
\newcommand{\fangsong}{\CJKfamily{fangsong}}
%% 自定义隶书
\newcommand{\lisu}{\CJKfamily{lisu}}
%% 自定义幼圆
\newcommand{\youyuan}{\CJKfamily{youyuan}}

%%%% xeCJK字体设置结束
%%%%%%%%------------------------------------------------------------------------

%%%%%%%%------------------------------------------------------------------------
%%%% 一些关于中文文档的重定义
\newcommand{\chntoday}{\number\year\,年\,\number\month\,月\,\number\day\,日}
%% 数学公式定理的重定义

%% 中文破折号,据说来自清华模板
\newcommand{\pozhehao}{\kern0.3ex\rule[0.8ex]{2em}{0.1ex}\kern0.3ex}

\makeatletter %
\@ifclassloaded{beamer}{

}{
\newtheorem{example}{例}
\newtheorem{theorem}{定理}[section]
\newtheorem{definition}{定义}
\newtheorem{axiom}{公理}
\newtheorem{property}{性质}
\newtheorem{proposition}{命题}
\newtheorem{lemma}{引理}
\newtheorem{corollary}{推论}
\newtheorem{remark}{注解}
\newtheorem{condition}{条件}
\newtheorem{conclusion}{结论}
\newtheorem{assumption}{假设}
}
\makeatother

\makeatletter %
\@ifclassloaded{beamer}{

}{
%% 章节等名称重定义
\renewcommand{\contentsname}{目录}
\renewcommand{\indexname}{索引}
\renewcommand{\listfigurename}{插图目录}
\renewcommand{\listtablename}{表格目录}
\renewcommand{\appendixname}{附录}
\renewcommand{\appendixpagename}{附录}
\renewcommand{\appendixtocname}{附录}
\@ifclassloaded{book}{

}{
\renewcommand{\abstractname}{摘要}
}
}
\makeatother

\renewcommand{\figurename}{图}
\renewcommand{\tablename}{表}

\makeatletter
\@ifclassloaded{book}{
\renewcommand{\bibname}{参考文献}
}{
\renewcommand{\refname}{参考文献}
}
\makeatother

\floatname{algorithm}{算法}
\renewcommand{\algorithmicrequire}{\textbf{输入:}}
\renewcommand{\algorithmicensure}{\textbf{输出:}}

\renewcommand{\today}{\number\year 年 \number\month 月 \number\day 日}

%%%% 中文重定义结束
%%%%%%%%------------------------------------------------------------------------


\usepackage{biblatex}
\addbibresource{ref.bib}

\usefonttheme[onlymath]{serif}
\numberwithin{subsection}{section}
%\usefonttheme[onlylarge]{structurebold}
\setbeamercovered{transparent}

\title{FEALPy 偏微分方程数值解程序设计与实现: 
    {\bf 参数拉格朗日有限元空间的构造}}
\author{魏华祎}
\institute[XTU]{
weihuayi@xtu.edu.cn\\
\vspace{5pt}
湘潭大学$\bullet$数学与计算科学学院\\
}
 
\date[XTU]
{
    \today
}


\AtBeginSection[]
{
  \frame<beamer>{ 
    \frametitle{Outline}   
    \tableofcontents[currentsection] 
  }
}

\AtBeginSubsection[]
{
  \frame<beamer>{ 
    \frametitle{Outline}   
    \tableofcontents[currentsubsection] 
  }
}

\begin{document}
\begin{frame}
  \titlepage
\end{frame}

\begin{frame}{Outline}
  \tableofcontents
\end{frame}

\section{关于重心坐标的导数}

\begin{frame}
    \frametitle{关于重心坐标的导数}
    考虑下面的函数序列
    $$
        g_i(\lambda) := \prod_{j=0}^{i-1}(p\lambda - j), i=0, 1, \cdots, p.
    $$
    其中 $g_0(\lambda) = 1$. 易知该函数序列有如下关系式成立
    $$
        g_i(\lambda) = (p\lambda - i + 1)g_{i-1}(\lambda),
    $$
    下面推导 $g_i$ 关于 $\lambda$ 的任意阶导数的计算公式.
\end{frame}

\begin{frame}
    \frametitle{各阶导数的计算公式}
    \begin{align*}
        \frac{\rmd g_i}{\rmd \lambda} =& (p\lambda - i + 1)\frac{\rmd g_{i-1}}{\rmd \lambda} + pg_{i-1}(\lambda) \\
        \frac{\rmd^2 g_i}{\rmd \lambda^2} =&
        (p\lambda - i + 1)\frac{\rmd^2 g_{i-1}}{\rmd \lambda^2} +
        2p\frac{\rmd g_{i-1}}{\rmd \lambda}\\ 
        \frac{\rmd^3 g_i}{\rmd \lambda^3} =&
        (p\lambda - i + 1)\frac{\rmd^3 g_{i-1}}{\rmd \lambda^3} +
        3p\frac{\rmd^2 g_{i-1}}{\rmd \lambda^2}\\ 
        \vdots & \\
        \frac{\rmd^p g_i}{\rmd \lambda^p} =&
        (p\lambda - i + 1)\frac{\rmd^p g_{i-1}}{\rmd \lambda^p} +
        p^2\frac{\rmd^p g_{i-1}}{\rmd \lambda^p}\\ 
    \end{align*}
\end{frame}

\begin{frame}
    \frametitle{导数的数组化计算公式}
    记
    $$
        \bfG = 
        \begin{bmatrix}
            g_0(\lambda) \\ g_1(\lambda) \\ \vdots \\ g_{p-1}(\lambda)
        \end{bmatrix}
    $$
    \begin{equation*}
        \frac{\rmd \bfG}{\rmd \lambda} = 
        \begin{bmatrix}
             \frac{\rmd g_0}{\rmd \lambda}    \\ 
             \frac{\rmd g_1}{\rmd \lambda}    \\ 
             \vdots \\ 
             \frac{\rmd g_{p-1}}{\rmd \lambda}
        \end{bmatrix}
        =
        \begin{bmatrix}
             0    \\ 
             (p\lambda - 0)\frac{\rmd g_{0}}{\rmd \lambda} + pg_{0}(\lambda)   \\ 
             (p\lambda - 1)\frac{\rmd g_{1}}{\rmd \lambda} + pg_{1}(\lambda)   \\ 
             \vdots \\ 
             (p\lambda - p + 2)\frac{\rmd g_{p-2}}{\rmd \lambda} + pg_{p-2}(\lambda)   \\ 
        \end{bmatrix}
    \end{equation*}
\end{frame}


\section{区间}
\begin{frame}
    \frametitle{参考区间单元}
    给定一个区间 $[0,1]$, 其所在坐标变量记为 $\xi$ 称这个区间为{\bf 参考区间单
    元}, 简称为{\bf 参考单元}.  易知, 参考单元重心坐标为
    \begin{align*}
        \lambda_0 &= 1 - \xi\\
        \lambda_1 &= \xi\\
    \end{align*}
\end{frame}

\begin{frame}
    \frametitle{任意次拉格朗日形函数}

    给定{\bf 非负多重整数}指标向量 $\bfm_\alpha = (m_0, m_1)$, 在参考单元上可以构造如下的 $p$ 次多项式函数:
    \begin{equation}
        \phi_{\alpha} = \frac{1}{\bfm_\alpha!}\prod_{i=0}^{1}\prod_{j_i =
        0}^{m_i - 1} (p\lambda_i - j_i).
        \label{eq:phi}
    \end{equation}
    其中
    \begin{align*}
        &m_0 + m_1 = p,\quad \bfm_\alpha! = m_0!m_1!, \\
        &\prod_{j_i=0}^{-1}(p\lambda_i - j_i) = 1,\quad i = 0, 1.
    \end{align*}
    \begin{remark}
        $\phi_{\alpha}$ 可以看成 $\xi$ 的函数, 也可以看成 $\blambda
        = (\lambda_0, \lambda_1)$ 的函数. 易知, 这样的函数可以定义
        $p+1$ 个.
    \end{remark}
\end{frame}

\begin{frame}
    对于每个形函数 $\phi_\alpha$ 对应的 {\bf 非负多重整数}指标向量 $\bfm_\alpha
    = (m_0, m_1)$, 都可定义唯一的重心坐标
    $$
        \blambda_\alpha = (\frac{m_0}{p}, \frac{m_1}{p}).
    $$
    满足
    \begin{equation*}
        \phi_\alpha(\blambda_\alpha) = 1.
    \end{equation*}
    给定一个不同的{\bf 非负多重整数}指标向量 $\bfm_\beta$, 有
    \begin{equation*}
        \phi_\alpha(\blambda_\beta) = 0.
    \end{equation*}
\end{frame} 

\begin{frame}
    \frametitle{高次拉格朗日区间单元}
    在参考单元上定义 $p+1$ 个 $p$ 次的拉格朗日形函数, 写成向量函数形式
    $$
        \bphi = [\phi_0, \phi_1, \cdots, \phi_p],
    $$

    给定 $\mbR$ 中的 $p$ 次拉格朗日曲线 $l$ 的 $p+1$ 个节点
    $\{\bfx_i\}_{i=0}^{p}$, 则可以建立起从参考单元到 $l$ 的一个一一映射:
    \begin{equation*}
        \bfx = \sum_{i=0}^{p}\bfx_i\phi_i.
    \end{equation*}
    则
    \begin{align*}
        \bfx_\xi = \sum_{i=0}^{p}\bfx_i(\phi_i)_\xi.
    \end{align*}
\end{frame}

\begin{frame}
    \frametitle{高次拉格朗日多项式空间}
    在曲线 $l$ 上定义 $q$ 次拉格朗日多项式空间, 基函数记为
    $$
        \bvarphi_q = [\varphi_0, \varphi_1, \cdots, \varphi_q],
    $$
    对应的插值点记为
    $$
        [\bfy_0, \bfy_1, \cdots, \bfy_q],
    $$
    其中, $\bfy_0=\bfx_0, \bfy_q=\bfx_p$.
\end{frame}

\begin{frame}
    对任意的 $\bfx \in l$, 一一对应一个 $\xi \in [0, 1]$
    $$
        [\varphi_0(\bfx), \varphi_1(\bfx), \cdots, \varphi_q(\bfx)] = 
        [\varphi_0(\xi), \varphi_1(\xi), \cdots, \varphi_q(\xi)].
    $$

    考虑 $\varphi_i$ 关于 $\bfx \in l$ 的导数
    \begin{equation*}
        \nabla_x \varphi_i = \bfx_\xi \bfG^{-1}(\varphi_i)_\xi.
    \end{equation*}
    \begin{remark}
        $$
            \bfx_\xi \bfG^{-1} = \frac{1}{\xi_\bfx} \cdot <\xi_\bfx, \xi_\bfx> 
            =\xi_\bfx,
        $$
        其中 $\bfG = <\bfx_\xi, \bfx_\xi>$.
    \end{remark}
\end{frame}




\section{三角形}

\begin{frame}
    \frametitle{参考三角形单元}
    给定一个三角形 $K$, 三个顶点分别为 (0, 0), (1, 0), (0, 1), 其所在坐标
    系的变量记为
    $$
        \bfu = \begin{bmatrix}
            \xi \\ \eta
        \end{bmatrix}
    $$
    称 $K$ 为{\bf 参考三角形单元}, 简称为{\bf 参考单元}.  易知, $K$ 上的重心坐标为
    \begin{align*}
        \lambda_0 &= 1 - \xi - \eta\\
        \lambda_1 &= \xi\\
        \lambda_2 &= \eta\\
    \end{align*}
\end{frame}

\begin{frame}
    \frametitle{任意次拉格朗日形函数}

    给定{\bf 非负多重整数}指标向量 $\bfm_\alpha = (m_0, m_1, m_2)$, 在参考元 $K$ 上可以构造如下的 $p$ 次多项式函数:
    \begin{equation}
        \phi_{\alpha} = \frac{1}{\bfm_\alpha!}\prod_{i=0}^{2}\prod_{j_i =
        0}^{m_i - 1} (p\lambda_i - j_i).
        \label{eq:phi0}
    \end{equation}
    其中
    \begin{align*}
        &m_0 + m_1 + m_2 = p,\quad \bfm_\alpha! = m_0!m_1!m_2!, \\
        &\prod_{j_i=0}^{-1}(p\lambda_i - j_i) = 1,\quad i = 0, 1, 2.
    \end{align*}
    \begin{remark}
        $\phi_{\alpha}$ 可以看成 $\bfu=(\xi, \eta)$ 的函数, 也可以看成 $\blambda
        = (\lambda_0, \lambda_1, \lambda_2)$ 的函数. 易知, 这样的函数可以定义
        $n=(p+1)(p+2)/2$ 个.
    \end{remark}
\end{frame}

\begin{frame}
    对于每个形函数 $\phi_\alpha$ 对应的 {\bf 非负多重整数}指标向量 $\bfm_\alpha = (m_0, m_1,
    m_2)$, 都可定义唯一的重心坐标
    $$
        \blambda_\alpha = (\frac{m_0}{p}, \frac{m_1}{p},\frac{m_2}{p}).
    $$
    满足
    \begin{equation*}
        \phi_\alpha(\blambda_\alpha) = 1.
    \end{equation*}
    给定一个不同的{\bf 非负多重整数}指标向量 $\bfm_\beta$, 有
    \begin{equation*}
        \phi_\alpha(\blambda_\beta) = 0.
    \end{equation*}
\end{frame} 

\begin{frame}
    \frametitle{高次拉格朗日三角形单元}
    把参考单元 $K$ 上定义 $n$ 个 $p$ 次的拉格朗日形函数, 写成向量函数形式
    $$
        \bphi = [\phi_0, \phi_1, \cdots, \phi_{n-1}],
    $$

    给定 $\mbR^d$ 空间中的 $p$ 次拉格朗日三角形 $\tau$ 的 $n$ 个节点
    $\{\bfx_i\}_{i=0}^{n-1}$, 则可以建立起从 $K$ 到 $\tau$ 的一个一一映射:
    \begin{equation*}
        \bfx = \sum_{i=0}^{n-1}\bfx_i\phi_i.
    \end{equation*}
    \begin{remark}
        注意空间中点 $\bfx$ 默认是列向量.
    \end{remark}
\end{frame}

\begin{frame}
    \frametitle{一一映射的 Jacobi 矩阵}
    \begin{align*}
        \nabla_\bfu \bfx = \bfx_0 (\nabla_\bfu \phi_0)^T + \bfx_1(\nabla_\bfu
        \phi_1)^T
        + \cdots + \bfx_{n-1}(\nabla_\bfu \phi_{n-1})^T
    \end{align*}
    所以求映射的 Jacobi 矩阵, 只需要求每个拉格朗日形函数 $\phi_i$ 关于 $\bfu$ 的导数。
    \begin{align*}
        \nabla_\bfu \phi_i = 
        \frac{\partial \phi_i}{\partial \lambda_0}\nabla_\bfu \lambda_0 + 
        \frac{\partial \phi_i}{\partial \lambda_1}\nabla_\bfu \lambda_1 + 
        \frac{\partial \phi_i}{\partial \lambda_2}\nabla_\bfu \lambda_2 
    \end{align*}
    \begin{remark}
        注意 $\nabla_\bfu \phi_i = \begin{bmatrix} \partial_\xi \phi_i \\
        \partial_\eta \phi_i \end{bmatrix}$ 是一个列向量.
    \end{remark}
\end{frame}

\begin{frame}
    \frametitle{一一映射的 Hessian 矩阵}
    \begin{align*}
        \nabla_\bfu^2 \bfx = \bfx_0 \nabla_\bfu[(\nabla_\bfu     
        \phi_0)^T] +
        \bfx_1\nabla_\bfu[(\nabla_\bfu \phi_1)^T]
        + \cdots + \bfx_{n-1}\nabla_\bfu[(\nabla_\bfu\phi_{n-1})^T].
    \end{align*}
    所以求映射的 Jacobi 矩阵, 只需要求每个拉格朗日形函数 $\phi_i$ 关于 $\bfu$ 的导数。
    \begin{align*}
        \nabla_\bfu \phi_i = 
        \frac{\partial \phi_i}{\partial \lambda_0}\nabla_\bfu \lambda_0 + 
        \frac{\partial \phi_i}{\partial \lambda_1}\nabla_\bfu \lambda_1 + 
        \frac{\partial \phi_i}{\partial \lambda_2}\nabla_\bfu \lambda_2 
    \end{align*}
\end{frame}

\begin{frame}
    考虑$\varphi_i$关于$\bfx \in l$的散度
    \begin{align*}
     \Delta_x \varphi_i 
     &=\frac{1}{\sqrt{|G|}}[\frac{\partial}{\partial \xi},\frac{\partial}{\partial \xi}][\sqrt{|G|}G^{-1}[\varphi_{i_\xi},\varphi_{i_\eta}]] \\
     &=G^{-1}:[\varphi_{i_{ab}}]
    \end{align*}

其中$[\varphi_{i_{ab}}]$为
	$$
	\begin{bmatrix}
	\varphi_{i_{\xi\xi}}-(\nabla_x \varphi_i)^T \cdot \bfx_{\xi\xi} &
	\varphi_{i_{\xi\eta}}-(\nabla_x \varphi_i)^T \cdot \bfx_{\xi\eta} \\
	\varphi_{i_{\xi\eta}}-(\nabla_x \varphi_i)^T \cdot \bfx_{\xi\eta} &
	\varphi_{i_{\eta\eta}}-(\nabla_x \varphi_i)^T \cdot \bfx_{\eta\eta} 
	\end{bmatrix}
	$$   
\end{frame}

\begin{frame}	
其中
	\begin{align*}
	\bfx_{\xi\xi} &= \sum_{i=0}^{n-1}\bfx_i\varphi_{i_{\xi\xi}}.\\
	\bfx_{\xi\eta} &= \sum_{i=0}^{n-1}\bfx_i\varphi_{i_{\xi\eta}}.\\
	\bfx_{\eta\eta} &= \sum_{i=0}^{n-1}\bfx_i\varphi_{i_{\eta\eta}}.
	\end{align*}


\end{frame}


\section{四边形}
\section{四面体}
\section{六面体}
\section{三棱柱}
\begin{frame}
    \frametitle{参考三棱柱单元}
    给定一个三棱柱单元 $\kappa$, 顶点分别为 (0, 0, 0), (1, 0, 0), (0, 1, 0),
    (0, 0, 1), (1, 0, 1), (0, 1, 1), 它是由一个三角形单元及一个区间单元张成的.
    
    记三角形为单元 $K$, 其所在坐标系的变量记为 $\begin{bmatrix} \xi \\ \eta
    \end{bmatrix}$, $K$ 上的重心坐标为
    \begin{align*}
        \lambda_0 = 1 - \xi - \eta,\quad \lambda_1 = \xi, 
        \quad \lambda_2 = \eta.
    \end{align*}
    
    同样地, 区间单元所在坐标的变量记为 $\zeta$, 其上的重心坐标为
    \begin{align*}
        \lambda_3 = 1-\zeta,\quad
        \lambda_4 =\zeta.
    \end{align*}
\end{frame}

\begin{frame}
    三棱柱 $\kappa$ 所在坐标系的变量可记为 $\bfu = \begin{bmatrix} \xi \\ \eta
    \\ \zeta \end{bmatrix}$ 
    称 $\kappa$ 为{\bf 参考三棱柱单元}, 简称为{\bf 参考单元}.  易知, $\kappa$ 上的重心坐标为
    \begin{align*}
        \begin{bmatrix}
            \Lambda_0 & \Lambda_1\\
            \Lambda_2 & \Lambda_3\\
            \Lambda_4 & \Lambda_5\\
        \end{bmatrix}
        =&
        \begin{bmatrix}
            \lambda_0 \\ \lambda_1 \\ \lambda_2 
        \end{bmatrix}
        \cdot
        \begin{bmatrix}
            \lambda_3 & \lambda_4 
        \end{bmatrix}
        =
        \begin{bmatrix}
            \lambda_0 \lambda_3 & \lambda_0 \lambda_4 \\
            \lambda_1 \lambda_3 & \lambda_1 \lambda_4 \\
            \lambda_2 \lambda_3 & \lambda_2 \lambda_4
        \end{bmatrix}\\
        =&
        \begin{bmatrix}
            (1-\xi-\eta)(1-\zeta) & (1-\xi-\eta)\zeta \\
            \xi(1-\zeta) & \xi \zeta \\
            \eta(1-\zeta) & \eta\zeta
        \end{bmatrix}
    \end{align*}
\end{frame}

\begin{frame}
    \frametitle{高次拉格朗日三棱柱单元}
    在参考三角形单元 $K$ 上定义 $m=(p+1)(p+2)/2$ 个 $p$ 次的拉格朗日形函数
    $$
        \Phi = \{\phi_0, \phi_1, \cdots, \phi_{m-1}\},
    $$
    
    同样地, 在参考区间单元上定义 $p+1$ 个 $p$ 次的拉格朗日形函数
    $$
        \Psi = \{\psi_0, \psi_1, \cdots, \psi_p\},
    $$
    \begin{remark}
        易知, 三棱柱单元上可以定义 $n=(p+1)^2(p+2)/2$ 个函数.
    \end{remark}
\end{frame}

\begin{frame}
    给定 $\mbR^d$ 空间中的 $p$ 次拉格朗日三棱柱 $\tau$ 的 $n$ 个节点
    $\kappa$ 到 $\tau$ 的一个一一映射: \begin{align*}
        \alpha = \Phi \otimes \Psi =&
        \begin{bmatrix}
            \phi_0 \psi_0 & \phi_0 \psi_1  & \cdots & \phi_0 \psi_p \\ 
            \vdots        &   \ddots       & \ddots &  \vdots       \\
            \phi_{m-1}\psi_0 & \phi_{m-1}\psi_1 & \cdots& \phi_{m-1}\psi_p  
        \end{bmatrix}\\
        &:=
        \begin{bmatrix}
            \alpha_{0,0}   & \alpha_{0,1}    & \cdots & \alpha_{0,p} \\ 
            \vdots        &   \ddots       & \ddots &  \vdots       \\
            \alpha_{m-1,0}   &  \alpha_{m-1,1}   & \cdots &  \alpha_{m-1,p} 
        \end{bmatrix}.
    \end{align*}
    那么三棱柱上任意一点可以表示为
    \begin{equation*}
        \bfx = \sum\alpha_{i,j}\bfx_{i,j}.
    \end{equation*}
\end{frame}

\begin{frame}
    \frametitle{一一映射的 Jacobi 矩阵}
    \begin{equation*}
        \nabla_\bfu \bfx = \sum\bfx_{i,j}(\nabla_\bfu \alpha_{i,j})^T.
    \end{equation*} 
    所以求映射的 Jacobi 矩阵, 只需要求每个拉格朗日形函数 $\alpha_{i,j}$ 关于
    $\bfu$ 的导数.
    \begin{align*}
       \alpha_\xi = \Phi_\xi \otimes \Psi = 
       \begin{bmatrix}
       (\phi_0)_\xi \psi_0 & (\phi_0)_\xi \psi_1  & \cdots & (\phi_0)_\xi \psi_p \\ 
       \vdots        &   \ddots       & \ddots &  \vdots       \\
       (\phi_{m-1})_\xi \psi_0 &  (\phi_{m-1})_\xi \psi_1 &  \cdots&
       (\phi_{m-1})_\xi\psi_p 
       \end{bmatrix}
   \end{align*}
\end{frame}

\begin{frame}
   \begin{align*}
       \alpha_\eta = \Phi_\eta \otimes \Psi =
       \begin{bmatrix}
      (\phi_0)_\eta \psi_0 & (\phi_0)_\eta \psi_1  & \cdots & (\phi_0)_\eta \psi_p \\ 
       \vdots        &   \ddots       & \ddots &  \vdots       \\
       (\phi_{m-1})_\eta \psi_0 &  (\phi_{m-1})_\eta \psi_1 &  \cdots&
       (\phi_{m-1})_\eta\psi_p  
       \end{bmatrix}\\
       \alpha_\zeta = \Phi_ \otimes \Psi_\zeta = 
       \begin{bmatrix}
       \phi_0 (\psi_0)_\zeta & \phi_0 (\psi_1)_\zeta  & \cdots & \phi_0
       (\psi_p)_\zeta \\ 
       \vdots        &   \ddots       & \ddots &  \vdots       \\
       \phi_{m-1} (\psi_0)_\zeta &  \phi_{m-1} (\psi_1)_\zeta &  \cdots&
       \phi_{m-1}(\psi_p)_\zeta
       \end{bmatrix} 
    \end{align*}
\end{frame}

\begin{frame}
    \frametitle{一一映射的 Hessian 矩阵}
    \begin{align*}
        \nabla_\bfu^2 \bfx = \sum\bfx_{i,j} \nabla_\bfu[(\nabla_\bfu
        \alpha_{i,j})^T]
    \end{align*}
    所以求映射的 Hessian 矩阵, 只需要求 $(\nabla_\bfu \alpha_{i,j})^T$ 关于
    $\bfu$ 的导数.
\end{frame}

\begin{frame}
    \frametitle{第一基本形式}
    $\tau$ 上的第一基本形式为:
    \begin{align*}
        I ==\bfv\bfG\bfv^T = <\rmd \bfx, \rmd \bfx> ,
    \end{align*}
    其中
    \begin{align*}
        \bfv =  
        \begin{bmatrix}
        d\xi & d\eta & d\zeta
    \end{bmatrix},
        \bfG = 
        \begin{bmatrix}
            g_{00} & g_{01} & g_{02} \\
            g_{10} & g_{11} & g_{12} \\
            g_{20} & g_{21} & g_{22}
        \end{bmatrix},
    \end{align*}
    \begin{align*}
        g_{00} = & <\bfx_\xi, \bfx_\xi>,\quad
        g_{01} =  <\bfx_\xi, \bfx_\eta> \\
        g_{02} = & <\bfx_\xi, \bfx_\zeta>,\quad
        g_{10} =  <\bfx_\eta,\bfx_\xi> ,\\
        g_{11} = & <\bfx_\eta,\bfx_\eta>,\quad
        g_{12} =  <\bfx_\eta,\bfx_\zeta> ,\\
        g_{20} = & <\bfx_\zeta,\bfx_\xi> ,\quad
        g_{21} =  <\bfx_\zeta,\bfx_\eta>, \\
        g_{22} = & <\bfx_\zeta,\bfx_\zeta>.
    \end{align*}
\end{frame}

\begin{frame}
    \frametitle{第二基本形式}
    $\tau$ 上的第二基本形式为:
    \begin{align*}
        I ==\bfv\bfB\bfv^T = -<\rmd \bfx, \rmd \bfn> ,
    \end{align*}
    \begin{align*}
        \bfB = 
        \begin{bmatrix}
            b_{00} & b_{01} & b_{02} \\
            b_{10} & b_{11} & b_{12} \\
            b_{20} & b_{21} & b_{22}
        \end{bmatrix},
    \end{align*}
    \begin{align*}
        b_{00} = & <\bfx_{\xi\xi}, \bfn> = -<\bfx_\xi, \bfn_\xi>,\\
        b_{01} = & <\bfx_{\xi\eta}, \bfn> = -<\bfx_\xi, \bfn_\eta> 
        = -<\bfx_\eta, \bfn_\xi> = b_{10},\\
        b_{02} = & <\bfx_{\xi\zeta}, \bfn> = -<\bfx_\xi, \bfn_\zeta>
        = -<\bfx_\zeta, \bfn_\xi> = b_{20},\\
        b_{11} = & <\bfx_{\eta\eta},\bfn> = -<\bfx_\eta, \bfn_\eta>,\\
        b_{12} = & <\bfx_{\eta\zeta},\bfn> = -<\bfx_\eta, \bfn_\zeta>
        = -<\bfx_\eta, \bfn_\zeta> = b_{21},\\
        b_{22} = & <\bfx_{\zeta\zeta},\bfn> = -<\bfx_\zeta, \bfn_\zeta>.
    \end{align*}
\end{frame}

\begin{frame}
    \frametitle{切梯度算子及体积}
    $\tau$ 上的切梯度算子:
    \begin{equation*}
        \nabla_\tau \alpha_{i,j} = 
        \nabla_\bfu\bfx G^{-1}\nabla_\bfu \alpha_{i,j} = 
        [\bfx_\xi, \bfx_\eta, \bfx_\zeta]G^{-1}\nabla_\bfu \alpha_{i,j}
    \end{equation*}

    $\tau$ 上的体积为:
    \begin{align*}
        |\tau| = \int_{\tau}1\rmd\bfx =
        \int_{\overline{\tau}}|\bfx_{\xi}\times\bfx_{\eta}\cdot\bfx_{\zeta}|
        \rmd \bfu
    \end{align*}
    \begin{remark}
        $\overline{\tau}$ 为标准三棱柱单元
    \end{remark}
\end{frame}

\end{document}
