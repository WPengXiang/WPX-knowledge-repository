
% !Mode:: "TeX:UTF-8"
\documentclass{beamer}

\usetheme{Darmstadt}
\useinnertheme{rounded}

\usecolortheme{beaver}
%\usecolortheme{albatross}
%\usecolortheme{beetle}
%\usecolortheme{crane}
%\usecolortheme{dolphin}
%\usecolortheme{dove}
%\usecolortheme{fly}
%\usecolortheme{lily}
%\usecolortheme{orchid}
%\usecolortheme{rose}
%\usecolortheme{seagull}
%\usecolortheme{seahorse}
%\usecolortheme{whale}
%\usecolortheme{wolverine}
%\usecolortheme{default}


\setbeamerfont*{frametitle}{size=\normalsize,series=\bfseries}
\setbeamertemplate{navigation symbols}{}

\input{../../en_preamble.tex}
\input{../../xecjk_preamble.tex}

\usepackage{biblatex}
\addbibresource{ref.bib}

\usefonttheme[onlymath]{serif}
\numberwithin{subsection}{section}
%\usefonttheme[onlylarge]{structurebold}
\setbeamercovered{transparent}

\title{FEALPy 偏微分方程数值解程序设计与实现: 
    {\bf 任意次任意维的拉格朗日有限元空间}}
\author{魏华祎}
\institute[XTU]{
weihuayi@xtu.edu.cn\\
\vspace{5pt}
湘潭大学$\bullet$数学与计算科学学院\\
}
 
\date[XTU]
{
    \today
}


\AtBeginSection[]
{
  \frame<beamer>{ 
    \frametitle{Outline}   
    \tableofcontents[currentsection] 
  }
}

\AtBeginSubsection[]
{
  \frame<beamer>{ 
    \frametitle{Outline}   
    \tableofcontents[currentsubsection] 
  }
}

\begin{document}
\begin{frame}
  \titlepage
\end{frame}

\begin{frame}{Outline}
  \tableofcontents
\end{frame}

\section{重心坐标函数}
\begin{frame}
    \frametitle{单纯形上的重心坐标函数}
\begin{onlyenv}<1>
    记 $\{\bfx_i:=[x_{i,0}, x_{i, 1}, \ldots, x_{i, d-1}]\}_{i=0}^d$ 为 $\mbR^d$ 空
间中的一组点, 假设它们不在同一个超平面上, 也即是说 $d$ 个向量 $\bfx_0\bfx_1$,
$\bfx_0\bfx_2$, $\cdots$, 和 $\bfx_0\bfx_d$ 是线性无关的, 等价于矩阵
\begin{equation*}
    \bfA =\begin{bmatrix}
        x_{0, 0} & x_{1, 0} & \cdots & x_{d, 0} \\
        x_{0, 1} & x_{1, 1} & \cdots & x_{d, 1} \\
        \vdots   & \vdots   & \ddots & \vdots \\
        x_{0, d-1} & x_{1, d-1} & \cdots & x_{d, d-1}\\
        1 & 1 & \cdots & 1
    \end{bmatrix}
\end{equation*}
是非奇异的.
\end{onlyenv}
\begin{onlyenv}<2>
    给定任一点 $\bfx=[x_0, x_1, \cdots, x_{d-1}]^T\in \mbR^d$, 求解如下线
性代数系统, 可得一组实数值 $\blambda := [\lambda_0(\bfx), \lambda_1(\bfx),
\cdots, \lambda_d(\bfx)]^T$:
\begin{equation*}
    \bfA \blambda= \bfx, 
\end{equation*}
满足如下性质
\begin{equation*}
    \bfx = \sum\limits_{i=0}^{d}\lambda_i(\bfx) \bfx_i,
    \quad\sum\limits_{i=0}^{d}\lambda_i(\bfx) = 1.
\end{equation*}
点集 $\{\bfx_i\}_{i=0}^d$ 形成的凸壳
\begin{equation*}
    \tau = \{\bfx = \sum_{i=0}^{d}\lambda_i\bfx_i | 0\leq \lambda_i \leq
    1, \sum_{i=0}^d\lambda_i = 1\}
\end{equation*}
称为一个{\bf 几何 $d$-单纯形}. $\blambda$ 称为 $\bfx$ 对应的{\bf 重心坐标}向量.
\end{onlyenv}
\begin{onlyenv}<3>
易知, $\lambda_0(\bfx)$, $\lambda_1(\bfx)$, $\cdots$, 和 $\lambda_{d}(\bfx)$
是关于 $\bfx$ 线性函数, 且 
\begin{equation*}
    \lambda_i(\bfx_j) = 
    \begin{cases}
        1, & i = j\\
        0, & i\not= j
    \end{cases}, 
    i, j = 0, \cdots, d
\end{equation*}
\end{onlyenv}
\end{frame}

\begin{frame}
    \frametitle{区间单元 $[x_0, x_1]$ 上的重心坐标函数}
给定区间单元上的一个重心坐标 $(\lambda_0, \lambda_1)$, 存在
$ x \in [x_0, x_1]$, 使得:
\[
\lambda_0 := \frac{x_1 - x}{x_1 - x_0}, \quad
\lambda_1 := \frac{x  - x_0}{x_1 - x_0}
\]
显然 
\[
\lambda_0 + \lambda_1 = 1
\]
重心坐标关于$ x $ 的导数为:
\[
\frac{\rmd \lambda_0}{\rmd x} = -\frac{1}{x_1 - x_0},\quad
\frac{\rmd \lambda_1}{\rmd x} = \frac{1}{x_1 - x_0}
\]
\end{frame}

\begin{frame}
    \frametitle{三角形单元 $[\bfx_0, \bfx_1, \bfx_2]$ 上的重心坐标函数}
因为 $\lambda_0, \lambda_1, \lambda_2$ 是关于 $\bfx$ 线性函数,它梯度分别为:
$$
\begin{aligned}
\nabla\lambda_0 = \frac{1}{2|\tau|}(\bfx_2 - \bfx_1)\bfW\\
\nabla\lambda_1 = \frac{1}{2|\tau|}(\bfx_0 - \bfx_2)\bfW\\
\nabla\lambda_2 = \frac{1}{2|\tau|}(\bfx_1 - \bfx_0)\bfW\\
\end{aligned}
$$
其中 
$$
\bfW = \begin{bmatrix}
0 & 1\\ -1 & 0 
\end{bmatrix}
$$
注意这里的 $\bfx_0$, $\bfx_1$, 和 $\bfx_2$ 是行向量。
\end{frame}

\begin{frame}[fragile]
    \frametitle{三角形单元上的重心坐标函数的梯度计算}
    \begin{listing}[H]
        \tiny
    \begin{pythoncode}
import numpy as np
from fealpy.mesh import MeshFactory
mf = MeshFactory()
box = [0, 1, 0, 1]
mesh = mf.boxmesh2d(box, nx=1, ny=1, meshtype='tri') 
NC = mesh.number_of_cells()

node = mesh.entity('node')
cell = mesh.entity('cell')
v0 = node[cell[:, 2], :] - node[cell[:, 1], :] # $x_2 - x_1$
v1 = node[cell[:, 0], :] - node[cell[:, 2], :] # $x_0 - x_2$
v2 = node[cell[:, 1], :] - node[cell[:, 0], :] # $x_1 - x_0$
nv = np.cross(v2, -v1)

Dlambda = np.zeros((NC, 3, 2), dtype=np.float64)
length = nv # 
W = np.array([[0, 1], [-1, 0]], dtype=np.int_)
Dlambda[:,0,:] = v0@W/length.reshape(-1, 1)
Dlambda[:,1,:] = v1@W/length.reshape(-1, 1)
Dlambda[:,2,:] = v2@W/length.reshape(-1, 1)
    \end{pythoncode}
    \caption{重心坐标梯度的计算代码。}
    \end{listing}
\end{frame}

\begin{frame}[fragile]
    \frametitle{作业}
    \begin{itemize}
        \item[(1)] 给定一个一维区间网格, 计算每个单元上的重心坐标函数的导数。
{\tiny
            \begin{minted}[autogobble]{python}
import numpy as np
from fealpy.mesh import IntervalMesh
node = np.array([[0.0], [0.5], [1.0]], dtype=np.float64)
cell = np.array([[0, 1], [1, 2]], dtype=np.int_)
mesh = IntervalMesh(node, cell)
            \end{minted}
}
        \item[(2)] 给定一个三维四面体网格,计算每个单元上的重心坐标函数的梯度。
{\tiny
            \begin{minted}[autogobble]{python}
import numpy as np
from fealpy.mesh import MeshFactory 

mf = MeshFactory()
mesh = mf.one_tetrahedron_mesh(meshtype='iso')
            \end{minted}
}
    \end{itemize}
\end{frame}

\section{任意维任意次拉格朗日有限元空间的构造}

\begin{frame}
    \frametitle{$d+1$ 维多重指标}
    记 $\bfm$ 为 $d+1$  维多重指标向量 $[m_0, m_1, \cdots, m_d]$, 满足  
\begin{equation*}
    m_i \geq 0, i=0, 1, \cdots, d, \text{ and } \sum_{i=0}^d m_i=p.
\end{equation*}
固定次数 $p$, $\bfm$ 的所有可能取值个数为 
\begin{equation*}
    n_p := \begin{pmatrix}
        d \\ p+d 
    \end{pmatrix}
    =\begin{cases}
        p+1, & 1D\\
        \frac{(p+1)(p+2)}{2}, & 2D\\
        \frac{(p+1)(p+2)(p+3)}{6}, & 3D
    \end{cases}
\end{equation*}
\end{frame}

\begin{frame}
    \frametitle{多重指标向量 $\bfm$ 编号规则}
记 $\alpha$ 为多重向量指标 $\bfm$ 的一个从 0 到 $n_p-1$ 一维编号, 编号规则如下: 
\begin{table}[H]
    \tiny
    \centering
    \begin{tabular}{| l | c | c | c | c | c|}
    \hline
    $\alpha$ & \multicolumn{5}{c|}{$\bfm_\alpha$} \\\hline
    0 & p   & 0 & 0 & $\cdots$ & 0 \\\hline
    1 & p-1 & 1 & 0 & $\cdots$ & 0 \\\hline
    2 & p-1 & 0 & 1 & $\cdots$ & 0 \\\hline
    $\vdots$ & $\vdots$ & $\vdots$ & $\vdots$ & $\ddots$ & $\vdots$ \\\hline
    d & p-1 & 0 & 0 & $\cdots$ & 1 \\\hline
    d+1 & p-2 & 2 & 0 & $\cdots$ & 0 \\\hline
    d+2 & p-2 & 1 & 1 & $\cdots$ & 0 \\\hline
    $\vdots$ & $\vdots$ & $\vdots$ & $\vdots$ & $\ddots$ & $\vdots$ \\\hline
    2d-1 & p-2 & 1 & 0 & $\cdots$ & 1 \\\hline
    2d & p-2 & 0 & 2 & $\cdots$ & 0 \\\hline
    $\vdots$ & $\vdots$ & $\vdots$ & $\vdots$ & $\ddots$ & $\vdots$ \\\hline
    $n_p$ & 0 & 0 & 0 & $\cdots$ & p \\
    \hline
    \end{tabular}
    \caption{多重指标 $\bfm_\alpha$ 的编号规则.}\label{tb:num}
\end{table}
\end{frame}

\begin{frame}
    \frametitle{高次拉格朗日形函数的一般公式}
给定第 $\alpha$ 个多重指标向量 $\bfm_\alpha$, 在 $d$-单纯形 $\tau$ 上可以构造如
下的 $p$ 次多项式函数:
\begin{equation}
    \phi_{\alpha} = \frac{1}{\bfm_\alpha!}\prod_{i=0}^{d}\prod_{j_i =
    0}^{m_i - 1} (p\lambda_i - j_i).
    \label{eq:phi0}
\end{equation}
其中
\begin{align*}
    \bfm_\alpha! = m_0!m_1!\cdots m_d!,\quad\quad \prod_{j_i=0}^{-1}(p\lambda_i -
    j_i) = 1,\quad i=0, 1,\cdots, d 
\end{align*}
\vspace{-0.5cm}
\begin{block}{Sylvester's Formula}
$$
R_i(p,\lambda)=
\begin{cases}
\frac{1}{i!}\prod_{j_i=0}^{i-1} (p\lambda-j_i),~& 1\leq i\leq p\\
1,& i=0\\
\end{cases}
$$
\end{block}
\end{frame}

\begin{frame}
    \frametitle{$d$-单纯形上的插值点}
每个多重指标 $\bfm_\alpha$, 都对应$d$-单纯形 $\tau$ 上的一个点 $\bfx_\alpha$, 
\begin{equation*}
    \bfx_\alpha = \sum_{i=0}^d \frac{m_i}{p} \bfx_i. 
\end{equation*}
其中 $m_i$ 是多重指标向量 $\bfm_\alpha$ 的第 $i$ 个分量. 易知 
$\bfx_\alpha$ 是 $\phi_{\alpha}$ 对应的插值点, 满足 
\begin{equation}
    \phi_\alpha(\bfx_\beta) = 
    \begin{cases}
        1, & \alpha = \beta\\
        0, & \alpha \ne \beta
    \end{cases}
    \text{ 和 }\alpha, \beta = 0, 1, \cdots, n_p - 1
    \label{eq:phi1}
\end{equation}
\end{frame}

\begin{frame}
    \frametitle{插值点的局部编号规则}
\begin{onlyenv}<1>
    $$
    \phi_{m,n,k} = \frac{1}{m!n!k!}\prod_{j_0 = 0}^{m - 1}
    (p\lambda_0 - j_0) \prod_{j_1 = 0}^{n-1}(p\lambda_1 -
    j_1) \prod_{j_2 = 0}^{k-1}(p\lambda_2 - j_2).
    $$
    \begin{figure}[h]
        \centering
        \includegraphics[scale=0.6]{./figures/dof3.png}
        \caption{三角形上的 $p=3$ 次形函数对应的编号规则.}
    \end{figure}
\end{onlyenv}
\begin{onlyenv}<2>
    \vspace{-0.8cm}
    $$
    \phi_{m,n,k} = \frac{1}{m!n!k!}\prod_{j_0 = 0}^{m - 1}
    (p\lambda_0 - j_0) \prod_{j_1 = 0}^{n-1}(p\lambda_1 -
    j_1) \prod_{j_2=0}^{k-1}(p\lambda_2 - j_2).
    $$
    \vspace{-0.5cm}
    \begin{figure}[h]
        \centering
        \includegraphics[scale=0.6]{./figures/dof4.png}
    \caption{三角形上的 $p=4$ 次形函数对应的编号规则.}
    \end{figure}
\end{onlyenv}
\begin{onlyenv}<3>
\begin{figure}[ht]
    \centering
    \includegraphics[scale=0.3]{figures/int.pdf}
    \includegraphics[scale=0.3]{figures/intdof4.pdf}
    \caption{区间上的 $p=4$ 次形函数对应的编号规则}
\end{figure}
\end{onlyenv}
\begin{onlyenv}<4>
\begin{figure}[ht]
    \centering
    \includegraphics[scale=0.3]{figures/tet.pdf}
    \includegraphics[scale=0.3]{figures/tetdof4.pdf}
    \caption{四面体上的 $p=4$ 次形函数对应的编号规则.}
\end{figure}
\end{onlyenv}
\end{frame}

\begin{frame}
    \frametitle{$\phi_\alpha$ 的数组化计算过程}
首先构造向量和矩阵
\begin{equation*}
    \tiny
    P = \left[\frac{1}{0!}, \frac{1}{1!}, \frac{1}{2!}, \cdots, \frac{1}{p!}\right],
\end{equation*}
$$
\tiny
\bfA :=                                                                            
\begin{bmatrix}  
1  &  1  & \cdot & 1 \\
p\lambda_0 & p\lambda_1 & \cdots & p\lambda_d\\                                             
p\lambda_0 - 1 & p\lambda_1 - 1 & \cdots & p\lambda_d - 1\\   
\vdots & \vdots & \ddots & \vdots \\                                                     
p\lambda_0 - (p - 1) & p\lambda_1 - (p - 1) & \vdots & p\lambda_d - (p - 1)
\end{bmatrix},
$$
\begin{block}{Remark}
$$
 \phi_{\alpha} = \frac{1}{\bfm_\alpha!}\prod_{j_0 = 0}^{m_0 - 1} (p\lambda_0 -
 j_0) \prod_{j_1 = 0}^{m_1 - 1} (p\lambda_1 - j_1)\cdots\prod_{j_d = 0}^{m_d -
 1} (p\lambda_d - j_d)
$$
\end{block}
\end{frame}

\begin{frame}
    \frametitle{$\phi_\alpha$ 的数组化计算过程}
$$
\tiny
\bfB = \mathrm{diag}(\bfP)
\begin{bmatrix}
1 & 1 & \cdots & 1 \\
p\lambda_0 & p\lambda_1 & \cdots & p\lambda_d\\
\prod_{j_0=0}^{1}(p\lambda_0 - j_0) & 
\prod_{j_1=0}^{1}(p\lambda_1 - j_1) &
\cdots &
\prod_{j_d=0}^{1}(p\lambda_d - j_d) \\
\vdots & \vdots & \ddots & \vdots \\
\prod_{j_0=0}^{p-1}(p\lambda_0 - j_0) &
\prod_{j_1=0}^{p-1}(p\lambda_1 - j_1) &
\cdots & 
\prod_{j_d=0}^{p-1}(p\lambda_d - j_d)
\end{bmatrix}
$$
\begin{onlyenv}<1>
\begin{block}{Remark}
$$
 \phi_{\alpha} = 
 \frac{1}{\bfm_\alpha!}
 \prod_{j_0 = 0}^{m_0 - 1} (p\lambda_0 - j_0)
 \prod_{j_1 = 0}^{m_1 - 1} (p\lambda_1 - j_1)
 \cdots
 \prod_{j_d = 0}^{m_{d} - 1} (p\lambda_d - j_d)
$$
\end{block}
\end{onlyenv}
\begin{onlyenv}<2>
注意 $\bfB$ 包含 $\phi_\alpha$ 的所有计算模块, 可重写  $\phi_\alpha$ 为如下形式
:
\begin{equation*}
    \phi_\alpha = \prod_{i=0}^d\bfB[m_i, i]
\end{equation*}
其中 $m_i$ 为 $\bfm_\alpha$ 的第 $i$ 个分量.
\end{onlyenv}
\end{frame}

\begin{frame}[fragile]{任意次任意维拉格朗日有限元基函数代码实现}
\vspace{-0.5cm}
\begin{listing}[H]
\tiny
    \begin{pythoncode}
def basis(self, bc): # bc.shape = (NQ, TD+1)
    p = self.p   # The degree of polynomial basis function
    TD = self.TD # The toplogy dimension of the mesh 
    multiIndex = self.dof.multiIndex # The multiindex matrix of $m_\alpha$

    # construct vector $P=(\frac{1}{1!}, \frac{1}{2!}, \cdots, \frac{1}{p!})$.
    c = np.arange(1, p+1, dtype=np.int)
    P = 1.0/np.multiply.accumulate(c)

    # construct the matrix $A$.
    t = np.arange(0, p)
    shape = bc.shape[:-1]+(p+1, TD+1)
    A = np.ones(shape, dtype=self.ftype)
    A[..., 1:, :] = p*bc[..., np.newaxis, :] - t.reshape(-1, 1)
    
    # construct matrix $B$ and here we still use the memory of $A$
    np.cumprod(A, axis=-2, out=A)
    A[..., 1:, :] *= P.reshape(-1, 1)

    # compute $\phi_\alpha$ 
    idx = np.arange(TD+1)
    phi = np.prod(A[..., multiIndex, idx], axis=-1)
    return phi[..., np.newaxis, :] # (NQ, 1, ldof)
    \end{pythoncode}
    \caption{拉格朗日有限元基函数计算展示代码。}
\end{listing}
\vspace{-0.5cm}
{\footnotesize
注意这里为清晰的展示代码,去掉了一些非必要的代码, 全部代码见 fealpy/functionspace/LagrangeFiniteELementSpace.py。
}
\end{frame}

\begin{frame}
    \frametitle{$\nabla\phi_\alpha$ 的数组化计算过程}
\begin{onlyenv}<1>
为计算 $\nabla\phi_\alpha$, 首先需要用到函数乘积求导法则来计算 $\prod_{j_i =
0}^{m_i - 1} (p\lambda_i - j_i)$ 的导数, 即
\begin{equation*}
    \nabla \prod_{j_i = 0}^{m_i - 1} (p\lambda_i - j_i)
    = p\sum_{j_i=0}^{m_i - 1}
    \prod_{0\le k \le m_i-1, k\not= j_i}(p\lambda_i - k)\nabla \lambda_i.
\end{equation*}
\begin{block}{Remark}
    这是一种标量的表达方式!
\end{block}
\begin{block}{Remark}
$$
 \phi_{\alpha} = 
 \frac{1}{\bfm_\alpha!}
 \prod_{j_0 = 0}^{m_0 - 1} (p\lambda_0 - j_0)
 \prod_{j_1 = 0}^{m_1 - 1} (p\lambda_1 - j_1)
 \cdots
 \prod_{j_d = 0}^{m_{d} - 1} (p\lambda_d - j_d)
$$
\end{block}
\end{onlyenv}
\begin{onlyenv}<2>
用数组化的方式, 需要首先构造 $d+1$ 阶矩阵 
$$
\bfD^i = 
\begin{pmatrix}
p & p\lambda_i-0 & \cdots & p\lambda_i-0 \\
p\lambda_i - 1 & p & \cdots & p\lambda_i - 1 \\
\vdots & \vdots & \ddots & \vdots \\
p\lambda_i - (p-1) & p\lambda_i - (p-1) & \cdots & p 
\end{pmatrix}
, \quad 0 \le i \le d, 
$$
把 $\bfD^i$  的每一列做累乘运算,然后取其下三角矩阵,再每一行求和, 即可得到矩阵
$\bfB$ 的每一列各个元素的求导后系数. 可得到矩阵 $\bfD$,其元素定义为
$$
D_{i,j} = \sum_{m=0}^j\prod_{k=0}^j D^i_{k, m},\quad 0 \le i \le d,
\text{ and }0 \le j \le p-1.
$$
\end{onlyenv}
\begin{onlyenv}<3>
进而可以计算 $\nabla\bfB$ 
$$
\begin{aligned}
\nabla \bfB = & \mathrm{diag}(\bfP)
\begin{pmatrix}
0 & 0 & \cdots & 0 \\
D_{0,0} \nabla \lambda_0 &
D_{1,0} \nabla \lambda_1 & \cdots&
D_{d,0} \nabla \lambda_d \\
\vdots & \vdots & \ddots & \vdots \\
D_{0, p-1} \nabla \lambda_0 &
D_{1, p-1} \nabla \lambda_1 & \cdots &
D_{d, p-1} \nabla \lambda_d
\end{pmatrix}\\
= & \mathrm{diag}(\bfP)
\begin{pmatrix}
\bfzero\\
\bfD
\end{pmatrix}
\begin{pmatrix}
\nabla \lambda_0 &  &  & \\
 & \nabla \lambda_1 & & \\
 & & \ddots & \\
 & & & \nabla \lambda_d
\end{pmatrix}\\
\end{aligned}
$$
\end{onlyenv}
\end{frame}


\begin{frame}
    \frametitle{数组化编程的一些总结}
    \begin{itemize}
        \item[$\bullet$] 算法的数组化表达是数组化编程的基础和关键。数组化的表达要明确:
            \begin{itemize}
                \item[$\square$] 算法的输入数组是什么?
                \item[$\square$] 算法最终期望得到的数组是什么?
                \item[$\square$] 算法中间需要的数组运算是什么?
            \end{itemize}
        \item[$\bullet$] 代数学是算法数组化表达的有力工具。
    \end{itemize}
\end{frame}

\section{FEALPy 中的 LagrangeFiniteELementSpace 类}

\begin{frame}
    \frametitle{有限维空间的实现关键}
    Galerkin 类型的离散方法,其程序实现的关键是{\bf 有限维空间的实现},而有限维
    空间的实现的关键是
    \begin{itemize}
        \item[$\bullet$] 基函数及其导数计算
            \begin{itemize}
                \item[$\square$] 主要是计算空间基函数及其导数在数值积分点处的值。
            \end{itemize}
        \item[$\bullet$] 全局和局部自由度的管理
            \begin{itemize}
                \item[$\square$] 全局自由度的编号规则,这与最终的离散代数系统中
                    矩阵的行号和列号相对应。 
                \item[$\square$] 局部自由度的编号规则,即基函数在每个单元上的
                    形函数的编号。 
                \item[$\square$] 局部自由度和全局自由度的对应规则,这是单元矩阵
                    组装与总矩阵需要知道的关键信息。 
            \end{itemize}
    \end{itemize}
\end{frame}

\begin{frame}
    \frametitle{LagrangeFiniteELementSpace 类的命名和接口约定}
    \begin{onlyenv}<1>
        \begin{table}[H]
            \scriptsize
            \centering
            \begin{tabular}{|l|l|}\hline
                mesh & 网格对象\\\hline
                p    & 空间的次数 \\\hline
                GD   & 几何空间的维数 \\\hline
                TD   & 几何空间的拓扑维数 \\\hline
                dof  & 函数空间的自由度管理对象, 用于管理函数空间的自由度 \\\hline
                ftype & 函数空间所用的浮点数类型 \\\hline
                itype & 函数空间所用的整数类型 \\\hline
                integralalg & 数值积分算法类  \\\hline
                spacetype & 函数空间的类型, 'C' 表示分片连续空间, 'D' 表示断空间\\\hline
            \end{tabular}
            \caption{ LagrangeFiniteElementSapce 的常用数据成员(属性)。}
        \end{table}
    \end{onlyenv}
    \begin{onlyenv}<2>
        \vspace{-0.5cm}
        \begin{table}[H]
            \scriptsize
            \centering
            \begin{tabular}{|l|l|}\hline
                ldof = space.number\_of\_local\_dofs() & 获得每个单元上的自由度个数
                \\\hline
                gdof = space.number\_of\_global\_dofs() & 获得所有单元上的总自由度个数
                \\\hline
                cell2dof = sapce.cell\_to\_dof() & 获得单元与自由度的对应关系数组
                \\\hline
                bdof = space.boundary\_dof() & 获得边界的自由度
                \\\hline
                phi = space.basis(bc, ... )  & 计算单元上积分点处的基函数值
                \\\hline
                gphi = space.grad\_basis(bc, ...) & 计算单元上积分点处的基函数梯度值
                \\\hline
                val = space.value(uh, bc, ...) & 计算有限元函数值
                \\\hline
                val = space.grad\_value(uh, bc, ...) & 计算有限元函数梯度值 \\\hline
                uI = space.interpolation(u) & 计算函数 u 在有限元空间中的插值函数
                \\\hline
                uh = space.function(...) &  创建拉格朗日有限元空间的函数对象
                \\\hline
                A = space.stiff\_matrix(...) & 构造刚度矩阵
                \\\hline
                M = space.mass\_matrix(...) & 构造质量矩阵
                \\\hline
                F = space.source\_vector(.. .) & 构造源项向量
                \\\hline
            \end{tabular}
            \caption{ LagrangeFiniteElementSapce 常用方法成员(属性),其中省略
            号表示有默认参数。}
        \end{table}
    \end{onlyenv}
    \begin{onlyenv}<3>
        \begin{remark}
            FEALPy 中的其它类型的有限维空间对象和拉格朗日有限元空间遵
            守几乎一样的命名和接口约定。
        \end{remark}
    \end{onlyenv}
\end{frame}

\begin{frame}
    \frametitle{LagrangeFiniteELementSpace 调用演示}
    \begin{itemize}
        \item[(1)] 基函数及其导数的计算
        \item[(2)] 自由度管理
        \item[(3)] PDE 模型定义
        \item[(4)] 有限元函数与插值
        \item[(5)] 误差计算 
        \item[(6)] 刚度矩阵组装
        \item[(7)] 质量矩阵组装
        \item[(8)] 载荷向量的组装
        \item[(9)] 边界条件的处理
    \end{itemize}
\end{frame}

\end{document}
