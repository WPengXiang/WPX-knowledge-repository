
% !Mode:: "TeX:UTF-8"
\documentclass{beamer}

\usetheme{Darmstadt}
\useinnertheme{rounded}

\usecolortheme{beaver}
%\usecolortheme{albatross}
%\usecolortheme{beetle}
%\usecolortheme{crane}
%\usecolortheme{dolphin}
%\usecolortheme{dove}
%\usecolortheme{fly}
%\usecolortheme{lily}
%\usecolortheme{orchid}
%\usecolortheme{rose}
%\usecolortheme{seagull}
%\usecolortheme{seahorse}
%\usecolortheme{whale}
%\usecolortheme{wolverine}
%\usecolortheme{default}

\usepackage{graphicx}
\usepackage{pythonhighlight}
\setbeamerfont*{frametitle}{size=\normalsize,series=\bfseries}
\setbeamertemplate{navigation symbols}{}

\input{../../../en_preamble.tex}
\input{../../../xecjk_preamble.tex}

\usepackage{biblatex}
\addbibresource{ref.bib}

\usefonttheme[onlymath]{serif}
\numberwithin{subsection}{section}
%\usefonttheme[onlylarge]{structurebold}
\setbeamercovered{transparent}

\title{FEALPy 偏微分方程数值解程序设计与实现: {\bf 网格数据结构}}
\author{魏华祎 \quad 李成新}
\institute[XTU]{
weihuayi@xtu.edu.cn\\
\vspace{5pt}
湘潭大学$\bullet$数学与计算科学学院\\
}
 
\date[XTU]
{
    \today
}

 

\AtBeginSection[]
{
  \frame<beamer>{ 
    \frametitle{Outline}   
    \tableofcontents[currentsection] 
  }
}

\AtBeginSubsection[]
{
  \frame<beamer>{ 
    \frametitle{Outline}   
    \tableofcontents[currentsection] 
  }
}

\begin{document}
\begin{frame}
  \titlepage
\end{frame}

\begin{frame}{Outline}
  \tableofcontents
\end{frame}


\section{FEALPy 中的网格模块}

\begin{frame}[fragile]{FEALPy 中的网格模块}
    在偏微分方程数值计算程序设计中,网格是最核心的数据结构,是下一步实现数值离散方法
    的基础。FEALPy 中核心网格数据结构是用{\bf 数组}表示。
\begin{itemize}
    \item[$\bullet$] 三角形、四边形、四面体和六面体等网格,因为每个单元顶点的个
        数固定,因此可以用{\bf 节点坐标数组} node 和{\bf 单元拓扑数组} cell 来表
        示,这是一种以{\bf 单元为中心的数据结构}。
    \item[$\bullet$] 其它的如{\bf 边数组} edge、{\bf 面数组} face 都可由 cell 生
        成。
    \item[$\bullet$] FEALPy 中把 node、edge、face 和 cell 统称为网格中的实体 entity。
    \item[$\bullet$] 在二维情形下,FEALPy 中的 edge 和 face 意义是相同的。
    \item[$\bullet$] FEALPy 中还有一种以{\bf 边中心的网格数据结构}, 称为{\bf 半边数
    据结构(Half-Edge data structure)},它具有更灵活和强大的网格表达能力。
\end{itemize}
\end{frame}

\begin{frame}
    \frametitle{FEALPy 网格模块中的网格对象}
\begin{table}[H]
    \tiny
\begin{tabular}[c]{|c|c|}\hline
        IntervalMesh      & 区间网格        \\\hline
		TriangleMesh      & 三角形网格       \\\hline
		QuadrangleMesh    & 四边形网格       \\\hline
		TetrahedronMesh   & 四面体网格       \\\hline
		HexahedronMesh    & 六面体网格       \\\hline
		PolygonMesh       & 多边形网格       \\\hline
		PolyhedronMesh    & 多面体网格       \\\hline
		StructureQuadMesh & 结构四边形网格   \\\hline
		StructureHexMesh  & 结构六面体网格   \\\hline
		Tritree           & 三角形树结构网格 \\\hline
		Quadtree          & 四叉树           \\\hline
		OCtree            & 八叉树           \\\hline
        HalfEdgeMesh2d    & 二维半边网格     \\\hline
        HalfEdgeMesh3d    & 三维半边网格     \\\hline
\end{tabular}
\caption{FEALPy 中的网格类。}
\end{table}
\end{frame}

\begin{frame}[fragile]
    \frametitle{FEALPy 中网格示例:三角形}
\begin{onlyenv}<1>
	\begin{listing}[H]
	 \tiny
     \caption{创建一个三角形网格。}
	 \begin{pythoncode}
import numpy as np
from fealpy.mesh import TriangleMesh

node = np.array([(0.0, 0.0), (1.0, 0.0), (1.0, 1.0), (0.0, 1.0)],dtype=np.float)
cell = np.array([(1, 2, 0), (3, 0, 2)], dtype=np.int)

mesh = TriangleMesh(node, cell)
mesh.uniform_refine(n=3)

node = mesh.entity('node')
edge = mesh.entity('edge')
cell = mesh.entity('cell')

fig = plt.figure()
axes = fig.gca()
mesh.add_plot(axes)
mesh.find_node(axes, showindex=True)
mesh.find_cell(axes, showindex=True)
plt.show()
	 \end{pythoncode}     
	\end{listing}
\end{onlyenv}
\begin{onlyenv}<2>
\begin{figure}[H]
	\begin{minipage}[t]{0.49\linewidth}
	\centering
    \includegraphics[scale=0.35]{./figures/Tri1.png}
    \caption{初始三角形网格。}
	\end{minipage}
	\hfill
	\begin{minipage}[t]{0.49\linewidth}
	\centering
    \includegraphics[scale=0.35]{./figures/Tri2.png}
    \caption{一致加密 2 次三角形网格。}
	\end{minipage}	
\end{figure}
\end{onlyenv}
\end{frame}

\begin{frame}[fragile]
    \frametitle{FEALPy 中网格示例:四边形}
\begin{onlyenv}<1>
	\begin{listing}[H]
	 \tiny
     \caption{创建一个四边形网格。}
	 \begin{pythoncode}
import numpy as np
from fealpy.mesh import QuadrangleMesh

node = np.array([(0,0),(1,0), (1,1),(0,1)],dtype=np.float)
cell = np.array([(0,1,2,3)],dtype = np.int)

mesh = QuadrangleMesh(node,cell)
mesh.uniform_refine(2)

node = mesh.entity('node')
edge = mesh.entity('edge')
cell = mesh.entity('cell')

fig = plt.figure()
axes = fig.gca()
mesh.add_plot(axes)
mesh.find_node(axes, showindex=True)
mesh.find_cell(axes, showindex=True)
plt.show()
	 \end{pythoncode}     
	\end{listing}
\end{onlyenv}
\begin{onlyenv}<2>
\begin{figure}[H]
	\begin{minipage}[t]{0.49\linewidth}
	\centering
    \includegraphics[scale=0.35]{./figures/Q1.png}
    \caption{初始四边形网格}
	\end{minipage}
	\hfill
	\begin{minipage}[t]{0.49\linewidth}
	\centering
    \includegraphics[scale=0.35]{./figures/Q2.png}
    \caption{一致加密2次四边形网格}
	\end{minipage}	
\end{figure}
\end{onlyenv}
\end{frame}

\begin{frame}[fragile]
    \frametitle{FEALPy 中网格示例:四边形}
\begin{onlyenv}<1>
	\begin{listing}[H]
	 \tiny
     \caption{创建一个半边数据结构网格。}
	 \begin{pythoncode}
import numpy as np
import matplotlib.pyplot as plt
from fealpy.mesh import PolygonMesh, HalfEdgeMesh2d 

node = np.array([
    (0.0, 0.0), (0.0, 1.0), (0.0, 2.0),
    (1.0, 0.0), (1.0, 1.0), (1.0, 2.0),
    (2.0, 0.0), (2.0, 1.0), (2.0, 2.0)], dtype=np.float)
cell = np.array([0, 3, 4, 4, 1, 0,
    1, 4, 5, 2, 3, 6, 7, 4, 4, 7, 8, 5], dtype=np.int)
cellLocation = np.array([0, 3, 6, 10, 14, 18], dtype=np.int)

mesh = PolygonMesh(node, cell, cellLocation)
mesh = HalfEdgeMesh2d.from_mesh(mesh)

mesh.uniform_refine(n=3)

node = mesh.entity('node')
edge = mesh.entity('edge')
cell, cellLocation = mesh.entity('cell')
halfedge = mesh.entity('halfedge')

fig = plt.figure()
axes = fig.gca()
mesh.add_plot(axes)
plt.show()
	 \end{pythoncode}     
	\end{listing}
\end{onlyenv}

\begin{onlyenv}<2>
\begin{figure}[H]
	\begin{minipage}[t]{0.49\linewidth}
	\centering
    \includegraphics[scale=0.35]{./figures/triquad.png}
    \caption{初始多边形网格。}
	\end{minipage}
	\hfill
	\begin{minipage}[t]{0.49\linewidth}
	\centering
    \includegraphics[scale=0.35]{./figures/triquadrefine.png}
    \caption{一致加密 3 次多边形网格。}
	\end{minipage}	
\end{figure}
\end{onlyenv}
\end{frame}

\begin{frame}
    \frametitle{FEALPy 中网格对象的命名与接口约定}
\begin{onlyenv}<1>
    \begin{table}[H]
    \scriptsize
    \centering
    \begin{tabular}{|l|l|}\hline
        变量名 & 含义\\\hline
        NN	& 节点的个数\\\hline
        NC	& 单元的个数\\\hline
        NE	& 边的个数\\\hline
        NF	& 面的个数\\\hline
        GD	& 空间维数\\\hline
        TD	& 拓扑维数\\\hline
        node& 节点数组,形状为 (NN, GD)\\\hline
        cell& 单元数组,形状为 (NC, NCV)\\\hline
        edge& 边数组,形状为 (NE, 2)\\\hline
        face& 面数组,形状为 (NF, NFV)\\\hline
        ds	& 网格的拓扑数据结构对象,所有的拓扑关系数据都由其管理和获取\\\hline
    \end{tabular}
    \caption{FEALPy 中网格对象数据成员(属性)的命名约定。}
    \end{table}
\end{onlyenv}

\begin{onlyenv}<2>
    \begin{table}[H]
    \tiny
    \centering
    \begin{tabular}{|l|l|}\hline
        成员函数名 &	功能\\\hline
        mesh.geo\_dimension()&	获得网格的几何维数\\\hline
        mesh.top\_dimension()&	获得网格的拓扑维数\\\hline
        mesh.number\_of\_nodes()&	获得网格的节点个数\\\hline
        mesh.number\_of\_cells()&	获得网格的单元个数\\\hline
        mesh.number\_of\_edges()&	获得网格的边个数\\\hline
        mesh.number\_of\_faces()&	获得网格的面的个数\\\hline
        mesh.number\_of\_entities(etype)&	获得 etype 类型实体的个数\\\hline
        mesh.entity(etype)&	获得 etype 类型的实体\\\hline
        mesh.entity\_measure(etype)	& 获得 etype 类型的实体的测度\\\hline
        mesh.entity\_barycenter(etype)&	获得 etype 类型的实体的重心\\\hline
        mesh.integrator(i)&	获得该网格上的第 i 个积分公式\\\hline
    \end{tabular}
    \caption{网格对象的常用方法成员(属性)列表。表格中 etype 值可以是 0, 1, 2, 3 或者字符串 ‘cell’, ‘node’, ‘edge’, ‘face’。对于
二维网格,etype 的值取 ‘face’ 和 ‘edge’ 是等价的,但不能取 3。}\label{tab:fun}
\end{table}
\end{onlyenv}

\begin{onlyenv}<3>
    \begin{table}[H]
        \tiny
        \centering
        \begin{tabular}{|l|l|}\hline
            成员函数名 & 功能\\\hline
            cell2cell = mesh.ds.cell\_to\_cell(...) &单元与单元的邻接关系\\\hline
            cell2face = mesh.ds.cell\_to\_face(...) &单元与面的邻接关系\\\hline
            cell2edge = mesh.ds.cell\_to\_edge(...) &单元与边的邻接关系\\\hline
            cell2node = mesh.ds.cell\_to\_node(...) &单元与节点的邻接关系\\\hline
            face2cell = mesh.ds.face\_to\_cell(...) &面与单元的邻接关系\\\hline
            face2face = mesh.ds.face\_to\_face(...) &面与面的邻接关系\\\hline
            face2edge = mesh.ds.face\_to\_edge(...) &面与边的邻接关系\\\hline
            face2node = mesh.ds.face\_to\_node(...) &面与节点的邻接关系\\\hline
            edge2cell = mesh.ds.edge\_to\_cell(...) &边与单元的邻接关系\\\hline
            edge2face = mesh.ds.edge\_to\_face(...) &边与面的邻接关系\\\hline
            edge2edge = mesh.ds.edge\_to\_edge(...) &边与边的邻接关系\\\hline
            edge2node = mesh.ds.edge\_to\_node(...) &边与节点的邻接关系\\\hline
            node2cell = mesh.ds.node\_to\_cell(...) &节点与单元的邻接关\\\hline
            node2face = mesh.ds.node\_to\_face(...) &节点与面的邻接关系\\\hline
            node2edge = mesh.ds.node\_to\_edge(...) &节点与边的邻接关系\\\hline
            node2node = mesh.ds.node\_to\_node(...) &节点与节点的邻接关系\\\hline
        \end{tabular}
        \caption{网格拓扑数据成员 ds 的方法成员。} 
    \end{table}
\end{onlyenv}
\begin{onlyenv}<4>
    \begin{table}[H]
    \tiny
    \centering
    \begin{tabular}{|l|l|}\hline
        成员函数名 & 功能\\\hline
        isBdNode = mesh.ds.boundary\_node\_flag() &	一维逻辑数组,标记边界节点\\\hline
        isBdEdge = mesh.ds.boundary\_edge\_flag() &	一维逻辑数组,标记边界边\\\hline
        isBdFace = mesh.ds.boundary\_face\_flag() &	一维逻辑数组,标记边界面\\\hline
        isBdCell = mesh.ds.boundary\_cell\_flag() &	一维逻辑数组,标记边界单元\\\hline
        bdNodeIdx = mesh.ds.boundary\_node\_index() &一维整数数组,边界节点全局编号 \\\hline
        bdEdgeIdx = mesh.ds.boundary\_edge\_index() &一维整数数组,边界边全局编号 \\\hline
        bdFaceIdx = mesh.ds.boundary\_face\_index() &一维整数数组,边界面全局编号 \\\hline
        bdCellIdx = mesh.ds.boundary\_cell\_index() &一维整数数组,边界单元全局编号 \\\hline
    \end{tabular}
    \caption{网格拓扑数据成员 ds 的方法成员。} \label{tab:ds}
\end{table}
\end{onlyenv}
\end{frame}

\section{FEALPy 中的网格生成示例}
\begin{frame}[fragile]
    \frametitle{ MeshFactory 模块}
    FEALPy 的 mesh 模块提供了一个 MeshFactory 的子模块,用于生成各种类型的常见网
    格,方便用户学习和使用 FEALPy。
	\begin{listing}[H]
	 \footnotesize
     \caption{创建一个半边数据结构网格。}
	 \begin{pythoncode}
from fealpy.mesh import MeshFactory

mf = MeshFactory()
	 \end{pythoncode}     
	\end{listing}
\end{frame}

\begin{frame}[fragile]
		\frametitle{基于FEALPy 中的MeshFactory模块生成各种类型的常见网格}
    FEALPy 的 mesh 模块提供了一个 MeshFactory 的子模块,用于生成各种类型的常见网
    格,方便用户学习和使用 FEALPy。
		\begin{listing}[H]
		\scriptsize
		\caption{ MeshFactory 模块的 boxmesh2d}
        \begin{pythoncode}
import matplotlib.pyplot as plt
from mesh import MeshFactory
mf = MeshFactory()
box = [0, 1, 0, 1]
mesh = mf.boxmesh2d(box, nx=4, ny=4, meshtype='tri') #三角形网格
mesh = mf.boxmesh2d(box, nx=4, ny=4, meshtype='quad')#四边形网格
mesh = mf.boxmesh2d(box, nx=4, ny=4, meshtype='poly')#多边形网格 
mesh = mf.special_boxmesh2d(box, n=10, meshtype='fishbone')
mesh = mf.special_boxmesh2d(box, n=10, meshtype='rice')
mesh = mf.special_boxmesh2d(box, n=10, meshtype='cross')
mesh = mf.special_boxmesh2d(box, n=10, meshtype='nonuniform')
mesh = mf.unitcirclemesh(0.1, meshtype='tri')
mesh = mf.unitcirclemesh(0.1, meshtype='poly')
mesh = mf.triangle(box, h=0.1, meshtype='tri')
mesh = mf.triangle(box, h=0.1, meshtype='poly')
		\end{pythoncode}
		\end{listing}
\end{frame}

\end{document}
