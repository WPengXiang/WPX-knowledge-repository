
% !Mode:: "TeX:UTF-8"
\documentclass{beamer}

\usetheme{Darmstadt}
\useinnertheme{rounded}

\usecolortheme{beaver}
%\usecolortheme{albatross}
%\usecolortheme{beetle}
%\usecolortheme{crane}
%\usecolortheme{dolphin}
%\usecolortheme{dove}
%\usecolortheme{fly}
%\usecolortheme{lily}
%\usecolortheme{orchid}
%\usecolortheme{rose}
%\usecolortheme{seagull}
%\usecolortheme{seahorse}
%\usecolortheme{whale}
%\usecolortheme{wolverine}
%\usecolortheme{default}

\usepackage{graphicx}
\usepackage{pythonhighlight}
\setbeamerfont*{frametitle}{size=\normalsize,series=\bfseries}
\setbeamertemplate{navigation symbols}{}


%%%%%%%%------------------------------------------------------------------------
%%%% 日常所用宏包

%% 控制页边距
% 如果是beamer文档类, 则不用geometry
\makeatletter
\@ifclassloaded{beamer}{}{\usepackage[top=2.5cm, bottom=2.5cm, left=2.5cm, right=2.5cm]{geometry}}
\makeatother

\makeatletter
\@ifclassloaded{beamer}{
\makeatletter
\def\th@mystyle{%
    \normalfont % body font
    \setbeamercolor{block title example}{bg=orange,fg=white}
    \setbeamercolor{block body example}{bg=orange!20,fg=black}
    \def\inserttheoremblockenv{exampleblock}
  }
\makeatother
\theoremstyle{mystyle}
\newtheorem*{remark}{Remark}

\newcommand{\propnumber}{} % initialize
\newtheorem*{prop}{Proposition \propnumber}
\newenvironment{propc}[1]
  {\renewcommand{\propnumber}{#1}%
   \begin{shaded}\begin{prop}}
  {\end{prop}\end{shaded}}

\makeatletter
\newenvironment<>{proofs}[1][\proofname]{%
    \par
    \def\insertproofname{#1\@addpunct{.}}%
    \usebeamertemplate{proof begin}#2}
  {\usebeamertemplate{proof end}}
\makeatother

}{
}
\makeatother
\usepackage{amsthm}

%\DeclareMathOperator{\sech}{sech}
%\DeclareMathOperator{\csch}{csch}
%\DeclareMathOperator{\arcsec}{arcsec}
%\DeclareMathOperator{\arccot}{arccot}
%\DeclareMathOperator{\arccsc}{arccsc}
%\DeclareMathOperator{\arccosh}{arccosh}
%\DeclareMathOperator{\arcsinh}{arcsinh}
%\DeclareMathOperator{\arctanh}{arctanh}
%\DeclareMathOperator{\arcsech}{arcsech}
%\DeclareMathOperator{\arccsch}{arccsch}
%\DeclareMathOperator{\arccoth}{arccoth}
%% 控制项目列表
\usepackage{enumerate}

%% Todo list
\usepackage{enumitem}
\newlist{todolist}{itemize}{2}
\setlist[todolist]{label=$\square$}
\usepackage{pifont}
\newcommand{\cmark}{\ding{51}}%
\newcommand{\xmark}{\ding{55}}%
\newcommand{\done}{\rlap{$\square$}{\raisebox{2pt}{\large\hspace{1pt}\cmark}}%
\hspace{-2.5pt}}
\newcommand{\wontfix}{\rlap{$\square$}{\large\hspace{1pt}\xmark}}

\usepackage[utf8]{inputenc}
\usepackage[english]{babel}

\usepackage{framed}

%% 多栏显示
\usepackage{multicol}

%% 算法环境
\usepackage{algorithm}
\usepackage{algorithmic}
\usepackage{float}

%% 网址引用
\usepackage{url}

%% 控制矩阵行距
\renewcommand\arraystretch{1.4}

%% 粗体
\usepackage{lmodern}
\usepackage{bm}


%% hyperref宏包,生成可定位点击的超链接,并且会生成pdf书签
\makeatletter
\@ifclassloaded{beamer}{
\usepackage{hyperref}
\usepackage{ragged2e} % 对齐
}{
\usepackage[%
    pdfstartview=FitH,%
    CJKbookmarks=true,%
    bookmarks=true,%
    bookmarksnumbered=true,%
    bookmarksopen=true,%
    colorlinks=true,%
    citecolor=blue,%
    linkcolor=blue,%
    anchorcolor=green,%
    urlcolor=blue%
]{hyperref}
}
\makeatother



\makeatletter % 如果是 beamer 不需要下面两个包
\@ifclassloaded{beamer}{
\mode<presentation>
{
}
}{
%% 控制标题
\usepackage{titlesec}
%% 控制目录
\usepackage{titletoc}
}
\makeatother

%% 控制表格样式
\usepackage{booktabs}

%% 控制字体大小
\usepackage{type1cm}

%% 首行缩进,用\noindent取消某段缩进
\usepackage{indentfirst}

%% 支持彩色文本、底色、文本框等
\usepackage{color,xcolor}

%% AMS LaTeX宏包: http://zzg34b.w3.c361.com/package/maths.htm#amssymb
\usepackage{amsmath,amssymb}
%% 多个图形并排
\usepackage{subfig}
%%%% 基本插图方法
%% 图形宏包
\usepackage{graphicx}


%%%% 基本插图方法结束

%%%% pgf/tikz绘图宏包设置
\usepackage{pgf,tikz}
\usetikzlibrary{shapes,automata,snakes,backgrounds,arrows}
\usetikzlibrary{mindmap}
%% 可以直接在latex文档中使用graphviz/dot语言,
%% 也可以用dot2tex工具将dot文件转换成tex文件再include进来
%% \usepackage[shell,pgf,outputdir={docgraphs/}]{dot2texi}
%%%% pgf/tikz设置结束


\makeatletter % 如果是 beamer 不需要下面两个包
\@ifclassloaded{beamer}{

}{
%%%% fancyhdr设置页眉页脚
%% 页眉页脚宏包
\usepackage{fancyhdr}
%% 页眉页脚风格
\pagestyle{plain}
}

%% 有时会出现\headheight too small的warning
\setlength{\headheight}{15pt}

%% 清空当前页眉页脚的默认设置
%\fancyhf{}
%%%% fancyhdr设置结束


%% 设置listings宏包的一些全局样式
%% 参考http://hi.baidu.com/shawpinlee/blog/item/9ec431cbae28e41cbe09e6e4.html
\usepackage{listings}
\lstloadlanguages{[LaTeX]TeX}

\usepackage{fancyvrb}

\newenvironment{latexample}[1][language={[LaTeX]TeX}]
{\lstset{breaklines=true,
    prebreak = \raisebox{0ex}[0ex][0ex]{\ensuremath{\hookleftarrow}},
    frame=single,
    language={[LaTeX]TeX},
    showstringspaces=false,              %% 设定是否显示代码之间的空格符号
    numbers=left,                        %% 在左边显示行号
    numberstyle=\tiny,                   %% 设定行号字体的大小
    basicstyle=\scriptsize,                    %% 设定字体大小\tiny, \small, \Large等等
    keywordstyle=\color{blue!70}, commentstyle=\color{red!50!green!50!blue!50},
                                         %% 关键字高亮
    frame=shadowbox,                     %% 给代码加框
    rulesepcolor=\color{red!20!green!20!blue!20},
    escapechar=`,                        %% 中文逃逸字符,用于中英混排
    xleftmargin=2em,xrightmargin=2em, aboveskip=1em,
    %breaklines,                          %% 这条命令可以让LaTeX自动将长的代码行换行排版
    extendedchars=false                  %% 这一条命令可以解决代码跨页时,章节标题,页眉等汉字不显示的问题
    basicstyle=\footnotesize\ttfamily, #1}
  \VerbatimEnvironment\begin{VerbatimOut}{latexample.verb.out}}
  {\end{VerbatimOut}\noindent
  \begin{minipage}{1.05\linewidth}
    \lstinputlisting[]{latexample.verb.out}%
  \end{minipage}\qquad
  \begin{minipage}{1\linewidth}
    \input{latexample.verb.out}
  \end{minipage}\\
}

\usepackage{minted}
\renewcommand{\listingscaption}{Python code} \newminted{python}{
    escapeinside=||,
    mathescape=true,
    numbersep=5pt,
    linenos=true,
    autogobble,
    framesep=3mm}
%%%% listings宏包设置结束


%%%% 附录设置
\makeatletter % 对 beamer 要重新设置
\@ifclassloaded{beamer}{

}{
\usepackage[title,titletoc,header]{appendix}
}
\makeatother
%%%% 附录设置结束





%% 设定行距
\linespread{1}

%% 颜色
\newcommand{\red}{\color{red} }
\newcommand{\blue}{\color{blue} }
\newcommand{\brown}{\color{brown} }
\newcommand{\green}{\color{green} }

\newcommand{\bred}{\bf\color{red} }
\newcommand{\bblue}{\bf\color{blue} }
\newcommand{\bbrown}{\bf\color{brown} }
\newcommand{\bgreen}{\bf\color{green} }
%% 1. 小写的英文或希腊字母表示 标量或标量函数
%% 2. 大写的英文或希腊字母表示 集合或空间
%% 3. 粗体的小写字母代表向量或向量形式的常量和函数
%% 4. 粗体的大写字母代表矩阵或张量形式的常量和函数
%% 5. 空心大写字母代表特殊的空间 \mbR 实数 \mbC 复数 \mbP 多项式
%% 6. 花体的大写字母代表算子

%% 粗体的小写字母代表向量或向量函数
\newcommand{\bfa}{{\boldsymbol a}}
\newcommand{\bfb}{{\boldsymbol b}}
\newcommand{\bfc}{{\boldsymbol c}}
\newcommand{\bfd}{{\boldsymbol d}}
\newcommand{\bfe}{{\boldsymbol e}}
\newcommand{\bff}{{\boldsymbol f}}
\newcommand{\bfg}{{\boldsymbol g}}
\newcommand{\bfh}{{\boldsymbol h}}
\newcommand{\bfi}{{\boldsymbol i}}
\newcommand{\bfj}{{\boldsymbol j}}
\newcommand{\bfk}{{\boldsymbol k}}
\newcommand{\bfl}{{\boldsymbol l}}
\newcommand{\bfm}{{\boldsymbol m}}
\newcommand{\bfn}{{\boldsymbol n}}
\newcommand{\bfo}{{\boldsymbol o}}
\newcommand{\bfp}{{\boldsymbol p}}
\newcommand{\bfq}{{\boldsymbol q}}
\newcommand{\bfr}{{\boldsymbol r}}
\newcommand{\bfs}{{\boldsymbol s}}
\newcommand{\bft}{{\boldsymbol t}}
\newcommand{\bfu}{{\boldsymbol u}}
\newcommand{\bfv}{{\boldsymbol v}}
\newcommand{\bfw}{{\boldsymbol w}}
\newcommand{\bfx}{{\boldsymbol x}}
\newcommand{\bfy}{{\boldsymbol y}}
\newcommand{\bfz}{{\boldsymbol z}}

%  算子
\newcommand{\mca}{{\mathcal a}}
\newcommand{\mcb}{{\mathcal b}}
\newcommand{\mcc}{{\mathcal c}}
\newcommand{\mcd}{{\mathcal d}}
\newcommand{\mce}{{\mathcal e}}
\newcommand{\mcf}{{\mathcal f}}
\newcommand{\mcg}{{\mathcal g}}
\newcommand{\mch}{{\mathcal h}}
\newcommand{\mci}{{\mathcal i}}
\newcommand{\mcj}{{\mathcal j}}
\newcommand{\mck}{{\mathcal k}}
\newcommand{\mcl}{{\mathcal l}}
\newcommand{\mcm}{{\mathcal m}}
\newcommand{\mcn}{{\mathcal n}}
\newcommand{\mco}{{\mathcal o}}
\newcommand{\mcp}{{\mathcal p}}
\newcommand{\mcq}{{\mathcal q}}
\newcommand{\mcr}{{\mathcal r}}
\newcommand{\mcs}{{\mathcal s}}
\newcommand{\mct}{{\mathcal t}}
\newcommand{\mcu}{{\mathcal u}}
\newcommand{\mcv}{{\mathcal v}}
\newcommand{\mcw}{{\mathcal w}}
\newcommand{\mcx}{{\mathcal x}}
\newcommand{\mcy}{{\mathcal y}}
\newcommand{\mcz}{{\mathcal z}}

% \rmd
\newcommand{\mra}{{\mathrm a}}
\newcommand{\mrb}{{\mathrm b}}
\newcommand{\mrc}{{\mathrm c}}
\newcommand{\mrd}{{\mathrm d}}
\newcommand{\mre}{{\mathrm e}}
\newcommand{\mrf}{{\mathrm f}}
\newcommand{\mrg}{{\mathrm g}}
\newcommand{\mrh}{{\mathrm h}}
\newcommand{\mri}{{\mathrm i}}
\newcommand{\mrj}{{\mathrm j}}
\newcommand{\mrk}{{\mathrm k}}
\newcommand{\mrl}{{\mathrm l}}
\newcommand{\mrm}{{\mathrm m}}
\newcommand{\mrn}{{\mathrm n}}
\newcommand{\mro}{{\mathrm o}}
\newcommand{\mrp}{{\mathrm p}}
\newcommand{\mrq}{{\mathrm q}}
\newcommand{\mrr}{{\mathrm r}}
\newcommand{\mrs}{{\mathrm s}}
\newcommand{\mrt}{{\mathrm t}}
\newcommand{\mru}{{\mathrm u}}
\newcommand{\mrv}{{\mathrm v}}
\newcommand{\mrw}{{\mathrm w}}
\newcommand{\mrx}{{\mathrm x}}
\newcommand{\mry}{{\mathrm y}}
\newcommand{\mrz}{{\mathrm z}}

%% 粗体的大写字母一般表示矩阵和张量
\newcommand{\bfA}{{\boldsymbol A}}
\newcommand{\bfB}{{\boldsymbol B}}
\newcommand{\bfC}{{\boldsymbol C}}
\newcommand{\bfD}{{\boldsymbol D}}
\newcommand{\bfE}{{\boldsymbol E}}
\newcommand{\bfF}{{\boldsymbol F}}
\newcommand{\bfG}{{\boldsymbol G}}
\newcommand{\bfH}{{\boldsymbol H}}
\newcommand{\bfI}{{\boldsymbol I}}
\newcommand{\bfJ}{{\boldsymbol J}}
\newcommand{\bfK}{{\boldsymbol K}}
\newcommand{\bfL}{{\boldsymbol L}}
\newcommand{\bfM}{{\boldsymbol M}}
\newcommand{\bfN}{{\boldsymbol N}}
\newcommand{\bfO}{{\boldsymbol O}}
\newcommand{\bfP}{{\boldsymbol P}}
\newcommand{\bfQ}{{\boldsymbol Q}}
\newcommand{\bfR}{{\boldsymbol R}}
\newcommand{\bfS}{{\boldsymbol S}}
\newcommand{\bfT}{{\boldsymbol T}}
\newcommand{\bfU}{{\boldsymbol U}}
\newcommand{\bfV}{{\boldsymbol V}}
\newcommand{\bfW}{{\boldsymbol W}}
\newcommand{\bfX}{{\boldsymbol X}}
\newcommand{\bfY}{{\boldsymbol Y}}
\newcommand{\bfZ}{{\boldsymbol Z}}

%% 花体大写字母
\newcommand{\mcA}{{\mathcal A}}
\newcommand{\mcB}{{\mathcal B}}
\newcommand{\mcC}{{\mathcal C}}
\newcommand{\mcD}{{\mathcal D}}
\newcommand{\mcE}{{\mathcal E}}
\newcommand{\mcF}{{\mathcal F}}
\newcommand{\mcG}{{\mathcal G}}
\newcommand{\mcH}{{\mathcal H}}
\newcommand{\mcI}{{\mathcal I}}
\newcommand{\mcJ}{{\mathcal J}}
\newcommand{\mcK}{{\mathcal K}}
\newcommand{\mcL}{{\mathcal L}}
\newcommand{\mcM}{{\mathcal M}}
\newcommand{\mcN}{{\mathcal N}}
\newcommand{\mcO}{{\mathcal O}}
\newcommand{\mcP}{{\mathcal P}}
\newcommand{\mcQ}{{\mathcal Q}}
\newcommand{\mcR}{{\mathcal R}}
\newcommand{\mcS}{{\mathcal S}}
\newcommand{\mcT}{{\mathcal T}}
\newcommand{\mcU}{{\mathcal U}}
\newcommand{\mcV}{{\mathcal V}}
\newcommand{\mcW}{{\mathcal W}}
\newcommand{\mcX}{{\mathcal X}}
\newcommand{\mcY}{{\mathcal Y}}
\newcommand{\mcZ}{{\mathcal Z}}

%% 空心大写字母
\newcommand{\mbA}{{\mathbb A}}
\newcommand{\mbB}{{\mathbb B}}
\newcommand{\mbC}{{\mathbb C}}
\newcommand{\mbD}{{\mathbb D}}
\newcommand{\mbE}{{\mathbb E}}
\newcommand{\mbF}{{\mathbb F}}
\newcommand{\mbG}{{\mathbb G}}
\newcommand{\mbH}{{\mathbb H}}
\newcommand{\mbI}{{\mathbb I}}
\newcommand{\mbJ}{{\mathbb J}}
\newcommand{\mbK}{{\mathbb K}}
\newcommand{\mbL}{{\mathbb L}}
\newcommand{\mbM}{{\mathbb M}}
\newcommand{\mbN}{{\mathbb N}}
\newcommand{\mbO}{{\mathbb O}}
\newcommand{\mbP}{{\mathbb P}}
\newcommand{\mbQ}{{\mathbb Q}}
\newcommand{\mbR}{{\mathbb R}}
\newcommand{\mbS}{{\mathbb S}}
\newcommand{\mbT}{{\mathbb T}}
\newcommand{\mbU}{{\mathbb U}}
\newcommand{\mbV}{{\mathbb V}}
\newcommand{\mbW}{{\mathbb W}}
\newcommand{\mbX}{{\mathbb X}}
\newcommand{\mbY}{{\mathbb Y}}
\newcommand{\mbZ}{{\mathbb Z}}

\newcommand{\mrA}{{\mathrm A}}
\newcommand{\mrB}{{\mathrm B}}
\newcommand{\mrC}{{\mathrm C}}
\newcommand{\mrD}{{\mathrm D}}
\newcommand{\mrE}{{\mathrm E}}
\newcommand{\mrF}{{\mathrm F}}
\newcommand{\mrG}{{\mathrm G}}
\newcommand{\mrH}{{\mathrm H}}
\newcommand{\mrI}{{\mathrm I}}
\newcommand{\mrJ}{{\mathrm J}}
\newcommand{\mrK}{{\mathrm K}}
\newcommand{\mrL}{{\mathrm L}}
\newcommand{\mrM}{{\mathrm M}}
\newcommand{\mrN}{{\mathrm N}}
\newcommand{\mrO}{{\mathrm O}}
\newcommand{\mrP}{{\mathrm P}}
\newcommand{\mrQ}{{\mathrm Q}}
\newcommand{\mrR}{{\mathrm R}}
\newcommand{\mrS}{{\mathrm S}}
\newcommand{\mrT}{{\mathrm T}}
\newcommand{\mrU}{{\mathrm U}}
\newcommand{\mrV}{{\mathrm V}}
\newcommand{\mrW}{{\mathrm W}}
\newcommand{\mrX}{{\mathrm X}}
\newcommand{\mrY}{{\mathrm Y}}
\newcommand{\mrZ}{{\mathrm Z}}


% 粗体的 Greek 字母
\newcommand{\balpha}{{\bm \alpha}}
\newcommand{\bbeta}{{\bm \beta}}
\newcommand{\bgamma}{{\bm \gamma}}
\newcommand{\bdelta}{{\bm \delta}}
\newcommand{\bepsilon}{{\bm \epsilon}}
\newcommand{\bvarepsilon}{{\bm \varepsilon}}
\newcommand{\bzeta}{{\bm \zeta}}
\newcommand{\bfeta}{{\bm \eta}}
\newcommand{\btheta}{{\bm \theta}}
\newcommand{\biota}{{\bm \iota}}
\newcommand{\bkappa}{{\bm \kappa}}
\newcommand{\blambda}{{\bm \lambda}}
\newcommand{\bmu}{{\bm \mu}}
\newcommand{\bnu}{{\bm \nu}}
\newcommand{\bxi}{{\bm \xi}}
\newcommand{\bomicron}{{\bm \omicron}}
\newcommand{\bpi}{{\bm \pi}}
\newcommand{\brho}{{\bm \rho}}
\newcommand{\bsigma}{{\bm \sigma}}
\newcommand{\btau}{{\bm \tau}}
\newcommand{\bupsilon}{{\bm \upsilon}}
\newcommand{\bphi}{{\bm \phi}}
\newcommand{\bvarphi}{{\bm \varphi}}
\newcommand{\bchi}{{\bm \chi}}
\newcommand{\bpsi}{{\bm \psi}}

\newcommand{\bAlpha}{{\bm \Alpha}}
\newcommand{\bBeta}{{\bm \Beta}}
\newcommand{\bGamma}{{\bm \Gamma}}
\newcommand{\bDelta}{{\bm \Delta}}
\newcommand{\bEpsilon}{{\bm \Psilon}}
\newcommand{\bVarepsilon}{{\bm \Varepsilon}}
\newcommand{\bZeta}{{\bm \Zeta}}
\newcommand{\bEta}{{\bm \Eta}}
\newcommand{\bTheta}{{\bm \Theta}}
\newcommand{\bIota}{{\bm \Iota}}
\newcommand{\bKappa}{{\bm \Kappa}}
\newcommand{\bLambda}{{\bm \Lambda}}
\newcommand{\bMu}{{\bm \Mu}}
\newcommand{\bNu}{{\bm \Nu}}
\newcommand{\bXi}{{\bm \Xi}}
\newcommand{\bOmicron}{{\bm \Omicron}}
\newcommand{\bPi}{{\bm \Pi}}
\newcommand{\bRho}{{\bm \Rho}}
\newcommand{\bSigma}{{\bm \Sigma}}
\newcommand{\bTau}{{\bm \Tau}}
\newcommand{\bUpsilon}{{\bm \Upsilon}}
\newcommand{\bPhi}{{\bm \Phi}}
\newcommand{\bChi}{{\bm \Chi}}
\newcommand{\bPsi}{{\bm \Psi}}

% \int_\Omega \bfx^2 \rmd \bfx
\newcommand{\rmd}{\,\mathrm d}
\newcommand{\bfzero}{\mathbf 0}

%% 算子
\newcommand{\ospan}{\operatorname{span}}
\newcommand{\odiv}{\operatorname{div}}
\newcommand{\otr}{\operatorname{tr}}
\newcommand{\ograd}{\operatorname{grad}}
\newcommand{\orot}{\operatorname{rot}}
\newcommand{\ocurl}{\operatorname{curl}}
\newcommand{\odist}{\operatorname{dist}}
\newcommand{\osign}{\operatorname{sign}}
\newcommand{\odiag}{\operatorname{diag}}
\newcommand{\oran}{\operatorname{Ran}} % 像空间
\newcommand{\oker}{\operatorname{Ker}} % 核空间
\newcommand{\ore}{\operatorname{Re}} % 实部
\newcommand{\oim}{\operatorname{Im}} % 虚部
\newcommand{\orank}{\operatorname{rank}}
\newcommand{\ovec}{\operatorname{vec}}
\newcommand{\odet}{\operatorname{det}}
\newcommand{\odim}{\operatorname{dim}}
\newcommand{\osym}{\operatorname{sym}}

\newcommand{\obcurl}{\operatorname{\bf curl}}
%%%% 个性设置结束
%%%%%%%%------------------------------------------------------------------------


%%%%%%%%------------------------------------------------------------------------
%%%% bibtex设置

%% 设定参考文献显示风格
% 下面是几种常见的样式
% * plain: 按字母的顺序排列,比较次序为作者、年度和标题
% * unsrt: 样式同plain,只是按照引用的先后排序
% * alpha: 用作者名首字母+年份后两位作标号,以字母顺序排序
% * abbrv: 类似plain,将月份全拼改为缩写,更显紧凑
% * apalike: 美国心理学学会期刊样式, 引用样式 [Tailper and Zang, 2006]

%\makeatletter
%\@ifclassloaded{beamer}{
%\bibliographystyle{apalike}
%}{
%\bibliographystyle{abbrv}
%}
%\makeatother


%%%% bibtex设置结束
%%%%%%%%------------------------------------------------------------------------

%%%%%%%%------------------------------------------------------------------------
%%%% xeCJK相关宏包

\usepackage{xltxtra, fontenc, xunicode}
\usepackage[slantfont, boldfont]{xeCJK}

\setlength{\parindent}{1.5em}%中文缩进两个汉字位

%% 针对中文进行断行
\XeTeXlinebreaklocale "zh"

%% 给予TeX断行一定自由度
\XeTeXlinebreakskip = 0pt plus 1pt minus 0.1pt

%%%% xeCJK设置结束
%%%%%%%%------------------------------------------------------------------------

%%%%%%%%------------------------------------------------------------------------
%%%% xeCJK字体设置

%% 设置中文标点样式,支持quanjiao、banjiao、kaiming等多种方式
\punctstyle{kaiming}

%% 设置缺省中文字体
\setCJKmainfont[BoldFont={Adobe Heiti Std}, ItalicFont={Adobe Kaiti Std}]{Adobe Song Std}
%\setCJKmainfont{Adobe Kaiti Std}
%% 设置中文无衬线字体
%\setCJKsansfont[BoldFont={Adobe Heiti Std}]{Adobe Kaiti Std}
%% 设置等宽字体
%\setCJKmonofont{Adobe Heiti Std}

%% 英文衬线字体
\setmainfont{DejaVu Serif}
%% 英文等宽字体
\setmonofont{DejaVu Sans Mono}
%% 英文无衬线字体
\setsansfont{DejaVu Sans}

%% 定义新字体
\setCJKfamilyfont{song}{Adobe Song Std}
\setCJKfamilyfont{kai}{Adobe Kaiti Std}
\setCJKfamilyfont{hei}{Adobe Heiti Std}
\setCJKfamilyfont{fangsong}{Adobe Fangsong Std}
\setCJKfamilyfont{lisu}{LiSu}
\setCJKfamilyfont{youyuan}{YouYuan}

%% 自定义宋体
\newcommand{\song}{\CJKfamily{song}}
%% 自定义楷体
\newcommand{\kai}{\CJKfamily{kai}}
%% 自定义黑体
\newcommand{\hei}{\CJKfamily{hei}}
%% 自定义仿宋体
\newcommand{\fangsong}{\CJKfamily{fangsong}}
%% 自定义隶书
\newcommand{\lisu}{\CJKfamily{lisu}}
%% 自定义幼圆
\newcommand{\youyuan}{\CJKfamily{youyuan}}

%%%% xeCJK字体设置结束
%%%%%%%%------------------------------------------------------------------------

%%%%%%%%------------------------------------------------------------------------
%%%% 一些关于中文文档的重定义
\newcommand{\chntoday}{\number\year\,年\,\number\month\,月\,\number\day\,日}
%% 数学公式定理的重定义

%% 中文破折号,据说来自清华模板
\newcommand{\pozhehao}{\kern0.3ex\rule[0.8ex]{2em}{0.1ex}\kern0.3ex}

\makeatletter %
\@ifclassloaded{beamer}{

}{
\newtheorem{example}{例}
\newtheorem{theorem}{定理}[section]
\newtheorem{definition}{定义}
\newtheorem{axiom}{公理}
\newtheorem{property}{性质}
\newtheorem{proposition}{命题}
\newtheorem{lemma}{引理}
\newtheorem{corollary}{推论}
\newtheorem{remark}{注解}
\newtheorem{condition}{条件}
\newtheorem{conclusion}{结论}
\newtheorem{assumption}{假设}
}
\makeatother

\makeatletter %
\@ifclassloaded{beamer}{

}{
%% 章节等名称重定义
\renewcommand{\contentsname}{目录}
\renewcommand{\indexname}{索引}
\renewcommand{\listfigurename}{插图目录}
\renewcommand{\listtablename}{表格目录}
\renewcommand{\appendixname}{附录}
\renewcommand{\appendixpagename}{附录}
\renewcommand{\appendixtocname}{附录}
\@ifclassloaded{book}{

}{
\renewcommand{\abstractname}{摘要}
}
}
\makeatother

\renewcommand{\figurename}{图}
\renewcommand{\tablename}{表}

\makeatletter
\@ifclassloaded{book}{
\renewcommand{\bibname}{参考文献}
}{
\renewcommand{\refname}{参考文献}
}
\makeatother

\floatname{algorithm}{算法}
\renewcommand{\algorithmicrequire}{\textbf{输入:}}
\renewcommand{\algorithmicensure}{\textbf{输出:}}

\renewcommand{\today}{\number\year 年 \number\month 月 \number\day 日}

%%%% 中文重定义结束
%%%%%%%%------------------------------------------------------------------------


\usepackage{biblatex}
\addbibresource{ref.bib}

\usefonttheme[onlymath]{serif}
\numberwithin{subsection}{section}
%\usefonttheme[onlylarge]{structurebold}
\setbeamercovered{transparent}

\title{FEALPy 偏微分方程数值解程序设计与实现: {\bf 网格数据结构}}
\author{魏华祎 \quad 李成新}
\institute[XTU]{
weihuayi@xtu.edu.cn\\
\vspace{5pt}
湘潭大学$\bullet$数学与计算科学学院\\
}
 
\date[XTU]
{
    \today
}

 

\AtBeginSection[]
{
  \frame<beamer>{ 
    \frametitle{Outline}   
    \tableofcontents[currentsection] 
  }
}

\AtBeginSubsection[]
{
  \frame<beamer>{ 
    \frametitle{Outline}   
    \tableofcontents[currentsection] 
  }
}

\begin{document}
\begin{frame}
  \titlepage
\end{frame}

\begin{frame}{Outline}
  \tableofcontents
\end{frame}


\section{FEALPy 中的网格模块}

\begin{frame}[fragile]{FEALPy 中的网格模块}
    在偏微分方程数值计算程序设计中,网格是最核心的数据结构,是下一步实现数值离散方法
    的基础。FEALPy 中核心网格数据结构是用{\bf 数组}表示。
\begin{itemize}
    \item[$\bullet$] 三角形、四边形、四面体和六面体等网格,因为每个单元顶点的个
        数固定,因此可以用{\bf 节点坐标数组} node 和{\bf 单元拓扑数组} cell 来表
        示,这是一种以{\bf 单元为中心的数据结构}。
    \item[$\bullet$] 其它的如{\bf 边数组} edge、{\bf 面数组} face 都可由 cell 生
        成。
    \item[$\bullet$] FEALPy 中把 node、edge、face 和 cell 统称为网格中的实体 entity。
    \item[$\bullet$] 在二维情形下,FEALPy 中的 edge 和 face 意义是相同的。
    \item[$\bullet$] FEALPy 中还有一种以{\bf 边中心的网格数据结构}, 称为{\bf 半边数
    据结构(Half-Edge data structure)},它具有更灵活和强大的网格表达能力。
\end{itemize}
\end{frame}

\begin{frame}
    \frametitle{FEALPy 网格模块中的网格对象}
\begin{table}[H]
    \tiny
\begin{tabular}[c]{|c|c|}\hline
        IntervalMesh      & 区间网格        \\\hline
		TriangleMesh      & 三角形网格       \\\hline
		QuadrangleMesh    & 四边形网格       \\\hline
		TetrahedronMesh   & 四面体网格       \\\hline
		HexahedronMesh    & 六面体网格       \\\hline
		PolygonMesh       & 多边形网格       \\\hline
		PolyhedronMesh    & 多面体网格       \\\hline
		StructureQuadMesh & 结构四边形网格   \\\hline
		StructureHexMesh  & 结构六面体网格   \\\hline
		Tritree           & 三角形树结构网格 \\\hline
		Quadtree          & 四叉树           \\\hline
		OCtree            & 八叉树           \\\hline
        HalfEdgeMesh2d    & 二维半边网格     \\\hline
        HalfEdgeMesh3d    & 三维半边网格     \\\hline
\end{tabular}
\caption{FEALPy 中的网格类。}
\end{table}
\end{frame}

\begin{frame}[fragile]
    \frametitle{FEALPy 中网格示例:三角形}
\begin{onlyenv}<1>
	\begin{listing}[H]
	 \tiny
     \caption{创建一个三角形网格。}
	 \begin{pythoncode}
import numpy as np
from fealpy.mesh import TriangleMesh

node = np.array([(0.0, 0.0), (1.0, 0.0), (1.0, 1.0), (0.0, 1.0)],dtype=np.float)
cell = np.array([(1, 2, 0), (3, 0, 2)], dtype=np.int)

mesh = TriangleMesh(node, cell)
mesh.uniform_refine(n=3)

node = mesh.entity('node')
edge = mesh.entity('edge')
cell = mesh.entity('cell')

fig = plt.figure()
axes = fig.gca()
mesh.add_plot(axes)
mesh.find_node(axes, showindex=True)
mesh.find_cell(axes, showindex=True)
plt.show()
	 \end{pythoncode}     
	\end{listing}
\end{onlyenv}
\begin{onlyenv}<2>
\begin{figure}[H]
	\begin{minipage}[t]{0.49\linewidth}
	\centering
    \includegraphics[scale=0.35]{./figures/Tri1.png}
    \caption{初始三角形网格。}
	\end{minipage}
	\hfill
	\begin{minipage}[t]{0.49\linewidth}
	\centering
    \includegraphics[scale=0.35]{./figures/Tri2.png}
    \caption{一致加密 2 次三角形网格。}
	\end{minipage}	
\end{figure}
\end{onlyenv}
\end{frame}

\begin{frame}[fragile]
    \frametitle{FEALPy 中网格示例:四边形}
\begin{onlyenv}<1>
	\begin{listing}[H]
	 \tiny
     \caption{创建一个四边形网格。}
	 \begin{pythoncode}
import numpy as np
from fealpy.mesh import QuadrangleMesh

node = np.array([(0,0),(1,0), (1,1),(0,1)],dtype=np.float)
cell = np.array([(0,1,2,3)],dtype = np.int)

mesh = QuadrangleMesh(node,cell)
mesh.uniform_refine(2)

node = mesh.entity('node')
edge = mesh.entity('edge')
cell = mesh.entity('cell')

fig = plt.figure()
axes = fig.gca()
mesh.add_plot(axes)
mesh.find_node(axes, showindex=True)
mesh.find_cell(axes, showindex=True)
plt.show()
	 \end{pythoncode}     
	\end{listing}
\end{onlyenv}
\begin{onlyenv}<2>
\begin{figure}[H]
	\begin{minipage}[t]{0.49\linewidth}
	\centering
    \includegraphics[scale=0.35]{./figures/Q1.png}
    \caption{初始四边形网格}
	\end{minipage}
	\hfill
	\begin{minipage}[t]{0.49\linewidth}
	\centering
    \includegraphics[scale=0.35]{./figures/Q2.png}
    \caption{一致加密2次四边形网格}
	\end{minipage}	
\end{figure}
\end{onlyenv}
\end{frame}

\begin{frame}[fragile]
    \frametitle{FEALPy 中网格示例:四边形}
\begin{onlyenv}<1>
	\begin{listing}[H]
	 \tiny
     \caption{创建一个半边数据结构网格。}
	 \begin{pythoncode}
import numpy as np
import matplotlib.pyplot as plt
from fealpy.mesh import PolygonMesh, HalfEdgeMesh2d 

node = np.array([
    (0.0, 0.0), (0.0, 1.0), (0.0, 2.0),
    (1.0, 0.0), (1.0, 1.0), (1.0, 2.0),
    (2.0, 0.0), (2.0, 1.0), (2.0, 2.0)], dtype=np.float)
cell = np.array([0, 3, 4, 4, 1, 0,
    1, 4, 5, 2, 3, 6, 7, 4, 4, 7, 8, 5], dtype=np.int)
cellLocation = np.array([0, 3, 6, 10, 14, 18], dtype=np.int)

mesh = PolygonMesh(node, cell, cellLocation)
mesh = HalfEdgeMesh2d.from_mesh(mesh)

mesh.uniform_refine(n=3)

node = mesh.entity('node')
edge = mesh.entity('edge')
cell, cellLocation = mesh.entity('cell')
halfedge = mesh.entity('halfedge')

fig = plt.figure()
axes = fig.gca()
mesh.add_plot(axes)
plt.show()
	 \end{pythoncode}     
	\end{listing}
\end{onlyenv}

\begin{onlyenv}<2>
\begin{figure}[H]
	\begin{minipage}[t]{0.49\linewidth}
	\centering
    \includegraphics[scale=0.35]{./figures/triquad.png}
    \caption{初始多边形网格。}
	\end{minipage}
	\hfill
	\begin{minipage}[t]{0.49\linewidth}
	\centering
    \includegraphics[scale=0.35]{./figures/triquadrefine.png}
    \caption{一致加密 3 次多边形网格。}
	\end{minipage}	
\end{figure}
\end{onlyenv}
\end{frame}

\begin{frame}
    \frametitle{FEALPy 中网格对象的命名与接口约定}
\begin{onlyenv}<1>
    \begin{table}[H]
    \scriptsize
    \centering
    \begin{tabular}{|l|l|}\hline
        变量名 & 含义\\\hline
        NN	& 节点的个数\\\hline
        NC	& 单元的个数\\\hline
        NE	& 边的个数\\\hline
        NF	& 面的个数\\\hline
        GD	& 空间维数\\\hline
        TD	& 拓扑维数\\\hline
        node& 节点数组,形状为 (NN, GD)\\\hline
        cell& 单元数组,形状为 (NC, NCV)\\\hline
        edge& 边数组,形状为 (NE, 2)\\\hline
        face& 面数组,形状为 (NF, NFV)\\\hline
        ds	& 网格的拓扑数据结构对象,所有的拓扑关系数据都由其管理和获取\\\hline
    \end{tabular}
    \caption{FEALPy 中网格对象数据成员(属性)的命名约定。}
    \end{table}
\end{onlyenv}

\begin{onlyenv}<2>
    \begin{table}[H]
    \tiny
    \centering
    \begin{tabular}{|l|l|}\hline
        成员函数名 &	功能\\\hline
        mesh.geo\_dimension()&	获得网格的几何维数\\\hline
        mesh.top\_dimension()&	获得网格的拓扑维数\\\hline
        mesh.number\_of\_nodes()&	获得网格的节点个数\\\hline
        mesh.number\_of\_cells()&	获得网格的单元个数\\\hline
        mesh.number\_of\_edges()&	获得网格的边个数\\\hline
        mesh.number\_of\_faces()&	获得网格的面的个数\\\hline
        mesh.number\_of\_entities(etype)&	获得 etype 类型实体的个数\\\hline
        mesh.entity(etype)&	获得 etype 类型的实体\\\hline
        mesh.entity\_measure(etype)	& 获得 etype 类型的实体的测度\\\hline
        mesh.entity\_barycenter(etype)&	获得 etype 类型的实体的重心\\\hline
        mesh.integrator(i)&	获得该网格上的第 i 个积分公式\\\hline
    \end{tabular}
    \caption{网格对象的常用方法成员(属性)列表。表格中 etype 值可以是 0, 1, 2, 3 或者字符串 ‘cell’, ‘node’, ‘edge’, ‘face’。对于
二维网格,etype 的值取 ‘face’ 和 ‘edge’ 是等价的,但不能取 3。}\label{tab:fun}
\end{table}
\end{onlyenv}

\begin{onlyenv}<3>
    \begin{table}[H]
        \tiny
        \centering
        \begin{tabular}{|l|l|}\hline
            成员函数名 & 功能\\\hline
            cell2cell = mesh.ds.cell\_to\_cell(...) &单元与单元的邻接关系\\\hline
            cell2face = mesh.ds.cell\_to\_face(...) &单元与面的邻接关系\\\hline
            cell2edge = mesh.ds.cell\_to\_edge(...) &单元与边的邻接关系\\\hline
            cell2node = mesh.ds.cell\_to\_node(...) &单元与节点的邻接关系\\\hline
            face2cell = mesh.ds.face\_to\_cell(...) &面与单元的邻接关系\\\hline
            face2face = mesh.ds.face\_to\_face(...) &面与面的邻接关系\\\hline
            face2edge = mesh.ds.face\_to\_edge(...) &面与边的邻接关系\\\hline
            face2node = mesh.ds.face\_to\_node(...) &面与节点的邻接关系\\\hline
            edge2cell = mesh.ds.edge\_to\_cell(...) &边与单元的邻接关系\\\hline
            edge2face = mesh.ds.edge\_to\_face(...) &边与面的邻接关系\\\hline
            edge2edge = mesh.ds.edge\_to\_edge(...) &边与边的邻接关系\\\hline
            edge2node = mesh.ds.edge\_to\_node(...) &边与节点的邻接关系\\\hline
            node2cell = mesh.ds.node\_to\_cell(...) &节点与单元的邻接关\\\hline
            node2face = mesh.ds.node\_to\_face(...) &节点与面的邻接关系\\\hline
            node2edge = mesh.ds.node\_to\_edge(...) &节点与边的邻接关系\\\hline
            node2node = mesh.ds.node\_to\_node(...) &节点与节点的邻接关系\\\hline
        \end{tabular}
        \caption{网格拓扑数据成员 ds 的方法成员。} 
    \end{table}
\end{onlyenv}
\begin{onlyenv}<4>
    \begin{table}[H]
    \tiny
    \centering
    \begin{tabular}{|l|l|}\hline
        成员函数名 & 功能\\\hline
        isBdNode = mesh.ds.boundary\_node\_flag() &	一维逻辑数组,标记边界节点\\\hline
        isBdEdge = mesh.ds.boundary\_edge\_flag() &	一维逻辑数组,标记边界边\\\hline
        isBdFace = mesh.ds.boundary\_face\_flag() &	一维逻辑数组,标记边界面\\\hline
        isBdCell = mesh.ds.boundary\_cell\_flag() &	一维逻辑数组,标记边界单元\\\hline
        bdNodeIdx = mesh.ds.boundary\_node\_index() &一维整数数组,边界节点全局编号 \\\hline
        bdEdgeIdx = mesh.ds.boundary\_edge\_index() &一维整数数组,边界边全局编号 \\\hline
        bdFaceIdx = mesh.ds.boundary\_face\_index() &一维整数数组,边界面全局编号 \\\hline
        bdCellIdx = mesh.ds.boundary\_cell\_index() &一维整数数组,边界单元全局编号 \\\hline
    \end{tabular}
    \caption{网格拓扑数据成员 ds 的方法成员。} \label{tab:ds}
\end{table}
\end{onlyenv}
\end{frame}

\section{FEALPy 中的网格生成示例}
\begin{frame}[fragile]
    \frametitle{ MeshFactory 模块}
    FEALPy 的 mesh 模块提供了一个 MeshFactory 的子模块,用于生成各种类型的常见网
    格,方便用户学习和使用 FEALPy。
	\begin{listing}[H]
	 \footnotesize
     \caption{创建一个半边数据结构网格。}
	 \begin{pythoncode}
from fealpy.mesh import MeshFactory

mf = MeshFactory()
	 \end{pythoncode}     
	\end{listing}
\end{frame}

\begin{frame}[fragile]
		\frametitle{基于FEALPy 中的MeshFactory模块生成各种类型的常见网格}
    FEALPy 的 mesh 模块提供了一个 MeshFactory 的子模块,用于生成各种类型的常见网
    格,方便用户学习和使用 FEALPy。
		\begin{listing}[H]
		\scriptsize
		\caption{ MeshFactory 模块的 boxmesh2d}
        \begin{pythoncode}
import matplotlib.pyplot as plt
from mesh import MeshFactory
mf = MeshFactory()
box = [0, 1, 0, 1]
mesh = mf.boxmesh2d(box, nx=4, ny=4, meshtype='tri') #三角形网格
mesh = mf.boxmesh2d(box, nx=4, ny=4, meshtype='quad')#四边形网格
mesh = mf.boxmesh2d(box, nx=4, ny=4, meshtype='poly')#多边形网格 
mesh = mf.special_boxmesh2d(box, n=10, meshtype='fishbone')
mesh = mf.special_boxmesh2d(box, n=10, meshtype='rice')
mesh = mf.special_boxmesh2d(box, n=10, meshtype='cross')
mesh = mf.special_boxmesh2d(box, n=10, meshtype='nonuniform')
mesh = mf.unitcirclemesh(0.1, meshtype='tri')
mesh = mf.unitcirclemesh(0.1, meshtype='poly')
mesh = mf.triangle(box, h=0.1, meshtype='tri')
mesh = mf.triangle(box, h=0.1, meshtype='poly')
		\end{pythoncode}
		\end{listing}
\end{frame}

\end{document}
