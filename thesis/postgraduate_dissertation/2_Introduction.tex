\pagestyle{plain}
\pagenumbering{arabic}    %设置页码格式为arabic,并把页计数器置为1.
\newpage
\chapter{绪~论}\label{Chap:Introduction}

\section{有限元方法}
有限元法($Finite\ Elements\ Method$)是力学,计算数学,计算机技术等多种学科综合发展和结合的产物。
在大多数工程研究领域,由于科学理论和科学实验的局限性,很多时候无法求得问题的解析解,因此科学计算成
为一种重要的研究手段,而有限元法是进行科学计算的极为重要的方法之一。

有限元法的实质是将复杂的连续体划分为有限多个简单的单元体,化无限自由度问题为有限自由度问题。在求解
过程中,通常以位移为基本变量,使用虚位移原理或最小势能原理来求解。其求解问题的一般步骤为
\begin{enumerate}
	\item 建立与实际问题相适应的数学模型(微分方程及边值条件);
	\item 寻找与数学模型相适应的变分问题;
	\item 通过伽辽金方法$(Galerkin\ method)$方法建立有限维空间,选择有限元类型和相应的形函数;
	\item 相关有限元矩阵的组装比如单元刚度矩阵、总刚度矩阵、载荷向量等;
	\item 边界条件的处理和有限元方程组求解;
	\item 回到实际问题中去(即解释实际问题)
\end{enumerate}

其中伽辽金方法里有限维空间的构造和选取是整个有限元算法核心,影响了整个算法的复杂度和精确性,在其它常见的微分方程数值解法中,有限维度空间的构造都有着重要的地位,比如有限体积法、谱方法、虚单元法,弱$Galerkin$法。

伽辽金方法是由俄罗斯数学家鲍里斯·格里戈里耶维奇·伽辽金(俄文:Борис Григорьевич Галёркин 英文:Boris Galerkin)发明的一种数值分析方法。应用这种方法可以将求解微分方程问题(通过方程所对应泛函的变分原理)简化成为线性方程组的求解问题。

我们以具有D氏边界条件的possion方程
\begin{eqnarray}
	-\Delta u(\mathbf x) &=& f(\mathbf x),\text{ on } \Omega.\label{eq:P}\\
	u|_{\partial\Omega} &=& 0.
\end{eqnarray}
为例来演示该方法的使用

我们首先建立其变分问题,对方程\eqref{eq:P}两边同时乘以$v \in H^1_0(\Omega) := \{v\in H^1(\Omega), v|_{\partial\Omega} = 0\}$,然后做分部积分可得:
\begin{align*}
	f(u,v) &\triangleq  \int_{\Omega} f v \mathrm{d}x = \int_{\Omega}- \Delta u v \mathrm{d}x  \\ 
	&=-v\nabla u \big|_{\partial \Omega} + \int_{\Omega} \nabla u\cdot\nabla v\mathrm{d}x \\
	&=\int_{\Omega} \nabla u\cdot\nabla v\mathrm{d}x\triangleq a(u,v)
\end{align*}

记$V \triangleq \{v \big| v,Dv \in L^2(\Omega) , a(u,v) < + \inf , v|_{\partial\Omega} = 0\}$
则问题\eqref{eq:P}的解可以有下列问题所描述:
\begin{equation}\label{eq:W}
	\text{求} u \in V, \text{使得}\quad f(u,v)=a(u,v) , \forall v \in V 
\end{equation}
上述\eqref{eq:W}称为\eqref{eq:P}的变分问题或弱形式

下面我们考虑 Galerkin 方法来求 \eqref{eq:W} 的逼近解。由于电脑无法处理无限维空间,因此我们需要构造有限维空间 $V=\mathrm{span}\{\phi_0,\phi_1,\ldots,
\phi_{N-1}\}$ 来逼近上面的无限维空间$H_0^1(\Omega)$。

将真解$u$用逼近的空间$V$里的基函数表示

\begin{equation}\label{eq:b}
	\tilde u= \sum_{i=0}^{N-1}u_i \phi_i
\end{equation}

因此我们只需要求出其系数向量$\bfu = (u_0,u_1,\cdots,n_{N-1})$,其满足
\begin{equation}
	\label{eq:d}
	a(\tilde u, v) = (f, v),
	\quad\forall~v\in V.
\end{equation}

方程 \eqref{eq:d} 中仍然包含有无穷多个方程。 但 $V$ 是一个有限维空间,因此只需要对所有基函数满足 \eqref{eq:d} 即可
$$
\begin{cases}
	a(\tilde u, \phi_0) = (f, \phi_0) \\
	a(\tilde u, \phi_1) = (f, \phi_1) \\
	\vdots \\
	a(\tilde u, \phi_{N-1}) = (f, \phi_{N-1})
\end{cases}
$$
将公式 \eqref{eq:b} 代入得到
$$
\begin{cases}
	a(\sum_{i=0}^{N-1}u_i \phi_i, \phi_0) = (f, \phi_0) \\
	a(\sum_{i=0}^{N-1}u_i \phi_i, \phi_1) = (f, \phi_1) \\
	\vdots \\
	a(\sum_{i=0}^{N-1}u_i \phi_i, \phi_{N-1}) = (f, \phi_{N-1})
\end{cases}
$$
上面的方程可以改写为下面的形式:
$$
\begin{pmatrix}
	a(\phi_0, \phi_0) & a(\phi_1, \phi_0) & \cdots & a(\phi_{N-1}, \phi_0) \\
	a(\phi_0, \phi_1) & a(\phi_1, \phi_1) & \cdots & a(\phi_{N-1}, \phi_1) \\
	\vdots & \vdots & \ddots & \vdots \\
	a(\phi_0, \phi_{N-1}) & a(\phi_1, \phi_{N-1}) & \cdots & a(\phi_{N-1},
	\phi_{N-1}) \\
\end{pmatrix}
\begin{pmatrix}
	u_0 \\ u_1 \\ \vdots \\ u_{N-1}
\end{pmatrix}
=
\begin{pmatrix}
	(f, \phi_0) \\ (f, \phi_1) \\ \vdots \\ (f, \phi_{N-1})
\end{pmatrix}
$$
上面的左端矩阵称为{\bf 刚度矩阵}(stiff matrix),简记为 $\mathbf A$;右端向量
称为 {\bf 载荷向量} (load vector), 简记为 $\mathbf f$。因此可将上面$N$个方程简记为
$$
\mathbf A\mathbf u = \mathbf f.
$$

求解得到系数向量$\bfu$代入 \eqref{eq:b} 的便得到解向量

\section{动机}
在通过$Galerkin$建立有限元空间中,很多学者提出了有限维空间V构造方法,相应的基函数也越来越丰富,但是分散在各种文献中,在有限元的相关教科书中也没有统一的说明,即使有整合在一起,其相应的理论和实现也没有具体给出,比如有限元周期表
\begin{figure}[H]
	\centering
	\includegraphics[width=0.7\linewidth]{figures/ptfe}
	\caption{}
	\label{fig:ptfe}
\end{figure}
一方面其不方便获取和理解,另一方面,大量不同的有限元类型,在编程实现方面存在冗余、复杂、结构混乱、接口不统一,对程序员不友好等情况。由于上述情况,本文一方面将一些常用的有限元构造给出系统的框架搭建,另一方面建立有限元的统一的结构,用一般化的方法来实对不同有限元的构造,使其更方便程序实现,也更易于用户理解。
\section{本文的主要工作与结构安排} 

在整个有限元过程中,第三步空间的构造最为重要,要找到适合的有限维空间来逼近原本的无限维空间,因此本文
主整理出有限元中不同空间的构造方法,再通过数组化的方法表达出来,以便于编程上的实现,最后通过数值案例
来验证空间构造的正确性。

本文共分为五章,具体安排如下:
\begin{itemize}
	\item 第一章是引言,主要介绍有有限元方法的背景和过程,以及本文的工作安排
	\item 第二章介绍网格基础知识,fealpy怎么实现
	\item 第三章介绍两个基本多项式空间(重心坐标,拉格朗日,缩放单项式)(先把每一维度的单纯形说明白)
	\item 第四章介绍混合元
	\item 第五张介绍等参元
	\item 介绍fealpy相关程序的使用方法
	\item 数值案例
	\item 本文的总结与展望
\end{itemize}
fealpy 空间构造  动机
要做什么,各种高次有限元的构造,基本空间,各种有限元,对应fealpy的程序(函数,参数,变量)。数组化的方式呈现出来,

第一节 要做什么
第二节 以possion方程为例有限元方法,
