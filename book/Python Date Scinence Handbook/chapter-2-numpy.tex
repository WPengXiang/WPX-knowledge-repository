% !Mode:: "TeX:UTF-8"
\documentclass{article}
\input{../../en_preamble.tex}
\input{../../xecjk_preamble.tex}
\begin{document}
\title{第二章:Numpy学习}
\author{王鹏祥}
\date{\chntoday}
\maketitle
\tableofcontents
\newpage
\section{numpy数组的几种操作}
\subsection{数组的属性}

\begin{itemize}
\item ndim:维度
\item shape:每个维度的大小
\item size:数组的总大小,每个维度的乘积
\item dtype:数据类型
\item itemsize:每个数组元素字节大小
\item nbytes:数组总字节大小,即itemsize和size乘积
\end{itemize}

\subsection{数组的索引}

\begin{itemize}
\item 可以用负值索引,-1就代表倒数第一个,0代表第一个
\item {\color{red} numpy数组类型固定,当数据类型为int时候,输入float型会自动截短}

\end{itemize}

\subsection{数组的切片}

\begin{itemize}
\item x[start:stop:step]:{\color{red}包含start.不包含stop,从0开始}
\item start默认为0,stop默认为维度的大小,step默认为1
\item 步长可以为负,
\item x[::-1]:逆序数组,这里start和stop默认交换
\item 多维数组同样可以采用这种方法来进行切片
\item 通过空切片(一个:),实现整行或者整列的提取
\item {\color{red}数组切片返回的是数组的视图,而不是副本,改变视图原始数据也会改变}
\item 可以通过.copy()方法创造副本
\end{itemize}

\subsection{数组的变形}

\begin{itemize}
\item .reshape()方法可以改变数组的形状
\item 该方法前提是原始数组大小(元素个数)

\subsection{数组的拼接和分裂}
\begin{itemize}
\item 拼接主要用np.concatenate([要拼接的数组],要按照哪个轴拼)
\item 拼接还有两个操作是np.vstack(垂直栈拼),np.hstack
\item 分裂主要使用np.split(要分裂的数组,(分裂点的位置))
\item {color{red}n个分点,会得到n+1个数组}
\item 分裂还有两个操作刚好与第二点相反np.hsplit,np.vsplit


\end{itemize}
\end{itemize}
%\cite{fem_2010}
%\bibliographystyle{abbrv}
%\bibliography{../../ref}
\end{document}
