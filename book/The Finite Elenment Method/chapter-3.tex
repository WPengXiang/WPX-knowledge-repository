% !Mode:: "TeX:UTF-8"
\documentclass{article}
\input{../../en_preamble.tex}
\input{../../xecjk_preamble.tex}
\begin{document}
\title{2维空间的分片多项式}
\author{王鹏祥}
\date{\chntoday}
\maketitle
\tableofcontents
\newpage

本章主要是将将分段多项式的逼近由以为变成二维,其最主要的变化是在对网格的剖分上,不再是一段一段的而是变为三角形,因此导致数据结构也产生相应的变化。
\section{三角形网格}
设$\Omega \subset \mathbb{R}^{2}$是一个二维有着光滑或者多边形区域$\partial \Omega$。在这个区域的三角形划分是由三角形的单元 $K$ 组成的集合 $\{K\}$, 即 $\Omega = \cup_{K\in\mcK}K$。两个三角形的交集可能为边(edges)、角(nodes)、或者不相交。边还要进一步划分为内部边(interior edges)用集合$\mcE_I$ 表示,边界边(boundary edges)用集合$\mcE_B$表示。
\subsection{判断标准}
网格的好坏通常会用chunkiness参数来判断,其对每个单元的定义为
$$
c_{K}=h_{K} / d_{K}
$$
其中$h_{K}$单元最大边长,$d_{K}$为内切圆的直径。如果存在一个常数$c_0$使得对于所有的$c_{K}$都有$c_{K} \geq c_{0}$则称网格是规则(regular)的。网格是否规则对之后的分段线性空间将会产生影响。

整体网格大小用 $h=\text{max}_{K\in\mcK}h_K$来评估.

准一致网格,意思是所有剖分单元的网格尺寸 $h_k$ 大致相等, 其对于任意两个剖分单元 $K$ 和 $K'$, 存在一个常数 $\rho$满足

$$
\rho<h_k/h_{k'}<\rho_{-1}
$$

\subsection{数据结构}
表示一个三角网格通常使用两个矩阵:结点矩阵(point matrix)和连接矩阵(connectivity matrix)。

结点矩阵:两行、结点个数列,每列分别是节点的横纵坐标。

连接矩阵:三行、单元个数列,每一列三个数字为结点在节点矩阵中所位于的列数,表示一个单元,顺序为逆时针方向,起始点可以随便选择。

\subsection{Matlab下生成网格}
MATLAB有一组非标准例程,称为“pde-Toolbox”,其中包括用于创建高质量二维几何三角形的Delaunay网格生成器.

用法为给定几何矩阵、算法模式和参数,最后声称节点矩阵、连接矩阵和边界矩阵。

[p, e t] = initmesh (0.1 g, hmax,)

几何矩阵形式为:第一行为维度,第二三行为结点横坐标值,第四五行为结点纵坐标值,第六七行为区域位于边的左侧还是右侧。每一列表示一个边界。

\section{分片多项式空间}
\subsection{线性多项式空间}
设K是一个三角形单元, 定义 K 上的多项式空间

\begin{equation}
    \mbP_1(K) = \{v: v = c_0 + c_1x_1 + c_2x_2, 
    (x_1, x_2) \in K, c_0, c_1, c_2 \in \mbR\}
\end{equation}

对于K上任意函数可以由三个结点值所决定,假设给定结点值$a_i$以及相应的节点 $N_i=(x_1^{(i)},x_2^{(i)})$,则由此计算函数$v$, 得到线
性方程组

\begin{equation}
\left[\begin{array}{ccc}
{1} & {x_{1}^{(1)}} & {x_{2}^{(1)}} \\
{1} & {x_{1}^{(2)}} & {x_{2}^{(2)}} \\
{1} & {x_{1}^{(3)}} & {x_{2}^{(3)}}
\end{array}\right]\left[\begin{array}{c}
{c_{0}} \\
{c_{1}} \\
{c_{2}}
\end{array}\right]=\left[\begin{array}{c}
{\alpha_{1}} \\
{\alpha_{2}} \\
{\alpha_{3}}
\end{array}\right]
\end{equation}

可以发现其系数矩阵行列式等于单元K面积的两倍。这意味着有唯一解。

其基函数可以自然地选择$\{1, x_1, x_2\}$,也可以是结点基函数

\begin{equation}
\lambda_{j}\left(N_{i}\right)=\left\{\begin{array}{ll}
{1,} & {i=j} \\
{0,} & {i \neq j}
\end{array} i, j=1,2,3\right.
\end{equation}


\subsection{连续分片多项式空间}

$\mcK$ 是三角剖分的集合, $K \in \mcK$, 定义$\mcK$ 上的分片连续多项式空间

\begin{equation}
    \mbV_h = \{v: v\in C^0(\Omega), v|_K \in \mbP_1(K), \forall K \in \mcK\}
\end{equation}

其中 $C^0(\Omega)$ 是区间 $\Omega$ 上的连续函数空间

$V_h$ 上的函数 $v$ 被节点值$\{v(N_j)\}_{j=1}^{n_p}$ 所决定,同样可做结点基。这组基 $\varphi_j$ 只在把点 $N_j$ 作为公共点的三角形组成的区域不为0。

\begin{equation}
    \varphi_j(x_i) = \begin{cases}
        1, \quad \text{if} \, i = j \\
        0, \quad \text{if} \, i \ne j \\
    \end{cases},
    \quad i, j = 0, 1, \cdots, n
\end{equation}

这组基 $\varphi_j$ 只在把点 $N_j$ 作为公共点的三角形组成的区域不为0,也叫做帽函数。

\subsection{分片常数空间}
分片常数空间定义为
\begin{equation}
W_{h}=\left\{w:\left.w\right|_{K} \in P_{0}(K), \forall K \in \mathcal{K}\right\}
\end{equation}
其中$P_{0}(K)$为单元K上的常值函数

为测量函数值在边上的大小,习惯上用两种算子平均算子(average operators)和跳跃算子(jump operator)

平均算子:符号表示为$\langle\cdot\rangle$,定义为
\begin{equation}
\langle w\rangle=\frac{w^{+}+w^{-}}{2}, \quad x \in E
\end{equation}
其中$w^{+}+w^{-}$在一条边上的两个单元的不同值。

跳跃算子:符号表示为$[\cdot]$,定义为
\begin{equation}
[w]=w^{+}-w^{-}, \quad x \in E
\end{equation}


\section{插值}
\subsection{线性插值}

在三角形单元 $K$ 上定义连续的线性插值 $\pi f \in \mbP_1(K)$ 

\begin{equation}
    \pi f(x) = \sum_{i=1}^{3} f(N_i) \varphi_i,
\end{equation}
其在三个节点$N_i$处与函数重合 $\pi f(N_i) = f(N_i)$.

令 $D$ 与 $D^2$ 为微分算子,定义为
\begin{equation}
D f=\left(\left|\frac{\partial f}{\partial x_{1}}\right|^{2}+\left|\frac{\partial f}{\partial x_{2}}\right|^{2}\right)^{1 / 2}, \quad D^{2} f=\left(\left|\frac{\partial^{2} f}{\partial x_{1}^{2}}\right|^{2}+2\left|\frac{\partial^{2} f}{\partial x_{1} \partial x_{2}}\right|^{2}+\left|\frac{\partial^{2} f}{\partial x_{2}^{2}}\right|^{2}\right)^{1 / 2}
\end{equation}

由此再定义$L^2(I)$ 范数
\begin{equation}
\|f\|_{L^{2}(\Omega)}=\left(\int_{\Omega} f^{2} d x\right)^{1 / 2}
\end{equation}

我们使用 $L^2(I)$ 范数测量插值误差.

\begin{proposition}
    插值误差满足下面估计
    \begin{align*}
        \|f- \pi f\|_{L^2(K)} &\leq 
    C h^2_K \|D^2 f\|_{L^2(K)} \\
        \|D(f - \pi f)\|_{L^2(K)} &\leq
        C h_K^2 \|D^2 f\|_{L^2(K)} 
    \end{align*}
    
\end{proposition}


\subsection{连续的分片线性插值}
定义 $\Omega$ 上的在网格 $\mcK$ 上的连续分段线性插值 $\pi f\in V_h$, 即

$$
\pi f=\sum_{i=1}^{n_p}f(N_i)\varphi_i
$$

其在节点处与函数值一样

\begin{proposition}
    插值误差满足下面估计
    \begin{align*}
        \|f- \pi f\|^2_{L^2(\Omega)} &\leq 
        C \sum_{K \in \mcK} h^4_K \|D^2 f\|^2_{L^2(K)} \\
        \|D(f - \pi f)\|^2_{L^2(\Omega)} &\leq
        C \sum_{K \in \mcK}h_K^2 \|D^2 f\|^2_{L^2(K)} 
    \end{align*}
\end{proposition}


\section{$L^2$ 投影}

给定函数 $f \in L^2(\Omega)$, 则它的 $L^2$ 投影 $P_hf \in \mbV_h$ 定义为

\begin{equation}\label{1}
    \int_\Omega (f - P_hf)v \mathrm{d} \bfx = 0, \quad \forall v \in \mbV_h
\end{equation}

\subsection{$L^2$ 投影求解推导}

对\ref{1}中的v特取为基函数$\varphi_i$,则有

\begin{equation}
    \int_{\Omega} (f - P_hf)\varphi_i \mathrm{d} \bfx = 0, 
    \quad i = 1, 2, \cdots, n_{p}.
\end{equation}

$n_{p}$ 是空间 $\mbV_h$ 的基函数的个数。

由于$P_hf \in \mbV_h$属于$\mbV_h$,因此可以表示为

\begin{equation}
P_{h} f=\sum_{j=1}^{n_{p}} \xi_{j} \varphi_{j}
\end{equation}

其中$\xi_{j}$为未知的系数,再带入\ref{1}便得到

\begin{equation}
\begin{aligned}
\int_{\Omega} f \varphi_{i} d x &=\int_{\Omega}\left(\sum_{j=1}^{n_{p}} \xi_{j} \varphi_{j}\right) \varphi_{i} d x \\
&=\sum_{j=1}^{n_{p}} \xi_{j} \int_{\Omega} \varphi_{j} \varphi_{i} d x
\end{aligned}
\end{equation}

这样便得到质量矩阵$M_i$
$$
M_{i j}=\int_{\Omega} \varphi_{j} \varphi_{i} d x, \quad i, j=1,2, \ldots, n_{p}
$$

过载向量$B_i$
$$
b_{i}=\int_{\Omega} f \varphi_{i} d x, \quad i=1,2 \ldots, n_{p}
$$

解方程组便可得到$\xi_{j}$,从而求出$L^2$ 投影。
\subsection{$L^2$ 投影的几个定理}
\begin{theorem}
    由式 (\ref{1}) 定义的 $L^2$ 投影 $\Pi_h f$ 存在且唯一.
\end{theorem}


\begin{theorem}
    由式 (\ref{1}) 定义的 $L^2$ 投影 $\Pi_h f$ 满足最优估计
    
    \begin{equation}
        \| f - \Pi_h f\|_{L^2(\Omega)} \le \|f - v\|_L^2(\Omega), 
        \forall v \in \mbV_h.
    \end{equation}
\end{theorem}


\begin{theorem}
   由式 (\ref{1}) 定义的 $L^2$ 投影 $\Pi_h f$  满足估计    
    \begin{equation}
        \| f - \Pi_h f\|^2_{L^2(\Omega)} \le C 
        \sum_{K\in \mcK}\|D^2 f\|^2_{L^2(\Omega)}
    \end{equation}
\end{theorem}

\section{三角单元上的几种积分公式}

三角形单元K的一般求积规则是这种形式

\begin{equation}
\int_{K} f d x \approx \sum_{j} w_{j} f\left(q_{j}\right)
\end{equation}

其中$q_{j}$为K上一组积分点,$\{\omega_j\}$是相应的权重

具体形式有以下几种

\subsection{center of gravity rule}
\begin{equation}
\int_{K} f d x \approx f\left(\frac{N_{1}+N_{2}+N_{3}}{3}\right)|K|
\end{equation}

\subsection{corner quadrature formula}
\begin{equation}
\int_{K} f d x \approx \sum_{i=1}^{3} f\left(N_{i}\right) \frac{|K|}{3}
\end{equation}

\subsection{two-dimensuonal Mid-point rule}
\begin{equation}
\int_{K} f d x \approx \sum_{1 \leq i<j \leq 3}^{3} f\left(\frac{N_{i}+N_{j}}{2}\right) \frac{|K|}{3}
\end{equation}


\end{document}
