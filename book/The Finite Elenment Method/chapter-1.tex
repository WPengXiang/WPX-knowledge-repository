% !Mode:: "TeX:UTF-8"
\documentclass{article}
\input{../../en_preamble.tex}
\input{../../xecjk_preamble.tex}
\begin{document}
\title{一维分段多项式的逼近}
\author{王鹏祥}
\date{\chntoday}
\maketitle
\tableofcontents
\newpage

这章主要讲的是如何用插值和投影对多项式进行逼近, 和几种常用的数值积分公式
\section{分片多项式空间(Piecewise Polynomial Space)}

\subsection{线性多项式空间(Space of Linear Polynomials)}

给定区间 $I = [x_0, x_1]$, 定义区间上的线性多项式空间

\begin{equation}\label{1}
    \mbP_1(I) = \{v: v(x) = c_0 + c_1x, x \in I, c_0, c_1 \in \mbR\}
\end{equation}

自由度就是看有几个变量, 具体来说就是看有几个系数, 线性多项式函数空间自由度就是2

线性多项式空间常用的的基函数有两个, 单项式基(monomial basis)和结点基(nodal basis)
\begin{itemize}
	\item monomial basis:$ \{ 1,x \}$
	\item nodal basis:$ \{ \frac{x_1 - x}{x_1 - x_0} , \frac{x - x_0}{x_1 - x_0} \}$
\end{itemize}
在结点基下函数值依赖于一个范德蒙行列式, 任意一个分段多项式空间里的元素都都可以被基线性表示, 一般常用结点基

\subsection{连续的分片线性多项式空间(The Space of Countinue Piecewise Linear Polynomials)}

把区间 $I = [0, L]$ 分成 $n$ 段, 设立n+1个结点(做剖分,变为网格)即

\begin{equation*}
    \mcI: 0 = x_0 < x_1 < x_2 < \cdots < x_{n-1} < x_n = L
\end{equation*}

并定义子区间 $I_i = [x_{i-1}, x_i], \, i = 1, 2, \cdots, n$, 区间长度固定 
$h_i = x_i - x_{i-1}$.

在网格 $\mcI$ 上, 定义连续分片线性多项式的空间 $\mbV_h$

\begin{equation*}
    \mbV_h = \{v: v\in C^0(I), v|_{I_i} \in \mbP_1(I_i)\}
\end{equation*}

其中 $C^0(I)$ 是区间 $I$ 上的连续函数空间, $\mbP_1(I_i)$ 是区间 $I_i$ 上的线性函数空间(\ref{1}).

连续的分片线性多项式空间$\mbV_h$ 上的基为帽函数(hat functions) $\{\varphi_j\}_{j = 0}^n$,

\begin{equation*}
    \varphi_j(x_i) = \begin{cases}
        1, \quad \text{if} \, i = j \\
        0, \quad \text{if} \, i \ne j \\
    \end{cases},
    \quad i, j = 0, 1, \cdots, n
\end{equation*}

简言之就是函数在一个结点处的值为1在其他结点出都为0的线性函数. 这种想法很自然, 因为很显然连续分片多项式空间是由结点处的值所唯一确定. 更直观的表达为

\begin{equation*}
	\varphi_j(x_i) = \begin{cases}
		\frac{x-x_{i-1}}{h_i}, & \quad \text{if}   x \in I_i \\
		\frac{x_{i+1}-x}{h_{i+1}},& \quad \text{if}  x \in I_{i+1} \\
		0, & \quad \text{otherwise}
		\end{cases}
\end{equation*}

\section{逼近连续分片线性多项式空间的方法}
\subsection{插值(Interpolation)}
\subsubsection{线性插值(Liner Interpolation)}

在区间 $I$ 上给定连续函数 $f$, 定义 $f$ 的线性插值函数为 $\pi f \in \mbP_1$ 

\begin{equation*}
    \pi f(x) = f(x_0)\varphi_0 + f(x_1)\varphi_1 
\end{equation*}

因此我们可以看出插值函数在结点处的值和原函数是一样的。

测量两者之间的误差可以通过两种范数来测量:无穷范数 $\|v|\|_{ \infty } $和 $L^2(I)$ 范数

$L^2(I)$ 范数是对函数整体上逼近程度好坏的一种度量,我们使用 $L^2(I)$ 范数测量插值误差

\begin{proposition}\label{2}
    插值误差满足下面估计
    \begin{align*}
        \|f- \pi_1 f\|_2 &\leq C h^2 \|f''\|_2 \\
        \|(f - \pi_1 f)'\|_2 &\leq  C h \|f''\|_2 
    \end{align*}
    其中 C 是常数.
\end{proposition}

从中可以看出误差是依赖于插值函数和剖分单元长度.

\subsubsection{连续的分片线性插值(Countinue Piecewise Liner Interpolation)}

在网格 $\mcI$ 上定义连续的线性插值 $\pi f \in \mbV_h$ 

\begin{equation}
    \pi f(x) = \sum_{i=1}^{n} f(x_i) \varphi_i(x)
\end{equation}

其实就是讲一个函数分成很多段, 在每一段上用线性插值来逼近函数.

我们使用 $L^2(I)$ 范数测量插值误差

\begin{proposition}
    插值误差满足下面估计
    \begin{align*}
        \|f- \pi_1 f\|^2_{L^2(I)} &\leq 
        C \sum_{i=1}^n h_i^4 \|f''\|^2_{L^2(I_i)} \\
        \|(f - \pi_1 f)'\|^2_{L^2(I)} &\leq
        C \sum_{i=1}^n h_i^2 \|f''\|^2_{L^2(I_i)} 
    \end{align*}
\end{proposition}

线性插值只适合光滑性比较好的函数, 若函数的二阶导的值很大的时候(就是函数图像凹凸很大时候),
线性插值就不能对函数有效进行逼近,因此我们引入$L^2$ 投影.

\subsection{$L^2$ 投影($L^2$Projection)}

给定函数 $f \in L^{I}$, 则它的 $L^2$ 投影 $P_hf \in \mbV_h$ 为

\begin{equation}
    \int_I (f - P_hf)v \mathrm{d} x = 0, \quad \forall v \in \mbV_h
\end{equation}

即$ P_hf $为 $ f $在$ \mbV_h $上的投影,并且两者间的误差$ (f - P_hf) $要与$ \mbV_h $
里任意向量正交

由于上式对任意的 $v$ 成立, 因此可以用 $\mbV_h$ 中的基函数$  \varphi_i $代替 $v$. 即

\begin{equation}\label{2}
    \int_I (f - P_hf)\varphi_i \mathrm{d} x = 0, \quad i = 1, 2, \ldots, n.
\end{equation}

将$  P_hf $用基函数表示$ P_hf= \sum_{j=0}^n\xi_J\varphi_i $然后带入\ref{2}可推出线性方程组

\begin{equation}
\begin{aligned}
\int_{I} f \varphi_{i} d x &=\int_{I}\left(\sum_{j=0}^{n} \xi_{j} \varphi_{j}\right) \varphi_{i} d x \\
&=\sum_{j=0}^{n} \xi_{j} \int_{I} \varphi_{j} \varphi_{i} d x, \quad i=0,1, \ldots, n
\end{aligned}
\end{equation}

定义
\begin{itemize}
\item 刚度矩阵(mass matrix):$ \int_{I}\varphi_{j} \varphi_{i} d x, \quad i=0,1, \ldots, n$
\item 载荷向量(load vector):$  \int_{I} f \varphi_{i} d x $
\end{itemize}

\begin{proposition}

$ P_{h} f $满足如下估计
\begin{align*}
\left\|f-P_{h} f\right\|_{L^{2}(I)}^{2} \leq C \sum_{i=1}^{n} h_{i}^{4}\left\|f^{\prime \prime}\right\|_{L^{2}\left(I_{i}\right)}^{2} \\
\left\|f-P_{h} f\right\|_{L^{2}(I)} \leq\|f-v\|_{L^{2}(I)}, \quad \forall v \in V_{h}
\end{align*}

\end{proposition}

最后$L^2$ 投影不适合对有很多震荡的函数来逼近。
\section{常用数值积分公式}
估计连续函数$f(x)$在区间$I = \left[x_0,x_1\right]$上的积分

\begin{equation*}
    J = \int_I f(x) dx
\end{equation*}

有以下几种方法
\begin{itemize}
\item 中间值公式: $ J \approx f(\frac{x_{0}+x_{1}}{2})h $
\item 梯形公式: $ J \approx \frac{f(x_0)+f(x_1)}{2}h $
\item Simpson's公式: $ J \approx \frac{f(x_0)+4f(\frac{x_{0}+x_{1}}{2})+f(x_1)}{6}h $
\end{itemize}

%\cite{fem_2010}
%\bibliographystyle{abbrv}
%\bibliography{../../ref}

\end{document}
