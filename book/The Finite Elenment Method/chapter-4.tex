% !Mode:: "TeX:UTF-8"
\documentclass{article}

%%%%%%%%------------------------------------------------------------------------
%%%% 日常所用宏包

%% 控制页边距
% 如果是beamer文档类, 则不用geometry
\makeatletter
\@ifclassloaded{beamer}{}{\usepackage[top=2.5cm, bottom=2.5cm, left=2.5cm, right=2.5cm]{geometry}}
\makeatother

\makeatletter
\@ifclassloaded{beamer}{
\makeatletter
\def\th@mystyle{%
    \normalfont % body font
    \setbeamercolor{block title example}{bg=orange,fg=white}
    \setbeamercolor{block body example}{bg=orange!20,fg=black}
    \def\inserttheoremblockenv{exampleblock}
  }
\makeatother
\theoremstyle{mystyle}
\newtheorem*{remark}{Remark}

\newcommand{\propnumber}{} % initialize
\newtheorem*{prop}{Proposition \propnumber}
\newenvironment{propc}[1]
  {\renewcommand{\propnumber}{#1}%
   \begin{shaded}\begin{prop}}
  {\end{prop}\end{shaded}}

\makeatletter
\newenvironment<>{proofs}[1][\proofname]{%
    \par
    \def\insertproofname{#1\@addpunct{.}}%
    \usebeamertemplate{proof begin}#2}
  {\usebeamertemplate{proof end}}
\makeatother

}{
}
\makeatother
\usepackage{amsthm}

%\DeclareMathOperator{\sech}{sech}
%\DeclareMathOperator{\csch}{csch}
%\DeclareMathOperator{\arcsec}{arcsec}
%\DeclareMathOperator{\arccot}{arccot}
%\DeclareMathOperator{\arccsc}{arccsc}
%\DeclareMathOperator{\arccosh}{arccosh}
%\DeclareMathOperator{\arcsinh}{arcsinh}
%\DeclareMathOperator{\arctanh}{arctanh}
%\DeclareMathOperator{\arcsech}{arcsech}
%\DeclareMathOperator{\arccsch}{arccsch}
%\DeclareMathOperator{\arccoth}{arccoth}
%% 控制项目列表
\usepackage{enumerate}

%% Todo list
\usepackage{enumitem}
\newlist{todolist}{itemize}{2}
\setlist[todolist]{label=$\square$}
\usepackage{pifont}
\newcommand{\cmark}{\ding{51}}%
\newcommand{\xmark}{\ding{55}}%
\newcommand{\done}{\rlap{$\square$}{\raisebox{2pt}{\large\hspace{1pt}\cmark}}%
\hspace{-2.5pt}}
\newcommand{\wontfix}{\rlap{$\square$}{\large\hspace{1pt}\xmark}}

\usepackage[utf8]{inputenc}
\usepackage[english]{babel}

\usepackage{framed}

%% 多栏显示
\usepackage{multicol}

%% 算法环境
\usepackage{algorithm}
\usepackage{algorithmic}
\usepackage{float}

%% 网址引用
\usepackage{url}

%% 控制矩阵行距
\renewcommand\arraystretch{1.4}

%% 粗体
\usepackage{lmodern}
\usepackage{bm}


%% hyperref宏包,生成可定位点击的超链接,并且会生成pdf书签
\makeatletter
\@ifclassloaded{beamer}{
\usepackage{hyperref}
\usepackage{ragged2e} % 对齐
}{
\usepackage[%
    pdfstartview=FitH,%
    CJKbookmarks=true,%
    bookmarks=true,%
    bookmarksnumbered=true,%
    bookmarksopen=true,%
    colorlinks=true,%
    citecolor=blue,%
    linkcolor=blue,%
    anchorcolor=green,%
    urlcolor=blue%
]{hyperref}
}
\makeatother



\makeatletter % 如果是 beamer 不需要下面两个包
\@ifclassloaded{beamer}{
\mode<presentation>
{
}
}{
%% 控制标题
\usepackage{titlesec}
%% 控制目录
\usepackage{titletoc}
}
\makeatother

%% 控制表格样式
\usepackage{booktabs}

%% 控制字体大小
\usepackage{type1cm}

%% 首行缩进,用\noindent取消某段缩进
\usepackage{indentfirst}

%% 支持彩色文本、底色、文本框等
\usepackage{color,xcolor}

%% AMS LaTeX宏包: http://zzg34b.w3.c361.com/package/maths.htm#amssymb
\usepackage{amsmath,amssymb}
%% 多个图形并排
\usepackage{subfig}
%%%% 基本插图方法
%% 图形宏包
\usepackage{graphicx}


%%%% 基本插图方法结束

%%%% pgf/tikz绘图宏包设置
\usepackage{pgf,tikz}
\usetikzlibrary{shapes,automata,snakes,backgrounds,arrows}
\usetikzlibrary{mindmap}
%% 可以直接在latex文档中使用graphviz/dot语言,
%% 也可以用dot2tex工具将dot文件转换成tex文件再include进来
%% \usepackage[shell,pgf,outputdir={docgraphs/}]{dot2texi}
%%%% pgf/tikz设置结束


\makeatletter % 如果是 beamer 不需要下面两个包
\@ifclassloaded{beamer}{

}{
%%%% fancyhdr设置页眉页脚
%% 页眉页脚宏包
\usepackage{fancyhdr}
%% 页眉页脚风格
\pagestyle{plain}
}

%% 有时会出现\headheight too small的warning
\setlength{\headheight}{15pt}

%% 清空当前页眉页脚的默认设置
%\fancyhf{}
%%%% fancyhdr设置结束


%% 设置listings宏包的一些全局样式
%% 参考http://hi.baidu.com/shawpinlee/blog/item/9ec431cbae28e41cbe09e6e4.html
\usepackage{listings}
\lstloadlanguages{[LaTeX]TeX}

\usepackage{fancyvrb}

\newenvironment{latexample}[1][language={[LaTeX]TeX}]
{\lstset{breaklines=true,
    prebreak = \raisebox{0ex}[0ex][0ex]{\ensuremath{\hookleftarrow}},
    frame=single,
    language={[LaTeX]TeX},
    showstringspaces=false,              %% 设定是否显示代码之间的空格符号
    numbers=left,                        %% 在左边显示行号
    numberstyle=\tiny,                   %% 设定行号字体的大小
    basicstyle=\scriptsize,                    %% 设定字体大小\tiny, \small, \Large等等
    keywordstyle=\color{blue!70}, commentstyle=\color{red!50!green!50!blue!50},
                                         %% 关键字高亮
    frame=shadowbox,                     %% 给代码加框
    rulesepcolor=\color{red!20!green!20!blue!20},
    escapechar=`,                        %% 中文逃逸字符,用于中英混排
    xleftmargin=2em,xrightmargin=2em, aboveskip=1em,
    %breaklines,                          %% 这条命令可以让LaTeX自动将长的代码行换行排版
    extendedchars=false                  %% 这一条命令可以解决代码跨页时,章节标题,页眉等汉字不显示的问题
    basicstyle=\footnotesize\ttfamily, #1}
  \VerbatimEnvironment\begin{VerbatimOut}{latexample.verb.out}}
  {\end{VerbatimOut}\noindent
  \begin{minipage}{1.05\linewidth}
    \lstinputlisting[]{latexample.verb.out}%
  \end{minipage}\qquad
  \begin{minipage}{1\linewidth}
    \input{latexample.verb.out}
  \end{minipage}\\
}

\usepackage{minted}
\renewcommand{\listingscaption}{Python code} \newminted{python}{
    escapeinside=||,
    mathescape=true,
    numbersep=5pt,
    linenos=true,
    autogobble,
    framesep=3mm}
%%%% listings宏包设置结束


%%%% 附录设置
\makeatletter % 对 beamer 要重新设置
\@ifclassloaded{beamer}{

}{
\usepackage[title,titletoc,header]{appendix}
}
\makeatother
%%%% 附录设置结束





%% 设定行距
\linespread{1}

%% 颜色
\newcommand{\red}{\color{red} }
\newcommand{\blue}{\color{blue} }
\newcommand{\brown}{\color{brown} }
\newcommand{\green}{\color{green} }

\newcommand{\bred}{\bf\color{red} }
\newcommand{\bblue}{\bf\color{blue} }
\newcommand{\bbrown}{\bf\color{brown} }
\newcommand{\bgreen}{\bf\color{green} }
%% 1. 小写的英文或希腊字母表示 标量或标量函数
%% 2. 大写的英文或希腊字母表示 集合或空间
%% 3. 粗体的小写字母代表向量或向量形式的常量和函数
%% 4. 粗体的大写字母代表矩阵或张量形式的常量和函数
%% 5. 空心大写字母代表特殊的空间 \mbR 实数 \mbC 复数 \mbP 多项式
%% 6. 花体的大写字母代表算子

%% 粗体的小写字母代表向量或向量函数
\newcommand{\bfa}{{\boldsymbol a}}
\newcommand{\bfb}{{\boldsymbol b}}
\newcommand{\bfc}{{\boldsymbol c}}
\newcommand{\bfd}{{\boldsymbol d}}
\newcommand{\bfe}{{\boldsymbol e}}
\newcommand{\bff}{{\boldsymbol f}}
\newcommand{\bfg}{{\boldsymbol g}}
\newcommand{\bfh}{{\boldsymbol h}}
\newcommand{\bfi}{{\boldsymbol i}}
\newcommand{\bfj}{{\boldsymbol j}}
\newcommand{\bfk}{{\boldsymbol k}}
\newcommand{\bfl}{{\boldsymbol l}}
\newcommand{\bfm}{{\boldsymbol m}}
\newcommand{\bfn}{{\boldsymbol n}}
\newcommand{\bfo}{{\boldsymbol o}}
\newcommand{\bfp}{{\boldsymbol p}}
\newcommand{\bfq}{{\boldsymbol q}}
\newcommand{\bfr}{{\boldsymbol r}}
\newcommand{\bfs}{{\boldsymbol s}}
\newcommand{\bft}{{\boldsymbol t}}
\newcommand{\bfu}{{\boldsymbol u}}
\newcommand{\bfv}{{\boldsymbol v}}
\newcommand{\bfw}{{\boldsymbol w}}
\newcommand{\bfx}{{\boldsymbol x}}
\newcommand{\bfy}{{\boldsymbol y}}
\newcommand{\bfz}{{\boldsymbol z}}

%  算子
\newcommand{\mca}{{\mathcal a}}
\newcommand{\mcb}{{\mathcal b}}
\newcommand{\mcc}{{\mathcal c}}
\newcommand{\mcd}{{\mathcal d}}
\newcommand{\mce}{{\mathcal e}}
\newcommand{\mcf}{{\mathcal f}}
\newcommand{\mcg}{{\mathcal g}}
\newcommand{\mch}{{\mathcal h}}
\newcommand{\mci}{{\mathcal i}}
\newcommand{\mcj}{{\mathcal j}}
\newcommand{\mck}{{\mathcal k}}
\newcommand{\mcl}{{\mathcal l}}
\newcommand{\mcm}{{\mathcal m}}
\newcommand{\mcn}{{\mathcal n}}
\newcommand{\mco}{{\mathcal o}}
\newcommand{\mcp}{{\mathcal p}}
\newcommand{\mcq}{{\mathcal q}}
\newcommand{\mcr}{{\mathcal r}}
\newcommand{\mcs}{{\mathcal s}}
\newcommand{\mct}{{\mathcal t}}
\newcommand{\mcu}{{\mathcal u}}
\newcommand{\mcv}{{\mathcal v}}
\newcommand{\mcw}{{\mathcal w}}
\newcommand{\mcx}{{\mathcal x}}
\newcommand{\mcy}{{\mathcal y}}
\newcommand{\mcz}{{\mathcal z}}

% \rmd
\newcommand{\mra}{{\mathrm a}}
\newcommand{\mrb}{{\mathrm b}}
\newcommand{\mrc}{{\mathrm c}}
\newcommand{\mrd}{{\mathrm d}}
\newcommand{\mre}{{\mathrm e}}
\newcommand{\mrf}{{\mathrm f}}
\newcommand{\mrg}{{\mathrm g}}
\newcommand{\mrh}{{\mathrm h}}
\newcommand{\mri}{{\mathrm i}}
\newcommand{\mrj}{{\mathrm j}}
\newcommand{\mrk}{{\mathrm k}}
\newcommand{\mrl}{{\mathrm l}}
\newcommand{\mrm}{{\mathrm m}}
\newcommand{\mrn}{{\mathrm n}}
\newcommand{\mro}{{\mathrm o}}
\newcommand{\mrp}{{\mathrm p}}
\newcommand{\mrq}{{\mathrm q}}
\newcommand{\mrr}{{\mathrm r}}
\newcommand{\mrs}{{\mathrm s}}
\newcommand{\mrt}{{\mathrm t}}
\newcommand{\mru}{{\mathrm u}}
\newcommand{\mrv}{{\mathrm v}}
\newcommand{\mrw}{{\mathrm w}}
\newcommand{\mrx}{{\mathrm x}}
\newcommand{\mry}{{\mathrm y}}
\newcommand{\mrz}{{\mathrm z}}

%% 粗体的大写字母一般表示矩阵和张量
\newcommand{\bfA}{{\boldsymbol A}}
\newcommand{\bfB}{{\boldsymbol B}}
\newcommand{\bfC}{{\boldsymbol C}}
\newcommand{\bfD}{{\boldsymbol D}}
\newcommand{\bfE}{{\boldsymbol E}}
\newcommand{\bfF}{{\boldsymbol F}}
\newcommand{\bfG}{{\boldsymbol G}}
\newcommand{\bfH}{{\boldsymbol H}}
\newcommand{\bfI}{{\boldsymbol I}}
\newcommand{\bfJ}{{\boldsymbol J}}
\newcommand{\bfK}{{\boldsymbol K}}
\newcommand{\bfL}{{\boldsymbol L}}
\newcommand{\bfM}{{\boldsymbol M}}
\newcommand{\bfN}{{\boldsymbol N}}
\newcommand{\bfO}{{\boldsymbol O}}
\newcommand{\bfP}{{\boldsymbol P}}
\newcommand{\bfQ}{{\boldsymbol Q}}
\newcommand{\bfR}{{\boldsymbol R}}
\newcommand{\bfS}{{\boldsymbol S}}
\newcommand{\bfT}{{\boldsymbol T}}
\newcommand{\bfU}{{\boldsymbol U}}
\newcommand{\bfV}{{\boldsymbol V}}
\newcommand{\bfW}{{\boldsymbol W}}
\newcommand{\bfX}{{\boldsymbol X}}
\newcommand{\bfY}{{\boldsymbol Y}}
\newcommand{\bfZ}{{\boldsymbol Z}}

%% 花体大写字母
\newcommand{\mcA}{{\mathcal A}}
\newcommand{\mcB}{{\mathcal B}}
\newcommand{\mcC}{{\mathcal C}}
\newcommand{\mcD}{{\mathcal D}}
\newcommand{\mcE}{{\mathcal E}}
\newcommand{\mcF}{{\mathcal F}}
\newcommand{\mcG}{{\mathcal G}}
\newcommand{\mcH}{{\mathcal H}}
\newcommand{\mcI}{{\mathcal I}}
\newcommand{\mcJ}{{\mathcal J}}
\newcommand{\mcK}{{\mathcal K}}
\newcommand{\mcL}{{\mathcal L}}
\newcommand{\mcM}{{\mathcal M}}
\newcommand{\mcN}{{\mathcal N}}
\newcommand{\mcO}{{\mathcal O}}
\newcommand{\mcP}{{\mathcal P}}
\newcommand{\mcQ}{{\mathcal Q}}
\newcommand{\mcR}{{\mathcal R}}
\newcommand{\mcS}{{\mathcal S}}
\newcommand{\mcT}{{\mathcal T}}
\newcommand{\mcU}{{\mathcal U}}
\newcommand{\mcV}{{\mathcal V}}
\newcommand{\mcW}{{\mathcal W}}
\newcommand{\mcX}{{\mathcal X}}
\newcommand{\mcY}{{\mathcal Y}}
\newcommand{\mcZ}{{\mathcal Z}}

%% 空心大写字母
\newcommand{\mbA}{{\mathbb A}}
\newcommand{\mbB}{{\mathbb B}}
\newcommand{\mbC}{{\mathbb C}}
\newcommand{\mbD}{{\mathbb D}}
\newcommand{\mbE}{{\mathbb E}}
\newcommand{\mbF}{{\mathbb F}}
\newcommand{\mbG}{{\mathbb G}}
\newcommand{\mbH}{{\mathbb H}}
\newcommand{\mbI}{{\mathbb I}}
\newcommand{\mbJ}{{\mathbb J}}
\newcommand{\mbK}{{\mathbb K}}
\newcommand{\mbL}{{\mathbb L}}
\newcommand{\mbM}{{\mathbb M}}
\newcommand{\mbN}{{\mathbb N}}
\newcommand{\mbO}{{\mathbb O}}
\newcommand{\mbP}{{\mathbb P}}
\newcommand{\mbQ}{{\mathbb Q}}
\newcommand{\mbR}{{\mathbb R}}
\newcommand{\mbS}{{\mathbb S}}
\newcommand{\mbT}{{\mathbb T}}
\newcommand{\mbU}{{\mathbb U}}
\newcommand{\mbV}{{\mathbb V}}
\newcommand{\mbW}{{\mathbb W}}
\newcommand{\mbX}{{\mathbb X}}
\newcommand{\mbY}{{\mathbb Y}}
\newcommand{\mbZ}{{\mathbb Z}}

\newcommand{\mrA}{{\mathrm A}}
\newcommand{\mrB}{{\mathrm B}}
\newcommand{\mrC}{{\mathrm C}}
\newcommand{\mrD}{{\mathrm D}}
\newcommand{\mrE}{{\mathrm E}}
\newcommand{\mrF}{{\mathrm F}}
\newcommand{\mrG}{{\mathrm G}}
\newcommand{\mrH}{{\mathrm H}}
\newcommand{\mrI}{{\mathrm I}}
\newcommand{\mrJ}{{\mathrm J}}
\newcommand{\mrK}{{\mathrm K}}
\newcommand{\mrL}{{\mathrm L}}
\newcommand{\mrM}{{\mathrm M}}
\newcommand{\mrN}{{\mathrm N}}
\newcommand{\mrO}{{\mathrm O}}
\newcommand{\mrP}{{\mathrm P}}
\newcommand{\mrQ}{{\mathrm Q}}
\newcommand{\mrR}{{\mathrm R}}
\newcommand{\mrS}{{\mathrm S}}
\newcommand{\mrT}{{\mathrm T}}
\newcommand{\mrU}{{\mathrm U}}
\newcommand{\mrV}{{\mathrm V}}
\newcommand{\mrW}{{\mathrm W}}
\newcommand{\mrX}{{\mathrm X}}
\newcommand{\mrY}{{\mathrm Y}}
\newcommand{\mrZ}{{\mathrm Z}}


% 粗体的 Greek 字母
\newcommand{\balpha}{{\bm \alpha}}
\newcommand{\bbeta}{{\bm \beta}}
\newcommand{\bgamma}{{\bm \gamma}}
\newcommand{\bdelta}{{\bm \delta}}
\newcommand{\bepsilon}{{\bm \epsilon}}
\newcommand{\bvarepsilon}{{\bm \varepsilon}}
\newcommand{\bzeta}{{\bm \zeta}}
\newcommand{\bfeta}{{\bm \eta}}
\newcommand{\btheta}{{\bm \theta}}
\newcommand{\biota}{{\bm \iota}}
\newcommand{\bkappa}{{\bm \kappa}}
\newcommand{\blambda}{{\bm \lambda}}
\newcommand{\bmu}{{\bm \mu}}
\newcommand{\bnu}{{\bm \nu}}
\newcommand{\bxi}{{\bm \xi}}
\newcommand{\bomicron}{{\bm \omicron}}
\newcommand{\bpi}{{\bm \pi}}
\newcommand{\brho}{{\bm \rho}}
\newcommand{\bsigma}{{\bm \sigma}}
\newcommand{\btau}{{\bm \tau}}
\newcommand{\bupsilon}{{\bm \upsilon}}
\newcommand{\bphi}{{\bm \phi}}
\newcommand{\bvarphi}{{\bm \varphi}}
\newcommand{\bchi}{{\bm \chi}}
\newcommand{\bpsi}{{\bm \psi}}

\newcommand{\bAlpha}{{\bm \Alpha}}
\newcommand{\bBeta}{{\bm \Beta}}
\newcommand{\bGamma}{{\bm \Gamma}}
\newcommand{\bDelta}{{\bm \Delta}}
\newcommand{\bEpsilon}{{\bm \Psilon}}
\newcommand{\bVarepsilon}{{\bm \Varepsilon}}
\newcommand{\bZeta}{{\bm \Zeta}}
\newcommand{\bEta}{{\bm \Eta}}
\newcommand{\bTheta}{{\bm \Theta}}
\newcommand{\bIota}{{\bm \Iota}}
\newcommand{\bKappa}{{\bm \Kappa}}
\newcommand{\bLambda}{{\bm \Lambda}}
\newcommand{\bMu}{{\bm \Mu}}
\newcommand{\bNu}{{\bm \Nu}}
\newcommand{\bXi}{{\bm \Xi}}
\newcommand{\bOmicron}{{\bm \Omicron}}
\newcommand{\bPi}{{\bm \Pi}}
\newcommand{\bRho}{{\bm \Rho}}
\newcommand{\bSigma}{{\bm \Sigma}}
\newcommand{\bTau}{{\bm \Tau}}
\newcommand{\bUpsilon}{{\bm \Upsilon}}
\newcommand{\bPhi}{{\bm \Phi}}
\newcommand{\bChi}{{\bm \Chi}}
\newcommand{\bPsi}{{\bm \Psi}}

% \int_\Omega \bfx^2 \rmd \bfx
\newcommand{\rmd}{\,\mathrm d}
\newcommand{\bfzero}{\mathbf 0}

%% 算子
\newcommand{\ospan}{\operatorname{span}}
\newcommand{\odiv}{\operatorname{div}}
\newcommand{\otr}{\operatorname{tr}}
\newcommand{\ograd}{\operatorname{grad}}
\newcommand{\orot}{\operatorname{rot}}
\newcommand{\ocurl}{\operatorname{curl}}
\newcommand{\odist}{\operatorname{dist}}
\newcommand{\osign}{\operatorname{sign}}
\newcommand{\odiag}{\operatorname{diag}}
\newcommand{\oran}{\operatorname{Ran}} % 像空间
\newcommand{\oker}{\operatorname{Ker}} % 核空间
\newcommand{\ore}{\operatorname{Re}} % 实部
\newcommand{\oim}{\operatorname{Im}} % 虚部
\newcommand{\orank}{\operatorname{rank}}
\newcommand{\ovec}{\operatorname{vec}}
\newcommand{\odet}{\operatorname{det}}
\newcommand{\odim}{\operatorname{dim}}
\newcommand{\osym}{\operatorname{sym}}

\newcommand{\obcurl}{\operatorname{\bf curl}}
%%%% 个性设置结束
%%%%%%%%------------------------------------------------------------------------


%%%%%%%%------------------------------------------------------------------------
%%%% bibtex设置

%% 设定参考文献显示风格
% 下面是几种常见的样式
% * plain: 按字母的顺序排列,比较次序为作者、年度和标题
% * unsrt: 样式同plain,只是按照引用的先后排序
% * alpha: 用作者名首字母+年份后两位作标号,以字母顺序排序
% * abbrv: 类似plain,将月份全拼改为缩写,更显紧凑
% * apalike: 美国心理学学会期刊样式, 引用样式 [Tailper and Zang, 2006]

%\makeatletter
%\@ifclassloaded{beamer}{
%\bibliographystyle{apalike}
%}{
%\bibliographystyle{abbrv}
%}
%\makeatother


%%%% bibtex设置结束
%%%%%%%%------------------------------------------------------------------------

%%%%%%%%------------------------------------------------------------------------
%%%% xeCJK相关宏包

\usepackage{xltxtra, fontenc, xunicode}
\usepackage[slantfont, boldfont]{xeCJK}

\setlength{\parindent}{1.5em}%中文缩进两个汉字位

%% 针对中文进行断行
\XeTeXlinebreaklocale "zh"

%% 给予TeX断行一定自由度
\XeTeXlinebreakskip = 0pt plus 1pt minus 0.1pt

%%%% xeCJK设置结束
%%%%%%%%------------------------------------------------------------------------

%%%%%%%%------------------------------------------------------------------------
%%%% xeCJK字体设置

%% 设置中文标点样式,支持quanjiao、banjiao、kaiming等多种方式
\punctstyle{kaiming}

%% 设置缺省中文字体
\setCJKmainfont[BoldFont={Adobe Heiti Std}, ItalicFont={Adobe Kaiti Std}]{Adobe Song Std}
%\setCJKmainfont{Adobe Kaiti Std}
%% 设置中文无衬线字体
%\setCJKsansfont[BoldFont={Adobe Heiti Std}]{Adobe Kaiti Std}
%% 设置等宽字体
%\setCJKmonofont{Adobe Heiti Std}

%% 英文衬线字体
\setmainfont{DejaVu Serif}
%% 英文等宽字体
\setmonofont{DejaVu Sans Mono}
%% 英文无衬线字体
\setsansfont{DejaVu Sans}

%% 定义新字体
\setCJKfamilyfont{song}{Adobe Song Std}
\setCJKfamilyfont{kai}{Adobe Kaiti Std}
\setCJKfamilyfont{hei}{Adobe Heiti Std}
\setCJKfamilyfont{fangsong}{Adobe Fangsong Std}
\setCJKfamilyfont{lisu}{LiSu}
\setCJKfamilyfont{youyuan}{YouYuan}

%% 自定义宋体
\newcommand{\song}{\CJKfamily{song}}
%% 自定义楷体
\newcommand{\kai}{\CJKfamily{kai}}
%% 自定义黑体
\newcommand{\hei}{\CJKfamily{hei}}
%% 自定义仿宋体
\newcommand{\fangsong}{\CJKfamily{fangsong}}
%% 自定义隶书
\newcommand{\lisu}{\CJKfamily{lisu}}
%% 自定义幼圆
\newcommand{\youyuan}{\CJKfamily{youyuan}}

%%%% xeCJK字体设置结束
%%%%%%%%------------------------------------------------------------------------

%%%%%%%%------------------------------------------------------------------------
%%%% 一些关于中文文档的重定义
\newcommand{\chntoday}{\number\year\,年\,\number\month\,月\,\number\day\,日}
%% 数学公式定理的重定义

%% 中文破折号,据说来自清华模板
\newcommand{\pozhehao}{\kern0.3ex\rule[0.8ex]{2em}{0.1ex}\kern0.3ex}

\makeatletter %
\@ifclassloaded{beamer}{

}{
\newtheorem{example}{例}
\newtheorem{theorem}{定理}[section]
\newtheorem{definition}{定义}
\newtheorem{axiom}{公理}
\newtheorem{property}{性质}
\newtheorem{proposition}{命题}
\newtheorem{lemma}{引理}
\newtheorem{corollary}{推论}
\newtheorem{remark}{注解}
\newtheorem{condition}{条件}
\newtheorem{conclusion}{结论}
\newtheorem{assumption}{假设}
}
\makeatother

\makeatletter %
\@ifclassloaded{beamer}{

}{
%% 章节等名称重定义
\renewcommand{\contentsname}{目录}
\renewcommand{\indexname}{索引}
\renewcommand{\listfigurename}{插图目录}
\renewcommand{\listtablename}{表格目录}
\renewcommand{\appendixname}{附录}
\renewcommand{\appendixpagename}{附录}
\renewcommand{\appendixtocname}{附录}
\@ifclassloaded{book}{

}{
\renewcommand{\abstractname}{摘要}
}
}
\makeatother

\renewcommand{\figurename}{图}
\renewcommand{\tablename}{表}

\makeatletter
\@ifclassloaded{book}{
\renewcommand{\bibname}{参考文献}
}{
\renewcommand{\refname}{参考文献}
}
\makeatother

\floatname{algorithm}{算法}
\renewcommand{\algorithmicrequire}{\textbf{输入:}}
\renewcommand{\algorithmicensure}{\textbf{输出:}}

\renewcommand{\today}{\number\year 年 \number\month 月 \number\day 日}

%%%% 中文重定义结束
%%%%%%%%------------------------------------------------------------------------

\begin{document}
\title{2维空间的分片多项式}
\author{王鹏祥}
\date{\chntoday}
\maketitle
\tableofcontents
\newpage

\section{知识预备}

\begin{theorem}[divergence therom]
设 $\Omega$ 是 $\mbR^2$ 上的域, 边界为 $\partial\Omega$, 外单位法线为 $n$, 则有

\begin{equation}
\int_{\Omega} \frac{\partial f}{\partial x_{i}} d x=\int_{\partial \Omega} f n_{i} d s, \quad i=1,2
\end{equation}

其中$n_{i}$为n在$x_i$上的分量
    
\end{theorem}

由散度定理可以推导出格林定理

\begin{theorem}[Green therom]

\begin{equation}
\int_{\Omega}-\Delta u v d x=\int_{\Omega} \nabla u \cdot \nabla v d x-\int_{\partial \Omega} n \cdot \nabla u v d s
\end{equation}
    
其中uv为向量场函数
\end{theorem}

\section{Poisson方程的有限元方法}

\subsection{possion方程}
找到满足下列条件的u

\begin{equation}\label{1}
    \begin{alignat}{2}
        -\triangle u & =f,\qquad \text{in}\Omega \\
        u & =0, \qquad \text{on}\partial\Omega
    \end{alignat}
\end{equation}

其中f为$\Omega$上的给定函数
\subsection{等价变分方程推导}

引入一类 Sobolev 空间:
\begin{equation}
\mbV^1(\Omega)=\left\{v:\|v\|_{L^{2}(\Omega)}+\|\nabla v\|_{L^{2}(\Omega)}<\infty\right\}
\end{equation}
两个基本要素

\begin{itemize}
	\item 试探函数空间 $u \in \mbV^1_{g_D}(\Omega) = \{u|u \in \mbV^1(\Omega)
        , u|_{\Gamma} = g_D\}$
    \item 检验函数空间 $v \in \mbV^1_0(\Omega) = \{v|v \in \mbV^1(\Omega) 
        , u|_{\Gamma} = 0\}$
\end{itemize}

该问题两个空间一样。通过green定理问题转化为

找出$u \in \mbV^1_{g_D}(\Omega)$使得下列方程成立
\begin{equation}\label{2}
\begin{aligned}
\int_{\Omega} f v d x &=-\int_{\Omega} \Delta u v d x \\
&=\int_{\Omega} \nabla u \cdot \nabla v d x-\int_{\partial \Omega} n \cdot \nabla u v d s \\
&=\int_{\Omega} \nabla u \cdot \nabla v d x  \quad \forall v \in \mbV^1_0(\Omega)
\end{aligned}
\end{equation}

注意:尽管泊松方程的解$\ref{1}$也是变分公式的解$\ref{2}$,但反过来通常是不正确的。这是因为变分形式的解是不需要二次可微的,因此,变分形式有时被称为弱形式。


\subsection{将空间离散}

设$\mcK$是$\Omega$区域的一种三角划分,$V_h$ 是 $\mcK$ 上的连续分片线性空间.$V_{h,0}=\{v\in V_h:v|_{\partial\Omega}=0\}$.

那么形式变为: 求解 $u_h \in \mbV_{h,0}$, 使得

\begin{equation}\label{4}
	\int_{\Omega} \nabla u_h\cdot \nabla v\mrd \bfx = \int_{\Omega} fv \mrd\bfx, \qquad \forall v\in V_{h,0}
\end{equation}

离散化:用$V_{h,0}$的基函数$\left\{\varphi_{i}\right\}_{i=1}^{n_{i}}$ 替代$V_{h,0}$,$u_h$也用这组基函数线性表处,便可以写成线性方程组的形式, 即

\begin{equation}
\begin{aligned}
\int_{\Omega} f \varphi_{i} d x &=\int_{\Omega} \nabla u_{h} \cdot \nabla \varphi_{i} d x \\
&=\int_{\Omega} \nabla\left(\sum_{j=1}^{n_{i}} \xi_{j} \varphi_{j}\right) \cdot \nabla \varphi_{i} d x \\
&=\sum_{j=1}^{n_{i}} \xi_{j} \int_{\Omega} \nabla \varphi_{j} \cdot \nabla \varphi_{i} d x, \quad i=1,2, \ldots, n_{i}
\end{aligned}
\end{equation}

可以简写为

\begin{equation}
b_{i}=\sum_{j=1}^{n_{i}} A_{i j} \xi_{j}, \quad i=1,2, \ldots, n_{i}
\end{equation}

其中

\begin{equation}
    \begin{aligned}
        A_{i, j} & = \int_{\Omega} \nabla \varphi_i \cdot \varphi_j \mrd \bfx,
        i, j = 1, 2, \cdots, n_i \\
        b_i & = \int_{\Omega} f \varphi_i \mrd \bfx, i = 1,2, \cdots, n_i, \\
    \end{aligned}
\end{equation}

用矩阵格式可以写为

\begin{equation}
\bfA \xi = \bfb
\end{equation}

其中$\bfA$为$n_i \times n_i$的刚度矩阵,$\bfb$为$n_i \times 1$的过载向量。 

\section{常用不等式}

\begin{theorem}[Poincar\'{e} Ineqality]

    已知 $\Omega \subset \mbR^2$ 是有界区域. 
    则有存在常数 $C = C(\Omega)$, 使得对 $\forall v \in \mbV_0$, 都有

    \begin{equation}
        \|v\|_{L^2(\Omega)} \le C \|\nabla v\|_{L^(\Omega)}
    \end{equation}
    
\end{theorem}


\begin{theorem}[Trace Ineqality]
    已知 $\Omega \subset \mbR^2$ 是有界区域
    并且边界 $\partial \Omega$ 是凸的光滑多边形. 
    则有存在常数 $C = C(\Omega)$, 使得对 $\forall v \in \mbV$, 都有

    \begin{equation}
        \|v\|_{L^2(\partial \Omega)} \le C 
        (\|v\|^2_{L^2(\Omega)} + \|\nabla v\|^2_{L^(\Omega)})^{\frac{1}{2}}
    \end{equation}
\end{theorem}


\begin{theorem}[Elliptic Regularity]
    已知 $\Omega \subset \mbR^2$ 是凸的有界区域. 
    并且边界是凸, 光滑的多边形边界. 
    则有存在常数 $C = C(\Omega)$, 使得对任意的足够光滑的函数 $v$, 在边界上都有
    $v = 0$ 或 $\bfn \cdot \nabla v = 0$, 且

    \begin{equation}
        \|D^2 v\|_{L^2(\Omega)} \le C \|\Delta v\|_{L^(\Omega)}
    \end{equation}

    若 $\Omega$ 是凸区域, $0 < C \le 1$.否则, $C > 1$.
\end{theorem}


\begin{theorem}[Inverse Estimate]
    在均匀 (quasi-uniform) 网格上, $\forall v \in \mbV_h$ 
    都满足逆估计

    \begin{equation}
        \|\nabla v\|_{L^2(\Omega)} \le C h^{-1} \|v\|_{L^(\Omega)}
    \end{equation}
\end{theorem}

\section{有限元的基本理论}

\subsection{存在性和唯一性}
\begin{theorem}
    由式 (\ref{4}) 定义的有限元数值解 $u_h$ 存在且唯一. 
\end{theorem}

\subsection{先验误差}
\begin{theorem}[Galerkin Orthogonality]
    由式 (\ref{4})定义的有限元数值解 $u_h$ 满足正交性
    \begin{equation}
        \int_{\Omega} \nabla(u - u_h) \cdot \nabla v \mrd \bfx = 0, \forall v
        \in \bfV_{h, 0}
    \end{equation}
\end{theorem}


\begin{theorem}[最优估计]
    由式 (\ref{4}) 定义的有限元数值解 $u_h$ 满足最优估计

    \begin{equation}
        |||u - u_h||| \le C |||\nabla v|||, \forall v \in \bfV_{h, 0}
    \end{equation}

    其中$|||v|||$为能量范数,定义为

    \begin{equation}
        |||v|||^2  = \int_{\Omega} \nabla v \cdot \nabla v \mrd \bfx.
    \end{equation}
   
\end{theorem}

\begin{theorem}
    由式(\ref{4})定义的有限元解 $u_h$, 满足如下估计:
    $$
    |||u-u_h|||^2\le C\sum_{K\in\mcK}h_k^2\|D^2u\|^2_{L^2(K)}
    $$
\end{theorem}

上面式子也可以用$L_2$形式的误差来写

\begin{equation}
\|u-u_h\|_{L^2(\Omega)}\le C\|u-u_h\|\le Ch\|D^2u\|_{L^2(\Omega)}
\end{equation}

\subsection{刚度矩阵性质}

\begin{theorem}
    刚度矩阵 A 是对称正定的, 并且条件数满足

    \begin{equation}
        \kappa(A) \le Ch^{-2}.
    \end{equation}
\end{theorem}

这里正定仅限于边界为0的Dirchlet条件,条件数为最大和最小特征值的比值。
\section{变系数的模型问题}
找到 $u$, 满足:

\begin{alignat}{2}
    -\nabla\cdot(a\nabla u) & =f,\quad\text{in}\quad\Omega \\
    -n\cdot(a\nabla u)=\kappa(u-g_D)-g_N, \quad\text{on}\quad\partial\Omega
\end{alignat}

其中 $a>0,f,\kappa>0$, $g_D$和$g_N$ 是已知函数.

在空间 $V=\{v:\|v\|_{L^2(\Omega)}+\|\nabla v\|_{L^2(\Omega)}<\infty\}$ 中寻找该
问题的解.

\\ \hspace*{\fill} \\

通过green公式,离散空间后可得到其有限元的模型

即找到 $u\in V_h\subset V$ 满足:

\begin{equation}
\int_{\Omega}a\nabla u_h\cdot\nabla vdx+\int_{\partial\Omega}\kappa
u_hvds=\int_{Omega}fvdx+\int_{\partial\Omega}(\kappa g_D+g_N)vds,\quad\forall
v\in V_h
\end{equation}

\\ \hspace*{\fill} \\

具体写成矩阵形式为

$$
(A+R)\xi=b+r
$$

\begin{alignat}{4}
	A_{ij} & =\int_{\Omega}a\nabla\varphi_i\cdot\nabla\varphi_jdx \\
	R_{ij} & =\int_{\partial\Omega}\kappa\varphi_i\varphi_jds \\
	b_{ij} & =\int_{\Omega}f\varphi_idx \\
	r_i & =\int_{\varphi\Omega}(\kappa g_D+g_N)\varphi_ids \\
\end{alignat}

\section{边界问题}

\subsection{Dirichlet 边界}
\subsubsection{问题模型}
\begin{equation}
    \begin{aligned}
        - \Delta u & = f, \text{in} \ \Omega, \\
        u & = g_D, \text{on}\ \partial \Omega.
    \end{aligned}
\end{equation}
其中f和$g_D$给定

\subsubsection{空间}
\begin{itemize}
	\item 试探函数空间 $V_{g_{D}}=\left\{v:\|v\|_{L^{2}(\Omega)}+\|\nabla v\|_{L^{2}(\Omega)}<\infty,\left.v\right|_{\partial \Omega}=g_{D}\right\}$
    \item 检验函数空间 $V_0  = \left\{v:\|v\|_{L^{2}(\Omega)}+\|\nabla v\|_{L^{2}(\Omega)}<\infty,\left.v\right|_{\partial \Omega}=0\right\}$
    \item 离散后
    \item $V_{h,g_{D}}=\left\{v \in V_h:\left.v\right|_{\partial \Omega}=g_{D}\right\}$
    \item $V_{h,0}  = \left\{v \in V_h:\left.v\right|_{\partial \Omega}=0\right\}$
\end{itemize}
     其中$V_h$为连续线性分段空间
     
\subsubsection{变分形式}
找到属于$u \in V_{g_{D}} $使得下式成立

\begin{equation}
\int_{\Omega} \nabla u \cdot \nabla v d x=\int_{\Omega} f v d x, \quad \forall v \in V_{0}
\end{equation}

\subsubsection{有限元形式}
找到属于$u \in V_{h,g_{D}} $使得下式成立

\begin{equation}
\int_{\Omega} \nabla u_{h} \cdot \nabla v d x=\int_{\Omega} f v d x, \quad \forall v \in V_{h, 0}
\end{equation}

使用 Galerkin 正交的性质, 我们可以把内部的点和边界的点分开, 
即内部的值和边界的值相互独立.

\subsection{Neumann 边界}

\subsubsection{问题模型}
\begin{equation}
    \begin{aligned}
        - \Delta u & = f, \text{in} \ \Omega, \\
        \nabla u \cdot \bfn & = g_N, \text{on}\ \partial \Omega.
          \int_{\Omega} u d x & =0
    \end{aligned}
\end{equation}
其中f和$g_N$给定


\subsubsection{空间}
\begin{itemize}
	\item $V = {v:\|v\|_{L^{2}(\Omega)}+\|\nabla v\|_{L^{2}(\Omega)}<\infty}$
    \item $\overline{V}=\{v\in V:\int_{\Omega}vdx=0\}$
    \item $\overline{V_h}$ 是所有平均值为0的连续分段线性空间.
\end{itemize}
     其中$V_h$为连续线性分段空间
     
\subsubsection{变分形式}
找到$u \in V $使得下式成立
\begin{equation}
\int_{\Omega} \nabla u \cdot \nabla v d x=\int_{\Omega} f v d x+\int_{\partial \Omega} g_{N} v d s, \quad \forall v \in V
\end{equation}

\subsubsection{有限元形式}
找到属于$u_h\in\overline{V_h}\subset\overline{V}$使得下式成立
\begin{equation}
\int_{\Omega} \nabla u_{h} \cdot \nabla v d x=\int_{\Omega} f v d x+\int_{\partial \Omega} g_{N} v d s, \quad \forall v \in \bar{V}_{h}
\end{equation}

在 Neumann 条件下, 拉格朗日乘子趋于零.

\subsection{特征值问题}

\subsubsection{问题模型}
解出函数u和数$\lambda$
\begin{equation}
    \begin{aligned}
        - \Delta u & = \lambda u, \text{in} \ \Omega, \\
        \nabla u \cdot \bfn & = 0, \text{on}\ \partial \Omega.
    \end{aligned}
\end{equation}


\subsubsection{变分形式}
找到 $u\in V$ 与 $\lambda\in\mbR$ 满足:

\begin{equation}
    \int_{\Omega}\nabla u\cdot\nabla vdx=\lambda\int_{\Omega}uvdx,\forall v\in V
\end{equation}

\subsubsection{有限元形式}
找到 $u_h\in V_h$且 $\lambda_k\in\mcR$满足:

\begin{equation}
\int_{\Omega}\nabla u_h\cdot\nabla vdx=\lambda_k\int_{\Omega}u_kvdx,\forall
v\in V_k
\end{equation}

最后我们需要求解的线性方程组是 

\begin{equation}
    \bfA \boldsymbol {\xi} = \lambda \bfM \boldsymbol{\xi}
\end{equation}

其中 $\boldsymbol \xi$ 是 节点处的向量. 我们需要求解的是特征对 $(\xi_n,
\lambda_n), n = 1, \cdots, n_p$. 

\section{有限元方法步骤总结}
\begin{itemize}
\item 在等式两边乘以测试函数并积分
\item 用散度定理简化微分次数
\item 带入边界条件
\item 将模型变为变分洗模型
\item 将空间都变为分片线性连续空间
\item 将方程变为向量矩阵形式,分别组装刚度矩阵和过载向量
\end{itemize}

\section{自适应有限元}

\begin{theorem}[后验误差估计]
    由式 (\ref{4}) 定义的有限元解 $u_h$, 满足估计

    \begin{equation}
        ||| u - u_h |||^2 \le C \sum_{K\in \mcK} \eta_K^2(u_h)
    \end{equation}

    其中, 定义单元残量 $\eta_K(u_h)$ 

    \begin{equation}
        \eta_K(u_h) = h_K\|f + \Delta u_h\|_{L^2(K)}
                    + \frac{1}{2}h_K^{1/2}\|[\bfn \cdot \nabla
                    u_h]\|_{L^2(\partial K \setminus \partial \Omega)}
    \end{equation}

    $[\cdot]$ 代表的是跳量, 由于在 $K$ 上 $u_h$ 是线性的, 因此$\Delta u_h = 0$

\end{theorem}

\subsection{自适应网格加密}

高维网格加密要考虑几个问题问题

\begin{itemize}
    \item 消除悬挂点, 尽可能少加密单元
    \item 要保证网格的最小角尽可能的大,以提高有限元解的质量
    \item 细化网格比未细化网格更能准确描述区域
\end{itemize}

以下是两种常用的方法

\subsubsection{Rivara 加密}

基本概念
\begin{itemize}
\item treminal edges:如果一条边是共享这条边的两个三角形中最长的边,那么这个变就叫treminal edges。
\item terminal star:共享terminal edges的两个三角形。
\item longest edge propagation path(LEPP):就是从K开始,依次移动到相邻的边最长的三角形,直到到达terminal edges的三角形序列
\end{itemize}

具体方法是,找到从k开始找到terminal edges然后用这条边的中点与相邻三角形(terminal star)的两个顶点相连,从而加密网格,每重新开始初始k都不变,最终当LEEP重复时候终止。

这种方法的缺点是:网格剖分较细时, 代价昂贵

\subsubsection{Regular 加密}

两种加密
\begin{itemize}
\item red refinement:将要细化的三角形各个边的中点相连接,划分成四个新的全等三角形
\item green refinement:因为red refinement会出现悬挂点,将悬挂点个相邻三角形的定点相连接。
\end{itemize}

具体方法是,选定要细化的单选,但后用red refinement加密,若出现悬挂点再用green refinement,迭代知道没有悬挂点出现为止。

显然red refinement加密质量更高,后续尽量不让green refinement过的三角形再次green refinemrnt.

\end{document}
