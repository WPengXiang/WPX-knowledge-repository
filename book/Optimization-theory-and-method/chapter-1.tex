% !Mode:: "TeX:UTF-8"

\documentclass{article}
\input{../../en_preamble.tex}
\input{../../xecjk_preamble.tex}
\begin{document}
\title{第一章引论}
\author{王鹏祥}
\date{\chntoday}
\maketitle
\tableofcontents
\newpage

\section{基本格式}
最优化问题一般形式为
\begin{align}
\min f(\bfx) \\
s.t. \bfx \in \bfX
\end{align}
其中$\bfx \in \mcR^n$是决策变量,$f(x)$为目标函数,
$\bfX \subset \mcR^n$为约束集或可行域。特别的,如果
约束集$\bfX = \mcR^n$,则为无约束最优化问题
$$
\min_{\bfx \in \mcR^n} f(\bfx)
$$
约束最优化问题通常写为
\begin{align*}
\min f(\bfx) \\
s.t. c_i(\bfx) &= 0 , i \in \mcE \\
     c_i(\bfx) &\geq 0 , i \in \mcI
\end{align*}
这里E,I是指标集

当目标函数和约束函数为现行函数,问题称为线性规划。当目标函数
和约束函数至少有一个是变量$\bfx$的非线性函数时,问题成为非线性规划。

\section{数学基础}
\subsection{范数}
\begin{definition}{范数}
映射$\| \cdot \| : R^n \rightarrow R$,具有下列性质,则称为范数:
\begin{itemize}
\item[1.] $\| \bfx \| \geq 0 , \forall \bfx \in R^n$ 
\item[2.] $\| \alpha \bfx  \| = \left| \alpha \right| \| \bfx \| 0 , \forall \alpha \in R,\forall \bfx \in R^n$ 
\item[3.] $\| \bfx + \bfy \| \leq \| \bfx \| +\| \bfy \| , \forall \bfx,\bfy \in R^n$ 
\item[4.] $\| \bfx \| = 0 \Leftrightarrow \bfx = 0$ 
\end{itemize}
\end{definition}

几种常用的向量范数
\begin{itemize}
\item $l_p$向量范数:$1 \leq p \leq \infty $
$$
\| \bfx \|_p = (\sum_{i=1}^n |\bfx_i|^p)^{\frac{1}{p}} 
$$
\item 椭球向量范数:设$x \in R^n,A \in R^{n \times n}$是对称正定矩阵
$$
\|x\|_A = (x^T A x)^{1/2}
$$
\end{itemize}

\begin{definition}{矩阵范数}
设$\bfA \in R^{n \times n}$,其诱导矩阵范数定义为
$$
\|\bfA\| = \max_{\bfx \neq 0}\{ \frac{\| \bfA \bfx \| }{\|\bfx\|} \}
$$
\end{definition}
几种常用矩阵范数
\begin{itemize}
\item 列和范数:每一列绝对值相加,取最大
	$$
	\|A\|_1=\max_j \sum^n_{i=1}|a_{ij}|
	$$
\item 行和范数:每一行绝对值相加,取最大
	$$
	\|A\|_{\infty}=\max_i \sum^n_{j=1}|a_{ij}|
	$$
\item 谱范数:$\lambda_{A^T A}$表示$A^T A$的最大特征值
	$$
	\|A\|_{2}=(\lambda_{A^T A})^{1/2}
	$$
\item Frobenius范数:
	$$
	\|A\|_{F}=[tr(A^t A)]^{1/2}
	$$
\end{itemize}

\subsection{矩阵}
$\|A^{-1}(B-A)\| < 1$代表矩阵B充分接近矩阵A
\begin{theorem}[von Neumann引理]

设$A,B \in R^{n \times n}$ A可逆,$\|A^{-1} \leq \alpha$,如果$\|A-B\| \leq \beta$,$\alpha \beta < 1$,则B可逆且
$$
\|B^{-1}\| \leq \frac{\alpha}{1-\alpha \beta}
$$
\remark{这个定理表明B充分接近可逆矩阵A时,B也可逆}
\end{theorem}

\begin{definition}{广义逆}
设$A \in C^{m \times n}$(复矩阵空间),则A的广义逆为$A^{+} \in C^{m \times n}$,它满足
$$
AA^{+}A = A, A^{+}AA^{+} = A^{+}, (AA^+)^*=AA^+, (A^+A)^*=A^+A
$$
\end{definition}
\end{document}
