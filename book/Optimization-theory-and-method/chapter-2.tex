% !Mode:: "TeX:UTF-8"
\documentclass{article}
\input{../../en_preamble.tex}
\input{../../xecjk_preamble.tex}
\begin{document}
\title{一维有限元方法}
\author{王鹏祥}
\date{\chntoday}
\maketitle
\tableofcontents
\newpage
\section{两点边值问题}
\subsection{问题描述}
\begin{align}
	-u'' & = f, x \in I = [0, L] \\
       u(0) & = u(L) = 0
\end{align}

由于很多微分方程的解析解很难求解甚至无法求解,因此我们需要用有限元的方法来求解数值解.

这种问题可适用于热方程传导或者杆形变两种等多种物理模型中。

\subsection{有限元求解推导}
有限元方法的思路是把微分方程改写为一个变分方程,再把变分方程离散化,把无限的空间变为有限的空间。
\subsubsection{变分公式}

\begin{equation}
\begin{aligned}
\int_{0}^{L} f v d x &=-\int_{0}^{L} u^{\prime \prime} v d x \\
&=\int_{0}^{L} u^{\prime} v^{\prime} d x-u^{\prime}(L) v(L)+u^{\prime}(0) v(0) \\
&=\int_{0}^{L} u^{\prime} v^{\prime} d x
\end{aligned}
\end{equation}

其中v被叫做检验函数,u被叫做试探函数。

v所在的空间

\begin{equation*}
V_{0}=\left\{v:\|v\|_{L^{2}(I)}<\infty,\left\|v^{\prime}\right\|_{L^{2}(I)}<\infty, v(0)=v(L)=0\right\}
\end{equation*}

被叫做检验函数空间.

u所在的空间

\begin{equation*}
V_{g_D}=\left\{v:\|v\|_{L^{2}(I)}<\infty,\left\|v^{\prime}\right\|_{L^{2}(I)}<\infty, v(0)=v(L)=g_D\right\}
\end{equation*}

被叫做试探函数空间(弱解所在的空间)

在这个问题中由于边界值都为0,所一两个空间一样

\subsubsection{将空间离散化(Galerkin method)}

将无限的空间变为有限的.

将函数空间离散化变为

\begin{equation}
V_{h, 0}=\left\{v \in V_{h}: v(0)=v(L)=0\right\}
\end{equation}

其中$V_{h}$为

\begin{equation}
V_{h}=\left\{v: v \in C^{0}(I),\left.v\right|_{I_{i}} \in P_{1}\left(I_{i}\right)\right\}
\end{equation}

那么离散形式为: 求解 $u_h \in V_{h,0}$, 使得

\begin{equation}\label{1}
\int_{I} u_{h}^{\prime} v^{\prime} d x=\int_{I} f v d x, \quad \forall v \in V_{h, 0}
\end{equation}

上述这种方法就叫做Galerkin方法。

\subsubsection{基函数带入}

之后我们讲$u_h \in V_{h,0}$的一组基函数带入\ref{1}中得到

\begin{equation}
\int_{I} u_{h}^{\prime} \varphi_{i}^{\prime} d x=\int_{I} f \varphi_{i} d x, \quad i=1,2, \ldots, n-1
\end{equation}

再讲$u_h$用基函数线性表出

\begin{equation}
u_{h}=\sum_{j=1}^{n-1} \xi_{j} \varphi_{j}
\end{equation}

其系数待定。

最后都带入到\ref{1}中 

\begin{equation}
\begin{aligned}
\int_{I} f \varphi_{i} d x &=\int_{I}\left(\sum_{j=1}^{n-1} \xi_{j} \varphi_{j}^{\prime}\right) \varphi_{i}^{\prime} d x \\
&=\sum_{j=1}^{n-1} \xi_{j} \int_{I} \varphi_{j}^{\prime} \varphi_{i}^{\prime} d x, \quad i=1,2, \ldots, n-1
\end{aligned}
\end{equation}

则写成矩阵形式为

\begin{equation}
	\begin{pmatrix}
		(\varphi'_1, \varphi'_1) & (\varphi'_1, \varphi'_2) & \cdots & (\varphi'_1, \phi'_{n-1}) \\
		(\varphi'_2, \varphi'_1) & (\varphi'_2, \varphi'_2) & \cdots & (\varphi'_2, \varphi'_{n-1}) \\
		\vdots & \vdots & \cdots & \vdots \\
	    (\varphi'_{n-1}, \varphi'_1) & (\varphi'_{n-1}, \varphi'_2) & \cdots & (\varphi'_{n-1}, \varphi'_{n-1}) \\
	\end{pmatrix}
	\begin{pmatrix}
		\xi_1 \\
		\xi_2 \\
		\vdots \\
		\xi_{n-1}
	\end{pmatrix} 
	= \begin{pmatrix}
		(f_1, \varphi_1) \\
		(f_2, \varphi_2) \\
		\vdots \\
		(f_{n-1}, \varphi_{n-1}) \\
	\end{pmatrix}
\end{equation}

其中左端项第一个矩阵为刚度矩阵,右端项向量为载荷向量


\subsection{误差分析}
\begin{proposition}
用有限元方法估计出的$u_h$满足如下公式

\begin{equation}
\int_{I}\left(u-u_{h}\right)^{\prime} v^{\prime} d x=0, \quad \forall v \in V_{h, 0}
\end{equation}

\begin{equation}
\left\|\left(u-u_{h}\right)^{\prime}\right\|_{L^{2}(I)} \leq\left\|(u-v)^{\prime}\right\|_{L^{2}(i)}, \quad \forall v \in V_{h, 0}
\end{equation}

\begin{equation}
\left\|\left(u-u_{h}\right)^{\prime}\right\|_{L^{2}(I)}^{2} \leq C \sum_{i=1}^{n} h_{i}^{2}\left\|u^{\prime \prime}\right\|_{L^{2}\left(I_{i}\right)}^{2}
\end{equation}

\end{proposition}

先验误差和后验误差,区别在于一个是对u的估计,一个是对$u_h$的估计

\section{三种边界条件}
三种边界条件主要是看$Ay+By'=C$的系数情况
\begin{itemize}
\item Diechlet边界条件(第一类边界条件):A$\neq$0,B=0,即给出未知函数在边界上的数值

\item Neumann边界条件(第二类边界条件):A=0,B$\neq$0,即给出未知函数在边界外法线的方向导数

\item Robin边界条件(第三类边界条件):A$\neq$0,B$\neq$0,即给出未知函数在边界上的函数值和外法线的方向导数的线性组合
\end{itemize}

%\cite{fem_2010}
%\bibliographystyle{abbrv}
%\bibliography{../../ref}
\end{document}
