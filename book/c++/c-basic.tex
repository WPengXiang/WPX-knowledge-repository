% !Mode:: "TeX:UTF-8"
\documentclass{article}
\input{../../en_preamble.tex}
\input{../../xecjk_preamble.tex}
\begin{document}
\title{c基础}
\author{王鹏祥}
\date{\chntoday}
\maketitle
\tableofcontents
\newpage

\section{基础}
\textbf{创建c++程序的步骤}
\begin{itemize}
\item 使用文本编辑器创建源代码
\item 使用编译器将源代码转换为目标文件
\item 使用链接器链接目标文件和必要的库,生成可执行的程序
\item 输入可执行文件的名称以运行它
\end{itemize}

\textbf{面向对象编程}
\begin{itemize}
\item 封装:c++支持创建用户定义的类型来封装属性,这种类型称为类。使用时候不需要知道工作原理。
\item 继承和重用:可将新类型声明为现有类型的拓展,新子类称为集成了现有类型。
\item 多态:用一样东西有多种形态
\end{itemize}

\textbf{程序的组成部分}
\begin{itemize}
\item 预处理:用\# 表示,include如果用的是<>,代表在标准位置寻找,“”代表在相对位置寻找
\item 源代码行:每个程序必有返回整形的main()函数.通常程序返回0表示成功,返回其他数字表示出现了某种故障
\end{itemize}

\textbf{函数} \par
程序从main()函数开头开始,逐行执行源代码。函数由函数头和函数体组成,函数头包含
\begin{itemize}
\item 函数的返回类型:void可以不用return
\item 函数名:不能包含空格,每个单词首字母大写
\item 函数接受的参数:实参是穿进来的数据,收到的实参存储在形参中
\end{itemize}
\section{常量}
当全局变量和局部变量的名字一样,在函数内局部变量的值会覆盖全局变量的值,局部变量不会自动初始化,全局变量会自动初始化。一个字节8位。定义常量最好用const,最好不要用预定义,编译器无法确定值是否合适。
\begin{itemize}
\item 整数常量
\begin{itemize}
\item 0x十六进制
\item 0表示8进制
\item 默认十进制
\item 后缀U表示无符号
\item 后缀L表示长整数
\end{itemize}
\item 浮点常量
\begin{itemize}
\item e或E表示指数
\end{itemize}
\item 字符常量
\begin{itemize}
\item $\setminus$ t 水平制表符
\item $\setminus$ v 垂直制表符
\item $\setminus$ f 换页
\item $\setminus$ n 换行
\end{itemize}
\item 定义常量
\begin{itemize}
\item \# define identifier value 预处理,定义在所有函数外
\item const type varible = value 定义在函数内
\end{itemize}
\end{itemize}

\section{运算符}
\begin{itemize}
\item 数学运算符中/,只有含有浮点型时候出的结果才是浮点数,要是都是整数相除,出来的是被舍弃小数的整数。
\item ++当放到变量前后是不一样的,放到前面先加到变量上再赋值,放到后面先赋值,再加到变量上。
\item 注意=是赋值,==是判断是否相等返回布尔值
\item 多层逻辑运算符最好加括号让程序变得更易读
\end{itemize}

\section{函数}
\begin{itemize}
\item 通常命名规则是除第一个单词,其他单词首字母大写
\item 函数定义在main()前面,则不需要声明,定义在后面需要声明一下
\item 局部变量:函数内创建的变量为局部变量,函数执行完后,局部变量将会消失
\item 全局变量:在函数(包括main)外定义的变量,在程序的任何地方都可以使用,尽量不要使用全局变量,出现问题难以查找
\item 形参:对函数形参的修改不会影响调用它的函数内部的值,这成为{\red 按值传递},这会创建每个实参的本地拷贝;形参可以设定默认值,如果某个形参没有默认值,那么按前面的任何形参都不能有默认值,如果摸个形参由默认值,那么他后面的都要有;可以在声明中给定默认参数。
\item 函数return后的语句都不会执行
\item 函数重载(多态):c++可有多个同名函数,只要形参(数量、类型)不同就可以,不可以通过只修改返回值类型来重载函数。
\item 内联函数:了解决程序中函数调用的效率问题,以空间代价换时间;声明时候加上inline,编译器不会创建该函数,而是将代码直接复制到代用他的地方,如果函数的语句比较短(1-5行),且使用频繁,这会减少程序调用的时间。内联函数内最好不要使用循环和开关语句
\item 可通过auto定义函数类型,减少思考量,增加程序的灵活性。使用auto定义函数,必须将函数定义放在main前面.
\end{itemize}

\section{控制程序流程}
\begin{itemize}
\item do while:再循环语句末尾检查条件,至少会执行一次
\item for循序:将设置计算器变量、检查计数器变量是都满足条件,修改计数器变量三步放到一起,比while更加方便。
\item switch:case部分只能进行相等比较,不能进行关系运算和布尔运算;switch中总应该包含default部分;每个case都应该包含break,否则会继续运行下一个。
\item countine跳过本次循环,进行下一次循环,break跳出循环体
\item switch if比较:switch让检查同一个变量的多种可能取值更简单,比if-else更容易开发和调试,多个else考虑用switch。进行比较测试如a>b时候只能用if-else
\item 如果循环体至少执行一次考虑do-while否则用while
\item 无需每次循环都递增计算器变量时候用while
\item foreach循环:for(int a:name),冒号前放的是用于存储数组各个元素的变量,冒号后面放的是数组名。
\end{itemize}

\section{数组和字符串}
数组是一系列类型相同的相关数据,可将数组视为一系列数据存储单元,其中每个存储单元都是数组的一个元素。
\begin{itemize}
\item 如果越界会将指定值存储到里第一元素开头16字节的地方,造成程序崩溃,{\red 注意不要越界}
\item sizeof():获得数组地址的长度
\item char name[]="dadad",来初始化字符串,编译器会自动添加空字符
\item cin.gitline:输入字符串,有三个参数(要赋予的量,读取字符个数,终止输入的分隔符)若省略第三个,默认为换行符。
\item strncpy():将整个字符窜复制到指定缓冲区,两个参数,第一个为复制到的,第二个为被复制的,第三个为最多复制数。strcpy只有两个参数
\item 
\end{itemize}

\section{类}
\begin{itemize}
\item class里所有成员变量和成员函数默认都是私有的
\item 私有成员只能在其所属累的函数内访问,共有成员可以再任何地方访问
\item 存取器:用于设置或获取私有成员变量值的函数
\item 如果在class外定义函数,要使用::符号来定义
\item 构造函数:职责为创造一个有效的对象,通常为初始化成员数据。构造函数与类同名切没有返回值,可以有参数;如果没有构造函数,编辑器提供默认构造函数(不执行任何操作)。
\item 析构函数:~加类名,执行清理工作并释放分配给对象的内存,不接受参数,也不返回值。
\item const成员函数:在参数括号后面加 const;代表该函数不可以修改任何类成员的值,否则会报错。
\item 函数定义放在类声明中,函数将自动变成内联的。
\end{itemize}

\section{指针}
\begin{itemize}
\item 指针:存储内存地址的特殊变量
\item 一定要初始化指针,若不知道指向什么可以=nullptr(或者=0,=NULL),其表示不指向任何东西的空指针.
\item 给指针传值的时候要加\& 符号。
\item 使用指针来访问变量的值用*,称为间接访问  
\item 指针的常用作用:管理堆中的数据,访问类的成员数据和成员函数,按引用将变量传递给函数
\item 5个内存区域
	\begin{itemize}
	\item 全局名称空间:全局变量
	\item 堆:自由存储区,不知道放在什么地方,但知道如何获取,程序结束时才会清理堆,在释放前都可以使用。
	\item 栈:局部变量和函数参数,每当函数返回时
	\item 代码空间:代码
	\item 寄存器:内部管理如跟踪栈顶和指令指针
	\end{itemize}
\item 使用关键字new并在后面指定要位置分配内存的对象类型,可以分配堆中的内存,其返回一个内存地址,必须将赋给指针。
\item 使用delete将释放指针指向的内存,将内存归还给堆,删除指针后最好在将其设置为nullptr,若果使用两次delete会使得程序需崩溃。
\item 使用指针时候瑶始终确保其有值,使用nullptr来定义空指针比0和NULL更好
\end{itemize}

\section{高级指针}
\begin{itemize}
\item 可以创建自己定义类的指针,使用delete是将调用析构函数,使用new将调用不接受任何参数的构造函数。程序结束的是时候也将调用析构函数。
\item 
\end{itemize}
\end{document}



































