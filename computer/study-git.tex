% !Mode:: "TeX:UTF-8"
\documentclass[12pt,a4paper]{article}
\input{../en_preamble.tex}
\input{../xecjk_preamble.tex}
\title{git 学习手册}
\author{王鹏详}
\date{2019.7.18}
\begin{document}
	\maketitle
	\tableofcontents
	\newpage

		\section{preface}\setlength{\parindent}{2em}
		之前在实习的时候学的有点匆忙,很多地方不是太明白,这回把我自己在学习git时候的体会和一些心得写一下。目的是希望之后学习git的同志能效率更快一点,不会像我一样耗费很多时间(也还好主要实习占了一些时间)。
		
		刚上手git的时候很懵,不知道这个东西是干什么,看了progit还是不是很了解,一开始照着progit上的操作一步一步打,出现了很多错误(后来才知道是路径没选对)。相对去初学者我更推荐去看廖雪峰的git教程,上面不止是有每一步的代码,还会有视频得出操作,更关键的是其教的东西都是最基本最关键的,很多用不上的里边也不会说,对于初学者来说学习的效率会更快。看完廖雪峰的教程再看progit会更加流畅。
		
		附:廖雪峰git网站https://www.liaoxuefeng.com/wiki/896043488029600

		\section{git基本知识}
		首先学习git前,要先知道为啥要学git,他能帮助我们干什么,我认为只有知道了这个,之后在学习git操作的时候才不会不知所措,一脸懵逼。
		\subsection{git是啥}
	    官方的话来说Git是目前世界上最先进的分布式版本控制系统。我自己的话来就是可以更好的对文件进行管理,外加还可以多人协调工作。
	    \subsection{为啥学git}
	    git的功能可能有很多,各种各样。但我自己认为git就是时光机+任意门。
	    
	    时光机意思是git可以将文件变回到任意时间下它所处的样子,举个例子就像玩游戏,你可以在打boss前保存一下,如果输了,不用从头开始,只要从存档点开始就可以。git就可给文件在任意时刻设定一个存档点,当你编写的文件出错的时候,可以随时从存档点开始。
	    
	    任意门的意思就是,git可以把文件瞬间传输到任意一台电脑上。配个图片易于理解
\begin{figure}[H]
	\centering
    \includegraphics[scale=1]{figures/0.jpg} 	
	\caption{}
	\label{fig:0}
\end{figure}
        每一台电脑间的文件都可以互通的,在实际操作的时候通常会有一个github,github相当于中央处理器,多人合作时候每个人把东西传到github上,再从github下载别人的文件。
        
        官方的话是这样的。git属于分布式版本控制系统,其根本没有“中央服务器”,每个人的电脑上都是一个完整的版本库,这样,你工作的时候,就不需要联网了,因为版本库就在你自己的电脑上。既然每个人电脑上都有一个完整的版本库,那多个人如何协作呢?比方说你在自己电脑上改了文件A,你的同事也在他的电脑上改了文件A,这时,你们俩之间只需把各自的修改推送给对方,就可以互相看到对方的修改了。
        
        \subsection{git工作流程}
        首先要先知道git的三个工作区域和四个工作状态。
        
        三个工作区域是,git仓库,git工作目录,git暂存区域。\\git工作目录这个最好理解,就是我们看文件夹里的文件。\\git仓库是,你是看不见的(可以在github上看),这里边放的是你完成一个项目,要分享给别人东西。\\暂存区域是指当你修改一个文件后,你要将他放到一个地方,这个地方放里的所用东西是准备要提交到git仓库中的。
\begin{figure}[H]
	\centering
	\includegraphics[width=0.7\linewidth]{figures/1.jpg}
	\caption{stage是暂存区域,master就是git仓库}
	\label{fig:1}
\end{figure}

        四个工作状态分别未追踪,未修改,已修改,已暂存            
\begin{figure}[H]
	\centering
	\includegraphics[width=0.7\linewidth]{figures/3.png}
	\caption{}
	\label{fig:3}
\end{figure}
       当你新建一个文件就是属于未追踪,你需要通过add命令把他加到暂存区,这时他就被git追踪到了,之后的每一次修改git都会记录。未修改,已修改很好理解,就是看文件你有没有改动过。修改过得你可以通过add加到加到暂存区。已暂存就是在暂存区里的东西喽。
       
       git的流程就是,将新建或者修改的文件通过add加到暂存区,然后通过commit将暂存区的东西放入git仓库。很简单的。

		\section{基本操作}
	    接下来说一下git的基本操作,可能有很多,我主要就写一些工作时候会用到的(其他的我也不会)
	    
	    \subsection{创建git}
	    两个命令一个是 git   init,就是初始一个git。使用方法为,你到一个文加夹下,右键,然后点击git bash here(你的命令是在当前文件夹下完成的),然后在命令窗口打git init。恭喜你,这个文件夹下的东西你都可以通过git来管理了
	    
	    第二个就是git clone。使用方法为,找一个github上你需要的repository(相当于git仓库),然后记住地址,通过这个命令,便可以将repos上的所有文件搞到你电脑上。
	    \subsection{提交到仓库}
	    首先下来说一下基本命令
	    
	    git status是看现在文件的状态,有啥状态就是之前讲的那些。
	    git add 将文件放到暂存区,也就是说将文件状态变为已暂存
	    git commit 将暂存区的东西都提交到git仓库中
		
		会着三步你就会将文件提交到git仓库里了,还有一个重要的一点git commit命令最好加上一个-m变为git commit -m“备注的内容”,这样每一次提交你就都知道自己干了些啥。还有要学会经常用git status随时知道文件的状态。通过git diff命令也可以知道暂存前后修改的地方在哪。
		
		还有一些额外的命令如忽未跟踪文件用ignore,从暂存区域移除文件git rm,对文件改名git mv。都不太常用就不介绍了。在progit在都用明确的讲解。
		
		\subsection{时光机操作}
		首先要学会用git log查看提交过的版本信息,之后显示的commit id后面带的一些列的数值就是你要退回到那个版本的索引。你也可以通过在git log后面加上--数字,表示之前几次的commit。还有一些--since,--until这里就不详细介绍了。
		
		然后就是回退命令git reset --hard [commit id(可以省略后面的,git会自动帮你填补,带是要唯一)],这个命令变回让你的文件回到你commit的那个版本。
		
		此外还有git reflog命令用来记录你的每一次命令
		
		命令git checkout -- filename意思就是,把文件在工作区的修改全部撤销,这里有两种情况:\\
		一种是readme.txt自修改后还没有被放到暂存区,现在,撤销修改就回到和版本库一模一样的状态;\\
		一种是readme.txt已经添加到暂存区后,又作了修改,现在,撤销修改就回到添加到暂存区后的状态。\\
		总之,就是让这个文件回到最近一次git commit或git add时的状态。
		
		\subsection{任意门:远程仓库的管理}
		这里远程仓库主要指的是github上的repository
		
		首先是添加远程仓库,命令为git remote add <hort name><url>,short name代表的是整个url的简写。这个命令我理解的是你可以和你github上的仓库关联在一起了,你可以通过get remote show <short name>来查看你远程仓库的信息。
		
		接着是往远程仓库推送文件,命令为git remote push <short name> <banch name> banch name代表的是分支的名称默认为master,之后你就可把你的git仓库里的文件推送到远程仓库了
		
		从远程仓库拉取数据是通过git pull <url>便可以将你远程的仓库里文件传到你的本地工作目录里。这里还可以用git fetch 但是这条命令只会将文件搞到你git 仓库中,并不会自动合并会修改你的当前工作区间,还需要通过git merge来合并。还有就是git clone已经在前面讲过这个是完全克隆到你自己的仓库,通常是初始用的。
		
		最后是远程仓库的重命名git remote rename。远程仓库的移除命令为git remote rm  <short name>
		
		注:第一次推送或克隆可能会遇到SSH警告,具体请查看廖雪峰的网站。
		
		

\section{常用操作流程}
\begin{itemize}
\item git status -s  查看状态
	\begin{itemize}
	\item 第一列是对staging区域而言,第二列是对working目录而言
    \item ?? 新添加的未跟踪文件	
	\item A 新添加的
	\item M 修改的
	\item D 你本地删除的文件(服务器上还在)
	\item U 文件没有被合并
	\item T 文件的类型被修改了	
	\item R 文件名被修改了
	\end{itemize}
\item git add . 提交所有文件
\item git diff 查看暂存区和工作区的变动
\item git commit -m 带有注释的添加到版本库
\item git reset HEAD 取消缓存区的文件
\item git rm 删除工作区里的文件
\item git rm -f 删除暂存区里的文件
\item git rm \verb|--|cached 从暂存区域移除,但保留在当前工作区中
\item git rm -r 递归删除
\item git mv 重命名
\\
\item git log \verb|--|oneline 查看版本历史
\item git reset \verb|--|hard 版本回退
\\
\item git branch 列出分支或者创建分支
\item git checkout 切换分支
\item git branch -d(D) 删除分支
\item git merge 融合分支
\\
\item git tag -a 添加一个带注解的标签
\item git log --decorate 带标签的查看版本历史
\end{itemize}
\section{一些问题和体会}\setlength{\parindent}{2em}
这里边还有很多的命令都没有写,只是列举了一些基础的命令,还有比如说分支的操作,标签的操作可以在progit中来学习,上面写的都比较清楚。由于是第一次写可能存在一些错误,和理解上的偏差。这次报告目的只是为git初学者来更好理解git,因为我第一回看progit时候真的很多地方都比较懵逼,花的时间也比较多,所以想写这次报告,把自己的一些心得写下来。

\begin{itemize}
\item git config --global core.quotepath false\\解决中文乱码
\end{itemize}

\end{document} 