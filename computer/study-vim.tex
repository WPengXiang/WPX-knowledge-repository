\documentclass{article}
\input{../en_preamble.tex}
\input{../xecjk_preamble.tex}
\begin{document}
	\title{vim学习}
	\author{王鹏祥}
	\date{\chntoday}
	\maketitle
	\newpage

\section{vi,vim优势}
\begin{itemize}
	\item 所有linux都内置vi文本编辑器
	\item {\color{red}巨多}软件编辑接口会主动调用vi(比如说texstudio很多时候编译不了,vim都可以编译)
	\item vim有程序编辑能力(以字体颜色辨别语法正确)
	\item 程序简单,编译快速
\end{itemize}
\section{使用}
参考网址:\url{https://www.cnblogs.com/uncle-qi/p/9356465.html}
参考网址:\url{https://www.imooc.com/learn/1129}
\begin{itemize}
	\item 一般模式 一打开文件的时候所处模式,可以进行删除、复制、黏贴等操作.
	\item 编辑模式  可以编辑内容,按i进入编辑模式,[esc]进入一般模式
	\item 命令行模式 进行数据查找,大量字符替换等操作,按"/""?"":"进入
\end{itemize}

\begin{figure}[H]
\centering
\includegraphics[scale=0.5]{figures/vim}
\end{figure}
\subsection{常用操作}
\begin{itemize}
	\item vim 文件名 \\打开文件 
	\item :wq \\保存并退出文件 
	\item :e! \\放弃修改,编辑区回复原样
	\item :w 路径/新文件名 另存为
	\item u \\撤销
	\item ctrl+r \\反撤销
	\item . \\执行上一个命令
	\item !shell命令 \\与shell命令行进行交互
	\item gq \\重新排版
\end{itemize}

\subsection{光标定位}
	\begin{itemize}
	\item :n \\跳转到第n行
	\item \$ \\光标定位到行尾
	\item 0 \\光标移动到行首
	\item pageup、pagedown \\向上向下翻页
	\item G \\光标移动到最后一行 
	\item 数字+上下左右 \\移动光标多少行 
	\end{itemize}
\subsection{删除}
一般要处于命令模式下
	\begin{itemize}
	\item d0 \\删除光标左边的文本
	\item d\$ \\删除光标右边的文本
	\item dd \\删除(剪切)光标所在行的文本
	\item dG \\删除光标所在行之后的所有行
	\end{itemize}
	\remark{上面的命令前加数字代表删除的范围扩大相应的倍数}
\subsection{剪切复制黏贴}
	\begin{itemize}
	\item yy \\复制光标所在的行
	\item p \\黏贴
	\item dd \\剪切光标所在的行
	\item 可视模式下加y,比较方便
	\item 
	\end{itemize}
\subsection{替换和插入}
	\begin{itemize}
	\item :[替换起始处,替换结束处] s/被替换的字符/替换的字符/[g][c]
	\item \$代表行尾	
	\item g选项表示替换目标行中所有匹配的字符
	\item c表示替换以互动形式进行
	\item \^代表行首,\$代表行尾
    \item 行首行尾插入只要在被替换的字符换上面的符号即可
	\end{itemize}
\subsection{查找}
	\begin{itemize}
	\item 向下(上)寻找一个名称为word的字符串 /word(?word)
	\item 重复前一个查找操作 n(N为反向)
	\end{itemize}
\subsection{多窗口编辑}
	\begin{itemize}
	\item :sp 文件名2 \\以水平打开第二个文件的窗口
	\item ctrl+w h(j、k、l) \\切换窗口
	\item close \\关闭当前窗口
	\item vi -O filenames \\在垂直分割的多个窗口中编辑多个文件。
	\item vi -o filenames \\在水平分割的多个窗口中编辑多个文件。
	\item ctrl+w H \\窗口调整为垂直分割
	\item ctrl+w K \\窗口调整为水平分割
	\end{itemize}
\end{document}








































