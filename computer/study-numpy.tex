% !Mode:: "TeX:UTF-8"
\documentclass[12pt,a4paper]{article}
\input{../en_preamble.tex}
\input{../xecjk_preamble.tex}
\title{Numpy}
\usepackage{pythonhighlight}
\author{王鹏详}
\date{\chntoday}
\begin{document}
	\maketitle
	\newpage
	\section{视图}
	\begin{itemize}
		\item reshape
		\item 切片索引
	\end{itemize}
	\section{广播}
	可以广播的情况
	\begin{itemize}
		\item  两个数组各维度大小从后往前比对均一致
		\item  两个数组存在一些维度大小不相等时,有一个数组的该不相等维度大小为1
	\end{itemize}
	\section{多维数组组装}
	\begin{itemize}
		\item np.ones
		\item np.zeros
		\item np.linspace
		\item np.add.at(a,indices,data),重复指标会累加(multiply,negative,divide,
				logical\_and,logicla\_or)
		\item a[indeces] += data ,中复指标不会累加,只会记录最后一个
	\end{itemize}
	\section{原则}
	\begin{itemize}
		\item 尽量避免复制数据
		\item 默认行优先
		\item 利用视图和广播
		\item 利用数组化编程
	\end{itemize}
	\section{爱因斯坦求和}
	\textbf{语法}
	\begin{figure}[H]
		\centering
		\includegraphics[width=0.7\linewidth]{figures/einsum}
		\caption{}
		\label{fig:einsum}
	\end{figure}
	\textbf{常见操作:}
	\begin{itemize}
		\item 转置$B_{ij} = A_{ij}$  
			\begin{python}
			a = np.arange(0, 9).reshape(3, 3)
			b = np.einsum('ij->ji', a)
			\end{python}		
		\item 全部元素求和$\sum\limits_{ij} A_{ij}$  
			\begin{python}
			a = np.arange(0, 9).reshape(3, 3)
			b = np.einsum('ij->', a)
			\end{python}
		\item 某一维度求和$\sum\limits_j A_{ij}$  
			\begin{python}
				a = np.arange(0, 9).reshape(3, 3)
				b = np.einsum('ij->i', a)
			\end{python}
		\item 矩阵相乘$C_{ij}=\sum\limits_{k}A_{ik}B_{kj}$
			\begin{python}
				a = np.arange(0, 12).reshape(3, 4)
				b = np.arange(0, 12).reshape(4, 3)
				c = np.einsum('ik,kj->ij', a, b)
			\end{python}
	\end{itemize}
	
	\section{稀疏矩阵}	
\end{document}



