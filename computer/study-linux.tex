% !Mode:: "TeX:UTF-8"
\documentclass{article}
\input{../en_preamble.tex}
\input{../xecjk_preamble.tex}
\begin{document}
\title{linux 学习}
\author{王鹏祥}
\date{\chntoday}
\maketitle
\tableofcontents
\newpage

\section{计算机概论}
计算机五大单元:输入单元,输出单元,cpu内部的控制单元和算数逻辑单元,内存

进制不一样到时硬盘的表示容量(10进制)和实际容量(2进制)不一样

操作系统是用来管理所有硬件,像linux,windows这样的都是操作系统

\section{Linux相关知识}
linux是一套操作系统用来提供一整组接口来给软件程序员开发用。

\subsection{linux特色}
\begin{itemize}
\item 自由和开放的使用和学习环境
\item 配备需求廉价
\item 内核(操作系统的核心,用于管理系统资源)功能强大而稳定
\item 独立作业
\end{itemize}

\subsection{linux优点}
\begin{itemize}
\item 稳定的系统
\item 免费或者少许费用
\item 安全和漏洞的快速修补
\item 多任务多用户
\item 相对比较不消耗资源
\end{itemize}

\subsection{linux缺点}
\begin{itemize}
\item 没有特定的支持厂商
\item 游戏的支持度不足
\item 专业软件的支持度不足
\end{itemize}

\section{命令行}
command  [-options]  parameter1  parameter2 ...

指令 \qquad 选项  \qquad   参数(1) \qquad  参数(2)
\begin{itemize}
\item 一行命令中第一个输入的部分绝对是『命令(command)』或『可运行文件案』
\item command 为命令的名称,例如变换路径的命令为 cd 等等; 
\item 中刮号[]并不存在于实际的命令中,而加入选项配置时,通常选项前会带 - 号, 
    例如 -h;有时候会使用选项的完整全名,则选项前带有 -- 符号,例如 --help; 
\item parameter1 parameter2.. 为依附在选项后面的参数,或者是 command 的参数;  
\item 命令, 选项, 参数等这几个咚咚中间以空格来区分,不论空几格 shell 都视为一格;  
\item 按下[Enter]按键后,该命令就立即运行。[Enter]按键代表着一行命令的开始启动。 
\item 命令太长的时候,可以使用反斜杠 ($\backslash$) 来跳脱[Enter]符号,使命令连续到下一行。 
     注意!反斜杠后就立刻接特殊字符,才能跳脱! 
\item  在 Linux 系统中,英文大小写字母是不一样的。举例来说, cd 与 CD 并不同。 
\end{itemize}
在命令行里执行语句主要有两种情况,一种是会直接显示结果回到命令提示符等待下一个命令输入,另一个是进入命令的环境直到结束该命令才回到命令提示符的环境。

\subsection{基础命令}
\begin{itemize}
\item reboot \\ 重启
\item poweroff、halt、shutdown -h \\关机
\item locate \\将所有与被查询的文件名相同的文件找出来
\item whereis \\找到可执行命令的局对路径
\item grep 关键字 文件名 \\查找制定文件总的特定字符串
\item man\\显示某个命令的格式用法
\end{itemize}

\subsection{常用热键}
\begin{itemize}
\item Tab \\补全命令和文件补齐
\item ctrl+c \\中断目前程序
\item ctrl+d \\ 键盘输入结束(相当于输入exit)

\end{itemize}

\section{linux文件管理}
\begin{itemize}
\item cd切换目录,pwd显示当前目录,mkdir新建一个新的目录,rmdir删除一个空目录
\item . 此层目录,..上一层目录,-前一个工作目录,$\~$ 目前用户所在的主文件夹
\item rm -r删除目录下所有文件,*为通配符,-i互动删除会先询问
\item ls -a列出全部文件包括隐藏文件,-d仅列出目录本身,-l列出长数据串(包括文件属性和权限)
\item cp -s复制成快捷方式,-r递归的持续复制用于目录复制,-p连用文件的属性一起复制过去,-i若目标文件已经存在覆盖时会询问操作的进行,-d若源文件为接文件这复制链接文件属性,-a相当于-pdr
\item mv -f若果目标已经不会询问直接覆盖,-i已经存在时会询问
\item cat -n现实文档内容会列出行号空白行也会有行号
\item tac 反向显示出内容
\item more和less翻页查询内容
\item whereis(寻找特定文件)或者which(寻找执行文件)或者locate(不分搜索)
\item nautilus .打开当前文件夹
\item echo snap >> ~/.hidden 将snap文件隐藏(就是在该目录下建立.hidden文本,在里边写上你要隐藏的文件即可)
\item ln -s [要创建的文件或文件夹] [软链接存放位置] \\建立软连接
\item ln [要创建的文件或文件夹] [软链接存放位置] \\建立硬连接
\end{itemize} 

%\cite{fem_2010}
%\bibliographystyle{abbrv}
%\bibliography{../../ref}

\end{document}
