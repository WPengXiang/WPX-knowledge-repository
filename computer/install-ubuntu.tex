% !Mode:: "TeX:UTF-8"
\documentclass[12pt,a4paper]{article}
\input{../en_preamble.tex}
\input{../xecjk_preamble.tex}
\title{windows下ubuntu的系统安装}
\author{王鹏详}
\date{\chntoday}
\begin{document}
	\maketitle
	\newpage
	\subsection{制作镜像}
	1.下载rufush镜像制作软件\\
	github.com/pbatard/rufus\\
	2.从ubuntu上下载最新LTS版本\\
	3.装上u盘制作ubuntu启动盘
	\subsection{ubuntu安装前期配置}
	1.进入bios界面,修改安全模式为AHCI(dell是f12)\\
	2.将bios设为uefi\\
	3.在advanced上选择enable\\
	\subsection{ubuntu安装}
	\begin{figure}[H]
		\centering
		\includegraphics[width=0.7\linewidth]{figures/IMG_2810}
		\caption{}
		\label{fig:img2810}
	\end{figure}
	
	\begin{figure}[H]
		\centering
		\includegraphics[width=0.7\linewidth]{figures/IMG_2799}
		\caption{}
		\label{fig:img2799}
	\end{figure}
	\begin{figure}[H]
		\centering
		\includegraphics[width=0.7\linewidth]{figures/IMG_2809}
		\caption{}
		\label{fig:img2809}	
	\end{figure}
	除了分区和efi其他全部默认
	\subsection{ubuntu配置}
	1.在软件更新中换源,最好换成华为的\\
	2.更新系统sudo apt upgrade,sudo apt updata\\
	3.安装驱动(如果有问题的话,否则不用)sudo ubuntu-drivers autoinstall	\\
	4.移除ibus 安装fcitx 安装搜狗输入法 然后重启,右上角有一个小面板,在配置里边添加搜狗输入法\\
	5.安装常用软件sudo sh ubuntu\_install.sh , wps,配置google访问助手,将插件拖进拓展程序中\\
	6.配置wps缺失字体,讲fonts放到usr/share/fonts里\\
	7.安装latex sudo apt-get install texlive-full \\
	8.安装texmaker sudo apt-get install texmaker\\
	9.安装mathpix sudo apt-get mathpix-snipping-tool\\
	10.安装git等,参阅老师仓库中的feapy中readme文件\\
	11.配置自己git,参阅老师git.md\\
	12.创建自己仓库repository\\
	13.安装vim,参阅vim文件中readme,git clone --recurse-submodules https://github.com/weihuayi/whyvim.git\\
	14.安装matlab\\
	15
	\begin{itemize}
	 \item sudo apt install git     \qquad        版本控制软件 
	 \item sudo apt install python3     \qquad    Python3 自带解释器 
	 \item sudo apt install python3-pip  \qquad   Python3 软件包安装、卸载和升级管理器. 
	 \item sudo apt install python3-tk   \qquad   Python interface to Tcl/Tk used by matplotlib 
	 \item sudo apt install spyder3     \qquad    集成开发环境 
	 \item sudo -H pip3 install numpy      \qquad     多维数组模块 
	 \item sudo -H pip3 install scipy       \qquad 科学计算模块 
	 \item sudo -H pip3 install matplotlib     \qquad 可视化模块
	 \item sudo -H pip3 install pandas         \qquad 数据分析和清洗模块 
	 \item sudo -H pip3 install jupyter \qquad Jupyter-notebook 
	 \item sudo -H pip3 install ipython
	 \item sudo -H pip3 uninstall prompt-toolkit \qquad 解决冲突问题 
	 \item sudo -H pip3 install jupyter\_ contrib\_ nbextensions \qquad Jupyter notebook 扩展功能模块 
	 \item sudo jupyter contrib nbextension install \qquad 把扩展功能模块安装入 Jupyter
	 \item sudo -H pip3 isntall RISE \qquad Jupyter notebook 的幻灯片扩展,可以把 Notebook 转化为 PPT 
	 \item sudo rm ~/.jupyter
	\end{itemize}

	
\section{anaconda安装}
\begin{itemize}
\item 从官网下载anaconda:https://www.anaconda.com/distribution/\# download-section
\item 到下载目录bash 文件名
\item 全程直接yes就可以了
\item 检查安装成功:到终端输入python3,conda --version,看是否有用
\item 添加环境变量 \\
echo 'export PATH="~/anaconda2/bin:echo 'export PATH=:\$PATH"' >> ~/.bashrc \\
source ~/.bashrc
\item conda update --all
\item conda update conda
\item conda update anconda
\end{itemize}

\subsection{打开可视化界面}
\begin{itemize}
\item 更新anaconda: conda update conda
\item 开启可视化界面:1 source ~/anaconda3/bin/activate root\\
           2 anaconda-navigator
\end{itemize}

\subsection{在桌面创建图标}
\url{https://blog.csdn.net/qq_43802597/article/details/98886365} 

注意命令中要改成自己的版本号和目录名
\begin{itemize}
\item 目录到/usr/share/applications
\item 创建anaconda.desktop
\item 修改文档,注意命令中要改成自己的版本号和目录名\\

\item 给权限 sudo chmod a+x anaconda-navigator.desktop
\end{itemize}
\subsubsection{参考网址}
	\url{https://www.dell.com/support/article/cn/zh/cnbsd1/sln301754/%E5%A6%82%E4%BD%95%E5%9C%A8-dell-pc-%E4%B8%8A%E5%AE%89%E8%A3%85-ubuntu-%E5%92%8C-windows-8-%E6%88%9610%E4%BD%9C%E4%B8%BA%E5%8F%8C%E5%BC%95%E5%AF%BC?lang=zh}
	\\
	\url{https://blog.csdn.net/lijiancheng0614/article/details/80866040}
	
	
\end{document}



